\renewcommand\bibname{Список литературы}
\begin{thebibliography}{199}
\addcontentsline{toc}{chapter}{\bibname}
\markboth{\sffamily Список литературы}{\sffamily Список литературы}

\subsection*{Общематематические книги}

\bibitem{Arnold} Арнольд~В.~И. Гюйгенс и Барроу, Ньютон и Гук. --- М.: Наука, 1989.
\bibitem{Knuth:Concrete} Грэхем~Р., Кнут~Д., Паташник~О. Конкретная математика. --- М.: Мир, 1998.
\bibitem{Klain} Клайн~М. Математика. Утрата неопределенности. --- М.: Мир, 1984.
\bibitem{Math} Математическая составляющая / Редакторы-составители Н.~Н.~Андреев, С.~П.~Коновалов, Н.~М.~Панюнин; Художник-оформитель Р.~А.~Кокшаров. --- М.: Фонд <<Мате­мати­ческие этюды>>, 2015.
\bibitem{Curant} Курант~Р., Роббинс~Г. Что такое математика? --- Изд. 7-е., стереот. --- М.: МЦНМО, 2015.
\bibitem{Problems} Проблемы Гильберта / под ред. П.~С.~Александрова --- М., Наука, 1969.
\bibitem{Reid} Рид~К. Гильберт. --- М.: Наука, 1977.



\subsection*{Логика и Теория множеств}

\bibitem{Beklemishev} Беклемишев~Л.~Д. \href{http://lpcs.math.msu.su/vml2008/}{Введение в математическую логику. Конспект лекций}. --- М.: МГУ, 2008.
\bibitem{Bekl} Беклемишев~Л.~Д. \href{http://www.mi-ras.ru/~bekl/Papers/goedel-uspehi.pdf}{Теоремы Гёделя о неполноте
и границы их применимости.} // Успехи Математических Наук. --- 2010. --- Т.65, N5.
\bibitem{Burbaki} Бурбаки~Н. Архитектура математики // Очерки по истории математики. --- М.: ИИЛ, 1963. --- С.245--259.
\bibitem{Vereschagin} Верещагин~Н.~К., Шень~А. \href{https://www.mccme.ru/free-books/shen/shen-logic-part1-2.pdf}{Начала теории множеств}. --- М.: МЦМНО, 2012. 
\bibitem{Vereschagin2} Верещагин~Н.~К., Шень~А. \href{http://math-info.hse.ru/f/2014-15/CompLingMasters/Vereshchagin-Shen2.pdf}{Лекции по математической логике и теории алгоритмов. Часть 2. Языки и исчисления}. --- М.: МЦНМО, 2012.
\bibitem{Vereschagin3} Верещагин~Н.~К., Шень~А. \href{http://www.mcnmo.ru/free-books/shen/shen-logic-part3-2.pdf}{Лекции по математической логике и теории алгоритмов. Часть 3. Вычислимые функции}. --- М.: МЦНМО, 2012.
\bibitem{Godel} Гёдель~К. Совместимость аксиомы выбора и обобщенной континуум-гипотезы с аксиомами теории множеств // Успехи мат.
наук. --- 1948. --- Т.8, вып.1. --- С.96--149.
\bibitem{Gilbert} Гильберт~Д., Бернайс~П. Основания математики. ---  в 2-х томах. --- М.: Наука, 1979--1982.
\bibitem{Gudstein} Гудстейн~Р.~Л. Математическая логика. --- М.: URSS, 2010.
\bibitem{Ershov} Ершов~Ю.~Л. $\Sigma$-определимость и теорема Гёделя о неполноте: Учебное
пособие. --- Новосибирск: Научная книга, 1995.
\bibitem{Jech-rus} Йех~Т. Теория множеств и метод форсинга. --- М. Мир, 1973.
\bibitem{Kiselev1} Киселев~А.~А. \href{https://arxiv.org/pdf/1110.0642.pdf}{Недостижимость и субнедостижимость},
Часть I. --- Минск.: Бел. гос. ун-т, 2011.
\bibitem{Kiselev2} Киселев~А.~А. \href{https://arxiv.org/pdf/1110.0643.pdf}{Недостижимость и субнедостижимость},
Часть II. --- Минск.: Бел. гос. ун-т, 2011.
\bibitem{Klini} Клини~С.~К. Математическая логика. --- М.: Мир, 1973.
\bibitem{Surreal-Knuth} Кнут~Д.~Э. Сюрреальные числа / Перевод Н. Шихова. --- М.: «Бином. Лаборатория знаний», 2014.
\bibitem{Dragalin} Колмогоров~А.~Н., Драгалин~А.~Г. Математическая логика. Дополнительные главы: Учеб. пособие. --- М.: Изд-во Моск. ун-та, 1984.
\bibitem{Kohen} Коэн~П.~Дж. Теория множеств и континуум-гипотеза. --- М.: Мир, 1969.
\bibitem{Kuratowski} Куратовский~К., Мостовский~А. Теория множеств. --- М.: Мир, 1970.
\bibitem{Petrovsky} Петровский~А.~Б. Пространства множеств и мультимножеств. --- М.: Едиториал УРСС, 2003.
\bibitem{Ponomarev} Пономарев~И.~Н. \href{http://inponomarev.ru/math.pdf}{Введение в математическую логику и роды структур}: Учебное пособие. --- М.:МФТИ, 2007.
\bibitem{Barwise} Справочная книга по математической логике в 4-х частях; под ред. Дж. Барвайса; Ч.2 Теория множеств. --- М., Наука, 1982.
\bibitem{Alling} Alling,~Norman~L. Foundations of Analysis over Surreal Number Fields. // Mathematics Studies. --- North-Holland, 1987. --- 141.
\bibitem{Conway} Conway~J.~H. \href{https://books.google.ru/books?id=tXiVo8qA5PQC&printsec=frontcover&hl=ru&source=gbs_ge_summary_r&cad=0#v=onepage&q&f=false}{On numbers and games}, second edition. --- A. K. Peters, 2001.
\bibitem{DushnikMiller} Dushnik~B. Miller~E.~W. Partially ordered sets // American Journal of Mathematics. --- 1941. --- Vol.63, N3. --- P.600--610.
\bibitem{Ehrlich} \href{https://www.ohio.edu/cas/philosophy/contact/profiles.cfm?profile=EAC10AF4-5056-A874-1DB87777A64C4268}{Ehrlich~Philip}. \href{http://lumiere.ens.fr/~dbonnay/files/talks/ehrlich.pdf}{The Absolute Arithmetic Continuum and the Unification of All Numbers Great and Small}. // The Bulletin of Symbolic Logic. --- 2012 --- V.18, N1. --- P.1--45.
\bibitem{Gentzen} Gentzen~G. Die Widerspruchsfreiheit der reinen Zahlentheorie // Mathematische Annalen. --- 1936. --- N112. --- P.493--565.
\bibitem{Gonshor} Gonshor~Harry. An Introduction to the Theory of Surreal Numbers. // London
Mathematical Society, Lecture Note Series 110. --- Cambridge University Press, 1986.
\bibitem{Henle} Henle~James~M. An Outline of Set Theory. --- New York etc.: Springer-Verlag, 1986. Русская версия: Хенл~Дж.~М. Введение в теорию множеств: Пер. с англ. --- М.: Радио и связь, 1993.
\bibitem{Jech} Jech~T.~J. The Axiom of Choice. --- Amsterdam etc.: North-Holland, 1973.
\bibitem{Paris} Kirby L., Paris J. \href{http://citeseerx.ist.psu.edu/viewdoc/summary?doi=10.1.1.107.3303}{Accessible independence results for Peano arithmetic} // Bulletin London Mathematical Society. --- 1982. --- V.14: P.285--293.
\bibitem{Levy} Levy~A. Basic Set Theory. --- Berlin etc.: Springer-Verlag, 1979.
\bibitem{Sierpinski} Sierpiński, Wacław, Cardinal and ordinal numbers. // Polska Akademia Nauk Monografie Matematyczne. --- Warsaw: 1958 --- N34. --- Państwowe Wydawnictwo Naukowe, MR 0095787.
\bibitem{Schwichtenberg} Schwichtenberg~H., Wainer~S.~S. Proofs and Computations Perspectives in Logic. --- Cambridge University Press, 2012.
\bibitem{Surreal} Tøndering~Claus \href{https://www.tondering.dk/download/sur.pdf}{Surreal Numbers --- An Introduction}, 2019.

\subsection*{Computer Science}

\bibitem{Baren} Барендрегт~Х. Ламбда-исчисление. Его синтаксис и семантика. --- М.: Мир, 1985.
\bibitem{Voevodin} Воеводин~В.~В., Воеводин~Вл.~В. Параллельные вычисления. СПб.: БХВ-Петербург, 2002.
\bibitem{Vorontsov} Воронцов~К.~В. \href{http://www.ccas.ru/voron/download/SVM.pdf}{Лекции по методу опорных векторов} [Электронный ресурс]. --- 2007.
\bibitem{Ershov2} Ершов~Ю.~Л. Определимость и вычислимость. Сибирская школа алгебры и логики.
--- Новосибирск: Научная книга, 1996; English transl., Ershov~Yu.~L.
Definability and computability, Siberian School of Algebra and Logic. --- New York: Consultants Bureau, 1996.
\bibitem{Knuth} Кнут~Д. Искусство программирования. Том 2. Получисленные алгоритмы. --- В 4-х томах. Пер. с англ. --- 3-е изд. --- М.: Вильямс, 2007.
\bibitem{MLbook} Мюллер~А., Гвидо~С. Введение в машинное обучение с помощью \Python. Руководство для специалистов по работе с данными. --- O'Reilly, 2017.
\bibitem{MLbook2} Орельен~Ж. Прикладное машинное обучение с помощью \texttt{Scikit-Learn} и \texttt{TensorFlow}. Концепции, инструменты и техники для создания интеллектуальных систем. --- O'Reilly, 2017.
\bibitem{Dongarra} Dongarra~J.~J., Duff~L.~S., Sorensen~D.~C., Vorst~H.~A.~V. Numerical Linear Algebra for High-Performance Computers (Software, Environments, Tools) // Soc. for Industrial \& Applied Math. --- 1999.

\subsection*{Алгебра и Теория чисел}

\bibitem{AIRLAND} Айерленд~К., Роузен~М.. Классическое введение в современную теорию чисел. --- М.: Мир, 1987.
\bibitem{Atlas} \href{http://brauer.maths.qmul.ac.uk/Atlas/v3/}{Атлас представлений конечных групп} [Электронный ресурс] --- Режим доступа: \href{http://brauer.maths.qmul.ac.uk/Atlas/v3/}{http://brauer.maths.qmul.ac.uk/Atlas/v3/}, свободный.
\bibitem{Artin} Артин~Э. Теория Галуа. Пер. с англ. А.~В.~Самохина. --- М.: МЦМНО, 2004.
\bibitem{BuLi1} Бурбаки~Н. Группы и алгебры Ли. Главы I—III. --- М.: Мир, 1976.
\bibitem{BuLi2} Бурбаки~Н. Группы и алгебры Ли. Глава IX. --- М.: Мир, 1986.
\bibitem{VanderVarden} Ван~дер~Варден~Б.~Л. Алгебра. --- М.: Наука, 1976.
\bibitem{Winberg} Винберг~Э.~Б. Курс алгебры --- М.: Факториал Пресс, 2001.
\bibitem{Gorod} Городенцев~А.~Л. Алгебра: Учебник для студентов-математиков. --- М.: факультет математики ВШЭ, 2011.
\bibitem{Korn} Корн~Г., Корн~Т. Алгебра матриц и матричное исчисление // Справочник по математике. --- 4-е издание. --- М: Наука, 1978.
\bibitem{Kostrikin} Кострикин~А.~И. Введение в алгебру. --- М. ФИЗМАТЛИТ, 2004.
\bibitem{Kurosh} Курош~А.~Г. Общая алгебра. --- М.: Наука, 1974.
\bibitem{Leng} Ленг~С. Алгебра. --- М.: Наука, 1971.
\bibitem{Pontriaghin} Понтрягин~Л.~С. Обобщения чисел. --- М.:Едиториал УРСС, 2018.
\bibitem{Postnikov} Постников~М.~М. Теория Галуа. --- М.: Факториал Пресс, 2003.
\bibitem{PrasolovAlg} Прасолов~В.~В. Многочлены. 4-е изд., испр. --- М.: МЦНМО, 2014.
\bibitem{Chovan} Хованский~А.~Г. Топологическая теория Галуа. Разрешимость и неразрешимость уравнений в конечном виде. --- М.: МЦНМО, 2008.
\bibitem{Chashkin} Чашкин~А.~В., Жуков~Д.~А. \href{http://ebooks.bmstu.press/catalog/117/book1467.html}{Элементы конечной алгебры}. --- М.: Изд. МГТУ им.Баумана, 2016.
\bibitem{NewScientist} \href{https://www.newscientist.com/article/2146647-baffling-abc-maths-proof-now-has-impenetrable-300-page-summary/}{Baffling ABC maths proof now has impenetrable 300-page ‘summary’} [Electronic Resourse]
\bibitem{GR1} Gorenstein~D., Lyons~R., Solomon~R. The classification of the finite simple groups. --- Providence, R.I.: American Mathematical Society, 1994. --- Vol.40.1. --- (Mathematical Surveys and Monographs).
\bibitem{GR2} Gorenstein~D., Lyons~R., Solomon~R. The classification of the finite simple groups. Number 2. Part I, chapter G: General group theory. --- Providence, R.I.: American Mathematical Society, 1996. --- Vol.40.2. --- (Mathematical Surveys and Monographs).
\bibitem{GR3} Gorenstein~D., Lyons~R., Solomon~R. The classification of the finite simple groups. Number 3. Part I, chapter A: Almost simple $\cal K$-groups. --- Providence, R.I.: American Mathematical Society, 1998. --- Vol.40.3. --- (Mathematical Surveys and Monographs).
\bibitem{GR4} Gorenstein~D., Lyons~R., Solomon~R. The classification of the finite simple groups. Number 4. Part II, chapters 1--4: Uniqueness theorems. --- Providence, R.I.: American Mathematical Society, 1999. --- Vol.40.4. --- (Mathematical Surveys and Monographs).
\bibitem{GR5} Gorenstein~D., Lyons~R., Solomon~R. The classification of the finite simple groups. Number 5. Part III, chapters 1--6: The generic case, stages 1–3a. --- Providence, R.I.: American Mathematical Society, 2002. --- Vol.40.5. --- (Mathematical Surveys and Monographs).
\bibitem{GR6} Gorenstein~D., Lyons~R., Solomon~R. The classification of the finite simple groups. Number 6. Part IV: The special odd case. --- Providence, R.I.: American Mathematical Society, 2005. --- Vol.40.6. --- (Mathematical Surveys and Monographs).
\bibitem{GR7} Gorenstein~D., Lyons~R., Solomon~R. The classification of the finite simple groups. Number 7. Part III, chapters 7--11: The generic case, stages 3b and 4a. --- Providence, R.I.: American Mathematical Society, 2018. --- Vol.40.7.
\bibitem{MilesReid} Reid~M. \href{https://homepages.warwick.ac.uk/~masda/MA3D5/Galois.pdf}{Galois Theory}. --- University of Warwick, Coventry, 2014.
\bibitem{Solomon} Solomon~R. \href{http://www.ams.org/bull/2001-38-03/S0273-0979-01-00909-0/S0273-0979-01-00909-0.pdf}{A brief history of the classification of the finite simple groups} // American Mathematical Society. Bulletin. New Series. --- 2001. --- Т.38, вып.3. --- С.315--352.


\subsection*{Анализ, Геометрия, Топология}

\bibitem{Borovkov} Боровков~А.~А. Теория вероятностей: Учеб. пособие для вузов. --- М.: Наука, 1986.
\bibitem{Berwald} Гарасько~Г.~И., Кокарев~С.~С., Тришин~В.~Н., Балан~В., Бринзей~Н., Сипаров~С.~В., Чернов~В.~М., Панчелюга~В.~А.  
\href{http://hypercomplex.xpsweb.com/articles/607/ru/pdf/book_lectures-opt.pdf}{Основы финслеровой геометрии и ее приложения в физике} // Материалы Международной школы-семинара для старшекурсников, аспирантов физико-математических
факультетов и молодых ученых. --- М.: МГТУ им.Н.Э.Баумана, 2010.
\bibitem{Infinitezim} Гордон~Е.~И., Кусраев~А.~Г., Кутателадзе~С.~С. Инфинитезимальный анализ. --- Новосибирск: Институт математики, 2006.
\bibitem{Domrin} Домрин~А.~В., Сергеев~А.~Г. \href{http://www.mi-ras.ru/books/pdf/ser1.pdf}{Лекции по комплексному анализу. В 2 частях}. --- М.: МИАН, 2004.
\bibitem{KolmFomin} Колмогоров~А.~Н., Фомин~С.~В. Элементы теории функций и функционального анализа. --- М.: Физматлит, 2009.
\bibitem{Kramer} Крамер~Г. Математические методы статистики. --- М.: Мир, 1975.
\bibitem{Landivschiz} Ландау~Л.~Д., Лифшиц~Е.~М. Квантовая механика (нерелятивистская теория). --- Издание 4-е. --- М.: Наука, 1989.
\bibitem{Massi} Масси~У., Столлингс~Дж. Алгебраическая топология. Введение. --- М.: Мир, 1977. \textit{English version}: Massey~W. Algebraic Topology: An Introduction. 1967; Stallings~J. Group Theory and Three-Dimensional Manifolds. 1971.
\bibitem{Kokarev} Павлов~Д.~Г., Кокарев~С.~С. Алгебра, геометрия и физика двойных чисел. // Сб. трудов РНОЦ "Логос". --- Вып.9. --- 2014. --- С.7--124.
\bibitem{Plahov} Плахов~А.~Ю. \href{http://www.etudes.ru/data/localdocs/umn_plahov.pdf}{Рассеяние в биллиардах и задачи ньютоновской аэродинамики} // Успехи математических наук. --- 2009. --- Т.64. Вып.5 (389). --- С. 97—166.
\bibitem{Prasol} Прасолов~В~.В. \href{https://yadi.sk/d/teZTJ7wC7L4r2}{Геометрия Лобачевского}. --- М.: МЦНМО, 2016.
\bibitem{Prasolov} Прасолов~В~.В., Тихомиров~В.~М. \href{http://prasolov.loegria.net/Geometry.pdf}{Геометрия}. --- М.: МЦНМО, 2007.
\bibitem{Sergeev} Сергеев~А.~Г. \href{http://www.mi-ras.ru/noc/14_15/ncgeom_2016.pdf}{Введение в некоммутативную геометрию}. [Электронный ресурс] --- 2016.
\bibitem{Siparov} Сипаров~С.~В. \href{https://siparov.com/sergey-siparov-books/}{Введение в анизотропную геометродинамику}. Нью Джерси --- Лондон --- Сингапур, World Scientific, 2011.
\bibitem{Sobolev} Соболев~В.~И. Лекции по дополнительным главам математического анализа. --- М.: Наука, 1968
\bibitem{Sosov} Сосов~Е.~Н. \href{https://dspace.kpfu.ru/xmlui/bitstream/handle/net/103716/SosovGeomLob!.pdf}{Геометрия Лобачевского и ее применение в специальной теории относительности}: учеб.-метод. пособие. --- Казань: Казан. ун-т, 2016.
\bibitem{Uspensky} Успенский~В.~А. Что такое нестандартный анализ. --- М.: Наука, 1987.
\bibitem{Feinman} Фейнман~Р. Лейтон~Р. Сэндс~М. Фейнмановские лекции по физике. Выпуск 8,9. Квантовая механика. Учебное пособие. --- М.: Либроком, 2014.
\bibitem{Fomenko} Фоменко~А.~Т. Фукс~Д.~Б. Курс гомотопической топологии: Учеб. пособие для вузов. --- М.: Наука, 1989.
\bibitem{Helstrom} Хелстром~К. Квантовая теория проверки гипотез и оценивания. --- М.: Мир, 1979.
\bibitem{Chebotarev} Чеботарев~А.~М. \href{http://test.nccse.ru/bayane/books/chebotarev_probabtheor.pdf}{Введение в теорию вероятностей и математическую статистику для физиков}. --- М., МФТИ, 2008.
\bibitem{Shashkin} Шашкин~Ю.~А. Неподвижные точки. --- М.: Наука, 1989.
\bibitem{Elsgolz} Эльсгольц~Л.~Э. Вариационное исчисление: Учебник. --- М.: Издательство ЛКИ, 2019.

\bibitem{Alexandroff1} Alexandroff~A.~D. Additive set-functions in abstract spaces I // Матем. сборник 1940. --- V.8(50), N 2. P.307-348.
\bibitem{Alexandroff2} Alexandroff~A.~D. Additive set-functions in abstract spaces II // Матем. сборник 1941. --- V.9(51), N 3. P.563-628.
\bibitem{Alexandroff3} Alexandroff~A.~D. Additive set-functions in abstract spaces III // Матем. сборник 1943. --- V.13(55), N 2. P.169-293.
\bibitem{Berwald1} Bejancu~A., Farran~H.~R. Geometry of Pseudo-Finsler Submanifolds, --- Kluwer, Dordrecht, 2000.
\bibitem{Konn} Connes.~А. \href{http://www.alainconnes.org/docs/book94bigpdf.pdf}{Noncommutative Geometry}. [Electronic Resourse]
\bibitem{Engelking} Engelking~R. General Topolgy. --- Warszawa.: PWN, 1977. \textit{Русское издание}: Энгелькинг~Р. Общая отпология. --- М.: Мир, 1986.
\bibitem{Kanovei} Kanovei~V., Reeken~M. Nonstandard Analysis, Axiomatically. --- Berlin: Springer-Verl., 2004.
\bibitem{Makarios} Makarios~T.~J.~M. \href{https://arxiv.org/pdf/1306.0066.pdf}{A further simplification of Tarski’s axioms of geometry}. [Electronic Resourse] --- 2013.
\bibitem{Nelson} Nelson~E. \href{https://web.math.princeton.edu/~nelson/books.html}{Books on Edward Nelson's Home Page on the Princetom University site} [Electronic Resourse]
\bibitem{Nica} Nica~E.. \href{https://arxiv.org/pdf/1306.2380.pdf}{The Mazur-Ulam Theorem.} --- G\"ottingen: Mathematisches Institut, Georg-August-Universit\"at, 2013.
\bibitem{Polyanin} Polyanin~A.~D., Manzhirov~A.~V., Handbook of Integral Equations. --- CRC Press, Boca Raton, 1998.


\subsection*{Графы и ветвящиеся процессы}

\bibitem{KazGC} Казимиров~Н.~И. \href{http://www.mathnet.ru/links/16a8a95d8309a4912790e5ee8a71e614/dm212.pdf}{Возникновение гигантской компоненты в случайной подстановке с известным числом циклов}. // Дискрет. матем. --- 2003. --- Т.15, Вып.3. --- С.145–159.
\bibitem{KD} Казимиров~Н.~И. \href{https://dlib.rsl.ru/01002618702}{Леса Гальтона--Ватсона и случайные подстановки}: дис. ... канд.физ.-мат.наук.: 01.01.09; Институт прикладных математических исследований Карельского научного центра РАН. --- Петрозаводск, 2003. --- 127 с.  (\href{https://dlib.rsl.ru/01002656449}{Автореферат доступен по ссылке})
\bibitem{Lando} Ландо~С.~К. \href{https://electives.hse.ru/data/2018/06/01/1150198748/GraphTopology18.pdf}{Графы и топология}. [Электронный ресурс] --- 2018.
\bibitem{Sevast} Севастьянов~Б.~А. Ветвящиеся процессы. --- М.: Наука, 1971.
\bibitem{Harari} Харари~Ф. Теория графов. --- М.: Мир, 1973.
\bibitem{Bollobas} Bollob\'as, B. Random Graphs. --- Cambridge University Press, 2001.
\bibitem{Erdos-Renyi} Erd\"os, P. and R\'enyi, A. \href{https://www.renyi.hu/~p_erdos/1959-11.pdf}{On Random Graphs.} // Publicationes Mathematicae. --- 1959. --- {\bfseries 6}. --- P.290-297.
\bibitem{Erdos-Renyi2} Erd\"os, P. and R\'enyi, A. On the Evolution of Random Graphs. // Publ. Math. Inst. Hungar. Acad. Sci. --- 1960. --- {\bfseries 5}. --- P.17-61.
\bibitem{GilbertGraph} Gilbert~E.~N. \href{https://projecteuclid.org/euclid.aoms/1177706098}{Random Graphs. // Annals of Mathematical Statistics}. --- 30 (4). --- P.1141--1144.
\bibitem{Harari-Palmer} Harary~F., Palmer~E.~M. \href{https://books.google.ru/books?hl=en&lr=&id=ZrvSBQAAQBAJ&oi=fnd&pg=PP1&ots=jBntDVBl4d&sig=NKBI28Bg6pjKqLwW3QUcrw7wuZ0&redir_esc=y#v=onepage&q&f=false}{Graphical Enumeration} --- New York, Academic Press, 1973.
\bibitem{KazRG} Kazimirow~N. \href{https://arxiv.org/pdf/1904.01263.pdf}{On some estimates for Erd\"os-R\'enyi random graph}. [Electronic Resourse] --- 2015.
\bibitem{Kolchin} Kolchin~V.~F. Random Graphs. --- New York: Cambridge University Press, 1998.
\bibitem{MeirMoon} Meir~A., Moon~J.~W. \href{https://doi.org/10.4153/CJM-1978-085-0}{On the altitude of nodes in random trees} // Can. J. Math. --- 1978. --- 30, N5. --- P.997–1015.
\bibitem{Pavlov} Pavlov~Yu.~L. Random Forests. --- Utrecht: VSP, 2000.
\bibitem{Pitman} Pitman~J. \href{https://statistics.berkeley.edu/tech-reports/482}{Enumerations of trees and forests related to branching processes and random walks}. // In. D. Aldous and J. Propp, editors, Microsurveys in
Discrete Probability. №41 in DIMACS Ser. Discrete Math. Theoret. Comp.
--- Sci. Providence RI, Amer. Math. Soc., 1998. P.163–180.
\bibitem{Polya} Polya~G. Kombinatorische Anzahlbestimmungen f\"ur Gruppen, Graphen und chemische Verbindungen. // Acta Math. --- 1937. --- 68. --- P.145--254.
\end{thebibliography}

