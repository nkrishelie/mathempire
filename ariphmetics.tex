\newchapter{Геометрическая алгебра}

\begin{figure}[htb!]
\begin{center}
\includegraphics[scale=0.3]{Gealgebra.png}
\end{center}
\caption{Вехи арифметики.}\label{ariphmetics}
\end{figure}

\section{Диэдральные группы}

\vrezka{Цель: знакомство с языком алгебры.}

\subsection{План}
\begin{enumerate}
\item Группа симметрий правильного треугольника, ее таблица Кэли.
\item Группа симметрий ромба (четверная группа Клейна), ее таблица Кэли.
\item Группа симметрий правильного многоугольника (снежинки).
\item \textit{Почему можно обойтись только одной симметрией для описания всех движений}?
\item Понятие группы $(G,\circ)$ и подгруппы, смежные классы, порядок элемента.
\item Несколько слов о базисе группы, порождающие эементы, эквивалентные базисы.
\item Базисы $\Sb_3$ и $V_4$.
\end{enumerate}

\subsection{Группа симметрий правильного треугольника}

Представим себе, что есть дверь и в ней замок треугольной формы (треугольник правильный). Вершины треугольника пронумерованы числами 1, 2, 3. Чтобы открыть дверь, нужно вставить в замок ключ (формы треугольной призмы) в правильном положении.
Углы ключа также пронумерованы цифрами 1, 2, 3, но это не означает корректного соответстия цифрам замка.
Вставить ключ можно как с одной стороны двери, так и с другой. Каковы шансы открыть дверь с первого раза?

Чтобы это описать математическим языком, рассмотрим все возможные соединения ключа и замка, которые сводятся к следующим действиям:
\begin{enumerate}[a)]
\item вставить ключ так, что его бородка вертикальна,
\item вынуть и повернуть ключ до совмещения следующих углов, снова вставить, и так далее,
\item те же действия с другой стороны двери.
\end{enumerate}

Таким образом, на треугольнике вводятся следующие элементарные операции, переводящие треугольник в себя (со сменой номеров вершин):
\begin{itemize}
\item $\id$ --- тождественное преобразование (ничего не меняем),
\item $R_\ph$ --- поворот на угол $\ph$, где $\ph\in\{120^o,240^o\}$,
\item $S_1$ --- симметрия относительно биссектрисы, проходящей через $1$-ю вершину треугольника (верхнюю).
\end{itemize}

Итого имеем 4 преобразования. Вопрос: \textit{могут ли быть еще какие-то преобразования и сколько их}?

Сразу же очевидно, что симметрию можно выполнять относительно двух оставшихся биссектрис, т.е. у нас добавляются симметрии $S_2$ и $S_3$.

\begin{thrm}
Преобразования правильного треугольника, при которых вершины переходят в вершины, а ребра в ребра с сохранением инцидентности (т.е. без разрушения треугольника), исчерпываются списком $\id, R_{120}, R_{240}, S_1, S_2, S_3$.
\end{thrm}
\pf
Заметим, что при указанных преобразованиях разные вершины всегда остаются разными и, кроме того, все вершины всегда переходят во все вершины (никакая не выпадает). Это значит, что всякое такое преобразование осуществляет перестановку вершин. Но все различные перестановки вершин таковы:
$$
\begin{pmatrix}
1 & 2 & 3 \\
1 & 2 & 3
\end{pmatrix},
\begin{pmatrix}
1 & 2 & 3 \\
2 & 3 & 1
\end{pmatrix},
\begin{pmatrix}
1 & 2 & 3 \\
3 & 1 & 2
\end{pmatrix},
\begin{pmatrix}
1 & 2 & 3 \\
1 & 3 & 2
\end{pmatrix},
\begin{pmatrix}
1 & 2 & 3 \\
3 & 2 & 1
\end{pmatrix},
\begin{pmatrix}
1 & 2 & 3 \\
2 & 1 & 3
\end{pmatrix}
$$
Нетрудно видеть, что эти перестановки в точности соответствуют преобразованиям $\id, R_{120}, R_{240}, S_1, S_2, S_3$.
\epf

Предполагая, что только одно положение ключа с одной стороны двери и одно --- с другой стороны соответствуют открыванию замка, мы теперь можем ответить на поставленный в начале вопрос: у нас ровно 2 шанса из 6, что замок будет открыт.

Предположим теперь, что ключом пользуется сразу несколько человек (например, этот ключ весит 100 кг, поэтому приходится меняться), и каждый из них проделывает строго одно из указанных преобразований (свое любимое). Назовем этих людей <<операторами>>. Какое преобразование получится, если два оператора подряд произведут одно из указанных преобразований?

Ясно, что ничего нового мы не получим, но мы хотим знать, сколько и каких операторов нужно, чтобы, чередуя их действия, получить все преобразования треугольника. Для этого составим таблицу Кэли (сначала выполняется преобразование в столбце, затем --- в строке):

\begin{table}\begin{center}
\begin{tabular}{c|c|c|c|c|c|}
$\id$     & $R_{120}$ & $R_{240}$ & $S_1$ & $S_2$ & $S_3$ \\
\hline
$R_{120}$ & $R_{240}$ & $\id$     & $S_2$ & $S_3$ & $S_1$ \\
\hline
$R_{240}$ & $\id$     & $R_{120}$ & $S_3$ & $S_1$ & $S_2$ \\
\hline
$S_1$     & $S_3$     & $S_2$     & $\id$ & $R_{240}$ & $R_{120}$ \\
\hline
$S_2$     & $S_1$     & $S_3$     & $R_{120}$ & $\id$ & $R_{240}$ \\
\hline
$S_3$     & $S_2$     & $S_1$     & $R_{240}$ & $R_{120}$ & $\id$\\
\hline
\end{tabular}
\caption{Таблица Кэли для треугольника.}\label{triangle}
\end{center}\end{table}

Таблица Кэли является аналогом таблицы умножения, если под умножением понимать композицию отображений, т.е. последовательное применение операторов. Заметим, что, в отличие от обычного умножения, операторы не коммутируют, т.е. результат их действий зависит от того, в каком порядке они работают. Например, $R_{120}S_1\ne S_1R_{120}$ (таблица не симметрична относительно главной диагонали).

Раскручивая эту таблицу, мы можем получить результат последовательного применения 3-х, 4-х и т.д. операторов. При этом, мы можем заметить, что произведение трех одинаковых поворотов подряд (их третья степень) дает $\id$, а произведение двух одинаковых симметрий подряд также дает $\id$. Условимся последовательное применение какого-либо оператора $k$ раз записывать как его степень.

Кроме того, можно заметить что $R_{120}^2=R_{240}$, что уже говорит о том, что один оператор $R_{240}$ можно заменить на степень $R_{120}$. Верно и обратное: $R_{240}^2=R_{120}$. Кроме того, все симметрии можно получить как комбинации одной симметрии и поворотов: $S_2=S_1R_{120}$, $S_3=S_1R_{240}=S_1R_{120}^2$.

Итак, видим, что для осуществления всех видов операций поворота и симметрии треугольника нам достаточно иметь двух операторов (одного поворота и одной симметрии) и применять их в различной последовательности. Любая пара поворот-симметрия образует базис всех преобразований треугольника в себя.

При этом, невозможно уменьшить базис, т.е. невозможно ограничиться только одной симметрией или только одним или двумя поворотами. Однако, композиция двух разных симметрий дает поворот ($R_{240}=S_1S_2$), а это значит, что в качестве базисных операторов можно брать и любые две различные симметрии!


\subsection{Группа симметрий ромба}





\addcommL{45 минут}



\section{Движения окружности}
 
\vrezka{Цель: разобраться с группой $O(2)$ и ее подгруппами.}

\textbf{Определение}: преобразование пространства (прямой/плоскости), сохраняющее размеры (попарные расстояния), называется \textbf{движением} (изометрией).

\textbf{План}:
\begin{enumerate}
\item Классификация движений окружности: лемма о гвоздях.
\item \textit{Почему можно обойтись только одной симметрией}? Все движения есть композиция вращений и одной выделенной симметрии.
\item Эквивалетность базисов группы движений: все вращения + одна симметрия, все симметрии.
\item Конечные подгруппы, соответствующие диэдральным и циклическим группам.
\item Бесконечные подгруппы: иррационалньость числа $\pi$ и группа $(\Z,+)$ (вращение на несоизмеримый с $\pi$ угол).
\item Арифметика остатков: конечные циклические группы и факторизация $\Z/n\Z$.
\end{enumerate}



\section{Движения и гомотетии вещественной прямой}

\vrezka{Цель: найти кольцо $(\R,+,\times)$.}


\textbf{План}:
\begin{enumerate}
\item Классификация движений прямой: аналог теоремы Шаля.
\item \textit{Почему можно обойтись только одной симметрией}? Все движения есть композиция смещений и одной выделенной симметрии (умножение на $-1$).
\item Эквивалетность базисов: все сдвиги + одна симметрия, все симметрии.
\item Все сдвиги образуют группу, изоморфную $(\R,+)$.
\item Действие группы $\Z$ на прямой. Понятие орбиты.
\item \textbf{Определение}: гомотетией с заданным центром и коэффициентом называется преобразование пространства (прямой/плоскости), при котором все векторы с началом в этом центре удлиняются на заданный коэффициент. Подобие на прямой --- это гомотетия + сдвиг.
\item Подобия на прямой можно описать с помощью кольца $(\R,+,\times)$.
\end{enumerate}



\section{Движения и подобия на плоскости}

\vrezka{Цель: найти кольцо $(\C,+,\times)$.}

\textbf{План}:
\begin{enumerate}
\item Классификация движений плоскости: теорема Шаля.
\item \textit{Почему можно обойтись только одной симметрией}? Все движения есть композиция параллельных переносов, поворотов и одного выделенного отражения (умножение на $-1$ вдоль одной оси).
\item Эквивалетность базисов: все параллельные переносы + все повороты + одна симметрия, все отражения.
\item Все параллельные переносы образуют группу, изоморфную $(\C,+)$.
\item Формула Эйлера и число $e$. Группа корней из 1. Связь умножения комплексных чисел со сложением в группе вычетов.
\item Мультипликаивная группа $|z|=1$, ее действие на комплексной плоскости. Орбиты.
\item Подобия на плоскости --- это поворотные гомотетии + параллельные переносы.
\item Подобия на плоскости описываются арифметикой кольца $(\C,+,\times)$.
\end{enumerate}



\section{Делимость в евклидовых кольцах}

\vrezka{Цель: общий вывод основной теоремы арифметики и ее следствий.}

\textbf{План}:
\begin{enumerate}
\item Понятие кольца.
\item Понятие нормы и обратимых элементов кольца.
\item Алгоритм Евклида деления с остатком.
\item Представление НОД двух чисел в виде линейной комбинации этих чисел.
\item Основная теорема арифметики. Факториальное кольцо.
\item Приложение к кольцам: многочленов, гауссовых чисел.
\item Примеры нефакториальных колец.
\item Несколько теорем теории делимости: МТФ, РТФ,...
\end{enumerate}














