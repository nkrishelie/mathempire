\newchapter{Геометрическая алгебра}

\begin{figure}[htb!]
\begin{center}
\includegraphics[scale=0.3]{Gealgebra.png}
\end{center}
\caption{Вехи арифметики.}\label{ariphmetics}
\end{figure}

\section{Диэдральные группы}

\vrezka{Цель: знакомство с языком алгебры.}

\textbf{План}:
\begin{enumerate}
\item Группа симметрий правильного треугольника, ее таблица Кэли.
\item Группа симметрий ромба (четверная группа Клейна), ее таблица Кэли.
\item Группа симметрий правильного многоугольника (снежинки).
\item \textit{Почему можно обойтись только одной симметрией для описания всех движений}?
\item Понятие группы $(G,\circ)$ и подгруппы, смежные классы, порядок элемента.
\item Несколько слов о базисе группы, порождающие эементы, эквивалентные базисы.
\item Базисы $\Sb_3$ и $V_4$.
\end{enumerate}



\section{Движения окружности}
 
\vrezka{Цель: разобраться с группой $O(2)$ и ее подгруппами.}

\textbf{Определение}: преобразование пространства (прямой/плоскости), сохраняющее размеры (попарные расстояния), называется \textbf{движением} (изометрией).

\textbf{План}:
\begin{enumerate}
\item Классификация движений окружности: лемма о гвоздях.
\item \textit{Почему можно обойтись только одной симметрией}? Все движения есть композиция вращений и одной выделенной симметрии.
\item Эквивалетность базисов группы движений: все вращения + одна симметрия, все симметрии.
\item Конечные подгруппы, соответствующие диэдральным и циклическим группам.
\item Бесконечные подгруппы: иррационалньость числа $\pi$ и группа $(\Z,+)$ (вращение на несоизмеримый с $\pi$ угол).
\item Арифметика остатков: конечные циклические группы и факторизация $\Z/n\Z$.
\end{enumerate}



\section{Движения и гомотетии вещественной прямой}

\vrezka{Цель: найти кольцо $(\R,+,\times)$.}


\textbf{План}:
\begin{enumerate}
\item Классификация движений прямой: аналог теоремы Шаля.
\item \textit{Почему можно обойтись только одной симметрией}? Все движения есть композиция смещений и одной выделенной симметрии (умножение на $-1$).
\item Эквивалетность базисов: все сдвиги + одна симметрия, все симметрии.
\item Все сдвиги образуют группу, изоморфную $(\R,+)$.
\item Действие группы $\Z$ на прямой. Понятие орбиты.
\item \textbf{Определение}: гомотетией с заданным центром и коэффициентом называется преобразование пространства (прямой/плоскости), при котором все векторы с началом в этом центре удлиняются на заданный коэффициент. Подобие на прямой --- это гомотетия + сдвиг.
\item Подобия на прямой можно описать с помощью кольца $(\R,+,\times)$.
\end{enumerate}



\section{Движения и подобия на плоскости}

\vrezka{Цель: найти кольцо $(\C,+,\times)$.}

\textbf{План}:
\begin{enumerate}
\item Классификация движений плоскости: теорема Шаля.
\item \textit{Почему можно обойтись только одной симметрией}? Все движения есть композиция параллельных переносов, поворотов и одного выделенного отражения (умножение на $-1$ вдоль одной оси).
\item Эквивалетность базисов: все параллельные переносы + все повороты + одна симметрия, все отражения.
\item Все параллельные переносы образуют группу, изоморфную $(\C,+)$.
\item Формула Эйлера и число $e$. Группа корней из 1. Связь умножения комплексных чисел со сложением в группе вычетов.
\item Мультипликаивная группа $|z|=1$, ее действие на комплексной плоскости. Орбиты.
\item Подобия на плоскости --- это поворотные гомотетии + параллельные переносы.
\item Подобия на плоскости описываются арифметикой кольца $(\C,+,\times)$.
\end{enumerate}



\section{Делимость в евклидовых кольцах}

\vrezka{Цель: общий вывод основной теоремы арифметики и ее следствий.}

\textbf{План}:
\begin{enumerate}
\item Понятие кольца.
\item Понятие нормы и обратимых элементов кольца.
\item Алгоритм Евклида деления с остатком.
\item Представление НОД двух чисел в виде линейной комбинации этих чисел.
\item Основная теорема арифметики. Факториальное кольцо.
\item Приложение к кольцам: многочленов, гауссовых чисел.
\item Примеры нефакториальных колец.
\item Несколько теорем теории делимости: МТФ, РТФ,...
\end{enumerate}














