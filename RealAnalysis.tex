\newchapter{Вещественный анализ}


\section{Структура поля вещественных чисел}


\vrezka{Цель: Изучить виды действительных чисел, научиться строить расширения полей, выйти на трансцендентные числа.}



\textbf{План}:
\begin{enumerate}
\item Формальное построение рациональных чисел.
\item Аксиомы вещественных чисел.
\item Аксиома полноты, принцип вложенных отрезков, архимедовость поля $\R$.
\item Алгебраические числа. Счетность $\A$. Примеры алгебры над $\Q$.
\item Приближения рациональными числами иррациональных. Цепные дроби. Примеры: григорианский и современный календарь, лист формата A4.
\item Теорема Лиувилля. Трансцендентность $e$ и $\pi$. Без доказательств.
\end{enumerate}


\section{Элементарные функции}
