Прикоснуться к бесконечности, научиться строить "циркуем и линейкой" и даже
одной линейкой, вычислять площади сферических треугольников, зная их углы – 
это и многое другое позволяет знание Математики, великолепие и красота которой 
превосходят великолепие и красоту всех остальных наук (даже вместе взятых!). 

Изучение математики развивает культуру рассуждения, умение доказывать и 
убеждать, которые необходимы каждому человеку, даже профессиональному 
гуманитарию. Курс рассчитан как на тех, кто интересуется математикой, так и 
на тех, кто не имеет о ней никакого представления.

ПРИМЕРНЫЕ ТЕМЫ ЭТОГО ГОДА:

Обычная и сферическая геометрии (куда надо лететь из Иркутска в Лиссабон?)
Что такое геометрия? Равные фигуры (треугольники и т.п.)? Понятие о движении.
Углы, равенство углов, описание группы движений сферы и плоскости. Теорема Шаля.
Основные приколы сферической геометрии, равенство треугольников по углам, площади.

Геометрическая алгебра: формулы x^2 - y^2, бином, сумма нечётных, подряд идущих.
Мотивация суммы квадратов - пифагоровы треугольники (геометрия!!), p= 4k+1 только. 
Рассуждение Спивака и Севы Воронова про суммы двух квадратов через инволюции.

Плоскость как комплексные числа (задача о разложении x^2 + y^2 на множители).
Умножение на комплексной плоскости через Шаля. Инверсия, сопряжение, картинки.
Вывод тригонометрических формул из геометрии. Конформные преобразования.
Числа (комплексные) как повороты. Кватернионы как повороты пространства.

Проективная геометрия и проективные преобразования. Школьная задача про
касательную одной линейкой. Дезарга, Паппа и три окружности на плоскости.
Проективная плоскость, связь со сферой. Бесконечная точка. Двойное отношение.
Сохранение двойного отношения это в точности дробно-линейные (по ситуации).

Пятый постулат Евклида, история вопроса. Геометрия Лобачевского. Модель
через логарифм двойного отношения, поиграться в это. Действие группы
перестановок на двойные отношения. Вообще о группе перестановок.

Замощения. Как устроен футбольный мяч. Формула Эйлера, запреты для
замощений на плоскости. Какие бывают паркеты. Открытые вопросы.

Эрлангенская программа Клейна.

ПРОШЛЫЙ ГОД:

Понятие о графах, обходах. Кёнигсберские мосты Эйлера.
Сколько шаров можно приставить к заданному? (И. Ньютон) Сколько шаров можно упаковать в пространство? А сколько семиугольников — на плоскости? Как именно? Задача о снабжении города пунктами доступа к благу.
Про Гренландских китов, для которых \pi=3.1. Что такое \pi? А что такое вообще число? Трансцендентное число? Рациональное? Почему \sqrt{2} иррационально? Соизмеримость, геометрическое доказательство несоизмеримости диагонали и стороны квадрата.
Треугольные и квадратные числа, уравнение Пелля, цепные дроби, приближения, биномиальные коэффициенты и степени.
Абсолютное доказательство. Пример с домино и шахматной доской (62 квадратика). Игра в 15 и теория групп. Пифагор и его доказательство (геометрическое).
Самый старший шахматист среди математиков и самый старший математик среди шахматистов — одно лицо? А самый сильный шахматист среди математиков и самый сильный математик среди шахматистов? О строгих логических рассуждениях.
Во сколько цветов красится плоскость? Карта? Две проблемы четырёх красок: одна решённая, другая – нет.
Сиракузы, старейшая загадка мира. Динамические системы, алгоритмическая разрешимость и перечислимость.
Целые числа, простые близнецы, совершенные числа, загадка Пьера Ферма и другие диофантовы уравнения. Специальные числовые системы, понятие о делимости и т.п. Зачем нужны комплексные числа, кто они такие? Геометрия чисел.
Теория вероятностей. Задача про козла и три двери. Карточные загадки "что чаще встречается?". Дуэли двух лиц. Дуэли трёх лиц. Теоретико-игровые аспекты. Арбитраж.
Насколько далеко видно с горы? Вывод формулы. Понятие о приближении, о производной и т.п. Функции в естествознании, безмасштабность физических явлений.
Шутник и пальто. Перестановки и неподвижные точки. Теорема Брауэра на примере мыльных плёнок и скомканных карт местности, история из поезда. Что такое топология, связанные с ней понятия (открытость, замкнутость и т.п.)
Культура рассуждений. Задачка Гаусса. Индукция. Примеры рассуждений "от противного" (как Брауэр). Out of the box (про полусферы и теорему Ферма)
Теорема Тарского и встреча Нового Года в поезде.
Сюжеты с бесконечностью, как её пощупать. Сумма половинок, четвертинок и т.п. яблок. Мячик ударяется за конечное время всего бесконечное число раз! Дерево падает бесконечно долго, точнее, бесконечно долго встаёт в положение равновесия.
Ещё раз об иррациональности. Пелль, постулат Бертрана, биномиальные коэффициенты и степени. $x^3 \ne 2y^3 \pm 1$. Разложение $\sqrt[3]{2}$ в цепную дробь, пять методов обращения алгебраических чисел.
Из Зорича, где про производную: задачи естествознания.
Арнольд: перестановка пределов в задаче о бочке с дыркой.
Комбинаторика игры в Дубль: сколько подмножеств мощности $8$, пересекающихся попарно точно по одному элементу?
О демографии. Что такое "рождаемость"? Миллион разных определений, источники манипуляций. Критика Вишневского с позиций строгой математики (а не только по делу).
Математическая формализация "русской рулетки", прогноз.

Автор курса: Савватеев Алексей Владимирович, д.ф-м.н, доцент МФТИ и профессор 
ИМЭИ ИГУ, в.н.с. ЛИСОМО и ЦЭМИ РАН, с.н.с. ОРЭСП ИНЦ СО РАН, эксперт отдела 
теоретических и прикладных разработок ООО Яндекс, научный руководитель 
Лаборатории социального анализа при Университете Дмитрия Пожарского


На картинках видно много чего. Формула 1+2+3+....+n = n(n+1)/2
видна из картинки с двумя лесенками, приставленными друг к
другу с поворотом на 180 градусов. Формула a^2 - b^2 тоже из
картинки замечательно читается, как и бином Ньютона (a+b)^2.
Сумма нечётных чисел равна квадрату - это тоже картинка!

Затем мы берём кубический кусок сыра. Режем его так, чтобы
получилось 8 разных кусков. Бином (a+b)^3 готов. А что будет
при степени 4? Можно догадаться? Можно !!!!!

Дальше - уже треугольник Паскаля. Число путей в прямоугольном
городе (путей одинаковой длины), число подмножеств данной 
мощности, бином в чистом виде, рекурсия, вычисление C_n^k.


Сходящиеся и расходящиеся ряды. Гармонический ряд, прогрессия
(причины, почему она сходится), обратные квадраты и формула Эйлера.
Обратные квадраты через площади и через оценку сверху очевидным.