±0. О проекте "ПАНМАТЕМАТИКА"
https://www.youtube.com/watch?v=qpmjTVZ7H6s

бесплатно, регулярно. Будем приучаться к регулярности


+1. Начальная математическая подготовка
https://www.youtube.com/watch?v=IACUFpRksFg

0-00 Хан Академия на русском
1-00 число, измерять и единица измерения, линейная шкала
2-00 градусник - нарисованная шкала чисел
3-10 целые и дробные значения
3-20 корень из двух, не дробь


+2. Целые числа
https://www.youtube.com/watch?v=THISzILbEek

0-00 обозначение целых - Z
0-30 кратные эталону
0-40 умеем складывать вычитать и умножать
1-10 что значит сложить два целых числа
2-00 вектор
2-40 3+(-4)
3-10 обратная операция 3+(-4)+4=3
3-50 группа, всё вводим сразу

+3. Что такое группа
https://www.youtube.com/watch?v=_5Ffp0pP57g

0-10 [Z,+] целые числа, где позволяется осуществлять сложение
0-35 свойства сложения
0-50 коммутативные группы а+в=в+а
1-40 (а+в)+с = а+(в+с)
2-10 ноль. 0+а=а+0=а
2-55 нейтральный элемент группы
3-05 противоположное число а+(-а)=0
4-30 произвольное множество G и операция "*"
5-00 одну игрушку спариваем со второй...
5-50 многие выводы для любых коробок с игрушками

+4. Целые числа как группа
https://www.youtube.com/watch?v=YB3YtOCFwQQ

0-00 есть ли подмножества относительно сложения
3-00 любая подгруппа содержит ноль
4-30 ассоциативность
6-00 умножение целых чисел - следствие групповой идеи

+5. Коммутативность умножения. Подгруппы целых чисел
https://www.youtube.com/watch?v=POcQwUAIBKA

0-00 bn=nb
1-00 рисуем прямоугольник n на b
2-30 все ли подгруппы мы опиали?
3-40 утверждение: других подгрупп нет, локазательство

+6. Подгруппы целых чисел
https://www.youtube.com/watch?v=_AMVDYE05LM

1-00 подгруппа bZ
2-40 берем самый маленький элемент и докажем что все остальные кратные ему

+7. Деление с остатком
https://www.youtube.com/watch?v=VzoHvCicu9w

1-00 d=b-nc остаток от деления
3-00 Остаток положительный и отрицательный, по модулю меньше c

+8. На подступах к ОТА
https://www.youtube.com/watch?v=je4WjNncYGc

1-00 36 чему кратно?
1-30 простые числа
4-20 а есть ли число, которое можно разложить на простые по разному - эксперементально

+9. Формулировка ОТА
https://www.youtube.com/watch?v=aVcXfFNNJSo

0-00 формулируем построже ОТА
1-50 одинаковое количество сомножителей
2-10 существует перестановка набора p(i)=q(j)

+10. ОТА: Лемма о кузнечике
https://www.youtube.com/watch?v=J0Gb0SvYQII

1-00 с левой ноги кузнечик прыгает на а, с правой на б
3-00 точки попадания - подгруппа
4-40 отрезать две ноги и приделать новую

+11. Доказательство леммы о кузнечике
https://www.youtube.com/watch?v=u9EhAvY5_xE

0-00 точки попадания - подгруппа
5-00 а=dl b=kl d=am+bn d - новая нога НОД

+12. Определение группы
https://www.youtube.com/watch?v=csVfvFFoddc
0-00 для взаимнопростых
5-00 вводим операцию
6-00 Таблица умножения
8-00 аксиомы операции
