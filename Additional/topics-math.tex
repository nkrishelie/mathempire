ТЕОРИЯ ВЕРОЯТНОСТЕЙ НА ПАЛЬЦАХ

Что такое ``парадокс Монти-Холла'' и с чем его едят?
Ошибка или болезнь? Как правильно интерпретировать анализы.
``Непрерывная вероятность'': оценка реальных шансов на встречу.
Как не попасть ``на сахар'' на зоне? Игра ``Тюремный покер''
Как посчитать шансы в дуэли, если мы знаем меткость игроков?
======================================================
НОВЫЕ СЮЖЕТЫ:

1. Математическое наследие Эйлера, Гаусса и Ферма

2. `Теория вероятностей ``на пальцах''

3. Фортификационная и инженерная математика

4. Избранные олимпиадные задачи с полным их решением

5. Знакомство с комплексными числами через формулу Кардано
======================================================
МОИ ЛЮБИМЫЕ ОЛИМПИАДНЫЕ ЗАДАЧКИ

Есть несколько олимпиадных задач, которые просто поражают меня
своей красотой! Я познакомлю с ними слушателей. На закуску дам ещё
некоторое количество задач, чтобы слушатели потренировались сами!
======================================================
ГИПОТЕЗА ГОЛЬДБАХА И ТЕОРЕМА ВИНОГРАДОВА

Все «известные» чётные числа являются суммой двух простых;
однако доказать общее утверждение про ВСЕ чётные числа пока
не получается. Эта проблема называется Гипотезой Гольдбаха.
В то же время в 1937 году Виноградов сумел доказать, что все
НЕЧЁТНЫЕ числа, начиная с достаточно большого, являются
либо простыми либо суммой трёх простых. Я расскажу про
эти две задачи, а также про некоторую их «окрестность».
======================================================
МАТЕМАТИЧЕСКИЕ ЗАРИСОВКИ ОКРУЖАЮЩЕГО МИРА

Хотите узнать:

- Какие удивительные перипетии случаются на футбольных чемпионатах
 мира и Европы (включая наш, прошлогодний), и как в них разбираться?

- Далеко ли видно с горы Эверест (8848 метров), высочайшей точки планеты?

- Сколько в вашем регионе автомобилей?

- Может ли быть так, что команда А сильнее команды Б, команда Б сильнее 
команды В, но при всём при том команда В сильнее, а не слабее команды А? 

-  Почему лист бумаги имеет размер именно 210 на 297?

-  Как правильно играть в ``Карандаш, огонь, вода и железная рука''?

Тогда приходите на лекцию Алексея Савватеева! 
===========================================================
ГЕОМЕТРИЧЕСКАЯ АЛГЕБРА

Многие формулы школьной математики, из наиболее ``(не)любимых'' учениками,
можно увидеть своими собственными глазами, если присмотреться. Знаменитую
теорему Пифагора так иногда и доказывают, но есть целый ряд других примеров
``священнодействия'': формула квадрата разности, формула суммы двух кубов,
правила раскрытия скобок, бином Ньютона (квадрат и куб суммы), многое ещё.

Из деликатесов: ``увидеть'' можно даже теорему Косинусов! Я подробно 
обо всём этом расскажу, ну и отвечу на вопросы зрителей, конечно же!
===========================================================
ТЕОРИЯ ИГР ВОКРУГ НАС (Балахна, 26 сентября)

Оглянемся вокруг. Почти все явления социальной, политической и
экономической жизни пестрят стратегическим взаимодействием.

В процессе взаимодействия "игроки" пытаются просчитать ходы
друг друга как можно дальше и глубже, как в шахматах и шашках.

Результат порой выглядит весьма неожиданным.

Мы сперва смоделируем "игру" прямо со слушателями 
в аудитории, а затем разберём следующие сюжеты:

0. Карандаш, огонь, вода и железная рука.
1. Задача о парковочных местах aka "Белый Аист".
2. Люксембург в Евросоюзе.
3. Парадокс Брайеса в Метрогородке (Москва).
4. Модель социального раскола.
5. Тюремный покер

Приходите - вы увидите мир вокруг вас в совершенно новом обличии!
====================================================
Игры: прятки двух видов; карандаш огонь вода; лотерея; тюремный покер;
покер Маша - Ваня; покер в казино; двусторонний аукцион; Белый Аист;
теория устойчивых бракосочетаний; налогообложение + биткойны и т.д.
====================================================
ЧТО ЕСТЬ ОБЩЕГО МЕЖДУ ШАРЫГИНСКИМИ ТРЕУГОЛЬНИКАМИ,
СЛОЖЕНИЕМ ЯБЛОК, ПОСЛЕДНЕЙ (ВЕЛИКОЙ) ТЕОРЕМОЙ ФЕРМА
И ФУНКЦИОНИРОВАНИЕМ КРИПТОВАЛЮТ, НАПРИМЕР БИТКОЙНА?

ОТВЕТ: ЭЛЛИПТИЧЕСКИЕ КРИВЫЕ! (Нижний Новгород, 26 сентября)

Мой доклад будет вводным, я расскажу об эллиптических кривых и
операции сложения точек на них. Интерес к ним связан с несколькими
не похожими друг на друга задачами:

- описание всех целочисленных "Шарыгинских" треугольников, у которых
  треугольник в основании биссектрис - равнобедренный (а исходный - нет);

- детская задачка про то, сколько должно быть яблок, бананов и апельсинов
  в трёх корзинах, чтобы сумма отношений количеств одного фрукта, делённых
  на суммарное количество двух других, равно 4 (или какому-то ещё числу).

- Великая, или последняя теорема Ферма;

- конструирование криптовалют.

Во всех случаях возникает нетривиальная и очень красивая наука
- вычисление ранга группы рациональных точек на невырожденных
кривых третьего порядка; получить же решения этих задач без этой
науки невозможно в принципе - в силу размера типичных решений.
=======================================================
МАТЕМАТИКА КВАДРАТНЫХ КОРНЕЙ (Перелучи, 27 сентября)

Насколько далеко видно с горы? Почему лист бумаги формата А4
имеет размер 297 на 210 сантиметров? Как построить корень из
заданного числа, используя циркуль и линейку? Как доказать
Теорему Пифагора, исходя из чисто геометрических соображений?
Что такое цепная дробь и когда она периодична? 

Эти и другие сюжеты, затрагивающие концепцию квадратного 
корня, мы разберём в течение двух отведённых нам уроков!
=======================================================
11:00 ``Как решать кубические уравнения: формула Кардано и вокруг''

Квадратные уравнения изучают в школе; а что с кубическими? Оказывается,
для них тоже существует формула, позволяющая выписать корни в общем виде.
Но! Но!! Но!!! бывает так, что её применение сталкивается с необходимостью
извлекать квадратные корни из отрицательных чисел (и это не всё - возникают
и иные сюрпризы с её трактовкой). Даже тогда, когда все три корня кубического
уравнения - вещественные! Как и почему? Я открою все тайны с помощью того,
что стало ``ядерной бомбой'' математики - комплексных чисел!

19:00 ``Аукционы и механизмы''

В лекции будет рассказано о самом востребованном разделе экономической
теории и теории игр - о теории конструирования механизмов. Будут примеры,
а также общая теория и блестящие достижения - за которые трижды (!) была
присуждена Нобелевская премия, в 2007, 2012 и 2016 годах. На закуску будет
краткое описание функционирования движка в компании Яндекс.
=======================================================
МАТЕМАТИКА, ИГРА, ЖИЗНЬ (Сириус, 17.09)

Хотите узнать,

- Какие удивительные перипетии случаются на футбольных чемпионатах 
  мира и Европы (включая наш, прошлогодний), и как в них разбираться?
  
- Далеко ли видно с горы Ачишхо (2300 метров), которая находится
  в 40 километрах от лагеря ``Сириус''?
  
- Сколько в вашем регионе автомобилей?

- Может ли быть так, что команда А сильнее команды Б, 
  команда Б сильнее команды В, но при всём при том 
  команда В сильнее, а не слабее команды А? 

-  Почему лист бумаги имеет размер именно 210 на 297?

-  Как правильно играть в ``Карандаш, огонь, вода и железная рука''?
  
Тогда приходите на лекцию Алексея Савватеева!
===========================================================
ВОЛШЕБНАЯ ШКОЛЬНАЯ ГЕОМЕТРИЯ (Сириус, 17.09)

"За душу каждого математика борются дьявол абстрактной алгебры
и ангел топологии", сказал кто-то из великих. На школьном уровне
тоже есть геометрия и алгебра (хотя математика - едина). На лекции
я расскажу о нескольких наиболее красивых школьных задачах:

1. Задача преследования (ученик плавает в бассейне, учитель ловит его);
2. Задача проведения касательной к окружности из данной точки одной линейкой;
3. Задача о замощении плоскости копиями выпуклого многоугольника;
4. Геометрия на сфере, или Что можно наблюдать во время перелёта 
    из Магадана в Москву за бортом самолёта?
5. Точка Ферма-Торичелли-Штейнера. Как оптимально связать 
    вершины квадрата друг с другом?
6. Магическая пентограмма: как строить правильный пятиугольник?

Предварительных сведений из высокой математики не требуется.
Приходите все, будет красиво, сложно и интересно!
===========================================================
Байкал через кран
Какой размер у листа?
Видимость с горы?
Лист 44 раза
какая из трёх команд сильнее? 249 357 168
Карандаш, огонь, вода
Преследование в бассейне
Азартные игры - покер тюремный
Очки и выходы из группы
Задача коммивояжёра
Задача о сборе снега и похожие
==========================================================
ЧТО ЕСТЬ ОБЩЕГО МЕЖДУ ШАРЫГИНСКИМИ ТРЕУГОЛЬНИКАМИ,
СЛОЖЕНИЕМ ЯБЛОК, ПОСЛЕДНЕЙ (ВЕЛИКОЙ) ТЕОРЕМОЙ ФЕРМА 
И ФУНКЦИОНИРОВАНИЕМ КРИПТОВАЛЮТ, НАПРИМЕР БИТКОЙНА?

ОТВЕТ: ЭЛЛИПТИЧЕСКИЕ КРИВЫЕ!

Мой доклад будет вводным, я расскажу об эллиптических кривых и 
операции сложения точек на них. Интерес к ним связан с несколькими 
не похожими друг на друга задачами: 

- описание всех целочисленных "Шарыгинских" треугольников, у которых
  треугольник в основании биссектрис - равнобедренный (а исходный - нет);

- детская задачка про то, сколько должно быть яблок, бананов и апельсинов 
  в трёх корзинах, чтобы сумма отношений количеств одного фрукта, делённых 
  на суммарное количество двух других, равно 4 (или какому-то ещё числу). 

- Великая, или последняя теорема Ферма;

- конструирование криптовалют.

Во всех случаях возникает нетривиальная и очень красивая наука
- вычисление ранга группы рациональных точек на невырожденных
кривых третьего порядка; получить же решения этих задач без этой 
науки невозможно в принципе - в силу размера типичных решений.
============================================================
МАТЕМАТИКА, ИГРА, ЖИЗНЬ (Байкал-Live)

Если вы хотите узнать:

- как надо играть в ``Карандаш, огонь, вода и железная рука'';
- что такое ``тюремный покер'', и как не попасть ``на сахар'' на зоне;
- какие удивительные перипетии случаются на футбольных чемпионатах 
  мира и Европы (включая наш, прошлогодний), и как в них разбираться,
  
а также поучаствовать в розыгрыше книги ``Математика для гуманитариев''
и узнать, как Алексей Савватеев применяет теорию игр при планировании
своих многочисленных перемещений по стране, то

НЕПРЕМЕННО ПРИХОДИТЕ НА ЕГО ЛЕКЦИЮ ВО ВРЕМЯ ФЕСТА!!!!!
============================================================
Кроме того, 26 августа в 18:00 состоится лекция в Молчановке!!!!

МАТЕМАТИКА ВОКРУГ НАС

Можно ли выиграть все 7 матчей в группе на чемпионате мира по
хоккею, но не пройти в финальную стадию чемпионата? Как надо
играть в ``Карандаш, огонь, вода и железная рука'', чтобы увеличить
вероятность победы? Что произошло в 2004-м году на футбольном
чемпионате Европы в группе С, и как это перекликается с ЧМ2018?

Какие удивительные математические тайны хранит детская игра 
СЕТ? Почему лист бумаги имеет размер 210 на 297? Ответы на 
все эти вопросы вы получите на лекции АЛЕКСЕЯ САВВАТЕЕВА,
которая пройдёт в Иркутске в Молчановской Библиотеке 26 августа!

В конце лекции автор расскажет о своей книге ``Математика для
гуманитариев'', и устроит автограф-сессию (ожидается большое
количество экземпляров этой книги)! Начало лекции в 18:00, 
вход на лекцию - свободный. Приходите и приводите друзей!
============================================================
МАТЕМАТИКА ВОКРУГ НАС

Далеко ли видно с горы Тебулосмта (4492), высочайшей точки 
Чеченской Республики? Сколько в Республике автомобилей?
Почему лист бумаги имеет размер 210 на 297 миллиметров? 
Что можно наблюдать во время перелёта из Магадана в Москву?
Какая математика лежит в основе систем защиты информации?

Ответы на все эти вопросы вы получите на лекции АЛЕКСЕЯ 
САВВАТЕЕВА, которая пройдёт в (.....)

В конце лекции автор расскажет о своей книге ``Математика для
гуманитариев'', и устроит автограф-сессию (ожидается большое
количество экземпляров этой книги)! 
============================================================
МАТЕМАТИЧЕСКИЕ ПАРАДОКСЫ СПОРТИВНЫХ СОСТЯЗАНИЙ

Может ли быть так, что команда А сильнее команды Б, 
команда Б сильнее команды В, но при всём при том 
команда В сильнее, а не слабее команды А? 

Сколько очков заведомо хватит, чтобы ``выйти из группы'' на 
чеспионате мира или Европы по футболу? В Лиге чемпионов?
Наоборот, насколько мало очков можно набрать, чтобы при 
том умудриться ``выползти'' из группы? А в хоккейном ЧМ?

Почему в дуэли трёх лиц, как правило, выигрывает слабейший? 

Ответы на все эти вопросы мы получим на лекции А.В.Савватеева!
============================================================
ВОЛШЕБНАЯ ШКОЛЬНАЯ ГЕОМЕТРИЯ

"За душу каждого математика борются дьявол абстрактной алгебры
и ангел топологии", сказал кто-то из великих. На школьном уровне
тоже есть геометрия и алгебра (хотя математика - едина). На лекции
я расскажу о нескольких наиболее красивых школьных задачах, вот
несколько примеров (сколько успеем - разберём):

1. Задача преследования (ученик плавает в бассейне, учитель ловит его);
2. Задача проведения касательной к окружности из данной точки одной линейкой;
3. Задача о замощении плоскости копиями выпуклого многоугольника;
6. Лемма о трёх окружностях;
8. Геометрия на сфере, или Что можно наблюдать во время перелёта 
    из Магадана в Москву за бортом самолёта?
9. Точка Ферма-Торичелли-Штейнера. Как оптимально связать 
    вершины квадрата друг с другом?

Предварительных сведений из высокой математики не требуется.
Приходите все, будет красиво и интересно!
============================================================
МАТЕМАТИКА ВОКРУГ НАС

Далеко ли видно с горы Тебулосмта (4492), высочайшей точки 
Чеченской Республики? Сколько в Республике автомобилей?
Почему лист бумаги имеет размер 210 на 297 миллиметров? 
Что можно наблюдать во время перелёта из Магадана в Москву?
Какая математика лежит в основе систем защиты информации?

Ответы на все эти вопросы вы получите на лекции АЛЕКСЕЯ 
САВВАТЕЕВА, которая пройдёт в (.....)

В конце лекции автор расскажет о своей книге ``Математика для
гуманитариев'', и устроит автограф-сессию (ожидается большое
количество экземпляров этой книги)! 
========================================================
МАТЕМАТИЧЕСКИЕ ПАРАДОКСЫ СПОРТИВНЫХ СОСТЯЗАНИЙ

Может ли быть так, что команда А сильнее команды Б, 
команда Б сильнее команды В, но при всём при том 
команда В сильнее, а не слабее команды А? 

Сколько очков заведомо хватит, чтобы ``выйти из группы'' на 
чеспионате мира или Европы по футболу? В Лиге чемпионов?
Наоборот, насколько мало очков можно набрать, чтобы при 
том умудриться ``выползти'' из группы? А в хоккейном ЧМ?

Почему в дуэли трёх лиц, как правило, выигрывает слабейший? 

Ответы на все эти вопросы мы получим на лекции А.В.Савватеева!
========================================================
ВОЛШЕБНАЯ ШКОЛЬНАЯ ГЕОМЕТРИЯ

"За душу каждого математика борются дьявол абстрактной алгебры
и ангел топологии", сказал кто-то из великих. На школьном уровне
тоже есть геометрия и алгебра (хотя математика - едина). На лекции
я расскажу о нескольких наиболее красивых школьных задачах, вот
несколько примеров (сколько успеем - разберём):

1. Задача преследования (ученик плавает в бассейне, учитель ловит его);
2. Задача проведения касательной к окружности из данной точки одной линейкой;
3. Задача о замощении плоскости копиями выпуклого многоугольника;
6. Лемма о трёх окружностях;
8. Геометрия на сфере, или Что можно наблюдать во время перелёта 
    из Магадана в Москву за бортом самолёта?
9. Точка Ферма-Торичелли-Штейнера. Как оптимально связать 
    вершины квадрата друг с другом?

Предварительных сведений из высокой математики не требуется.
Приходите все, будет красиво и интересно!
========================================================
ПЕРВООБРАЗНЫЕ КОРНИ И ПОСТРОЕНИЕ ПРАВИЛЬНЫХ 
МНОГОУГОЛЬНИКОВ С ПОМОЩЬЮ ЦИРКУЛЯ И ЛИНЕЙКИ

Ненулевые остатки по модулю простого числа всегда могут быть 
перечислены путём возведения в квадрат, куб, четвёртую и более
высокие степени одного из них; это - очень красивое и весьма 
нетривиальное утверждение. Более того, указать на тот остаток, 
который таким образом породит все остальные, тоже непросто. 

Я расскажу пару доказательств существования такого остатка, 
который называется первообразным корнем по модулю данного 
простого числа. На основании полученных результатов я покажу, как 
строить с помощью циркуля и линейки правильные многоугольники.
========================================================
МАТЕМАТИЧЕСКИЕ ПАРАДОКСЫ СПОРТИВНЫХ СОСТЯЗАНИЙ

Почему в дуэли трёх лиц выигрывает слабейший? Может ли быть так,
что команда А сильнее команды Б, команда Б сильнее команды В, но
при всём при том команда В сильнее, а не слабее команды А? Сколько
очков заведомо хватит, чтобы ``выйти из группы'' на чеспионате мира?
Тот же вопрос, но про Лигу чемпионов? А насколько мало очков можно 
набрать, чтобы всё-таки умудриться ``выползти'' из группы? Почему ФК
``Москвоский Локомотив'' систематически лучше играет с московскими 
командами, чем с остальными? На все эти вопросы я отвечу на лекции!
================================================================
МАТЕМАТИКА В СПОРТЕ И МУЗЫКЕ

Почему в дуэли трёх лиц выигрывает слабейший? Может ли быть так, что команда А 
сильнее команды Б, команда Б сильнее команды В, но при всём при том команда В 
сильнее, а не слабее команды А? Сколько очков заведомо хватит, чтобы ``выйти 
из группы'' на чемпионате мира? Как приблизить разницу в 3/2 ("кварта") по высоте 
ноты с помощью целого числа полутонов? Логарифмы в работе!
================================================================
АВТОРСКОЕ ВВЕДЕНИЕ В АРИФМЕТИКУ

С чего начинается арифметика? С таблиц умножения "по модулю данного числа",
то есть с умножения остатков друг на друга. Например, если оставить в таблице
умножения (обычной, детской) только первые цифры всех чисел, то получится
умножение остатков по модулю 10. Особенно интересны свойства таких таблиц
по простым модулям: в них всегда можно не только умножить, но и разделить
(если не на ноль, естественно). Следствий из этого - масса: малая теорема
Ферма, теорема Вильсона, "рождественская теорема Ферма" и многие другие.
Я расскажу, как, задача за задачей, можно вводить школьников в эту наиболее
важную и содержательную ветвь олимпиадно-школьной математики.
================================================================
ОТ ТАБЛИЦЫ УМНОЖЕНИЯ ДО ТРЁХ ТЕОРЕМ ФЕРМА

В лекции я расскажу (и покажу на конкретных задачках!), как изучение «детской» 
таблицы умножения может привести к самым интересным загадкам математики: 
малой и рождественской теоремам Ферма, теореме Вильсона и многим другим. 
Вы также сможете узнать, как - задача за задачей - можно вводить школьников 
в эту важную и содержательную ветвь олимпиадной математики. 
================================================================
МАТЕМАТИКА ФУТБОЛА

С каким количеством очков можно выйти из группы в Лиге Чемпионов,
если благоприятно для Локомотива сложатся игры конкурентов? Наоборот,
с каким числом очков можно умудриться не выйти? Почему Локомотив
сравнительно лучше играет с московскими клубами, чем "в целом" в
чемпионате России? Какие интересные перипетии случились на ЧМ18
с точки зрения математики турниров? Об этом всём я расскажу на уроке!
================================================================
ФЕРМА, ЭЙЛЕР, ГАУСС: АРИФМЕТИЧЕСКОЕ НАСЛЕДИЕ

В лекции я расскажу о трёх судьбах, полных математических прозрений и 
открытий - о жизни, вероятно, величайших учёных-алгебраистов ``догрупповой''
эпохи нового времени: Пьере Ферма, Леонарде Эйлере и Карле-Фридрихе Гауссе.

Великая Теорема Ферма, ``Рождественская теорема Ферма'' о суммах двух 
квадратов, квадратичный закон взаимности и первообразные корни, вплоть
до построения правильного 17-угольника с помощью циркуля и линейки - вот
те темы, которых мы с той или иной степенью подробности разберём.
================================================================
КАК ЗАПИСЫВАЮТСЯ ЧИСЛА: ПОЗИЦИОННЫЕ СИСТЕМЫ СЧИСЛЕНИЯ,
ЦЕПНЫЕ ДРОБИ, РАЦИОНАЛЬНОСТЬ И ПРИБЛИЖЁННЫЕ ВЫЧИСЛЕНИЯ

ЦЕПНЫЕ ДРОБИ И МАТЕМАТИЧЕСКИЕ ЧУДЕСА
Калейдоскоп сюжетов вокруг (ир)рациональности чисел

Можно ли утверждать что все "правовое поле" математики (я имею в виду 
рассуждения, теоремы, аксиомы и т.д., которые справедливы в десятичной 
системе), так же справедливо в любых других системах исчисления? Может 
быть, есть какая-то норма, которая "приводит" все системы исчисления к 
"унифицированной"? Короткий ответ таков: практически вся интересная
математика формулируется инвариантно по отношению к выбору систем
счисления, то есть никоим образом не зависит от произвола в выборе.
Я объясню это на лекции; кроме того, я расскажу про альтернативный 
способ записи чисел - а именно, посредством цепных дробей. Сквозь
``призму'' цепных дробей мы посмотрим на обычные числа под новым,
порой совершенно неожиданным углом.
======
ЧИСЛО Е: какую массу надо стартануть, чтобы на орбиту вышло столько-то.
за сколько дней заростает пруд? как меняется давление на высоте? распад
радиоактивности, рождаемость, кролики и числа Фибоначчи. Задача про
пьяницу, идущего домой, и про шутника и пальто
======
ДУЭЛЬ ТРЁХ ЛИЦ (ЛЕНИН, КЕРЕНСКИЙ И КОРНИЛОВ)
======
ФРУКТОВАЯ МАТЕМАТИКА
======
ЧИСЛА ПИ, Е
Множество забавных фактов вокруг двух главных констант
======
КОМБИНАТОРИКА 2013-2018: ОБЗОР ГЛАВНЫХ ДОСТИЖЕНИЙ
Что нового произошло в "математике на пальцах" за эти годы
======
ИНЖЕНЕРНО-МАТЕМАТИЧЕСКИЕ ЗАРИСОВКИ
Задачи а-ля Штейнер в огромном количестве;
районирование; транспортные парадоксы
======
ТЕОРИЯ ВЕРОЯТНОСТЕЙ НА ПАЛЬЦАХ
- Пьяница;
- Шутник и пальто;
- Вакцина;
- Дуэль двух;
- Совпадения дней рождений;
- Встреча в метро;
- Два козла и авто;
- Расклады в покере и префе, игра на костях (Ростов);
- Игла Бюффона;
- Теорвер и теория игр
======
-21. ПОСЛЕДНИЕ ПРОРЫВЫ В МАТЕМАТИКЕ, 2013 - 2018

-20. КОМБИНАТОРИКА 2013-2018: ОБЗОР ГЛАВНЫХ ДОСТИЖЕНИЙ

-19. ИНЖЕНЕРНО-МАТЕМАТИЧЕСКИЕ ЗАРИСОВКИ

-18. ТЕОРИЯ ВЕРОЯТНОСТЕЙ НА ПАЛЬЦАХ

-17. ЧТО ТАКОЕ ``ЧИСЛО Е'': УВЛЕКАТЕЛЬНОЕ ВВЕДЕНИЕ В ЛОГАРИФМЫ

-16. ПЯТЫЙ ПОСТУЛАТ: КАКИЕ БЫВАЮТ ГЕОМЕТРИИ И С ЧЕМ ИХ ЕДЯТ

-15. КАК ЗАПИСЫВАЮТСЯ ЧИСЛА: ПОЗИЦИОННЫЕ СИСТЕМЫ СЧИСЛЕНИЯ,
       ЦЕПНЫЕ ДРОБИ, РАЦИОНАЛЬНОСТЬ. ЛИКБЕЗ НАЧАЛЬНОЙ МАТЕМАТИКИ

-14. ВСЁ О ЧИСЛЕ ПИ (БОНУС-ТРЕК: СФЕРИЧЕСКАЯ ГЕОМЕТРИЯ)

-13. ШКОЛЬНАЯ МАТЕМАТИКА: ЧТО ЖЕ ДЕЛАТЬ РОДИТЕЛЯМ И ИХ ДЕТЯМ?

-12. МЕЖДУ МАТЕМАТИКОЙ И ИСКУССТВОМ:
ЭСКИЗ УДИВИТЕЛЬНОГО ДОКАЗАТЕЛЬСТВА ДИ ГРЕЯ

-11. СРЕДНИЕ ИЗ N ЧИСЕЛ: АРИФМЕТИЧЕСКОЕ И ГЕОМЕТРИЧЕСКОЕ

-10. ЭТИ ВОСХИТИТЕЛЬНЫЕ ЦЕПНЫЕ ДРОБИ: ТЕОРИЯ 
И ПРИМЕРЫ ИХ ПРОЯВЛЕНИЯ В МАТЕМАТИКЕ

-9. ПОКЕР: МАТЕМАТИКА ПРИНЯТИЯ ИГРОВЫХ РЕШЕНИЙ

-8. КВАДРАТИЧНЫЙ ЗАКОН ВЗАИМНОСТИ: ЗОЛОТАЯ ТЕОРЕМА К.Ф.ГАУССА

-7. МАТЕМАТИКА ИЗЛЮБЛЕННЫХ НАСТОЛЬНЫХ ИГР:
SET, ДОББЛЬ, ПОКЕР, КАРКАССОН, ПРЕФЕРАНС....

-6. ЭЛЛИПТИЧЕСКИЕ КРИВЫЕ СПЕЦИАЛЬНОГО ВИДА  И ИХ ПРИЛОЖЕНИЯ

-5. ЗАДАЧА ФЕРМА-ТОРРИЧЕЛЛИ И ЕЁ ОКРЕСТНОСТИ

-4. ПЕРВООБРАЗНЫЙ КОРЕНЬ ПО МОДУЛЮ ПРОСТОГО ЧИСЛА: 
     ЧТО ЭТО ТАКОЕ, И ПОЧЕМУ ОН НЕПРЕМЕННО СУЩЕСТВУЕТ

-3. МАТЕМАТИКА ПОКЕРА, ПРЯТОК И ЖЕЛЕЗНОЙ РУКИ

-2. МНОГОГРАННИКИ

-1. ЦИКЛ ЛЕКЦИЙ А.В.САВВАТЕЕВА "100 УРОКОВ МАТЕМАТИКИ"

0. ЧТО НОВОГО В МАТЕМАТИКЕ?

1. ЗАДАЧА ЭРДЁША О РАВНЫХ РАССТОЯНИЯХ

2. ВОКРУГ ВЕЛИКОЙ ТЕОРЕМЫ ФЕРМА

3. ВОЛШЕБНАЯ ШКОЛЬНАЯ ГЕОМЕТРИЯ

4. МАТЕМАТИКА В ЖИЗНИ, МАТЕМАТИКА В РЕТРОСПЕКТИВЕ

5. ЗАГАДКИ ПРОСТЫХ ЧИСЕЛ

6. ТЕОРИЯ ИГР И ПРОБЛЕМЫ БОЛЬШОГО ГОРОДА

7. АУКЦИОНЫ: ТЕОРИЯ И ПРАКТИКА

8. НОВЕЙШИЕ МАТЕМАТИЧЕСКИЕ ДОСТИЖЕНИЯ МИРОВОЙ ЦИВИЛИЗАЦИИ

9. ГЛАВНЫЕ ОТКРЫТИЯ В МАТЕМАТИКЕ: ЗНАМЕНИТЫЕ ЗАДАЧИ, 
    КАК ИХ РЕШАЛИ И К ЧЕМУ ЭТО ПРИВЕЛО

10. ЖИЗНЬ ПОСЛЕ ВЕЛИКОЙ ТЕОРЕМЫ ФЕРМА: ГИПОТЕЗА ABC

11. ПРОСТЫЕ ЧИСЛА В АРИФМЕТИЧЕСКИХ ПРОГРЕССИЯХ

12. О ЗАМОЩЕНИЯХ ПЛОСКОСТИ И СФЕРЫ

13. ВОКРУГ ЭКСПОНЕНТЫ, ИЛИ КАК ПРЕПОДАВАТЬ ВЫСШУЮ МАТЕМАТИКУ?

14. РЕПЬЮНИТЫ И КУБИЧЕСКИЕ ВЫЧЕТЫ

15. ТЕОРЕМА ФЕРМА ПРИ n=3

16. КОНЕЧНЫЕ ПОЛЯ

17. КВАДРАТИЧНЫЙ МИР И УРАВНЕНИЕ ПЕЛЛЯ

18. ШКОЛЬНАЯ ТЕОРИЯ ГРУПП

19. МАТЕМАТИЧЕСКИЕ СЮЖЕТЫ ИЗ АСТРОНОМИИ И ЖИЗНИ

20. ПОПУЛЯРНАЯ МАТЕМАТИКА

21. БИССЕКТРАЛЬНО-ПИФАГОРОВЫ ТРЕУГОЛЬНИКИ (О ЗАДАЧЕ ШАРЫГИНА)

22. ВОЛШЕБСТВО К.Ф.ГАУССА: ПОСТРОЕНИЕ ПРАВИЛЬНОГО 17-УГОЛЬНИКА

23. НОВЫЙ РЕЗУЛЬТАТ ПРО ПЛОСКИЕ ЗАМОЩЕНИЯ

24. ГЕОМЕТРИЯ, ЕВКЛИДОВА И НЕЕВКЛИДОВА

25. ПЛАН МИНИКУРСА "МАЛОМЕРНАЯ ТОПОЛОГИЯ"

26. АРИФМЕТИКА И ГЕОМЕТРИЯ КУБИЧЕСКИХ КРИВЫХ

27. ДИОФАНТОВЫ УРАВНЕНИЯ

28. ДИОФАНТОВЫ УРАВНЕНИЯ И ГАУССОВЫ ЧИСЛА

29. ДОКАЗАТЕЛЬНАЯ ГЕОМЕТРИЯ

30. МОЗАИКИ, УКЛАДКИ, РАССКРАСКИ И УЗЛЫ

31. ЦЕПНЫЕ ДРОБИ И АЛГЕБРАИЧЕСКИЕ ЧИСЛА

32. ПОПУЛЯРНАЯ МАТЕМАТИКА

33. ЗНАМЕНИТЫЕ НЕРЕШЁННЫЕ ПРОБЛЕМЫ ШКОЛЬНОЙ МАТЕМАТИКИ
*********************************************************************************************************
34. ЗАКОН ЦИПФА И КОАЛИЦИОННАЯ УСТОЙЧИВОСТЬ ЗАЗБИЕНИЙ НА ГРУППЫ

35. ТЕОРЕМА СКАРФА-ДАНИЛОВА

36. ОСНОВНЫЕ ТЕОРЕМЫ ТЕОРИИ ИГР

37. ТЕОРЕМА ЭРРОУ О ДИКТАТОРЕ С ПОЛНЫМ ЕЁ ДОКАЗАТЕЛЬСТВОМ

38. ТЕОРИЯ ОБЩЕГО ПРОСТРАНСТВЕННОГО РАВНОВЕСИЯ

39. ЭВОЛЮЦИОННЫЕ ИГРЫ И СТАТИСТИЧЕСКАЯ ФИЗИКА

40. РАВНОВЕСИЕ ДИСКРЕТНОГО ОТКЛИКА

41. ЖАН ТИРОЛЬ, НОЕЛЕВСКИЙ ЛАУРЕАТ 2014 ГОДА: ОБЗОР ДОСТИЖЕНИЙ

42. ТЕОРИЯ МНОГОМЕРНОГО РАЗМЕЩЕНИЯ

43. ТРИ НАИБОЛЕЕ ЗНАКОВЫХ ТЕОРЕМЫ КООПЕРАТИВНОЙ ТЕОРИИ ИГР

44. BREXIT НА ЯЗЫКЕ МАТЕМАТИКИ: ИГРЫ ТЕРНАРНОГО ВЫБОРА НА ГРАФАХ
 
45. ДУЭЛЬ N ЛИЦ 

46. ЗАДАЧА О КОЛЛЕКТИВНОЙ ОТВЕТСТВЕННОСТИ

47. ЗАДАЧА ГЕЙЛА И ШЕПЛИ О СТАБИЛЬНЫХ МАРЬЯЖАХ

48. ТЕОРИЯ ИГР ВОКРУГ НАС

49. КОНЕЧНЫЕ ПОДГРУППЫ ГРУПП ПРЕОБРАЗОВАНИЙ
=====================================================================
ПОСЛЕДНИЕ ПРОРЫВЫ В МАТЕМАТИКЕ, 2013 - 2018

Математика - это Стелла, вечно юная волшебница Розовой Страны.
Действительно, вроде бы ей 3000 лет или даже больше, но каждый
год происходят какие-то новые открытия, прорывы. Как залечить
раненого ежа? Как замостить плиткой Вашу ванную комнату?
Как покрасить плоскость в несколько цветов, если требуется,
чтобы никакие две точки на фиксированном расстоянии не
оказались одноцветными? Как разделить равносторонний
треугольник на пять равных частей? Эти тривиальные по
формулировке и сложнейшие по содержанию вопросы
были решены в самые последние годы (и то не до конца).
На лекции я расскажу про эти и другие достижения.
Будет понятно всем, кто интересуется математикой!

В последние пять лет произошёл целый ряд прорывов в фундаментальной 
математике. Я расскажу про некоторые из них, конкретно про следующие:
0. Простые близнецы, Чжанг и Виноградов (компьютерный счёт до 10^30).
1. Проблема Данцера - Грюнбаума (про остроугольные треугольники), или о том, 
как простой ученик московской школы 179 принял эстафету у Пола Эрдёша. 
2. Решена проблема заплаток на сфере, стоявшая перед математиками полвека. 
3. Француз Мишель Рао объявил о доказательстве несуществования никаких 
выпуклых пятиугольных паркетных плиток, кроме 15-ти известных на сей день. 
4. ABC-гипотеза продолжает находиться в стутасе гипотезы - идёт проверка 
Шольце доказательства Мошидзуки. Последние события на этом фронте!
5. Прорыв ДиГрея для хроматического числа плоскости.
6. Пифагорова комната и треугольник Патракеева
7. Как за конечное число итераций разделить пирог на n частей без зависти
=====================================================================
КОМБИНАТОРИКА 2013-2018: ОБЗОР ГЛАВНЫХ ДОСТИЖЕНИЙ

1. Оценку числом F_n я придумал не на следующий день после доклада, а 
где-то через месяц. Видимо это спуталось из-за того, что подготовка текста 
сильно затянулась тогда, и подавал работу я уже с улучшенной оценкой.

2. История задачи такова: в 50-х Эрдеш поставил задачу о максимальном 
числе точек в R^d без тупых углов и предположил, что куб оптимален. 

В 1962 товарищи Данцер и Грюнбаум это доказали (доказательство: возьмем выпуклую 
оболочку P наших точек, сожмем P в 2 раза в каждую вершину, полученные многогранники 
попарно не пересекаются, сравнивая объемы, видим, что вершин не больше 2^d) и поставили 
задачу про остроугольные множества. Они доказали, что 2d-1 <= f(d) <= 2^d-1 (!)

1983 -- Эрдеш и Фюреди построили 1.1^d.
2011 -- Харанги сделал 1.2^d
2017, весна 1.4^d, 1.6^d. 
2017, осень, Харанги и команда 2^{d-1}+1. 

3. Семереди и Тао тут ни при чем. Публикацию мне пообещал 
Янош Пах, а Боллобаш (шутя) пригласил учиться в Кембридж.

Главные прорывы в комбинаторике за 2013-2018, мне кажется, такие:
2013 - Крут-Лев-Пах, n-мерных сетов не больше, чем 2.8^n
2013 - Бондаренко, контрпример к Борсуку в размерности 65.
2014 - Киваш решил одну из самых старых задач комбинаторики 
(поставлена в 1853 Штейнером) -- дизайны существуют!  
https://ru.wikipedia.org/wiki/Система_Штейнера
2015 - Гут-Кац, решена задача о различных расстояниях! 
n точек на плоскости определяют хотя бы n/log n различных расстояний.
2015 - Тао решил задачу Эрдеша об отклонении. Для любой последовательности 
a(n) состоящей из 1 и -1 и любого числа C существуют n, d: |a(d)+a(2d)+...+a(nd)| > C.
=====================================================================
ВСЁ О ЧИСЛЕ ПИ

Что такое число Пи? По определению, это отношение длины окружности к её
диаметру. Постулат о неискривлённости нашего пространства позволяет не
уточнять, к какой именно окружности мы обращаемся за поиском числа Пи.

Однако на поверхности нашей планеты (а также на плоскости Лобачевского)
ситуация иная: отношение длины окружности к диаметру меняется с размером
самой окружности. Понимание этого факта открывает перед человеком двери
современной геометрии и приглашает его в увлекательное путешествие по ней!

В лекции будет рассказано, как быстро оценить число Пи (будет приведён ряд
формул, улучшающих одна другую с точки зрения скорости вычисления), ясно
сформулирована его трансцендентная сущность (это - строгое математическое
понятие!), а также изложена его связь с числом Е, если позволит время.
=====================================================================
ШКОЛЬНАЯ МАТЕМАТИКА: ЧТО ЖЕ ДЕЛАТЬ РОДИТЕЛЯМ И ИХ ДЕТЯМ?

Когда я был маленьким, во многих школах учили математике хорошо и бесплатно.
Теперь хорошо и бесплатно учат только самых сильных школьников в нескольких 
знаменитых матшколах. Что же тогда делать тем родителям, которые мечтают о
хорошем (математическом и не только) образовании для своих детей, но не могут
мотивировать своих любимых чад на подвиг поступления в лучшие школы страны?

"Мой ответ Чемберлену" заключаются в моём собственном усилии а направлении
просвещения. Конкретно, я решил начитать на видео и выложить в интернет все
основные темы школьной математики, с точки зрения математики высшей.

Ключевые три идеи, вокруг которых происходит развитие программы 100 уроков:

1. Основная теорема арифметики с полным её доказательством;
2. Движения прямой, окружности и плоскости, теоремы классификации;
3. Комплексные числа - арифметика, алгебра и геометрия.

Переплетаясь, эти три идеи образуют то, что я называю "начальной математической
подготовкой". Освоение моей программы абсолютно необходимо каждому, кто будет
в жизни заниматься точными и/или естественными науками, а также инженерным 
делом. Для всех остальных школьников моя программа желательна в качестве
тренировки для мозгов и всестороннего развития и образования личности.

Родителям придётся "попотеть": в среднем возрасте эти темы даются нелегко.
Однако делать нечего - в отличие от времён СССР, сейчас ребёнок на вас. Если
вы не справляетесь с вопросами, которые вам задают ваши дети, увлекающиеся
моими 100 уроками, то бросайте всё и читайте "Математику для гуманитариев"!
=====================================================================
МЕЖДУ МАТЕМАТИКОЙ И ИСКУССТВОМ:
ЭСКИЗ УДИВИТЕЛЬНОГО ДОКАЗАТЕЛЬСТВА ДИ ГРЕЯ

В лекции будет рассказано в общих чертах, что позволило геронтологу
Обри Ди Грею в возрасте 51 год продвинуться в нашем понимании мира.
Разумеется, речь пойдёт именно о его математических конструкциях, 
хотя сам факт, что Обри Ди Грей в столь почтенном возрасте сумел 
сделать нечно новое и красивое, на мой взгляд, представляет собой
лучшую из возможных апробаций его профессиональной активности
по замедлению старения. Будут также поставлены некоторые задачи,
естественно вытекающие из этих конструкций, например, во сколько
цветов красится кольцо Z[\sqrt{3},\sqrt{11}] или поле Q[\sqrt{3},\sqrt{11}].
=====================================================================
СРЕДНИЕ ИЗ N ЧИСЕЛ: АРИФМЕТИЧЕСКОЕ И ГЕОМЕТРИЧЕСКОЕ

В популярной лекции будут изложены три доказательства теоремы
о средних (арифметическом и геометрическом) - школьное, а также 
два "высших". Кроме того, мы коснёмся вопроса о других "средних",
устанавливая общее неравенство между всеми ними.
=====================================================================
ЭТИ ВОСХИТИТЕЛЬНЫЕ ЦЕПНЫЕ ДРОБИ: ТЕОРИЯ 
И ПРИМЕРЫ ИХ ПРОЯВЛЕНИЯ В МАТЕМАТИКЕ

Теорема Хинчина о цепных дробях утверждает, что предел среднего
геометрического подряд идущих знаменателей у почти любой цепной
дроби всегда один и тот же. Я объясню смысл всех терминов, а потом
ошарашу слушателей тем фактом, что _ни одного_ числа, цепная дробь
которого обладала бы этим свойством, науке не известно! Впрочем, тут
нечему удивляться - мы "почти ничего" не знаем о цепных дробях. Жаль,
ибо они являются крутейшим инструментом в теории приближений, в
решении диофантовых уравнений, при проведении алгоритма Евклида
и во многих других разделах математики. На закуску я сформулирую
теорему Лагранжа о квадратичных иррациональностях (и докажу её,
если позволит время и аудитория).
=====================================================================
ПОКЕР: МАТЕМАТИКА ПРИНЯТИЯ ИГРОВЫХ РЕШЕНИЙ

Покеру посвящена последняя, 12-я глава удивительной и неповторимой
книги Кена Бинмора "Fun and games". В простейшей же форме про покер 
можно говорить с аудиторией, ничего не знающей о теории игр. Лекция
будет посвящена двум вариациям игры в покер, вполне передающим дух
как игры, так и ажиотажа вокруг неё. Мы найдём игровое равновесие в
каждой из двух вариаций, и обсудим идею "смешанной стратегии" в игре.
=====================================================================
КВАДРАТИЧНЫЙ ЗАКОН ВЗАИМНОСТИ: ЗОЛОТАЯ ТЕОРЕМА К.Ф.ГАУССА

Квадратичный закон взаимности является утверждением о том, что из
разрешимости сравнения $p=x^2 (mod q)$ можно вывести разрешимость
или неразрешимость сравнения $q=y^2 (mod p)$, где $p,q$~--- различные 
нечётные простые числа (то есть, не равные $2$). Конкретно, достаточно
хотя бы одному из этих простых чисел давать остаток $1$ при делении
на 4, чтобы разрешимость этих двух сравнений наступала одновременно;
если же оба имеют вид $4k+3$, то разрешимость одного сравнения будет,
напротив, эквивалентна неразрешимости второго. Дополнительно нечто
утверждается про квадратичный характер числа 2, после чего возможно
быстро и эффективно проверить квадратичный характер любого числа
по модулю любого другого. Мы рассмотрим примеры, а также докажем
все сформулированные утверждения (воздав честь и хвалу Гауссу!).
===============================================================
МАТЕМАТИКА ИЗЛЮБЛЕННЫХ НАСТОЛЬНЫХ ИГР:
SET, ДОББЛЬ, ПОКЕР, КАРКАССОН, ПРЕФЕРАНС....

Может ли математика помочь чаще обыгрывать своих партнёров?
Честный ответ таков: да, но только на "стартовом отрезке" - если
математик вдумывается в правила игры, с которой он только что 
познакомился и играет с такими же новичками, он их обыграет
"почти наверное". Однако с "зубрами" любой из игр этот номер
не пройдёт: они давно "на собственной шкуре" твёрдо усвоили
все уроки математики, хоть сами и не догадываются об этом.

В то же время математическое наполнение популярных игр,
с моей точки зрения, бывает интересным и поучительным
само по себе, вне контекста желания в них побеждать.

Будучи (через своих детей) знакомым с отдельными популярными
настольными играми, я в лекции попробую раскрыть слушателям ту
удивительную математическую красоту, которая сопровождает их.
=====================================================================
ЭЛЛИПТИЧЕСКИЕ КРИВЫЕ СПЕЦИАЛЬНОГО ВИДА  И ИХ ПРИЛОЖЕНИЯ

31.10 Мой доклад "Эллиптические кривые специального вида и их приложения" будет
посвящён исследованию эллиптических кривых специального полу-симметрического
вида $Q[\frac{a}{b+c} + \frac{b}{a+c}] + \frac{c}{a+b} = S$. Интерес к ним связан с двумя
не похожими друг на друга задачами: описание всех "Шарыгинских" треугольников,
и детская задачка про то, сколько должно быть яблок, бананов и апельсинов в трёх
корзинах, чтобы сумма отношений количеств одного фрукта, делённых на суммарное
количество двух других, равно 4 (или какому-то ещё числу). Возникает нетривиальная
и очень красивая наука; получить же решения этих задач без этой науки невозможно.
=====================================================================
ЗАДАЧА ФЕРМА-ТОРРИЧЕЛЛИ И ЕЁ ОКРЕСТНОСТИ

Задача Ферма-Торричелли состоит в том, чтобы найти точку, удалённую от трёх
вершин треугольника минимальным суммарным расстоянием. Ферма эффектно и
красиво её решил, решение очень наглядное (мы его получим). Эту задачу можно 
переформулировать и таким образом, что требуется соединить три точки между 
собой системой дорог минимальной длины. 

У рассмотренной выше задачи довольно интересная судьба. Прежде всего, уже 
для 4-х вершин квадрата оказалось, что задача поиска медианы (то есть точки, 
минимизирующей суммарное расстояние до всех четырёх вершин) отличается
от задачи поиска минимальной системы дорог! Как это может быть, мы с вами
увидим на лекции. Я расскажу о дальнейшей судьбе обеих этих задач.
=====================================================================
ПЕРВООБРАЗНЫЙ КОРЕНЬ ПО МОДУЛЮ ПРОСТОГО ЧИСЛА: 
ЧТО ЭТО ТАКОЕ, И ПОЧЕМУ ОН НЕПРЕМЕННО СУЩЕСТВУЕТ

Остатки по модулю любого числа образуют то, что математики называют
``кольцом'': их можно складывать, вычитать и делить по обычным правилам.
При этом их всего конечное количество. Самое же интересное, это системы
остатков по модулю простого числа: их ещё и всегда можно делить (кроме
как на ноль, но на ноль мы никогда не делим). Ненулевые остатки по модулю
простого числа всегда могут быть перечислены путём возведения в квадрат,
куб, четвёртую и так далее степени одного из них; это - очень красивое и 
весьма нетривиальное утверждение. Характерно, что указать на тот остаток,
который таким образом породит все остальные, весьма непросто. Я расскажу
одно из доказательств существования такого остатка, который называется
{\em первообразным корнем по модулю данного простого числа}.
=====================================================================
МАТЕМАТИКА ПОКЕРА, ПРЯТОК И ЖЕЛЕЗНОЙ РУКИ

Аннотация: мы разберём три популярных игры, в которые
играют как дети, так и взрослые: расширенную версию
"камня, ножниц и бумаги", простейшие варианты покера,
а также обычные прятки, которые "под микроскопом"
окажутся совсем нетривиальными с точки зрения
поиска и свойств смешанных равновесий Нэша!

МАТЕМАТИКА ИГРЫ В ПОКЕР: 
БЛЕФОВАТЬ НЕЛЬЗЯ ПАСОВАТЬ
(где поставить запятую?)

Аннотация: мы разберём простейший вариант игры в покер
"по косточкам" (полнометражный покер разобран в главе 12
замечательной книги Ken Binmore "Fun and Games"). Окажется,
что в любом равновесии, когда игроки "хорошо угадывают друг 
друга", блеф неизбежен, но не тотален. Блефовать надо "иногда".
А насколько часто? Математика даёт точный ответ на вопрос.

Заодно будет вскрыта математическая цена информированности,
когда Вы знаете свои карту, а партнёр - нет. Даже простейший 
вариант покера, тем самым, оказывается совсем нетривиальным 
с точки зрения поиска и свойств "смешанных равновесий Нэша"!
=====================================================================
МНОГОГРАННИКИ

Разговор строится вокруг самой древней и в то же время вечно живой 
формулы начальной топологии, датируемой концом 18 века - о формуле 
Эйлера. Мы её выведем для сферы, и наметим вывод в общем случае.

Приложений будет два:

1. Как устроен футбольный мяч, у которого в каждой вершине сходятся
три пяти- или шестиугольника? Сколько в нём может быть пятиугольников?
2. Какие бывают многогранники, у которых каждая грань граничит с каждой 
по ребру? Заметим, что ответ зависит от топологического типа многогранника.

Во втором случае мы выведем комбинаторную формулу Г(Г-7) = 12(g-1),
где g - число дырок внутри многогранника. В этом бесконечном ряду
только первые два (тетраэдр и многогранник Силаши) подтверждённо
существуют. Про остальные такие многогранники это не знает никто!

На закуску - две лекции, полностью характеризующие платоновские тела -
правильные многогранники с максимально возможной группой симметрий.

На основе анализа орбит и неподвижных точек у элементов подгруппы
вращений многогранника выводится чисто комбинаторная формула
2(1-1/n) = \sum_C (1-1/n_C), где n - порядок группы, C перечисляет
классы сопряжённости нетривиальных неподвижных точек, n_C -
количество вращений в подгруппах, оставляющих оные на месте.

Аккуратный анализ этой формулы позволяет затем заключить, что 
платоновских тел бывает только пять (в трёхмерном пространстве).
=====================================================================
ЦИКЛ ЛЕКЦИЙ А.В.САВВАТЕЕВА "100 УРОКОВ МАТЕМАТИКИ"

Когда я был маленьким, многих из нас учили математике хорошо и бесплатно. 

Теперь хорошо и бесплатно учат только самых сильных школьников в нескольких 
знаменитых матшколах. Чтобы частично восполнить этот пробел, я решил начитать 
на видео и выложить в интернет все основные темы школьной математики, так 
сказать, с точки зрения математики высшей.

Ключевые три идеи, вокруг которых происходит развитие программы 100 уроков:

1. Основная теорема арифметики с полным её доказательством;
2. Движения прямой, окружности и плоскости, теоремы классификации;
3. Комплексные числа - арифметика, алгебра и геометрия.

Переплетаясь, эти три идеи образуют то, что я называю "начальной математической
подготовкой". Освоение моей программы абсолютно необходимо каждому, кто будет
в жизни заниматься точными и/или естественными науками, а также инженерным 
делом. Для всех остальных школьников моя программа желательна в качестве
тренировки для мозгов и всестороннего развития и образования личности.
=====================================================================
ЧТО НОВОГО В МАТЕМАТИКЕ?
1. Проблема Данцера - Грюнбаума (про остроугольные треугольники), или о том,
как простой ученик московской школы 179 принял эстафету у Пола Эрдёша.
2. Решена проблема заплаток на сфере, стоявшая перед математиками полвека.
3. Француз Мишель Рао объявил о доказательстве несуществования никаких
выпуклых пятиугольных паркетных плиток, кроме 15-ти известных на сей день.
Однако его доказательство очень сложное и запутанное.
4. ABC-гипотеза продолжает находиться в стутасе гипотезы - идёт проверка 
доказательства Мошидзуки.
5. Прорыв ДиГрея для хроматического числа плоскости.
==========================================================
ЗАДАЧА ЭРДЁША О РАВНЫХ РАССТОЯНИЯХ

Допустим, что у вас есть 100 фишек-точек, которые вы можете как угодно 
расставлять на плоскости. Ваша задача - как можно большее количество
отрезков, соединяющих пары точек, сделать равными друг другу. То есть
из набора 4950 чисел, равных попарным расстояниям между точками, надо
путём расстановки точек как можно больше чисел уравнять друг с другом.

Как вы их будете расставлять? Сколько отрезков окажутся в результате
оптимальной расстановки равными друг другу? Что, если точек будет 1000
вместо 100? Или 1 000 000 вместо 1 000? Или 1 000 000 000?

Ответ ни на один из перечисленных выше вопросов науке неизвестен. 

Тем не менее, как это часто бывает, в асимптотике можно установить 
определённые факты. Например, такой: при растущем числе точек, $n$
можно получить более, чем $C\times n$ равных отрезков при {\em любом} 
значении константы $C$. Я расскажу, как это делается - оказывается, что
конструкция тривиальна. Искомый результат является следствием самой
красивой формулы школьной арифметики, выражающей число решений 
уравнения $m = x^2 +y^2$ в зависимости от $m$. Гауссовы числа помогут
нам вывести соответствующую теорему и оценку.

А теперь - самое интересное. До сих пор не доказана и не опровергнута
гипотеза о том, что ни при каком положительном $\tau$ невозможно в
асимптотике расположить $n$ точек на плоскости, чтобы количество
равных расстояний оказалось бОльшим, чем (константа) на $n^{1+\tau}$!
==========================================================
ВОКРУГ ВЕЛИКОЙ ТЕОРЕМЫ ФЕРМА

Я расскажу про то, как возникла Великая Загадка Математики, что этому
предшествовало, и что началось потом. В фокусе внимания будут несколько
различных познавательных сюжетов:

- Нахождение тремя способами формулы для описания Пифагоровых троек;
- Метод бесконечного спуска Пьера Ферма и его доказательство при n=4;
- некоторые "целочисленные" соображения;
- Сведение к простым показателям;
- Анализ случая n=3 на основании исследования кольца чисел Эйзенштейна.

К сожалению, полученное Эндрю Уайлзом полное доказательство пока не
освоено мной, так что я лишь поверхностно опишу, на каких идеях оно стоит.
=====================================================================
ВОЛШЕБНАЯ ШКОЛЬНАЯ ГЕОМЕТРИЯ

"За душу каждого математика борются дьявол абстрактной алгебры
и ангел топологии", сказал кто-то из великих. На школьном уровне
тоже есть геометрия и алгебра (хотя математика - едина). На лекции
я расскажу о нескольких наиболее красивых школьных задачах, вот
несколько примеров (сколько успеем - разберём):

1. Задача преследования (ученик плавает в бассейне, учитель ловит его);
2. Задача проведения касательной к окружности из данной точки одной линейкой;
3. Задача о замощении плоскости копиями выпуклого многоугольника;
4. Геометрическое решение задачи о пифагоровых тройках;
5. Задача Шарыгина о биссектрисе;
6. Лемма о трёх окружностях;
7. Несколько сюжетов из комбинаторной геометрии.
8. Геометрия на сфере, или Что можно наблюдать во время перелёта 
    из Магадана в Москву за бортом самолёта?
9. Точка Ферма-Торичелли-Штейнера. Как оптимально связать 
    вершины квадрата друг с другом?

Предварительных сведений из высокой математики не требуется.
Приходите все, будет красиво и интересно!
=====================================================================
МАТЕМАТИКА В ЖИЗНИ, МАТЕМАТИКА В РЕТРОСПЕКТИВЕ

Миникурс будет посвящён "авторскому введению в математику". Первая лекция
25 октября стоит особняком: она открыта для всех, и посвящена математике вокруг
нас, то есть в нашей обыденной жизни (а также простейшим математическим фактам
и соображениям об устройстве нашей вселенной, наших городов и социального мира).

Во время этой первой лекции я презентую слушателям свою книгу
"Математика для гуманитариев. Живые лекции".

Вторая и третья лекции представят развитие математике "в ретроспективе". Начиная
с 3000-летней давности, люди формулировали и решали (а иногда и не решали!) задачи,
которые постепенно сформировали современный облик нашей науки. Мы повторим путь
познания математики, стартуя от теоремы Пифагора и задачи о целых (Пифагоровых) 
треугольниках (с полным её решением!), переходя к задаче о решениях уравнений в
радикалах, и затем - к четырём знаменитым проблемам в теории построений циркулем
и линейкой. По пути нам придётся очень близко познакомиться с простыми числами и
целым рядом их свойств. На последней лекции все секреты великих загадок будут
раскрыты - в общих чертах, конечно (а некоторые из задач до сих пор остаются 
неразрешёнными!). Я надеюсь, что по окончании миникурса слешатели смогут
уже пуститься в самостоятельное плавание по "морю математики".
=====================================================================
ЗАГАДКИ ПРОСТЫХ ЧИСЕЛ

В лекции я расскажу про старейшие нерешённые задачи математики -
задачка о простых близнецах, существование нечётных совершенных
чисел, проблема Гольдбаха-Виноградова. Про каждую задачку будет
дан небольшой исторический экскурс, в котором в увлекательной и
понятной нематематику форме я объясню, почему данная задачка
оказалась в центре внимания. К концу лекции станет понятно, чем
живут математики, какими смыслами наполнены их жизни и судьбы.

Завершим мы на самых последних математических достижениях, 
буквально наших дней. Дерево математики является вечнозелёным!
=====================================================================
ТЕОРИЯ ИГР И ПРОБЛЕМЫ БОЛЬШОГО ГОРОДА

Почему люди предпочитают селиться большими группами, а именно жить в городах? 
Зачем нужна такая концентрация людей в одном месте, в то время как расселение 
людей могло бы быть куда более равномерным? Зачем мы терпим перенаселение, 
пробки, очереди, давку на транспорте, проблемы с экологией и тому подобные 
напасти? Один из основных ответов состоит в том, что в городах можно проще 
и дешевле получить доступ к качественным общественным благам: медицине, 
образованию, безопасности, спорту, культуре и т.п.

Однако предоставление общественных благ приводит к самым разнообразным 
конфликтам: зачастую сложно достичь согласия о том, как именно должны 
предоставляться общественные блага и, самое главное, кто за них должен 
платить. Одно из основных открытий теоретической экономики середины 
20-го века, "проблема безбилетника", как раз и состоит в том, что эти конфликты 
приводят к недостаточному количеству и качеству общественных благ.

Одним из основных инструментов подобных исследований выступает теория игр, 
как раз благодаря способности делать прогнозы относительно исхода взаимодействия 
большого числа людей с различными интересами. В своей лекции я расскажу, как 
именно теория игр применяется для моделирования предоставления общественных 
благ и каковы возможные последствия "голосования ногами", когда люди выбирают 
города или районы согласно своим предпочтениям, средствам и способностям.
=====================================================================
АУКЦИОНЫ И МЕХАНИЗМЫ: ТЕОРИЯ И ПРАКТИКА

Доклад будет посвящён обзору основных достижений в теории аукционов: решению в
доминирующих стратегиях для аукциона второй цены; выводу формулы равновесного
поведения в аукционах первой цены; и принципу стратегической эквивалентности, что
является, несомненно, венцом данной теории. Также будет продемонстрирован целый ряд
примеров проведения аукционов на практике - Яндекс, продажа мобильного спектра и т.п.

Во второй части доклада я расскажу о механизмах в более широком смысле слова.
Механизмы контроля и наказания, устойчивые к сговору участников, будут в самом
центре внимания. Как организовать контроль над воровством меди из выплавок на
заводе Северсталь? Как эффективно собирать налоги в условиях коррупции? Вот
лишь некоторые из вопросов, которые получат освещение.
=====================================================================
НОВЕЙШИЕ МАТЕМАТИЧЕСКИЕ ДОСТИЖЕНИЯ МИРОВОЙ ЦИВИЛИЗАЦИИ

В последние годы достигнут значительный прогресс в решении 
некоторых задач, старых как мир, а также целого ряда проблем
в математике, логически связанных со знаменитыми задачами,
уже решёнными в общих чертах. Мы поговорим о простых
близнецах, о замощениях плоскости и о гипотезе ABC,
пришедшей на смену Великой Теореме Ферма. 

На закуску будет пирог (не в прямом, а в переносном смысле
этого слова: в 2016 году решена задача о справедливом дележе
пирога на $n$ гостей). Я ознакомлю слушателей с формулировкой
этой задачи, стоящей на стыке чистой математики и экономики.
=====================================================================
ГЛАВНЫЕ ОТКРЫТИЯ В МАТЕМАТИКЕ: ЗНАМЕНИТЫЕ ЗАДАЧИ, 
КАК ИХ РЕШАЛИ И К ЧЕМУ ЭТО ПРИВЕЛО

Около 200 лет назад произошла великая революция в математике, когда люди 
разобрались с возможностями решения алгебраических уравнений. Всего два 
десятка лет назад была доказана Великая теорема Ферма, а несколько лет 
назад случился сдвиг с мёртвой точки в самой старой загадке математики. 

На лекции разберем часть знаменитых проблем, над которыми бились лучшие умы 
последние три тысячелетия. В общих чертах обсудим, каким образом удалось решить 
некоторые из этих проблем, и какие новые задачи возникли на основе найденных решений.

Мы разберем ряд гипотез (а также решённых не так давно задач) о простых числах, 
гипотезу ABC, пришедшую на смену Великой Теореме Ферма, гипотезу Пуанкаре, 
доказанную Григорием Перельманом, задачи о замощениях плоскости, проблему 
четырех красок, а также коснёмся решённой в 2016 году задачи о дележе пирога. 
=====================================================================
ЖИЗНЬ ПОСЛЕ ВЕЛИКОЙ ТЕОРЕМЫ ФЕРМА: ГИПОТЕЗА ABC

Теорема Ферма, сформулированная в 1637 году и якобы доказанная 
самим Пьером Ферма, получила полное доказательство в 1993-1994
годах благодаря семилетнему отшельничеству гениального английского
математика Эндрю Уайлза (и дальнейшему латанию дыр Эндрю Уайлзом
вместе с Ричардом Тейлором). 

С точки зрения человека, далёкого от математики, это ознаменовало 
конец целой эпохи: теорема Ферма всегда позиционировалась в печати
как самая главная математическая загадка. Что же теперь делать в
математике, если Великая Теорема Ферма доказана?

Оказывается, что "жизнь только начинается" - ибо в доказательстве
Уайлса-Тэйлора разработан ряд революционно новых методов, которые 
с тех пор уже были применены для доказательства ряда других глубоких 
гипотез современной математики. Эти методы, правда, выходят за рамки 
того, о чём можно рассказать простым языком.

В то же время, за 8 лет до объявления Эндрю Уайлзом о завершении 
эпохального доказательства, в 1985 году, учёные Массер и Остерле 
сформулировали гипотезу, "покрываюшую" единой крышей целый 
ряд уже решённых и нерешённых до сих пор диофантовых уравнений. 

Я расскажу, в чём эта гипотеза заключается, и проясню её связь как
с теоремой Ферма, так и с некоторыми другими известными задачами. 

Арифметика - по-прежнему важнейшая из наук!
=====================================================================
ПРОСТЫЕ ЧИСЛА В АРИФМЕТИЧЕСКИХ ПРОГРЕССИЯХ

Более 100 лет назад Дирихле доказал следующую теорему: 

"В любой арифметической прогрессии со взаимно-простыми начальным 
членом и разностью содержится бесконечное количество простых чисел."

Мы докажем кустарными методами бесконечность числа простых вида
(4k+1) и (4k+3), а затем на этом примере познакомимся с методом Дирихле.
Слушатели узнают, что такое ряд Дирихле, гипотеза Римана, бесконечное 
произведение и характер Дирихле. Будет дан набросок доказательства
общей теоремы, а также краткое введение в комплексный анализ.
=====================================================================
О ЗАМОЩЕНИЯХ ПЛОСКОСТИ И СФЕРЫ

Наблюдение за самыми обычными окружающими нас объектами наводит
на размышления, приводящие к весьма запутанным загадкам. Например,
почему футбольный мяч сшит из двенадцатити пятиугольников и двадцати 
шестиугольников, и можно ли его сшить иным способом? Какие виды плиток 
годятся для замощения пола, а какие - нет? Какие узлы можно развязать, не 
разрывая верёвки, а какие - нельзя?

В лекции мы подробно разберём футбольный мяч "по клеточкам", заодно
выводя и анализируя знаменитую формулу Эйлера. Затем применим её к
анализу возможных плоских замощений. В минувшем году в теории плоских
замощений достигнут значительный прогресс, в том числе и благодаря 
исследованиям выпускника СУНЦа Алексея Тарасова. Я расскажу о его
идеях, которые пока не получается довести до ума - возможно, это как
раз и удастся кому-то из слушателей!

На закуску останутся узлы, бильярды и хроматические числа пространств.
=====================================================================
ВОКРУГ ЭКСПОНЕНТЫ, ИЛИ КАК ПРЕПОДАВАТЬ ВЫСШУЮ МАТЕМАТИКУ?

Высшая математика вообще и математический анализ в частности
изобилуют новыми и непривычными для простого школьника понятиями.

Часто не хватает внутренней мотивации, чтобы в них не запутаться, 
всё расставив в голове по местам. Мне всегда казалось, что аппарат 
усваивается проще при наличии некой сквозной идеи, под которую 
потом цепочкой выстраиваются возникающие конструкции и 
многочисленные технические результаты.

Я попробую убедить слушателей в том, что такая идея существует -
а именно, построение экспоненты как отображения, переводящего
сложение в умножение (гомоморфизм между двумя операциями в 
поле, говоря научным языком). Доклад поможет преподавателям
первых двух курсов университетов сделать свои лекции интереснее!
=====================================================================
РЕПЬЮНИТЫ И КУБИЧЕСКИЕ ВЫЧЕТЫ

Если число, состоящее из одних единиц ("десятичный репьюнит"), 
делится на 2017, то оно делится также и на 9. Это удивительно -
если учесть тот факт, что 2017 на 9 не делится. Однако верно.

Почему?

Объяснение кроется в свойствах мультипликативной группы
конечного поля остатков по модулю 2017 (наш год - простой!).
Кроме того, если решать "вручную", возникает необходимость
формулировать и доказывать "кубический закон взаимности"
Эйзенштейна. Это всё - красивейшие разделы математики.

На лекции мы пройдёмся по всем основным пунктам решения
поставленной выше задачки, заодно делая экскурс в кубический
мир Эйзенштейна. До кучи, может быть, вспомним и квадратичный
закон взаимности Гаусса, его "Золотую теорему".
=====================================================================
ТЕОРЕМА ФЕРМА ПРИ n=3

Первый содержательный случай Последней, или Великой теоремы Ферма
состоит в том, что для ненулевых целых чисел x,y,z невозможно равенство
x^3 + y^3 = z^3
Доказательство проводится путём рассмотрения специальной системы
чисел, так называемых {\em чисел Эйзенштейна}, и теории делимости в 
них. Числа Эйзенштейна~--- это те комплексные числа, которые попадают
в узлы треугольной сетки на плоскости всех комплексных чисел. Подобная
интерпретация позволяет сделать доказательство максимально наглядным.
Мы постараемся пройти через тернии к звёздам, разобравшись во всех его
деталях и ``подружившись'' с системой чисел Эйзенштейна.
=====================================================================
КОНЕЧНЫЕ ПОЛЯ

"Конечная математика" намечает границы применимости повседневной интуиции 
при работе с математическими абстракциями. Сколько точек на плоскости? Сколько
всего многочленов пятой степени? Сколько раз надо сложить единицу с самой собой,
чтобы получить ноль? Эти, на первых взгляд абсурдные, вопросы являются прелюдией
к материалу нашего миникурса из трёх лекций. Вот - примерная программа всего курса:

1. Таблицы сложения и умножения остатков. Многочлены с коэфффициентами в
остатках. Теорема Безу над любой системой остатков. Парадоксы числа корней.

2. Таблицы умножения по простому модулю. Простейшие конечные поля.
Основная теорема о корнях многочленов с коэффициентами в поле. 

3. Поля из $p$ элементов ($p$ - простое). Теоретико-групповые методы: теорема 
Лагранжа и Малая теорема Ферма. Бином Ньютона, автоморфизм возведения в 
$p$-ю степень и второе доказательство теоремы Ферма. Теорема Вильсона.

3. Некоторые применения основной теоремы: $q$-е степени и их количество.
Для простых $p=qr+1$ остаток является $q$-й степенью тогда и только тогда,
когда его $r$-я степень равна единице. Квадраты по модулю $p$.

4. Критерий квадратичности $(-1)$ (два доказательства). Описание всех простых,
являющихся суммой двух квадратов. Кубические вычеты и представимость целых
чисел в форме $x^2 - xy + y^2$. Задача Эрдёша о равных расстояниях, всё о ней.

5. Проективная плоскость над конечным полем. Кубические кривые. Сложение
точек и группа кривой. Кривая Шарыгина над полем из трёх элементов. Её группа.

6. Если останется время, то конечные поля из $p^r$ элементов, мультипликативная
группа и структура их вложимости друг в друга. Единственность конечного поля.
======================================================================
КВАДРАТИЧНЫЙ МИР И УРАВНЕНИЕ ПЕЛЛЯ

Рассмотрим уравнение $y^2 - 61 x^2 = \pm 1$. Его решения самым лучшим
теоретически возможным образом приближают $\sqrt{61}$, как легко поймёт
любой из вас (в смысле, что рациональное число $y/x \approx 61$).

Но существуют ли они?
Оказывается, да~--- но первое из них имеет вид (226153980, 1766319049) !

Как можно (было) его найти? 
Но сперва, как вообще можно понять, что решения существуют?

Оказывается, что на первый вопрос ответ даёт разложение числа $\sqrt{61}$
в цепную дробь. Ответом на второй вопрос служит красивейшая ветвь алгебры,
посвящённая гиперболическим поворотам. Ключевым утверждением при этом
служит знаменитая Лемма Минковского о выпуклом теле. 

В этом месте сходятся и теснейшим образом переплетаются алгебра, 
арифметика и геометрия. На самом деле, существование решения 
можно доказать, как минимум, для любого уравнения Пелля вида 
$y^2 - m x^2 = \pm 1$, где $m$ свободно от квадратов.
==================================================
ШКОЛЬНАЯ ТЕОРИЯ ГРУПП

Я считаю, что теорию групп нужно изучать в средних классах - примерно
тогда же, когда вводится символьное обозначение (буквы x,y,u и т.п.).

Потому что ступень абстракции, ведущая к общему понятию группы от 
систем остатков по данному модулю (с одной стороны) и перестановок
(с другой), не выше, чем ступень абстракции от чисел 3,4,5 к символам.

Перестановки же легко понять и освоить уже во втором-третьем классе, 
точно так же, как и системы остатков по данному модулю (основанию).

В миникурсе я ликвидирую пробелы школьного образования, относящиеся
к теории групп (и к конкретным примерам групп). Будут установлены базовые
факты про вычеты, доказана малая теорема Ферма, исследованы подгруппы
групп перестановок на трёх и четырёх символах, введено понятие нормальной
подгруппы данной группы и простоты группы. 

Если будет время, то мы далее докажем, что группа чётных перестановок на 
пяти символах - простая (что откроет желающим дорогу к изучению вопросов 
о разрешимости алгебраических уравнений в радикалах), теорему Шаля и др.

Добавочки:

а также что подгруппа переносов плоскости (пространства) - нормальная в
группе всех (аффинных) движений соответствующего объекта. Маломерные
группы движений получат полную характеризацию (теорема Шаля и законы
композиции движений разных видов). Если останется время, мы коснёмся
более интересных примеров групп, а также начал теории расширений полей.
============================================
МАТЕМАТИЧЕСКИЕ СЮЖЕТЫ ИЗ АСТРОНОМИИ И ЖИЗНИ

Далеко ли видно с горы? А с самолёта? А с вышки сотовой связи? 
А с высоты человеческого роста? Можно ли увидеть одновременно
Москву и Петербург с борта одно и того же самолёта?

Что наблюдает за бортом самолёта человек, вылетевший из Магадана 
в Москву примерно в середине дня? А из Магадана в Хабаровск?

Можно ли перегнать на самолёте тень от луны во время полного затмения?
Как оценить "на глаз", далеко ли до берега озера, видя на берегу вышку?

Как оценить "вручную" радиус земли, расстояние до луны, до солнца?
Можно ли "замахнуться" на бОльшее, и вручную прикинуть скорость света?

С каким количеством очков можно выйти из группы в Лиге Чемпионов?
А с каким количеством очков, наоборот, можно умудриться не выйти?

Сколько человек надо набрать в коллектив для того, чтобы вероятность 
совпадения хотя бы двух дат рождения перевалила через половину?

Эти, а также некоторые другие чисто бытовые вопросы, часто помогают
ориентироваться в окружающем нас мире, лучше его понимая и ощущая.

На все эти вопросы элементарная математика даёт исчерпывающий ответ.
На лекции мы в увлекательной форме разберём несколько таких сюжетов.
============================================
ПОПУЛЯРНАЯ МАТЕМАТИКА

"Законы сохранения" в математике: инвариант как мощное орудие 
доказательства теорем. Несколько примеров: игра в пятнадцать, 
устройство футбольного мяча, а также несколько простых задач 
(про домино и про круглый стол, за которым сидит шесть ребят).

После детального разбора некоторых из этих сюжетов я вкратце 
расскажу о ветви математики, рождённой два века назад и которую 
можно было бы назвать "Доказательной геометрией". 

Если останется время, слушатели узнают об истории нескольких
замечательных теорем, доказанных к настоящему времени:

1. Невозможность вывести общую формулу для корней многочлена;
2. Трансцендентность чисел "e" и "пи";
3. Невозможность осуществления  определённых построений 
    с помощью циркуля и линейки;
4. Различение узлов, топологические сюжеты и победа Перельмана;
5. Великая Теорема Ферма.
============================================
БИССЕКТРАЛЬНО-ПИФАГОРОВЫ ТРЕУГОЛЬНИКИ (О ЗАДАЧЕ ШАРЫГИНА)

В журнале "Квант" номер 8 за 1983 год в статье "Вокруг биссектрисы" на 
странице 36 И.Ф.Шарыгин формулирует такую задачу (входящую также 
под номером 500 в его известный задачник): 

"Про данный треугольник известно, что треугольник, образованный 
основаниями его биссектрис - равнобедренный. Можно ли утверждать, 
что и данный треугольник равнобедренный?"

Ответ отрицательный, но в статье далее сказано: 

"К сожалению, автор не сумел построить конкретный пример треугольника 
(то есть точно указать величины всех его углов или длины сторон) со столь 
экзотическим свойством. Может быть, это удастся сделать читателям журнала?"

Решая эту задачу вместе с Сергеем Маркеловым, я понял, как здесь
выйти на теорию эллиптических кривых и операцию сложения точек.

После этого мы с Игорем Нетаем применили технику анализа таких кривых
и группы рациональных точек на них (относительно так определённой операции).
Соответствующая техника позволяет доказать существование бесконечного
семейства неподобных друг другу целочисленных треугольников, обладающих 
требуемым свойством. На докладе я расскажу, как выводится этот результат.

Школьная задача, таким образом, привела нас в самое сердце одной 
из красивейших ветвей современной математики. Первый целочисленный
треугольник имеет стороны, равные 18800081, 1481089 и 19214131.
============================================
ВОЛШЕБСТВО К.Ф.ГАУССА: ПОСТРОЕНИЕ ПРАВИЛЬНОГО
17-УГОЛЬНИКА С ПОМОЩЬЮ ЦИРКУЛЯ И ЛИНЕЙКИ

Геометрические построения с помощью циркуля и линейки интересовали
математиков с самых древних времён. Античная наука не справилась со
следующими четырьмя классическими задачами:

- Квадратура круга (построение квадрата, имеющего 
  площадь равную площади заданного круга);

- Удвоение куба (построение стороны куба вдвое большего
  объёма, нежели заданный);

- Трисекция угла (разделение заданного угла на три равные части);

- Построение правильных многоугольников.

"Первая математическая революция" времён Эвариста Галуа открыла
двери к доказательству НЕВОЗМОЖНОСТИ первых трёх построений.
Она же помогла установить тот факт, что правильный 7-угольник не
может быть построен с помощью циркуля и линейки.

Однако ещё за несколько десятков лет до великих свершений, в 1772
году, 17-летний Карл Фридрих Гаусс сумел ПОСТРОИТЬ с помощью 
циркуля и линейки правильный 17-угольник! Таким образом, молодой
Гаусс поставил "мат" всей античной школе математики, которая умела
строить из "простоугольников" лишь только правильные треугольник и 
пятиугольник. Построение Гаусса - это ода полю комплексных чисел.
============================================
НОВЫЙ РЕЗУЛЬТАТ ПРО ПЛОСКИЕ ЗАМОЩЕНИЯ

Давно известно на основании формулы Эйлера: замостить плоскость
одинаковыми выпуклыми семиугольниками нельзя. Можно немного
уточнить этот результат: если плоскость замощена даже различными
выпуклыми шестиугольниками и семиугольниками согласно принципу 
face-to-face (то есть, нет вершин на сторонах), и многоугольники эти -
"соизмеримых размеров" (точнее, существуют два круга радиусов R 
и r такие ,что любой многоугольник содержится в первом из кругов и 
любой многоугольник содержит второй круг), то доля семиугольников 
на всей плоскости "стремится к нулю" (в понятном смысле слова). 

Этот факт также доказывается с помощью формулы Эйлера.

Оказалось, что эти результаты улучшаются следующим образом:
в ЛЮБОМ разбиении плоскости на выпуклые шестиугольники и
семиугольники с выполнением перечисленных требований общее 
количество семиугольников обязано быть КОНЕЧНЫМ.
============================================
ГЕОМЕТРИЯ, ЕВКЛИДОВА И НЕЕВКЛИДОВА
Алексей Савватеев, февраль 2015 года

Андре Вейль сказал однажды, что за душу каждого математика борются ангел 
топологии и дьявол абстрактной алгебры. Все предыдущие созывы Школы я был
формально на стороне дьявола, хотя и не очень ему потакал (в любой алгебре
всегда прячется красивая геометрия). Нынче я буду целиком на белой стороне!

Конкретно, мы рассмотрим группы движений трёх основных геометрий - обычной
плоской евклидовой геометрии, сферической геометрии и геометрии Лобачевского.

По пути к последней из трёх мы пройдём через проективные преобразования нашей
обычной плоскости. Геометрические постулаты Евклида, история создания геометрии
Лобачевского, теорема о трёх гвоздях для всех трёх геометрий, порождение групп
движений отражениями, классификация движений (теорема Шаля), арифметика
композиций движений, углы и площади, двойное отношение - вот круг вопросов,
который будет в идеале освещён в течение миникурса. Насколько мы сможем
продвинуться, зависит от аудитории. Приходите все!!!
============================================
ПЛАН МИНИКУРСА "МАЛОМЕРНАЯ ТОПОЛОГИЯ"

I. РАССЛОЕНИЕ ХОПФА

1. Гомотопические группы сфер (обзор)
2. Комплексная проективная плоскость: построение
3. Действие группы поворотов на трёхмерной сфере, орбиты и факторпространство
4. Описание расслоения Хопфа одной формулой
5. Топологическая нетривиальность расслоения Хопфа
6. Геометрия расслоения Хопфа
7. Индекс зацепления любой пары окружностей равен единице

II. СООТНОШЕНИЯ МЕЖДУ ГРУППАМИ ВРАЩЕНИЙ

1. Комплексные числа как средство описания вращений плоскости
2. Задача Гамильтона о вращениях пространства
3. Пьяное откровение Гамильтона: кватернионы
4. Анализ вращений пространства как пар кватернионов: формулы
5. Отождествление группы единичных кватернионов с $SU(2)$
6. Знаменитое отображение $SU(2) \to SO(3)$ на этом языке
7. Геометрия построенного отображения, сравнение с Хопфом

Если добавить к трёхмерному пространству одну точку на бесконечности, то 
получится трёхмерная сфера $S^3$. Памятуя об этом, выберем вертикальную
прямую и окружность, её опоясывающую. Станем постепенно ``наращивать''
вокруг нашей окружности всё более толстые торы, которые постепенно будут
заполнять всё трёхмерное пространство, за исключением выбранной прямой.
В конечном итоге торы будут всё более походить на опоясывающие именно
прямую, а не окружность, а с учётом бесконечной точки, они будут опоясывать
просто ещё одну окружность. 

Далее, каждый тор является объединением семейства окружностей, каждая 
из которых зацеплена с каждой, а также с исходной (и финальной). Теперь 
сопоставим каждой точке сферы ту окружность, на которой она лежит
(бесконечной точке~--- вертикальную прямую, которая тоже окружность).
Получим знаменитое {\em расслоение Хопфа}.

На лекции будет подробно рассказано об этой конструкции, а также о её
связи с двумерной проективной полскостью, гомотопическими группами
сфер и другими классическими топологическими сюжетами.
============================================
АРИФМЕТИКА И ГЕОМЕТРИЯ КУБИЧЕСКИХ КРИВЫХ

Доклад будет посвящён неособым неприводимым кривым третьего порядка~--- 
{\em эллиптическим кривым}. Исторически сложилось так, что эллиптические
кривые находились в центре внимания алгебры и теории чисел, и именно их
анализ открыл двери к доказательству Великой Теоремы Ферма.

Я расскажу про групповой закон на эллиптической кривой, и методами
алгебраической геометрии докажу его ассоциативность. Затем, после 
разбора нескольких примеров, будет сформулирована теорема Морделла 
о конечной порождённости группы рациональных точек на эллиптической 
кривой. Я также расскажу о последних результатах и открытых вопросах
в этой центральной области арифметики.
============================================
ДИОФАНТОВЫ УРАВНЕНИЯ

Когда я был маленький, меня страшно увлекали уравнения, которые надо
решать в целых числах~--- {\em диофантовы уравнения}. Действительно, вот 
пишешь ты в тетрадке $y^2 = x^3 - 1$, и сидишь думаешь: какие целые числа
этому уравнению удовлетворяют? Когда куб на единичку больше квадрата?

А когда куб на единичку меньше? Удивительным образом, ответ на первый
вопрос гораздо проще ответа на второй (но всё равно не вполне тривиален~---
попрбуйте-ка решить эту задачку самостоятельно!).

Казалось бы, какая может быть польза в решении таких уравнений? Но как
всегда, математика щедра~--- дарит вдвое больше, чем от неё ждёшь. Мало
того, что такие вопросы манят своей кажущейся простотой~--- любой школьник
средних классов поймёт, что в задаче требуется~--- так ещё и аппарат, который
был разработан в процессе решения подобных ``детских головоломок'', оказался
центральным чуть ли не во всей математике. (Между прочим, то же касается и
вопроса о том, как собирать кубик рубика, и задачи об игре в ``пятнадцать''.)

Я постараюсь завлечь читателя в фантастический и удивительный мир
арифметики~--- колец, делимостей в них, полей, многочленов и матриц над 
числовыми полями~--- начиная каждый раз с простого и понятного вопроса 
или элементарной на вид задачки. Примерная программа курса:

Лекции 1-2. Пифагоровы тройки и суммы двух квадратов: приглашение в 
мир Гауссовых чисел. Делимость, основная теорема арифметики. Решение 
диофантова уравнения $y^2 = x^3 - 1$ с помощью гауссовых чисел.

Лекции 3-4. Великая теорема Ферма, показатели $4$ и $3$. Кольцо
Эйзенштейна и делимость в нём. Уравнение $y^2 = x^3 + 1$. Уравнение
Пелля: цепные дроби, гиперболические повороты и принцип Минковского.
Множество решений как группа обратимых элементов некоторого кольца.

Лекции 5-6. Построения циркулем и линейкой. Поле ``построимых чисел''.
Расширения полей, простейшие следствия из линейной алгебры. Задачи
о $17$-угольнике, трисекции угла, квадратуре круга и удвоении куба.

Кронекер сказал: "Бог создал натуральные числа, всё остальное - дело рук
человеческих". Но всё, к чему прикоснулся человек, уже тронуто человеческою
порчею. Натуральные числа - совершеннейший из объектов, с которыми мы
имеем дело как в математике, так и в жизни вообще. Соотношения между
ними удивительны и неожиданны, а уравнения, связывающие натуральные
числа (которые называются {\em диофантовыми уравнениями}), как правило,
очень сложны для решения, даже если выглядят на вид совсем тривиальными
(достаточно вспомнить Великую теорему Ферма!). 

В курсе, состоящем из натурального числа лекций, будет проведён анализ 
и дано полное решение нескольких наиболее известных и красивых 
диофантовых уравнений (уравнение Пелля, $y^2 = x^3 + k$ и другие).

Также, если останется время, мы коснёмся вопроса приближения
действительных чисел рациональными, и важными следствиями 
для трансцендентности знаменитых чисел $\pi$ и $e$.
============================================
ДИОФАНТОВЫ УРАВНЕНИЯ И ГАУССОВЫ ЧИСЛА
Лектор: Алексей Савватеев

В курсе, состоящем из трёх лекций, будут решены следующие, на первый
взгляд простейшие, но на проверку весьма нетривиальные проблемы:

\begin{enumerate}

\item Когда квадрат целого числа отличается от куба целого числа  
на единицу? (Знаменитые диофантовы уравнения $y^2 = x^3 \pm 1$);

\item Сколькими способами можно представить данное натуральное
число в виде суммы двух квадратов натуральных чисел? Какие простые 
числа представляются в виде суммы двух квадратов?

\item Когда ``квадратное число'' является одновременно ``треугольным''?

\end{itemize}

При решении поставленных выше задач мы выйдем за пределы области 
обычных целых чисел, рассмотрев так называемые Гауссовы числа, а также
другие сходные с ней системы, являющиеся {\em кольцами}. Свойства
делимости во введённых кольцах позволят решить все три задачи.
============================================
ДОКАЗАТЕЛЬНАЯ ГЕОМЕТРИЯ

На рубеже XVIII-XIX веков произошла "первая математическая революция"
(вторая происходит на рубеже XX-XXI веков на наших с вами глазах). А именно,
был решен целый ряд задач, стоявших с античных времён. Среди них: как найти
общую формулу для корней уравнения пятой (и более высокой) степени; как
построить с помощью циркуля и линейки куб, вдвое бОльший данного; как
разделить данный угол на три равные части; и некоторые другие проблемы.

Общей характерной чертой полученных решений было то, что они содержали
_доказательства_невозможности_ требуемого в задачах. На этом пути произошло
становление науки, которую я бы назвал "доказательной геометрией". Речь идёт о
сведении геометрических формулировок к задачам из области чистой алгебры, и
затем, после наращивания "военной техники", прихода к противоречию с логикой
изучаемых алгебраических конструкций.

Я докажу наиболее простые теоремы доказательной геометрии, относящиеся
к построениям циркулем и линейкой. С этой целью я познакомлю слушателей с
базовыми понятиями теории расширений полей и теории алгебраических чисел.
Никаких предварительных знаний ни в какой области математики не требуется!
============================================
МОЗАИКИ, УКЛАДКИ, РАССКРАСКИ И УЗЛЫ

ВНИМАНИЕ !!! 8 апреля 2018 года английский математик Ди Грей, приглашённый
профессор МФТИ, объявил об улучшении нижней оценки на хроматическое число
плоскости! Он построил дистанционный граф, который не красится в 4 цвета!!!!!

Ниже даётся описание оставшейся части лекции (сперва, конечно, про Ди Грея!!).

Наблюдение за самыми обычными окружающими нас объектами наводит
на размышления, приводящие к весьма запутанным загадкам. Какие узлы
можно ``развязать'', а какие --- нельзя? Почему футбольный мяч сшит из 
12-ти пятиугольников и 20-ти шестиугольников, можно ли его сшить иным 
способом? Сколько цветов потребуется, чтобы раскрасить карту мира 
(граничащие страны должны быть разноцветными)? Что можно, а чего 
нельзя построить циркулем и линейкой? Одинаковыми плитками какой 
формы можно замостить пол, а какой~--- нельзя? Какие траектории у
мяча могут реализоваться при игре в бильярд, а какие~--- не могут?

Попытки ответить на (почти) все поставленные вопросы привели к
красивой науке, причём похожей не на геометрию, а на алгебру (хотя
математика едина, она не знает ни о каком разделении на области). 

На этих и других примерах я поведаю о становлении ``доказательной 
геометрии'', а также объясню слушателям, в чём именно заключается
знаменитая гипотеза Пуанкаре, доказанная русским математиком
Григорием Перельманом в самом начале XXI века.
============================================
ЦЕПНЫЕ ДРОБИ И АЛГЕБРАИЧЕСКИЕ ЧИСЛА
Алексей Савватеев, Школа Райгородского-2013, лето

Курс посвящён очень красивой, но практически не затрагиваемой
в школьной программе теме: разложению вещественных чисел в
цепные дроби. В первой части курса речь пойдёт о разложении
квадратичных иррациональностей, и в качестве приложения мы
получим полное решение любых диофантовых уравнений второй 
степени от двух переменных (и уравнения Пелля в частности),
а также докажем знаменитую теорему Лагранжа о периодичности.

Во второй части курса мы попробуем разложить в цепную дробь
кубические иррациональности (на примере $\sqrt[3]{2}$), после
чего естественным образом выйдем на поле $Q[\sqrt[3]{2}]$ и
арифметику в нём. В третьей части курса, если останется время,
мы докажем некоторые теоремы о приближении вещественных 
чисел рациональными. Курс не требует никаких пререквизитов,
и без проблем доступен для понимания ученикам 9-11 классов.
============================================
ПОПУЛЯРНАЯ МАТЕМАТИКА

0. Картинка, открывающая курс математики: теорема Пифагора в картинках.

1. Фейнман и две культуры Сноу. "Великий Архитектор Вселенной" был математиком,
и никакого "гуманитарного" пути к постижению Его творения нет. Ибо там диффуры
прямо с лёту - силы зависят от положений и влияют на ускорения.

2. Эксперимент, закон сохранения и абсолютное доказательство: шахматная доска с 
вырезанными из неё противоположными клеточками не может быть замощена домино.

3. Инвариант (закон сохранения чётности перестановки) для игры "Пятнадцать".

4. Невозможность построений и формул для решений уравнений. Трансцендентность
чисел e и \pi. Комплексные числа как ключ к построению семнадцатиугольника. Галуа!

5. Инварианты в топологии: Эйлер и его характеристика. Формула Эйлера и виды
многогранников. Различение тора и сферы. Трёхмерная ситуация: узлы и Пуанкаре,
теорема Перельмана. Планарные графы и хроматическое число плоскости.

6. О теореме Пифагора снова: пифагоровы треугольники и теорема Ферма.

7. Арифметика: совершенные числа, простые числа Ферма и Мерсенна, связь
с построениями Гаусса. Простые близнецы. Проблема Виноградова. Суммы
двух квадратов - изящное доказательство через простоту Гауссовых чисел.

8. О чём наука "геометрия"? Кратчайшие перелёты, площадь сферических
треугольников, движения, проективное доказательство построения одной
линейкой касательной к окружности. Сколько сфер касается данной одной?

9. Паркеты: парадоксы и запреты. Связь с формулой Эйлера. Лёша Тарасов.

10. Вероятности: игла Бюффона, встреча в метро (вероятность встретиться),
вероятность того, что две наугад взятые дуги на сфере пересекаются.
============================================
ЗНАМЕНИТЫЕ НЕРЕШЁННЫЕ ПРОБЛЕМЫ ШКОЛЬНОЙ МАТЕМАТИКИ

Самая простая, "школьная" математика таит в себе целый ряд логиических 
загадок, часть из которых до сих пор не разгадана человечеством. В лекции 
будет приоткрыта дверь в этот "заколдованный мир" математической красоты.

Конкретно, я расскажу про несколько знаменитых и очень красивых 
задач, формулируемых языком буквально детского сада:

1. "Проблема равных расстояний" П.Эрдёша;
2. Хроматическое число плоскости;
3. Нечётные совершенные числа;
4. Простые близнецы (и другие загадки простых чисел);
5. Алгоритмические загадки Кнута и игры "Гексагон".
============================================
============================================
============================================
ЗАКОН ЦИПФА И КОАЛИЦИОННАЯ УСТОЙЧИВОСТЬ ЗАЗБИЕНИЙ НА ГРУППЫ

Люди делятся на коалиции разного рода (страны и регионы, политические партии,
клубы по интересам и т.д.). Этот процесс изучался в ряде работ применительно к
образованию стран, к поставке пространственного блага, а также к принятию
решения об объёме перераспределения. Среди прочего, анализировался аспект
коалиционной устойчивости по Ауманну-Дрезу: можно ли так разбить агентов на
коалиции, что никакая коалиция не захочет выделиться из структуры?

В ранних работах первого и третьего авторов было показано, что в дискретной
модели возможны распределения, в которых нет устойчивых конфигураций.

Эти контрпримеры легко обобщаются на непрерывную модель.

В настоящей работе, во-первых, анализируется случай монотонно убывающей
плотности населения. Оказывается, что контрпример можно построить и для
этого случая. Но при дополнительном условии "плавного убывания" доказывается
теорема о существовании устойчивого разбиения. Более того, такое разбиение
строится конструктивно. Во-вторых, анализируется размер и масса юрисдикций
в построенном разбиении. Оказывается, чем меньше плотность населения, тем
меньше масса и тем больше размер коалиций. Более того, при начальном
степенном распределении результат соответствует известному эмпирическому
закону Ципфа: распределение по массе также оказывается степенным.

ТЕОРЕМА СКАРФА-ДАНИЛОВА

Кооперативная теория игр - наука специфическая: формализация
игрового взаиможействия в достаточно общем виде подсказывает
концепцию решения, в простейших случаях оказывающуюся пустой:
ни одного решения, удовлетворяющего разумным требованиям, нет.

Я расскажу, в каком ключе может быть снята эта проблема. А именно,
сформулирую и докажу теорему Скарфа о непустоте ядра в обобщённом 
смысле слова. Оригинальное доказательство Скарфа (1967) запутанное и
неподъёмное; я же расскажу красивейшее доказательство В.И. Данилова,
который вывел этот результат из теоремы Какутани о неподвижной точке.
============================================
ОСНОВНЫЕ ТЕОРЕМЫ ТЕОРИИ ИГР

Как это ни парадоксально, но "сухой математический остаток" 
от чрезвычайно разросшейся в последние 50-100 лет теории игр 
составляет сравнительно небольшой объём. В первую очередь
это, безусловно, триумфальная теорема Нэша о существовании
смешанного равновесия в любой конечной игре. Далее, имеется
теорема существования сильного секвенциального равновесия
в произвольной динамической игре на конечном дереве. Нельзя 
также не упомянуть теорему Скарфа из области кооперативной 
теории игр, а также оптимальный аукцион Майерсона. Немного
в сторонке лежит общая теорема существования конкурентного 
равновесия, доказанная Эрроу и Дебрэ в 1951 году. Практически
все эти теоремы (за исключением результата Маерсона) держатся
на теореме Какутани о неподвижной точке.

В первой части доклада будет сформулирована и строго доказана
(возможно, по модулю совсем уж технических деталей) теорема 
Нэша - путём сведЕния к теореме Какутани, последней - к теореме
Брауэра и, наконец, последней - к лемме Шпернера. (В качестве
побочного любопытного результата будет показано, что теорема
Брауэра эквивалентна нетривиальности n-й гомотопической группы 
n-мерной сферы.) Во второй части выступления я охарактеризую,
с моей точки зрения, перспективы будущего развития теории игр
и всей математической экономики, в целом.
============================================
ТЕОРЕМА ЭРРОУ О ДИКТАТОРЕ С ПОЛНЫМ ЕЁ ДОКАЗАТЕЛЬСТВОМ

О знаменитой теореме Эрроу многие знают "понаслышке": демократия 
нереализуема, и единственный способ непротиворечивого принятия 
решений - диктаторский. Но мало кто может чётко сформулировать, что 
же конкретно было доказано величайшим математическим экономистом 
всех времён, ныне здравствующим учёным, нобелевским лауреатом 
по экономике Кеннетом Эрроу - не говоря уже собственно о способе 
доказательства этой теоремы. 

В лекции будет дана точная формулировка и самое короткое на 
сегодня доказательство этого замечательного факта. Кроме того,
я опишу вкратце современные направления теории коллективного
выбора - науки, самый первый результат которой был отрицательным.
============================================
ТЕОРИЯ ОБЩЕГО ПРОСТРАНСТВЕННОГО РАВНОВЕСИЯ

Чарльз Тьебу, географ, высказал в 1956 году идею децентрализации принятия
политических решений и решений в области экономики общественного сектора.

На проверку речь шла о математической задаче самоорганизации некоего
пространства с заданной плотностью расселения - о задаче разбиения на
группы, когда сами ``жители'' выбирают, в какой группе проживать. 

Концепция равновесия предусматривает, что в предложенном разбиении
никто не хочет никуда перебегать ("равновесие свободной миграции").

Целый ряд исследователей пытался по-своему её трактовать, и в рамках 
той или иной математической модели - доказать либо опровергнуть. Точки,
то есть окончательного решения этой задачи, никогда не было поставлено.

Мы предлагаем модель, полностью имитирующую нестрогую конструкцию
Чарльза Тьебу. Пространство характеристик при этом предполагается
существенно многомерным (то есть размерность не ниже двух). Как ни
странно, это предположение снимает все прошлые проблемы: пользуясь
леммой Кнастера-Куратовского-Мазуркевича, мы доказываем наиболее
общую теорему существования ``равновесия по Тьебу''.
============================================
ЭВОЛЮЦИОННЫЕ ИГРЫ И СТАТИСТИЧЕСКАЯ ФИЗИКА

А.В. Леонидов, А.В. Савватеев, А.Г. Семенов

В докладе обсуждается связь между эволюционными играми с дискретным 
пространством выбора и кинетической теорией в статистической физике. 
Особое внимание уделяется обобщению известных результатов на случаи 
общего распределения вероятностей для дискретного выбора и эволюционных 
игр на разреженных графах. Обсуждается точное выражение для времени 
перехода из метастабильного в стабильный минимум в изинговской 
эволюционной динамике.
============================================
РАВНОВЕСИЕ ДИСКРЕТНОГО ОТКЛИКА:
В ПОИСКАХ АДЕКВАТНОЙ МОДЕЛИ ПОВЕДЕНИЯ ЛЮДЕЙ

Общеизвестно, что целый ряд предсказаний классической теории игр, 
основанных на поиске равновесий Нэша и даже на решении задач по 
доминированию, идёт в разрез с результатами "полевых экспериментов", 
когда в те же игры подопытные люди играют даже на настоящие деньги. 
В первую очередь это такие игры, как "Ультиматум", "Производство 
общественного блага" и "Сороконожка". 

В то же время введённая в конце 20 века в работах Палфри и Маккельви
концепция равновесия дискретного отклика очень неплохо прогнозирует 
результаты эксперимента, например, для игры "Сороконожка". 

По своему духу эта концепция несёт в себе элементы статистической 
механики, умело встроенные в теоретико-игровой контекст.

В лекции будет дано определение новой концепции решения игр,
доказана теорема существования равновесия, а также приведён
целый ряд результатов компьютерного поиска равновесий этого
класса, соотнесённых с экспериментальными данными.

Доклад основан на следующих двух статьях:

McKelvey, R. D., & Palfrey, T. R. (1995). Quantal response equilibria for normal 
form games. Games and economic behavior, 10(1), 6-38. 

McKelvey, R. D., & Palfrey, T. R. (1998). Quantal response equilibria for extensive 
form games. Experimental economics, 1(1), 9-41. 
============================================
ЖАН ТИРОЛЬ, НОЕЛЕВСКИЙ ЛАУРЕАТ 2014 ГОДА: ОБЗОР ДОСТИЖЕНИЙ

Нобелевскую премию по экономике в 2014 году получил французский
экономист Жан Тироль, с формулировкой "за анализ несовершенных 
рынков и их регулирования". Однако в данном случае премия, скорее
всего, была вручена по совокупности работ, ибо трудно представить
экономиста со столь разноплановыми научными интересами.

В обзорной лекции будут рассмотрены два его главных достижения:

А. Теория коллективной репутации и "оптимального наказания";
Б. Анализ несовершенных рынков: современное состояние;

Также, если позволит время, будет затронута такая актуальнейшая
на сегодня тема, как теория регулирования и построения механизмов.

Некоторые ссылки:

1. Тироль Ж. Рынки и рыночная власть: теория организации 
промышленности (в 2 томах). СПб.: Экономическая школа, 2000.

2. Laffont J.-J., Tirole J. A Theory of Incentives in Regulation and 
Procurement. — Cambridge, MA: MIT Press, 1993.

3. Tirole J. (1996) A Theory of Collective Reputations (with applications to the 
persistence of corruption and to firm quality // Review of Economic Studies,
volume 63, issue 1, pages 1-22.
============================================
ОБЩИЙ ДЛЯ НАШИХ СО ШЛОМО

Рассмотрим модель Тьебу с несколькими агентами, различающимися в 
оценке разновидностей локального общественного блага. Предполагается,
что множество разновидностей общественного блага одномерное; данную
разновидность принято называть локализацией. Любую разновидность
можно произвести, с фиксированными затратами. В отличие от стандартной
модели Тьебу, предполагается эндогенный выбор локализации блага в
каждом сообществе (далее: группе), путём голосования по большинству.
Предпочтения агентов на множестве разновидностей предполагаются 
однопиковыми, поэтому выбор разновидности ограничивается множеством
медианных выборов в любой группе.

В рамках этой модели исследуется процесс формирования групп, целью
которого является разбиение множества всех агентов на непересекающиеся 
группы. Члены групп стоят перед дилеммой: с одной стороны, участие
в большей группе влечёт меньшую долю издержек на создание общественного
блага, а с другой стороны~--- в большей группе выбор локализации может 
быть далёк от идеального, с точки зрениянетипичных для данной группы 
агентов. Исследуются вопросы существования устойчивого разбиения на 
группы, где устойчивость может пониматься как в духе Нэша, так и в
сильном (коалиционном) смысле. Показано, что коалиционно-устойчивое 
разбиение может не существовать, независимо от конкретного способа выбора
локализации из множества медиан. В то же время демонстрируется, что 
устойчивость можно восстановить с помощью системы трансфертов (то есть,
механизма компенсации) внутри каждой группы.
============================================
ТРИ НАИБОЛЕЕ ЗНАКОВЫХ ТЕОРЕМЫ КООПЕРАТИВНОЙ ТЕОРИИ ИГР:
СТАБИЛЬНЫЕ МАРЬЯЖИ, ТЕОРЕМА О ДИКТАТОРЕ И ТЕОРЕМА СКАРФА

Некооперативную теорию игр знают многие в наши дни. Теорема Нэша,
различные классы игр, динамика. А вот с кооперативной теорией игр
знакомы далеко не все - а те, кто о ней слышал, обычно плохо себе
представляют, каков её размах и что в неё включается. 

Я сделаю обзор этой теории, с упором на самые знаменитые её результаты -
теорему о стабильном марьяже, теорему Эрроу о диктаторе (часто относимую 
к отдельной ветви теоретической экономики - теории выбора), а также общую 
теорему Скарфа. У последней есть фантастически красивое доказательство 
через теорему Какутани - доказательство, найденное живым классиком 
теории игр, московским учёным Владимиром Ивановичем Даниловым.
============================================
BREXIT НА ЯЗЫКЕ МАТЕМАТИКИ: ИГРЫ ТЕРНАРНОГО ВЫБОРА НА ГРАФАХ
 
Рассмотрим какую-нибудь страну,  например Великобританию, и стоящий 
перед ней бинарный выбор. В этих условиях каждый житель страны должен 
выбрать одну из трёх (не двух) стратегий поведения: (-1) - "топить" против 
выхода, (0) - воздержаться, (1) - агитировать за выход из ЕС.

Предполагается, что соответствующее поведение реализуется и в выборе 
стратегии на самом референдуме (прийти/не прийти, и как голосовать). 
Но самое главное - это что игроки соотносят свои стратегии поведения с 
теми стратегиями, которые принимают их друзья, соседи и т.д.
 
Следует заметить, что математически нижеописанная модель годится не 
только для изучения Brexit, но и для описания любого глубокого социального 
конфликта, в том числе и "украинского раскола" в русскоязычном мире.

Выигрыш игрока зависит не только от того, каково соотношение выбранного 
поведения с его внутренними экзогенными установками, но также и от степени 
конформности стратегии этого индивида со стратегиями его окружения.

Записывая соответствующую модель на произвольном графе социальных 
связей, мы выводим уравнения, которые поразительным образом совпадают 
(или почти совпадают), для равновесия дискретного отклика с логистической 
функцией распределения случайной компоненты полезности, с уравнениями, 
выводимыми в статфизических моделях Изинга и Поттса на графах. Получить 
точное решение последних можно только для полных графов; для графов 
произвольной топологии описаны только приближенные методы решения 
(подход Бете-Пайерлса). Отметим, что статфизические модели содержат 
в себе единый для всех целевой функционал.

Всё это совершенно невероятно, и пока до конца нами не понято и не осознано.

Кроме того, следует отметить, что на  полном графе похожие модели изучались 
Гранноветтером, Броком, Дурлауфом, Блюмом, а также целым рядом исследователей 
из ИПУ РАН (отдела член-корр. Д.А.Новикова). 

Пора навести полный порядок в этой области знания. Для начала нужно попытаться 
вывести характеристики системы уравнений, определяющих в поставленной задаче 
равновесие дискретного отклика (наиболее разумный вид решения, применяемый 
для игр очень большого числа участников вместо Нэша), и запустить их на компьютере. 

Или исхитриться и решить данную систему в явной форме после введения разумных 
упрощающих (агрегирующих) предположений.
 
Поставленные задачи - это программа для новых и, как нам кажется, в высшей 
степени перспективных и актуальных исследований в области теории игр.
============================================
ДУЭЛЬ N ЛИЦ 
(Савватеев, Ильинский, Измалков)

Рассматривается следующая динамическая игра. В игре участвует произвольное 
количество игроков, у каждого из которых есть фиксированная вероятность 
промахнуться при стрельбе. 

На каждом шаге игры все оставшиеся стреляют одновременно друг в друга, 
после чего оставшиеся одновременно стреляют друг в друга, и.т.д.до тех пор, 
в живых не останется меньше двух людей. Также игрокам позволяется 
стрелять в воздух, но с некоторыми ограничениями. 

Выигрышем каждого игрока является вероятность выжить.  Ищутся все равновесия 
в стационарных стратегиях, совершенные на подыграх. В докладе будут описаны все 
равновесия при количестве игроков, не превосходящих 3, а также для любого 
количества игроков в случае, когда у всех игроков равные вероятности промаха.
============================================
ЗАДАЧА О КОЛЛЕКТИВНОЙ ОТВЕТСТВЕННОСТИ

Рассмотрим следующую задачу: есть начальник, и несколько его 
подчинённых. Начальник объявляет {\em схему наказания} за 
отлынивание от работы, которая представляет собой функцию из 
множества всех {\em векторов отлыниваний} во множество {\em векторов 
штрафов}, причём последнее предполагается ограниченным сверху (точнее, 
мы предполагаем ограниченным {\em суммарный размер штрафа}. На
множество допустимых схем наказания накладывается ограничение
{\em коалиционной устойчивости}, которое формально выражается в 
требовании имплементации сильного равновесия Нэша в игре 
подчинённых между собой, при выборе степеней отлынивания. 
Требуется описать все такие схемы.

Полное решение этой задачи на сегодня мне представляется необозримым. 
В то же время, можно выделить и описать достаточно широкий класс схем 
наказания с требуемыми свойствами, обратившись к теории построения
коалиционно-устойчивых механизмов распределения издержек 
коллективного производства (Э.Мулен и др.). Таким образом, подчинённые 
как-бы делят между собой {\em символические издержки} отлынивания, 
функциональная форма которых предписывается начальником. При 
определённых дополнительных условиях можно доказать коалиционную 
устойчивость так построенных схем, применяя теорему существования 
сильного равновесия, которую я формулирую и доказываю в двух 
вариантах. Эта теорема позволяет существенно расширить многообразие 
схем наказания, по сравнению с простанством схем, непосредственно 
позаимствованных из теории Мулена.

Или по-простому, для популяризации:
ЗАДАЧА О КОЛЛЕКТИВНОЙ ОТВЕТСТВЕННОСТИ

Представьте себе, что Вы - дежурный милиционер в турникетном зале. Безбилетники
пытаются прыгать через турникеты, Вы их ловите. Вы один, их - много. Возможно ли 
задать такие "правила игры", чтобы они не смели пытаться перепрыгивать, даже 
если заранее известно, что поймаете Вы в любом случае только одного из них?

Оказывается, что да, это возможно. Но не очень тривиально. Подобные схемы борьбы
с массовыми нарушениями могут применяться (и применяются в некоторых странах)
при борьбе с налогоуклонением, списыванием на экзаменах и т.д. Я расскажу о том,
какие теоретико-игровые основания стоят за изобретением различных хитроумных
схем контроля. Иными словами, я планирую популярное введение в так называемую
"теорию построения механизмов" - науку, за которую уже не один раз дали Нобеля
(премию Нобелевского комитета) по экономике.

или ещё:
Представьте себе, что Вы - начальник, и у вас есть несколько подчинённых. Они все
пытаются отлынивать от работы, Вы за ними следите. Вы один, их - много. Каждый
из них сидит за своим компом в своей каморке. Вы видите, кто чем занят, однако не
можете нанести больше одного визита с "волшебным пендалём".

Возможно ли задать такие "правила игры", чтобы они не смели пытаться отлынивать?

Оказывается, что да, это возможно. Но не очень тривиально. Подобные схемы борьбы
с массовыми нарушениями могут применяться (и применяются в некоторых странах)
при борьбе с налогоуклонением, списыванием на экзаменах и т.д. Я расскажу о том,
какие теоретико-игровые основания стоят за изобретением различных хитроумных
схем контроля. Иными словами, я планирую популярное введение в так называемую
"теорию построения механизмов" - науку, за которую уже не один раз дали Нобеля
(премию Нобелевского комитета) по экономике.
============================================
ЗАДАЧА ГЕЙЛА И ШЕПЛИ О СТАБИЛЬНЫХ МАРЬЯЖАХ

На лекции будет рассказано о замечательном достижении экономической теории -
а именно, об алгоритме достижения устойчивой системы бракосочетаний (мэтчинга)
при произвольных начальных условиях, заданных в виде списков предпочтений 
участников взаимодействия.

Более подробно, имеется некоторое количество "мужчин" и "женщин", при чём
каждый "мужчина" имеет строго сформулированные предпочтения относительно
потенциальной "жены", и наоборот. Требуется найти систему "браков" без "разводов"
(все указанные понятия формализуются в строгом смысле, соответствующим духу 
рассматриваемой задачи). Оказывается, что это всегда возможно сделать.

Применения данной задачи многочисленны. Наиболее известное из них - это
правило размещения абитуриентов по ВУЗам. Международное сообщество 
оценило математическую красоту результата и его приложения в 2012 году,
когда за него была вручена премия Нобелевского комитета по экономике.
============================================
ТЕОРИЯ ИГР ВОКРУГ НАС

Оглянемся вокруг. Почти все явления социальной, политической и
экономической жизни пестрят стратегическим взаимодействием.

В процессе взаимодействия "игроки" пытаются просчитать ходы
друг друга как можно дальше и глубже, как в шахматах и шашках.

Результат порой выглядит весьма неожиданным.

Мы сперва смоделируем "игру" прямо со слушателями 
в аудитории, а затем разберём следующие сюжеты:

1. Телеигра или задача о парковочных местах 
2. Люксембург в Евросоюзе
3. Синдзо Абэ и Северная Корея
4. Парадокс Брайеса в Метрогородке (Москва) 
5. Два парадокса Дональда Трампа
6. Рациональное безумие (снова Северная Корея!)
============================================
КОНЕЧНЫЕ ПОДГРУППЫ ГРУПП ПРЕОБРАЗОВАНИЙ

Теорема Шаля доказывается так: пускай у нашего движения столько-то точек
остаётся на месте, тогда докажем, что оно является (тождественным, поворотом,
отражением и так далее). На самом деле это достаточно общий приём изучения
групп более хитрых преобразований множества, не обязательно движений. 

Довольно большое внимание всегда уделяется конечным подгруппам. Решая
задачу классификации таких подгрупп, мы ставим вопрос о том, как устроено 
множество точек, неподвижных хотя бы для одного из преобразований нашей
конечной подгруппы.

Соответствующий на редкость красивый анализ будет сперва проведён для 
случая конечных подгрупп движений прямой, окружности, плоскости и сферы 
(в последнем случае мы придём к движениям, сохраняющим какое-то из 
платоновских совершенных тел).

Затем я расскажу про два очень специальных случая конечных подгрупп
рациональных преобразований прямой и плоскости - один из них основан
на изучении двойного отношения точек, а второй связан с формулой
$$
x^3 + y^3 + z^3 - 3xyz = (x+y+z)(......),
$$
известной каждому школьнику математического класса (остальных прошу
вывести её самостоятельно, или разузнать у знакомых либо в интернете!).

Возможно, мы успеем рассмотреть родственный вопрос о функциональных
выражениях для сложения точек на эллиптических кривых, но это неточно.
============================================
============================================
&&&&&&&&&&&&
ENGLISH PART:
============================================
A general equilibrium approach to the multidimensional Tiebout hypothesis.
(Savvateev A., Sorokin K., and S.Weber)

Consider the following problem called `multidimentional group formation'. We are given 
a probabilistic distribution over a finite-dimentional convex compactum, $X \in {\bf R}^d$.
This distribution is assumed to admit continuous density $f: X \to {\bf R}$, which is in
addition bounded away from zero: $\exists \delta: \quad \forall x \in X \,\, f(x) \ge \delta$.

We look for a stable, `migration-proof' partition of $X$ into a prescribed number $n$ of
nonempty measurable compacta: $X=S_1 \cup \dots \cup S_n$, almost mutually exclusive.
Stability is meant in a game-theoretic sense, when the `centers' of those groups are given
via $m_1,\dots,m_n$ and each point $x \in X$ is interpreted as a citizen choosing 
between those $n$ jurisdictions.

In searching for the stable configuration, citizen $x$ compares cost functions in all the
jurisdictions, cost functions which split into `monetary' and `transportation' parts:
$$
\frac{Const}{\int_{S_i} f(y) dy} + l(x,m_i),
$$
where $l(\cdot,\cdot)$ is a given metric over $X$, and picks up one of jurisdictions which
minimize costs. Above, a spectacular feature is the inverse proportionality between the 
monetary part of the cost and the measure of a jurisdiction, its `population'. By using 
a technique borrowed from general-equilibrium theory, we prove the existence result.
