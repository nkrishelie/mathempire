МАТЕМАТИКА ВОКРУГ НАС (&& И ЖИЗНЬ, && В ОБЫДЕННОЙ ЖИЗНИ И ТП)

1. Можно ли купить квартиру, экономя на автобусных билетах?
(Есть вариации - "сколько лет певец Сюткин ездил в метро, сложив 
минуты в года?", и ещё несколько аналогичных сюжетов с подсчётом);

2. Как планировать частые перелёты, исходя из представления о размере 
земли? (Можно ли, вылетая в 10 утра из Магадана в Москву, успеть на встречу
в 16:00 в Санкт-Петербург?). Дополнительный вопрос: что будет наблюдать за
окном пассажир упомянутого рейса из Магадана в Москву 13 декабря?

3. Чисто бытовые вопросы, связанные с походами: 
- Успеваешь ли ты на электричку, когда идёшь на лыжах по льду Байкала?
   (Впереди видна заводская труба в Слюдянке с известной высотой).
- Нужно ли брать в поход фонарик (с 3 по 20 августа на Таймыр)?

4. Прикладная геометрия:
- Как быстро измерить площадь дачного участка, не вызывая землемеров?
- Где больше площадь земной поверхности: между тропиками, или севернее
  /южнее них? (Формула Герона и цилиндрическая проекция)
связь площади треугольника с суммой углов на сфере

5. Математика парадоксов в политике и экономике 
- Парадоксы американских выборов
- Парадоксы американской финансовой системы

6. Проценты и шансы. Несколько зарисовок.

6.1. Читая прессу....

"Согласно данным проведенной в 2009 году переписи населения, 99% этнических 
казахов и 74% жителей страны заявили о том, что понимают разговорный казахский. 
Хотя возможно, некоторые из опрошенных и приукрасили свои лингвистические 
способности, есть основания полагать, что показатели владения казахским 
значительно выросли. Примечательно, что если в 1989 году лишь 0,9% этнических 
русских заявили о владении казахским, то к 2009 году эта цифра взлетела до 25%. 
В то же время в 2009 году 94% жителей Казахстана, включая 92% этнических 
казахов, заявили о том, что понимают русский язык, что значительно выше,
чем в 1989 году, когда этот показатель находился на уровне 64%.
- Нет ли в этих цифрах какого-либо противоречия? 
- Сколько этнических русских в Казахстане? 
Считаем для простоты, что Казахстан двунационален.

6.1.1. "На севере Антарктиды зафиксирована рекордно низкая температура..."

6.2. Задача про вакцину на планете Миранда

6.3. Вероятность встречи в метро

7. Паркеты для вашей ванной. (Теория замощений на практике)

8. Как выиграть Бегущий Город?

9. Парадокс Брайеса: управление транспортными потоками.

10. Далеко ли видно с горы? Это в пункт 3

Далеко ли до луны? прочая астрономия типа "где какие звёзды видны"

дерево падает бесконечное время; задача про бассейн

как поймать шпиона? проецирования и вокруг

Вадим Сюткин "Ежедневно 42 минуты под землёй
Я, наконец, сложу в года" - сколько надо ездить?

МАТЕМАТИКА ЖИЗНЕННЫХ ПАРАДОКСОВ

Шутник и пальто; задача про пбяницу и про число e
Номера машин !!!
Лига чемпионов !!
Длина окружности на сфере, сферическая геометрия, (не)пропорции
Полёт Алисы сквозь землю
Где надо пилить дерево? Куда оно прогибается?
Загадки про метро
Лунные и солнечные затмения
Перелёты, белые ночи, чёрные дни и т.п.
Плитки и замощения
Касательная одной линейкой

Сумма квадратов биномиальных \sum_{k=0}^n (C_n^k)^2 = C_{2n}^n,
(C_2^2)^2 + (C_3^2)^2 + \dots = ?

Герон!!! обобщения на многомерие
Совпадающие дни рождения
Календари (когда в следующий раз ДР в четверг?)
вдвое старше и втрое старше?
МММ - когда он рухнет?
Фаренгейт и Цельсий
Как падает высокое дерево?
Логарифм: шесть рукопожатий + что-то там было ещё

ГЕОМЕТРИЧЕСКАЯ АЛГЕБРА

Геометрическое изображение теоремы Пифагора, операций умножения и 
сложения. Алгоритм Евклида на геометрическом языке, цепные дроби.

Соизмеримость отрезков, рациональные и иррациональные числа. Дроби,
их сократимость, величины, выражаемые ими. Площади и сходимость рядов.

МАТЕМАТИКА ЖИЗНЕННЫХ ПАРАДОКСОВ

Может ли открытие новой дороги привести к ухудшению дорожной ситуации?
Может ли страна быть должна всему остальному миру, но богатеть от года к году?
Можно ли увидеть одновременно Санкт-Петербург и Москву, не будучи космонавтом?
Можно ли построить касательную к окружности, имея при себе одну линейку?
Можно ли дважды за один день встретить восход солнца?

Эти, а также некоторые другие чисто жизненные вопросы, будут рассмотрены
на популярной лекции по математике автора "Математики для гуманитариев"!
======================================================
На 44 деревьях, расположенных по окружности, сидели 44 весёлых чижа (на каждом 
дереве по чижу). Время от времени два чижа одновременно перелетают на соседние 
деревья в противоположных направлениях (один по часовой стрелке, другой - против). 
Докажите, что чижи никогда не соберутся на одном дереве.
======================================================
Задачка из Успенского про комитет (занести в слайды!!!)

"Кто раньше всех заснёт, тот может не спать!"

ДНИ РОЖДЕНИЯ И ИХ СОВПАДЕНИЕ
РАССТОЯНИЕ ДО ЛУНЫ, ДО СОЛНЦА, РАЗМЕР ЗЕМЛИ И СКОРОСТЬ СВЕТА
ФОРМУЛА ГЕРОНА
ВЕРОЯТНОСТЬ ВСТРЕЧИ В МЕТРО
ЛЮКСЕМБУРГ И ТРАМП
ДЕРЕВО И ЕГО ИЗЛОМ (ТРИГОНОМЕТРИЯ)
ТРИГОНОМЕТРИЯ И КОМПЛЕКСНЫЕ ЧИСЛА
МАТАНАЛИЗ (ПАЛЬТО, ЗАДАЧА ПРО ПЬЯНИЦУ, ЧТО ТАКОЕ ПИ)
======================================================
МАТЕМАТИКА ЕЩЁ:

Фаренгейт и Цельсий: $9(c+40) = 5(f+40)$

В течение какого промежутка времени отец старше сына вдвое по полным годам? Втрое?

Задача Эрдёша о равных расстояниях: что мы знаем, куда надо двигаться и т.п.

Формула Герона с выводом и примерами из жизни (участок в Больших Котах)

Как падает дерево? Два сюжета - с каким ускорением, и где потом ломается

Где какие звёзды видны?

Календари и задачки на даты (Пасха, Гумиров и т.д., день рождения в четверг)

Когда рухнет МММ? линейное уравнение

Куча сюжетов вокруг представительной демократии

Когда быстрее темнеет? четыре строго упорядоченные даты - равноденствия и солнцестояния

Вакцина и эксперимент - на применение формулы Байеса в жизни

Перелёты, Магадан - Москва и т.п. + в связке про то, откуда на сколько далеко видно;
Эратосфен, расстояние до Солнца и скорость света; сферические треугольники и "пи"

Встреча в метро и другие сюжеты из жизненного тервера; вероятность совпадений ДР

Логарифмы: шесть рукопожатий и заростающий пруд

Лига чемпионов - с какими очками выходят из группы?

Задача Штейнера и всё, что с ней связано

Сколько машин в данном регионе?

Нотная грамота и конечная арифметика
======================================================
ПОЛЕЗНЫЕ ССЫЛКИ ДЛЯ МАЛЫШЕЙ ОТ АРКАШИ КАЦА:

Ты Лёша сам посоветовал мне когда-то книгу Звонкина Малыши и Математика, и я до сих пор чту её как катехизис, хотя как ни странно многие преподаватели с которыми я знаком её не особенно понимают. Есть конечно и другие книги, и куча видео на youtube,. На вскидку список ресурсов который я составлял для своей работы. Но все эти ресурсы конечно, в первую очередь нужны что бы заинтересоваться самому. Пособия непосредственно для детей 5-и летнего возраста конечно есть, но математику в них самому ребёнку не распознать.
 
Книжки:

    Л. Кэрол “Алиса в зазеркалье”

    А. Звонкин “Малыши и математика”

    Смаллиан “Принцесса или тигр”, "Как же называется эта книга"

    Ежи Цвирко-Годыцки "Как победить колдунью"

    М. Клайн “Математика утрата определённост

    Златко Шпорер “Ох, эта математика!” 

    Maria Droujkova and Yelena McManaman “Moebius Noodles: Adventurous Math for the Playground Crowd”
     
Веб  сайты и приложения
 
    http://brilliant.org 

    http://mech.math.msu.su/~shvetz/54/

    http://csunplugged.org/activities/

    https://pelicanbook.ru/

    https://www.euclidea.xyz/

    http://ncase.me/trust/

    https://www.desmos.com/

    https://teacher.desmos.com/ 

    https://www.youtube.com/channel/UCYO_jab_esuFRV4b17AJtAw 

    https://brilliant.org/ 

    http://www.cutoutfoldup.com/index.php 

    https://www.youtube.com/channel/UC6nSFpj9HTCZ5t-N3Rm3-HA 

    https://www.youtube.com/channel/UCjwOWaOX-c-NeLnj_YGiNEg

    https://plus.google.com/communities/104964145698135252427 
