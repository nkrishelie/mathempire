1 

2014_03_13 - 1-я лекция А.В.Савватеева "Числа, символы и фигуры" 4-5 классы ч. 1/2
https://www.youtube.com/watch?v=j2ZZou7AanI

2-30 Рисуем квадрат и ставим точку, делим некотором отношении. Откладываем равные отрезки на каждой стороне.
3-50 почему это квадрат?
5-50 еще один квадрат
6-50 чьё имя?
7-19 теорема Пифагора визуальное доказательство
8-00 сложение дробей на примере прямоугольника (офтопик)
9-30 запись дробей
10-40 это рациональные числа (дети хором вздохнули)
12-40 отрезки и "сколько квадратиков в картинке"
13-25 площадь квадратика со стороной 1/12
16-16 какие числа кроме дробных?
16-50 некоторые длины отрезков нельзя записать в виде дроби
17-00 иррациональное число корень из двух
19-43 равенство треугольников (2 стороны и угол между ними)
20-48 2ху выкинута из квадрата
22-33 сумма квадратов катетов
24-00 а прямой ли там угол?
25-00 доказываем прямоту угла в гипотенузном квадрате
26-30 корень. "Знаете, но не проходили - хорошее состояние ума".
27-30 видеть числа как длины и площади фигур
28-00 три горы и три чайки - разные для примитивных народов (Фосс "Сущность математики")
30-15 бином Ньютона для 2


2014_03_13 - 1-я лекция А.В.Савватеева "Числа, символы и фигуры" 4-5 классы ч. 2/2
https://www.youtube.com/watch?v=bygKx1GnxyU

0-0 бином Ньютона для 3
0-30 рисуем куб
2-45 режем куб-сыр, сколько кусков?
3-20 разрезать торт тремя разрезами на 8 частей
4-00 как считать объемы частей?

2
2014_04_01 - 2-я Лекция Алексея Владимировича Савватеева 4-5 классы ч. 1/2
https://www.youtube.com/watch?v=yrPr-h-_zyg

1-00 Соизмеримость и несоизмеримость отрезков
1-40 знаете про циркуль линейку
2-40 возьмем два отрезка
3-20 два кузнечика прыгают на эти две длины (а и в)
4-30 встретятся ли эти кузнечики где-нибудь кроме общего начала
5-20 зависит ли ответ от а и в?
6-50 встретятся там, где длина делится нацело на а и на в
8-00 n*a=m*b
8-50 может ли быть, что они не встретятся никогда?
11-50 поделим обе части на n*m
13-00 a/m=b/n
13-40 посмотрим на а и в под микроскопом
14-30 "в том и только в том" случае
15-30 если а и в получены из третьего отрезка
17-50 Определение: а и в называются соизмеримыми, если существует такой 
          отрезок с, который укладывается как в а так и в в целое число раз.
19-50 откуда начинаются натуральные числа?
20-40 N Z Q R C
21-50 "Комплексные числа лучше знать с детского сада"
23-00 образуем прямоугольник из двух соизмеримых отрезков
23-50 математик сомневается всегда, не верит даже президенту
24-30 геометрическое деление с остатком
25-30 алгоритм Евклида
27-00 если отрезки соизмеримы, то процесс остановится
29-00 тайная сетка, которая разбивает а и в на целое число частей

2014_04_01 - 2-я Лекция Алексея Владимировича Савватеева 4-5 классы ч. 2/2
https://www.youtube.com/watch?v=MUhkydMwRUA

1-40 за конечное число ходов алгоритм остановится
3-20 Обратное тоже верно: если процесс остановится за конечное число шагов - отрезки соизмеримы
7-20 квадрат со стороной а и диагональю в. Эти отрезки несоизмеримы. Алгоритм Евклида никогда не остановится
8-50 1 и корень из 2
11-30 бесконечно подобные прямоугольники
13-30 значит есть числа кроме дробных
3


2014_09_17 - 3-я Лекция А. Савватеева "Визуальное представление бинома Ньютона" ч. 1/3
https://www.youtube.com/watch?v=mBx2yRTbfjM

2-30 важность визуализации
3-00 (а-в)(а+в)=аа-вв что это значит
3-20 проверка
5-20 рисуем квадраты
9-30 равные площади заштриховываем
11-00 раскрываем скобки
13-40 самма подряд идущих чисел - геометрическая задача
18-30 почему n(n+1) всегда нацело разделится на 2
20-00 произведение нескольких подряд идущих чисел делится на их количество
21-00 1+3+5+...2n-1 сумма нечетных
23-00 геометрическое доказательство суммы нечетных
24-30 бином Ньютона
25-00 (а+в)^n
27-00 а четвертая степень что такое?
28-00 четвертое измерение

2014_09_17 - 3-я Лекция А. Савватеева "Визуальное представление бинома Ньютона" ч. ⅔

https://www.youtube.com/watch?v=hzhZ-onNKHA

0-00 бином Ньютона для 2. Рисуем квадрат, рассекаем
2-00 н-мерный куб рассекаем н-1 мерным ножом
3-40 сколько кусков сыра после 3х разрезаний?
4-40 режем. Какие будут объемы
8-00 считаем куски

2014_09_17 - 3-я Лекция А. Савватеева "Визуальное представление бинома Ньютона" ч. 3/3
https://www.youtube.com/watch?v=-iGPlMofsfA

0-00 бином для 3
2-00 дз - раскрытие скобок для 3
2-30 четырехмерная интуиция, тессеракт
4-30 по аналогии 1 2 3 4
7-00 объяснение коэффициентов
10-00  це из 4 по 2

4-5
2014_12_03 - Лекция А.В. Савватеева - движение ч. 1/2
https://www.youtube.com/watch?v=WAJr5BTGJLg

4-50 физический и математический смысл движения
7-00 движение прямой
8-30 движение - это преобразование, которое сохраняет расстояние
13-20 два движение для перехода а в в - подвинуть и перевернуть
18-00 разница между перевернул и передвинул и передвинул-перевернул
21-00 арифметика движений
23-00 сдвинуть и перевернуть = перевернуть
24- значки T S и композиции
27-00 последовательность выполнений T S и S T


2014_12_03 - Лекция А.В. Савватеева - движение ч. 2/2
https://www.youtube.com/watch?v=EMpAm6ANQ7M

0-00 классификация движений прямой
3-00 есть ли что то кроме переноса и поворота?
7-20 наша программа - доказать теорему, что все преобразования исчерпываются
8-00 дз - рассмотрите композицию из двух переворачиваний (в обоих порядках)
9-00 табличка S T
9-30 id

6-7
2014_12_25 - Лекция 6 А. В. Савватеева - теоремы о движениях прямых ч. 1/2
https://www.youtube.com/watch?v=7dt-0M8wBzo

2-40 Теорема о двух гвоздях
7-10 Теорема 2: если движение g оставляет на месте одну точку, то это симметрия
12-30 теорема 3: рассмотрим любое движение, сколько точек оно оставляет на месте?
14-00 если нет ни одной точки на месте, то движение - перенос
27-00 теорема Шаля


2014_12_25 - Лекция 6 А. В. Савватеева - теоремы о движениях прямых ч. 2/2
https://www.youtube.com/watch?v=nbjMxCxd6v4

1-00 четные и нечетные отображения
3-50 визуальное представление композиции S(0)*S(A)

8
2015_02_11 - А.В.Савватеев "Движение окружности" ч. 1/2
https://www.youtube.com/watch?v=f8xwW2d2jWU

0-00 вращаем обруч
5-00 R(90)*R(90)*R(90)*R(90) = id
6-50 R(a)R(b)=R(a+b)
8-00 неподвижность центра
9-40 отражение относительно прямой L
10-00 теорема: других движенией нет
13-00 доказательство
15-00 расстояние на окружности
17-00 пи
18-00 радиан
20-30 если движение сохраняет на месте одну точку, что с противоположной?
22-00 доказательство
26-00 а что с другими точками?
27-30 а если две точки? Два гвоздя
30-00 Лемма 1 о 2х непротивоположных точках

2015_02_11 - А.В.Савватеев "Движение окружности" ч. 2/2
https://www.youtube.com/watch?v=Yi3-F4ISsrY
0-00 тождественное преобразование
1-00 если две неподвижные - то отражение
3-00 если ни одной - то это поворот

9-10
2015_03_16 - лекция А. В. Савватеева - Конечные подгруппы движений прямой и плоскости
https://www.youtube.com/watch?v=pr9Ye-1hfGg
0-00 таблица умножения движений по окружности
4-50 поворот, ели ни одной неподвижной точки
8-00 композиция либо id либо отражение
9-00 доказательство
14-00 одно движение повторим несколько раз
16-00 R(50)^7 = R(-10)
17-30 R(50)^252=id, а может это произойти раньше?
21-00 а отражение повторить несколько раз? Подгруппа
23-00 поворот на 90 - тоже подгруппа
25-00 какие конечные подгруппы есть?
26-40 поворот на 1 радиан
27-00 больше 60 градусов?
31-00 почему пи одинаково для всех окружностей на плоскости?
32-00 получи ли id если много раз буду двигать на 1 радиан?
33-40 никогда, свойство числа пи (несоизмеримость отрезков)

11
2015_04_06 - лекция А. В. Савватеева Арифметика остатков -1: таблицы сложения
https://www.youtube.com/watch?v=KNZkKEGzy3A

0-30 сколько времени будет через 20 часов? 9+20=5
2-15 12-часовое время 9+5=2
5-03 какой день недели 1 июля 2037
8-50 число дней в году делится на 7? 365=1(7)
12-00 через какое количество лет повториться дата-день недели?
24-30 друг математик, который не помнил таблицу умножения
25-00 что такое последняя цифра числа?
27-30 что такое мажорный аккорд?
32-50 таблица сложения в группе остатков по модулю семи Z/7Z
36-30 дз х+х=0 (8)

12
2015_04_08 - лекция А. В. Савватеева Арифметика остатков - 2
https://www.youtube.com/watch?v=wekZyALPICI

0-00 х+х=0 (8) решение
3-20 а при произвольном модуле?
6-50 таблица умножения
7-30 умножение чет/нечет
9-00 таблица умножения по модулю три
11-00 изоморфизм разных групп 
11-50 таблица умножения по модулю 4
12-50 какие свойства этой таблицы?
13-30 центрально симметричны
14-30 чем отличаются? Есть ноль
15-30 таблица умножения по модулю 5
19-00 таблица умножения по модулю 6
21-20 общие свойства
23-00 дз - доказать центральную симметрию
24-30 дописать таблицы для всех модулей 7 8 9 
25-30 таблица умножения не содержит 0 ⇔ модуль простой
27-00 какой статус этого утверждения?
30-00 обратное утверждение (сложное)
32-00 простых чисел бесконечно много (утверждение Евклида)

?
2015_05_12 - лекция А. В. Савватеева - Основная теорема арифметики. Продолжение ч. 1/2
https://www.youtube.com/watch?v=_6Pu0LE04p4

4-00 существуют ли в строке нули? А единица? Все остатки?
10-00 если встречаются все остатки, то нуля там нет
17-00 если два числа не делятся на число, то и произведение не делится
21-00 существует только единственное разложение на простые числа


13
100 лекций по математике для детей. Лекция 13.
https://www.youtube.com/watch?v=hSMPHJmxeC0

0-00 Таблицы умножения где есть нули
3-00 эмпирика показывает что если число простое, нулей нет. А как доказать?
4-30 101 - простое? Как проверить? До какого момента искать делители?
6-40 докажем что для 101 нулей в таблице нет (100х100)
7-50 подготовительная работа, рисуем линию с кузнечиками
10-00 скачок на 101 и 62, вопрос, где его можно обнаружить?
11-00 в какую ближайшую к нулю точку он может попасть?
14-30 в любой точке
17-30 доказали, что в строке есть единица, а есть ли ноль?
17-50 доказательство от противного
24-30 в любой другой строке тоже найдется 1 и не найдется 0
26-00 заменяем 62 на любое число от 0 до 100
27-00 докажем что любой кузнечик может прыгнуть в точку 1
36-00 алгоритм Евклида

15
2015_09_10 - 15-я лекция д. ф.-м.н. А. В. Савватеева Основная теорема арифметики: следствия ч. ¼
https://www.youtube.com/watch?v=znkH8vFwy6s

2-30 для чего нужна основная теорема арифметики?
4-00 пифагоровы треугольники
6-00 5 12 13
8-30 разбиваем плоскость на кватдратики, рисуем прямую

2015_09_10 - 15-я лекция д. ф.-м.н. А. В. Савватеева Основная теорема арифметики: следствия ч. 2/4
https://www.youtube.com/watch?v=S9G6DLcB4as

0-40 какие целые координаты попадают на прямую
3-00 аб делится на ц, а и ц взаимопросты, то б делится на ц
4-40 произведение взаимнопростых чисел равно степени, то каждое является той же степенью какого-то числа
9-00 два целых числа

2015_09_10 - 15-я лекция д. ф.-м.н. А. В. Савватеева Основная теорема арифметики: следствия ч. ¾
https://www.youtube.com/watch?v=6Dve5JifC7M

0-00 кузнечик прыгает на а или в, куда он может попасть?

2015_09_10 - 15-я лекция д. ф.-м.н. А. В. Савватеева Основная теорема арифметики: следствия ч. 4/4
https://www.youtube.com/watch?v=OocW2VaNzOE
0-00 математически запишем точки куда может попасть кузнечик

16
2015_09_23 - 16-я лекция д. ф.-м.н. А. В. Савватеева из цикла "100 лекций по математике" ч ¼
https://www.youtube.com/watch?v=q5hyCa59RsQ
1-00 систематизация задач из разных областей
2-00 линейное уравнение
5-00 точки линии

2015_09_23 - 16-я лекция д. ф.-м.н. А. В. Савватеева из цикла "100 лекций по математике" ч 2/4
https://www.youtube.com/watch?v=F8pOEEJnTq4

0-00 кузнечик
2-30 кузнечика можно заменить на НОД (аб)

2015_09_23 - 16-я лекция д. ф.-м.н. А. В. Савватеева из цикла "100 лекций по математике" ч ¾
https://www.youtube.com/watch?v=sjOQAQjpEJo
4-10 НОД = ам+бн

015_09_23 - 16-я лекция д. ф.-м.н. А. В. Савватеева из цикла "100 лекций по математике" ч 4/4
https://www.youtube.com/watch?v=KIddxhgnY0Y
0-00 Нашли одну точку 7х-4у=3, как найти все?

17-18
2015_10_23 - 17, 18 лекцииА. В. Савватеева - Линейные уравнения. Окончание ч1/8
https://www.youtube.com/watch?v=_FPfuDGBifI

2-30 5х-9у=2  и общий вид ах+ву=с
6-00 возрастает или убывает прямая? Какой наклон?
8-00 10х-18у=3

2015_10_23 - 17, 18 лекцииА. В. Савватеева - Линейные уравнения. Окончание ч2/8
https://www.youtube.com/watch?v=sXCmzh10gbk

1-30 делим на 2, рисуем прямую, которая избегает все целые точки
6-30 если с не делится на НОД(а в), то решений целочисленных нет
8-20 если с делится на НОД, то решение всегда есть

2015_10_23 - 17, 18 лекцииА. В. Савватеева - Линейные уравнения. Окончание ч3/8
https://www.youtube.com/watch?v=o-lkEv4Iyzk

0-00 кузнечик 5 и 9 
4-50 как найти все решения
9-30 общий вид решений уравнения

2015_10_23 - 17, 18 лекцииА. В. Савватеева - Линейные уравнения. Окончание ч4/8
https://www.youtube.com/watch?v=oh01aBMU9wA

2015_10_23 - 17, 18 лекцииА. В. Савватеева - Линейные уравнения. Окончание ч5/8
https://www.youtube.com/watch?v=WgDFw5NT7Bw

0-00 прямые не пересекаются, почему? Одинаковые левые части. Нет общих точек
4-00 прибавление вектора
6-00 (4+9к, 2+5к)
9-20 кузнечик и евклид

2015_10_23 - 17, 18 лекцииА. В. Савватеева - Линейные уравнения. Окончание ч6/8
https://www.youtube.com/watch?v=nmwLRwcuSTk

1-00 шаги алгоритма евклида для кузнечика
7-30 решение однородного + любое решение
9-00 три шага 1) делим на НОД

2015_10_23 - 17, 18 лекцииА. В. Савватеева - Линейные уравнения. Окончание ч7/8
https://www.youtube.com/watch?v=3vXdg0DHyik

1-00 цепные дроби
10-00 9/5 - в цепную дробь, в чем фокус?

2015_10_23 - 17, 18 лекцииА. В. Савватеева - Линейные уравнения. Окончание ч8/8
https://www.youtube.com/watch?v=FHVr3IiHBTg

1-00 18х+11у=1

19-20
2015_10_28 - 19-я и 20-я лекция д. ф.-м.н. А. В. Савватеева ч. ⅛
https://www.youtube.com/watch?v=m_N1Jc3HapU

2-30 соизмеримость отрезков
4-00 алгебраическая запись а=md b=nd
6-50 соотношение отрезков - рациональное число ⇔ соизмеримость

2015_10_28 - 19-я и 20-я лекция д. ф.-м.н. А. В. Савватеева ч. 2/8
https://www.youtube.com/watch?v=PsAxdahrv1Q

0-00 что если а и в несоизмеримы?
1-20 задача про кузнечика
2-50 ах+ву=с, а=корень из двух, в=1, сложное множество
6-00 если соизмеримы, то…
9-00 d(НОД(mn)z)

2015_10_28 - 19-я и 20-я лекция д. ф.-м.н. А. В. Савватеева ч. ⅜
https://www.youtube.com/watch?v=po68il5wqp0

1-30 все кратные отрезка d*НОД(mn)

2015_10_28 - 19-я и 20-я лекция д. ф.-м.н. А. В. Савватеева ч. 4/8
https://www.youtube.com/watch?v=9z8reMyp8ls

0-00 НОД (17 12)
2-00 геометрическая иллюстрация 17 на 12

2015_10_28 - 19-я и 20-я лекция д. ф.-м.н. А. В. Савватеева ч. ⅝
https://www.youtube.com/watch?v=Wi4R9Y-0XsI

0-00 геометрический алгоритм Евклида

2015_10_28 - 19-я и 20-я лекция д. ф.-м.н. А. В. Савватеева ч. 6/8
https://www.youtube.com/watch?v=vE6lFlpaVcE

0-00 цепные дроби 17/12
3-00 три ипостаси одного и того же
6-00 пишем любую цепную дробь

2015_10_28 - 19-я и 20-я лекция д. ф.-м.н. А. В. Савватеева ч. ⅞
https://www.youtube.com/watch?v=pHhQ_xoT1gA

Проверяем цепную дробь
Феномен цепных дробей - дроби с отличием на 1

2015_10_28 - 19-я и 20-я лекция д. ф.-м.н. А. В. Савватеева ч. 8/8
https://www.youtube.com/watch?v=xUPSD4zOPOM


0-00 ad-bc=+-1 GL2(Z)
2-00 кузнечик с соизмеримыми отрезками
3-00 геометрическое представление
7-00 "... словами и так до бесконечности!"

21-22
015_11_30 - лекция А. В. Савватеева - Перестановки ч. ¼
https://www.youtube.com/watch?v=-xtfF5TJUgs

00-00 перестановки и отображение множеств
01-00 задача о пальто
06-00 объяснение вероятности
08-00 перестановка
09-00 сколько всего перестановок
09-40 перебираем для 2, 3, 4
11-20 n!
15-20 принцип матиндукции
16-20 для 4
22-30 композиция
27-20 пример композиции шутников
32-00 случайно взяли перестановки которые не зависят от порядка :)
33-00 другой пример, где порядок важен
36-50 квадрат перестановки
https://www.youtube.com/watch?v=CrH3heukYPo

2015_11_30 - лекция А. В. Савватеева - Перестановки ч. ¾
https://www.youtube.com/watch?v=v0jx-_DGNDs

02-00 тау и сигма, расставлем скобки и получаем то же t=s(ss)=(ss)s почему?
05-00 ассоциативность - порядок скобок не важен
11-00 считаем сигму в четвертой степени
12-00 “очевидность в тумане”, “отображения становятся родными”
14-00 тождественная перестановка
15-00 для любой перестановки t существует n что t в степени n равно тождественной
21-00 полная таблица композиций
23-00 для любой перестановки есть обратное действие (и порядок не важен)
25-00 рисуем и заполняем табличку
28-30 важно хоть раз руками записать таблицу умножения
https://www.youtube.com/watch?v=5LyoWNk9fxw
В каждой строчке ровно 1 ид и в каждом столбце тоже, 
Все шесть разных, как и с остатками

23-24
Перестановки: циклы, чётность, порядок ч. ⅓
https://www.youtube.com/watch?v=UH3xQ6ewFoY&t=5s
02-00 возьмем перестановку из 8 символов
03-00 возведем в квадрат
04-00 греческий алфавит на мехмате
06-00 сигма в кубе
09-00 циклы внутри
11-00 коммутативность циклов
13-30 в пятой степени
14-00 запись циклов кругом
17-00 пятнадцатая степень (НОК)
18-30 сложение по модулю 15
19-00 перестановки образуют группу
20-00 ассоциативность, тождественный элемент, обратный элемент
23-30 любая перестановка имеет стпепень, в которой она тождественна
25-00 доказательство: разложение в циклы
31-00 произведение циклов коммутативно

Часть 2
https://www.youtube.com/watch?v=sVcXmm6kEeU
0-00 альтернативное доказательство
6-40 более сильный результат(самая маленькая степень), порядок
11-00 четность
14-00 сколько инверсий
15-00 четность композиции - по четности каждой из них (складывается)
17-30 умножаем перестановку на транспозицию и наооборот

Часть3
https://www.youtube.com/watch?v=iBV-l5rlzCo
2-00 любая перестановка - произведение транспозиций
5-40 доказательство по циклам
9-00 при перемножении четность складывается
10-00 записываем на языке алгебры, гомоморфизм групп, композиция переходит в композицию

25-26
А. В. Савватеева - задачки про перестановки ч. ⅓
https://www.youtube.com/watch?v=0wDEIWHDAQ0
00-00 свойства 2016 года (делится на 2 в пятой, подмножество попарных клеток шахматной доски)
3-00 четность и гомоморфизм
8-00 обратная перестановка и ее четность
11-00 подгруппа четных перестановок
15-00 хитрая таблица умножения(нечетные+четные)
21-00 игра пятнашки и почему нельзя поменять два символа местами
26-00 доказательство. Сопоставим каждому положению - перестановку
29-00 ошибки кострикина и сопоставление змейкой

Савватеева - задачки про перестановки ч. 3/3
https://www.youtube.com/watch?v=-VfkQn-pm1Q&t=6s
1-00 что происходит когда двигаем фишку(в 15)
13-00 инвариант
15-00 сопряжение
25-00 все перестановки с одинаковой структурой по циклам сопряжены друг с другом
26-00 классы сопряженности
28-00 группа S4, какие есть структуры циклов
31-00 связь с корнями кубических и 4х степеней
33-00 математика стала наукой в 18 веке

27-28
Группа Клейна ч. ⅓
https://www.youtube.com/watch?v=DrJeIgyT-Os
00-00 словарик (ядро, подгруппа, сопряжение, гомоморфизм)
2-00 S4, мощность, циклическая структура
5-00 перечисляем все циклы
11-00 выписываем все элементы и домножаем на сопряженное
20-30 Sn-Sn! Является гомоморфизмом групп
26-00 расслаивается на более простые
33-00 изоморфизм

Группа Клейна ч. ⅔
https://www.youtube.com/watch?v=xH6zQ2POq1I&t=10s
3-00 перестановка “знак”
5-30 утверждение. Других гомоморфизмов нет
7-00 ядра и какие бывают
9-30 перечислим все подгруппы в S3
14-00 если (12) лежит в ядре, то наш гомоморфизм тривиален
20-30 ядром может служить только та подгруппа, за пределы которой мы не выходим операцией сопряжения
21-00 нормальная подгруппа - та что может служить ядром гомоморфизма
22-30 порождающие
27-00 S4-S1, S4-S6, S4-S8, S4-S3
29-30 поэтому существует формула для корней
30-00 S5
32-20 самое интересное - S4-S3
33-00 принцип Дирихле
https://www.youtube.com/watch?v=uJsNjGx0KfM
 01-00 4 перестановки являются нормальной подгруппой
2-50 коммутативность
4-00 четверная группа

29-30
29, 30 лек. А. В. Савватеева ч. ½ перестановки-деликатесы
https://www.youtube.com/watch?v=YI1CESDmfrM
00-00 повторение пройденного (12)(34), группы клейна и пр
7-00 проецирование
16-00 что сохраняется для 4х точек, двойное отношение четырех точек на прямой
25-00 “проверочные перестановки”
26-00 задача: порождающие элементы
29-00 S3 :  λ 1-λ  λ^(-1)
34-00 теорема - два порождающих

Часть2
https://www.youtube.com/watch?v=6372Er-XFas
00-00 таблица умножения  λ
2-00 доказательство про два порождающих для Sn
10-00 теорема An порождается “трёшкой”
14-00 возвращаемся к деликатесам
20-00 определитель, площадь параллелограмма
27-50 при чем здесь перестановки
29-00 пареллелипипед

31-32
Движения плоскости. Начало ч. ½
https://www.youtube.com/watch?v=0d2uMhmF1NI
00-00 вспоминаем прямую и окружность
4-30 перенос плоскости на вектор (вектор=паралельный перенос)
6-40 сумма двух векторов
8-30 поворот
9-30 композиция поворотов относительно разных точек
11-00 ориентация
11-30 прасолов
15-30 ориентация пространства
19-00 отражения
21-00 все ли это движения?
23-00 теорема: неподвижные 2 различные точки - то неподвижной является прямая через эти точки. Если есть еще неподвижная вне этой прямой - то это ID. (о трех гвоздях)

Часть 2
https://www.youtube.com/watch?v=jrDuBiFC2io&t=14s
00-00 Теорема - любое движение плоскости является композицией не более чем трёх отражений
23-00 4 варианта
39-00 скользящая симметрия

33-34
Скользящая симметрия ч. ½
https://www.youtube.com/watch?v=ACpV2PJZfdM
00-00 1- 2- 3- отражения
5-00 L(k,v)
9-00 отражение+перенос
14-00 таблица композиции
22-00 гомоморфизмы
28-00 перенос не меняет угол поворота (T*R)

Часть 2
https://www.youtube.com/watch?v=EBe5eVzIOAM
00-00 R*T
5-00 R*R
14-00 интересная задача, если линии окажутся параллельными
26-00 задача: рассмотрим любой треугольник на плоскости, на каждой стороне строим равносторонний, треугольник, соединяющий центры - тоже равносторонний

35-36
Комплексные числа ч. ½
https://www.youtube.com/watch?v=34XPQUkqytU
1-00 является ли ноль натуральным числом
5-00 целые числа для решения уравнений, чтоб и складывать и вычитать
6-20 умножение
9-00 рациональные числа
13-00 х*х+1=0
16-20 рисуем комплексную плоскость
23-00 умножение комплексных чисел
27-00 гауссовы числа (представление в виде суммы двух квадратов)

Часть 2
https://www.youtube.com/watch?v=B0XQQ_KEuew
1-30 сопряженное число
3-30 произведение сопряженных = сопряженному произведения
7-00 сопряжение - это автоморфизм
10-00 умножение числа на сопряженное
16-00 деление
21-00 уравнение единичной окружности
25-00 косинус и синус
30-00 умножаем на окружности
31-30 расстояние между двумя числами
38-00 углы складываются, модули перемножаются


37-38
Геометрия комплексных чисел ч. ¼
https://www.youtube.com/watch?v=HhxL2NaS4TQ
00-00 повторяем комплексные числа, вводим новые числа, когда старых не хватает
5-00 пару слов про линейную алгебру
12-00 гомотетия
15-00 разница “отображения” и “преобразования”
16-30 умножение на вещественное число
21-00 уравнение единичной окружности, тригонометрическая окружность
29-00 умножение двух комплексных чисел
31-00 сопряжение

https://www.youtube.com/watch?v=4de2P0Xtr_M
Произведение нормы равно норме произведения

Часть 3
https://www.youtube.com/watch?v=20HDjlZ99Kk
00-00 доказательства про норму произведения
16-00 Теорема: эф от зэд с чертой является движением, то есть сохраняет расстояние
24-00 если движение имеет одну неподвижную точку - это поворот
32-00 тригонометрия комплексных чисел, синус и косинус

https://www.youtube.com/watch?v=z9cz_6mztNw
При умножении углы складываются, теорема косинуса и синуса суммы

39-40
Комплексные числа и их арифметика (повторение) ч. ½
https://www.youtube.com/watch?v=i3f0jCEqm9k
01-00 натуральные числа
03-00 целые числа
06-00 дроби и делимость
15-00 корень из 2
26-00 х*х+1=0
32-00 единичная окружность на комплексной плоскости

Часть 2
https://www.youtube.com/watch?v=yzjV-RJrvMs
03-00 сопряжение
5-30 правила арифметики - сложение, умножение
14-00 деление
32-00 расстояние до нуля произведение равно произведению расстояний

41
Умножение комплексных чисел
https://www.youtube.com/watch?v=JIUc0-ssdXM
01-00 рисуем координатные оси
02-00 z=3+2i
04-00 z=hw, w - единичной длины
09-00 про углы в математике
11-20 скользящая симметрия
13-00 берем все точки и домножаем на z
14-00 ноль переходит в ноль, а единица?
17-00 куда перейдет i
19-30 при умножении на скаляр плоскость растягивается
23-00 как изменится расстояние между двумя точками?
25-00 есть ли еще точки которые остаются на месте?
29-00 если одна точка остается неизменной - тогда и только тогда когда поворот
34-00 сухой остаток: углы складываем и модули умножаем
42
Умножение и корень из комплексного числа
https://www.youtube.com/watch?v=Oe2i19tWONU
01-00 z/|z|=w
03-00 раскладываем на две операции - сначала умножаем на модуль z
05-00 расстояние между двумя точками
07-00 какие точки остаются на месте
08-40 если одна неподвижная точка - значит поворот
09-30 модули умножаются, углы складываются (поворотная гомотетия)
10-30 корень из комплексного числа
11-40 что значит извлечь корень - решить уравнение х*х=у
13-00 логарифм
16-20 корень квадратный. q*q=z
18-30 это единственное решение? Нет
20-30 кубический корень
23-00 где лежат еще два корня?
26-30 корень третьей степени из 1
28-00 находим любой корень числа, а потом умножаем его на корень из 1 (произвольной степени)
30-00 корень четвертой степени из 1
31-00 правильный н-угольник - решения корня н-ной степени
36-00 косинус и синус

43
Знакомство с гауссовыми числами
https://www.youtube.com/watch?v=sS42R8ReYtU
00-00 Ферма заметил а Гаусс добил
01-00 сколькими способами можно представить число в виде суммы двух квадратов
04-00 3, 5 ,6
05-00 прямоугольный треугольник
06-00 упрощение при помощи комплексных чисел (произведение сопряженных чисел)
07-00 где живут все целые комплексные числа (сетка с шагом 1)
09-00 числа из сетки можно складывать вычитать и умножать - они остаются на сетке (кольцо)
14-00 рисуем название и объясняем его Z[i]
15-00 гауссово число - целое комплексное число. Z[i]
22-00 пятерка перестает быть простым числом 5=(2+i)(2-i), а тройка осталась
25-00 а двойка?
26-00 число 7 остается простым
29-00 супертеорема Гаусса (остаток 1 при делении на 4)

44
https://www.youtube.com/watch?v=S15xTfUwD9Q
00-00 на какое число делятся все числа?
07-00 таблица умножения 1,i,-1,-i
15-00 простые гауссовы числа, ассоциированные
18-00 докажем что из второго утверждения теоремы Гаусса следует первое
30-00 докажем что из второго следует третье
34-40 Основная теорема арифметики в гауссовых числах

45
Основная теорема арифметики для Гауссовых чисел
https://www.youtube.com/watch?v=G_At-NjZPW8
00-00 формулировка трёх эквивалентных утверждений
04-40 Теорема: любое гауссово число представляется в виде произведения простых чисел единственным образом с точностью до ассоциированного
12-00 если p простое вида 4к+1 то существует целое, такое, что его квадрат+1 делится на р
21-40 доказательство от противного
46
Деление с остатком и идеалы.
https://www.youtube.com/watch?v=2ReZhJb5bls
02-10 7+i поделить на 3+2i
03-00 почему не делится нацело?
06-00 домножаем на сопряженные
09-00 рисуем решетку 
21-00 другие варианты деления с остатком (еще 2 штуки, иногда 3)
29-00 еще один способ деления с остатком (домножаем числитель и знаменатель на сопряженное)

47
Основная теорема арифметики для Гауссовых чисел 2
https://www.youtube.com/watch?v=iqQSP45KPtc
02-00 для любых двух чисел существует еще два числа со следующими двумя свойствами
07-00 алгоритм Евклида и комплексный кузнечик
07-20 рассмотрим алгоритм Евклида
27-00 делим 7 на 3+i
35-00 делим 10+i на 4+3i

48
Основная теорема арифметики для Гауссовых чисел 2 (часть2)
https://www.youtube.com/watch?v=hI_liJTGhDg
00-00 продолжаем предыдущее деление
09-00 кузнечик может прыгать на 10+i на 4+3i
12-00 теорема: он попадет в любую точку
15-00 обобщенная теорема
24-00 Ключевой ход ОТА
29-00 доказательство
33-00 однозначность разложения с точностью до ассоциированности
39-00 числа Эйнштейна

49
Пифагоровы тройки, общая формула. ч. ½
https://www.youtube.com/watch?v=RuNMBUi-T4k
00-00 как описать все целые тройки
02-00 3 4 5
03-00 локальные решения
8-40 более сложная серия троек
09-00 уравнение “Перья”
12-00 есть ли подобный прямоугольный подобный треугольник?
15-00 требование: попарно простые числа
17-00 ABC-гипотеза
20-00 замечания про четность
24-00 Гауссовы числа
25-00 рассмотрим случай когда целое делится на гауссово число
32-00 если m делится на l+ki, то  m делится на l*l+k*k

50
Пифагоровы тройки, общая формула. ч. 2/2
https://www.youtube.com/watch?v=VfQgmICeP-g

00-00 продолжение доказательства если m делится на l+ki, то  m делится на l*l+k*k
00-02 делимость на 1+i
10-00 теорема НОД (a+bi, a-bi)=1
21-00 раскладываем a+bi на простые
28-00 a+bi - равно квадрату гауссова числа с точностью до умножения на {1 i -i -1}
32-00 существует гауссово m+ni такое, что либо a+bi = (m+ni)^2 либо i (m+ni)^2
38-00 a=m^2-n^2 b=2nm c=m^2+n^2

51
Конечная арифметика. Теорема Безу
https://www.youtube.com/watch?v=SoA5uIdqMLU
01-40 таблицы умножения и сложения остатков (mod 2 3 4 5 6)
08-00 Z/2Z и “переименование”
21-00  таблица 1-3-5-7
26-00 Z/4Z и (Z/2Z)+(Z/2Z)
28-00 решение 2х=3 в различных числовых системах
36-00 mod 10  таблица 1-3-7-9

52
Часть2
https://www.youtube.com/watch?v=o4UYwo8ozAw
00-00 Z/4Z
02-00 график для умножения конечных множеств
05-00 многочлены - центральная роль в математике
08-00 график с другими остатками x^2-1 mod 8
11-00 в “нормальной” арифметике многочле имеет корней не больше чем его степень
21-00 деление многочлена в столбик
31-00 теорема Безу
33-00 арифметика “хорошая”, если корень многочлена “наследуется”

53
Теорема о корнях многочленов ч. ½
https://www.youtube.com/watch?v=PukfSyLT-TA
01-00 позиционные системы счисления, остатки по модулю 10
05-00 хитрый способ умножения
11-00 многочлены с коэффициентами из кольца/поля
15-40 бывают разные приколы, например по модулю 6
20-00 Теорема Безу
29-00 mod 8 рассмотрим x^2-1 (4 корня: 1 3 5 7)

54
Теорема о корнях многочленов Часть2
https://www.youtube.com/watch?v=wzmlAEkxbVM
05-00 разгадка - два эквивалентных разложения
06-00 абсурдные явления
10-00 исследуем многочлен на количество корней
28-00 вывод о многочленах н-ной степени, которые имеют н-корней
32-00 основная теорема алгебры

55-56
Теоремы Виета, Вильсона и Ферма (малая)
https://www.youtube.com/watch?v=8bpqTvADN8Q

04-00 теорема Вильсона
06-00 теорема Виета
16-00 МТФ
24-00 рассмотрим частные случаи (делимость на 7)
40-00 рисуем диаграмму степеней 3 при делении на 7
56-00 зацикливание на 1
59-00 рассмотрим 3*2, 3*2*2, 3*2*2*2, …,
1-07-00 рассмотрим для 4
1-13-00 для 5
1-14-00 первообразный корень

57-58
Малая теорема Ферма: осн. следствие. Теорема Вильсона
https://www.youtube.com/watch?v=9M5w4PDwJZg
0-00 МТФ
05-00 для многочленов
27-00 теорема Вильсона

53-30 проверяем для р=5: 4!
54-00 р=7
56-00 р=13
57-20 при произвольном р=4к+1, (р-1)!= (2к!)^2
59-00 доказали, что для р=4к+1 существует остаток с, что с^2+1 делится на р
1-02-00 для гауссовых чисел
1-17-00 альтернативные рассуждения о существовании “с”

59
Геометрия, арифметика и алгебра преобразований ч. 1/2
https://www.youtube.com/watch?v=gXT7X_Zkl3A
03-00 ставим треугольнику в соответствие его самый большой угол
05-30 сюръекция
07-00 Какой диапазон будет покрываться всевозможными треугольниками [60, 180)
09-20 инъекция
14-00 биекция
16-30 композиция отображений
18-00 ассоциативность композиций
21-00 каждому человеку ставим в соответствие отца. Это что?
22-30 проекция плоскости на прямую
24-00 числу сопоставляется остаток от деления на 7
26-00 подобие
34-00 гомотетия
37-00 гомотетия на прямой

60
Геометрия, арифметика и алгебра преобразований ч. 2/2
https://www.youtube.com/watch?v=cnRsQvjv57c
00-00 композиция H(0,A) * H(0,R), вращение сферы
05-00 расширение класса переворачиваний
07-00 групповые изоморфизмы
15-00 история математика в деревне
18-00 Гомотетии с центром 0  изоморфны группе
20-00 числа - это гомотетии и это перенос
22-30 Числа - это преобразования (сложение - перенос, умножение - гомотетия)
25-00 гомотетии с центрами в разных точках
30-00 доказательство что это перенос, если произведение λ и μ равно 1
==

81-82
https://www.youtube.com/watch?v=JM_7-JcD4IQ
00-00 кузнечик с ногами 1 и альфа
2-30 альфа - иррациональная => слепых зон нет
3-50 всюду плотное множество
4-40 доказательство. Всю прямую наматываем на окружность
12-50 для любого эпсилон можно найти точки кратные альфа в пределах эпсилон, сколь угодно малого
13-10 доказательство от противного
18-30 иррациональные числа бывают разные
20-20 приделаем кузнечику еще одну ногу альфаквадрат
29-00 степень алгебраического числа
32-50 подход 1, альфа алгебраическое, если существует многочлен с целыми коэф
34-30 второй подход
37-00 многочлен самой маленькой степени, как вычислить?
39-30 доказательство что корень кубической степени из двух имеет степень 3
40-00 ступень 1
41-10 ступень 2 (лемма Гаусса)
45-00 ступень 3
47-00 докажем ступень 3
57-00 докажем ступень 1
1-05-00 что если в общем случае (а не корень кубический из 2)
1-09-00 многочлен нечетной степени всегда имеет корень


83-84
Всё про корень кубический из двух
https://www.youtube.com/watch?v=vzQrVQT6M7A
00-00 Всё о корне кубическом из 2
1-00 удвоение куба
2-00 x^3-2=(x-p)(x^2+) - неразложим на множители
7-00 утверждение: все выражения вида p+q∛2 +r∛4 различны
13-30 следовательно размерность поля равна 3, нужно 3 параметра, чтобы описать элемент
15-00 почему это поле
16-00 ∛2 - цепная дробь
16-30 что происходит при умножении
20-00 деление
50-00 ∛2 - цепная дробь с нечетными числами идущими по порядку (гипотеза)
52-00 ∛2 нельзя построить циркулем и линейкой (доказательство)
52-20 как циркуль и линейка связаны с полями
53-00 античные задачи: трисекция, квадратура, многоугольники
54-00 лемма. ∛2 не умеет находиться внутри полей, размерность которых не делится на 3
55-00 произведение размерностей полей, башня полей
1-07-00 построить циркулем и линейкой означает построить последовательность полей 2^n
1-12-00 немного про лемму гаусса
1-13-00 центральная теорема начальной алгебры (о неприводимых многочленах)

85-86
Теория построений циркулем и линейкой ч.1
https://www.youtube.com/watch?v=2A_eHukU3RI

01-00 2500 лет назад
1-50 делим угол пополам
2-30 перпендикуляр
3-00 если заданы 2 оси и масштаб, какие точки можно построить?
6-00 все рациональные точки мы можем построить
8-40 главный постулат построения циркулем и линейкой
10-30 все построени ЦЛ - это вычисление на калькулятое +-*√
10-50 доказательство
13-00 есть отрезок а, как получить 1/а?
16-50 извлекаем корень
19-30 утверждение наоборот тоже верно: все что строится ЦЛ можно вычислить на квадратичном калькуляторе
22-00 доказательство
28-00 как выглядит уравнение окружности
31-30 пересечение двух прямых
32-50 прямая пересекает прямую

Теория построений циркулем и линейкой ч.2
https://www.youtube.com/watch?v=rskahqqen8s
00-00 пересечение прямой и окружности
5-00 пересечение двух окружностей
13-30 четыре античные задачи - трисекция, удвоение, квадратура построение
14-50 синус - это число, арифметическая природа которого запрещает вычислять на классическом калькуляторе
21-00 построение многоугольника=построение синуса
22-00 построение правильного пятиугольника
31-30 строятся синусы 18, 15 и 3 градусов
34-00 девятиугольник
36-00 синус 10 градусов
38-00 косинус и синус суммы углов

87-88
Вычислимость на квадратичном калькуляторе
https://www.youtube.com/watch?v=MSenKHJXbuo

00-30 Хотим понять, какие числа можно, а какие нельзя получить ЦЛ/квадратичным калькулятором
1-30 есть ∛2, есть синус 10, число пи
4-50 многоугольники, cos 2pi/7
6-30 рассмотрим семиугольник
8-50 сумма 7 векторов равна 0
13-30 используем это и произведем некоторые вычисления
19-40 2пи/7 является корнем х^3+x^2-2x-1=0, этот многочлен неразложим
21-30 докажем неразложимость
37-00 получаем что размерность поля равна 3, и следовательно невычислимо на квадратичном калькуляторе
39-30 покажем что нельзя построить 10 градусов
41-00 косинус и синус суммы углов
42-00 синус 3а
50-30 рисуем 2sin10 на комплексной плоскости
57-00 вся тригонометрия - перепись комплексных чисел на языке вещественных и мнимых чисел
59-00 неразложимость x^5-3x+3=0
1-00-30 идея Галуа
1-02-00 размерность - 5!
1-08-00 автоморфизмы


89-90
Волшебство К.Ф.Гаусса: построение правильного 17-угольника
https://www.youtube.com/watch?v=vk3xDVkHG-Q
0-00 напоминание что делали в прошлом году
9-30 спустя 2000 лет, Гаусс ответил на вопрос о построении
10-20 ключ к разгадке - комплексные числа
10-50 начинаем рисовать 17 угольник
16-20 Гаусс начал группировать числа и построил двойной косинус
17-20 257-угольник и 65537 угольник, что это за числа?
20-00 числа Ферма
24-40 общий метод построения и частный
25-40 поле и кольцо деления круга
26-00 продолжаем построение
30-10 теорема: произведение ню равно сумме ню
35-00 лемма: сумма всех ню равна -1
37-30 лемма2: АВ=-4
39-30 умножаем и заносим результаты в таблицу
46-46 зачем все это было. Рассмотрим многочлен (х-А)(х-В) и раскроем скобки
49-30 получаем (2x+1)^2=17
51-30 Вичислили А и  В
55-00 вводим альфа бета гамма тэта  - суммы ню
55-20 лемма: альфа*бета=гамма*тэта
1-00-30 решаем упавнение
1-02-30 вычислили альфа бета, гамма тэта
1-10-00 пишем двойной косинус
1-16-00 алгоритм построения 17угольника  ЦЛ

91-92
Канторова теория множеств: первые наблюдения
https://www.youtube.com/watch?v=jQFcM7YDyvo
00-00 бесконечные множества бывают разные
2-30 все целые можно вложить в натуральные
5-00 бесконечное - значит можно вложить в себя и еще что-то останется
6-00 опр: множества равномощные если сущетсвует взаимнооднозначное отображение
7-50 теорема: Q=N
9-00 доказательство
17-50 если множество равномощно натуральном - оно “счетно”
20-00 корни уравнений - счетные, доказательство
27-10 гауссовы числа- счетны, все узлы сетки - счетны
29-00 лемма: подмножество счетного множества всегда счетно
31-00 объединение счетного количества счетных множеств тоже счетно
33-30 Теорема Кантора
38-00 доказательство
46-30 множество точек отрезков
47-00 теорема Кантора-Берштейна
51-40 существует способ сказать что одно множество больше другого
1-07-00 мощность континуума
1-08-00 интервал равномощен бесконечной прямой, изгибаем чашу
1-10-00 отрезок - вырезаем точки

93-94
Вещественные числа: аксиома полноты
https://www.youtube.com/watch?v=1GAIDF8TPYQ
00-00 аксиома полноты, вещественные числа
1-00 в любой ли точке живет число?
3-00 алгоритм нахождения любого числа/точки
6-50 соответствие N и R
11-50 на прямой есть точки к которым можно обратиться только аз бесконечное количество итераций
13-20 любая последовательность вложенных отрезков имеет непустое пересечение
14-30 последовательность приближений
16-00 Любая монотонная ограниченная последовательность имеет предел
18-30 монотонность
20-30 ограниченность
25-30 предел монотонной последовательности
31-00 последовательность вычисления корня
33-00 определение предела
41-00 последовательность, которая обращается к числам
42-00 определить поведение последовательности
42-30 определение Коши
47-00 как учат матан в плохом вузе
48-00 А3: любая фундаментальная последовательность имеет предел
50-30 подходы к исследованию отсутствия дыр :)
52-00 последовательность n^k/2^n
54-00 экспонента забивает степенную
55-00 последовательность a^n/n!
58-00 подмножество R - ограничено если…
59-00 уточнение: ограничено сверху, если..
1-00-00 ограничено снизу, если…
1-01-00 идея полноты
1-03-30 верхняя грань множества А
1-09-00 модель вещественных чисел
1-12-30 человечество пришло к вещественным числам 150 лет назад
1-15-00 Сечение Дедекинда

95-96
сечения Дедекинда и другие модели вещественных чисел
https://www.youtube.com/watch?v=eFIWK1NVJcA
00-01 аксиомы чисел
2-30 ответы начинаются с Кантора: множества всех точек прямой больше чем натуральные
3-00 число по Д: множество рациональных, такое что…
6-00 какие сечения Д отвечают требованиям рациональных чисел?
13-00 очевидность заменяем на аксиому
13-30 для любого эпсилон, существует эн, что 1/n<e
16-00 что такое сумма двух чисел (сумма минковского раньше минковского)
17-40 нерешенная задача математики - об одновременном “падении” множеств
24-10 три аспекта вещественных чисел
25-10 коммутативная группа по сложению
26-00 группа по умножению
26-30 отдельно стоящий закон - распределитель
27-00 означает поле
27-05 упорядоченность поля
27-50 еще две аксиомы связанные с порядком
29-30 доказываем что 1>0
31-10 модель бесконечных -ичных дробей
37-00 почему 0,99999...999 в точности равно 1
38-50 сложение и умножение, выяснение числа какого-то разряда
40-00 сравнение последовательностей
42-10 теорема: точки отрезка 0-1  то же самое, что и последовательности 0 и 1 (без хвостов)
48-00 2^N=R
50-30 проблемы Гильберта
56-00 о корнях, существует ли?
1-00-00 теорема: корни корней корней... стремится к 1
1-10-00 о степенях
1-10-30 n^k/f^n=>0
1-15-00 зажимание последовательности двумя другими
1-16-30 a^n/n!

97-98
https://www.youtube.com/watch?v=7fS1ED5wUEE
Экспонента как решение функционального уравнения

00-00 что такое экспонента
2-00 ищем функции f(x+y)=f(x)+f(y)
15-40 ответ - любая прямая через ноль
17-00 ищем функции f(x+y)=f(x)*f(y)
24-40 лемма: функция не принимает значения ноль
32-00 рисуем a^n
35-00 что происходит с дробями
48-00 непрерывна ли эта функция
53-00 выясняем наклон функции
59-30 радиоактивность, распад урана
1-01-30 размножение бактерий
1-02-00 давление воздуха с высотой
1-03-30 функция которая равна производной
1-05-00 какая функция стартует с 0 под углом 45 градусов?
1-07-00 год рождения Льва Толстого и е
1-10-30 Теорема:е=1+х+х2/2!+...
1-25-00 e=lim(1+x/n)^n

99-100
https://www.youtube.com/watch?v=hUrYMf5eKMI
0-00 Урок на вырост
1-20 доказать e^ipi=-1
3-00 экспонента и производная
4-30 ряд 1+х + x^2/2! + +
5-30 бином ньютона для 1+(х+у)+(х+у)^2/(х+y)!
7-30 бином это (х+у)^n= Cnk+...
11-30 запишем в столбик и увидим
13-30 можно ли умножать бесконечные суммы
17-00 запишем еще одним способом е^х = lim(1+x/n)^n
19-00 сложный процент
19-30 разложим по биному ньютона
28-00 для любого n последовательность меньше чем e^n
29-50 берем такое большое к, что эта сумма отличается от е^x на 1%, фиксируем к и берем n сильно сильно больше чем к
33-50 вещественная - однозначно задавла функцию, комплексная - не однозначно
34-30 комплексная дифференцируемость
37-40 рисуем на комплексной плоскости сумму 1+z+z^2/2!+z^3/3!+...
40-00 всё сходится к конкретному комплексному числу - комплексной экспоненте
41-30 бином ньютона для матриц не годится, экспонента есть, а формула не действует
43-00 подставим комплексное число и посмотрим что происходит
46-00 фундаментальная лемма
50-00 доказательство
57-50 доказательство предел()*предел()= предел ()*()
59-00 раскрываем скобки
59-30 дополнение к лемме
1-04-30 доказательство формулы Эйлера
1-12-00 что из этого можно извлечь (cos+sin)
1-16-00 матричная экспонента




















оглавление
1 лек. "Числа, символы и фигуры" 4-5 классы ч. 1:2 (2014-03-13) (30:26)
1 лек. "Числа, символы и фигуры" 4-5 классы ч. 2:2 (2014-03-13) (06:09)
2 Лек. "Соизмеримость и несоизмеримость отрезков" 4-5 классы ч. 1:2 (2014-04-01) (30:27)
2 Лек. "Соизмеримость и несоизмеримость отрезков" 4-5 классы ч. 2:2 (2014-04-01) (15:08)
3 Лек. "Визуальное представление бинома Ньютона" для 5-6 классов ч. 1:3 (2014-09-17) (30:28)
3 Лек. "Визуальное представление бинома Ньютона" для 5-6 классов ч. 2:3 (2014-09-17) (08:46)
3 Лек. "Визуальное представление бинома Ньютона" для 5-6 классов ч. 3:3 (2014-09-17) (16:54)
5 Начальное представление о движении (36:32)
6 Классификация движений прямой (37:41)
7 Таблица умножения движений прямой (33:44)
8 Движения окружности (33:29)
9 Таблица умножения движений окружности (34:58)
10 Конечные подгруппы движений прямой и окружности (36:07)
11 Введение в арифметику остатков (37:29)
12 Таблицы умножения остатков (37:04)
13 Основная теорема арифметики (начало) (39:02)
14 лекция Основная теорема арифметики. Начало ч. 1 2 (доска) (37:14)
14 лекция Основная теорема арифметики. Начало ч. 1:2 (37:14)
14 лекция Основная теорема арифметики. Начало ч. 2 2 (доска) (01:59)
14 лекция Основная теорема арифметики. Начало ч. 2:2 (01:42)
15 Основная теорема арифметики. Следствия 1:4 (10:39)
15 Основная теорема арифметики. Следствия 2:4 (10:40)
15 Основная теорема арифметики. Следствия 3:4 (10:40)
15 Основная теорема арифметики. Следствия 4:4 (04:02)
16. Решение линейных уравнений в целых числах 1:4 (10:39)
16. Решение линейных уравнений в целых числах 2:4 (10:40)
16. Решение линейных уравнений в целых числах 3:4 (10:40)
16. Решение линейных уравнений в целых числах 4:4 (07:55)
17 : 18 "Линейные уравнения. Окончание" ч1:8 (10:39)
17 : 18 "Линейные уравнения. Окончание" ч2:8 (10:40)
17 : 18 "Линейные уравнения. Окончание" ч3:8 (10:40)
17 : 18 "Линейные уравнения. Окончание" ч4:8 (01:33)
17 : 18 "Линейные уравнения. Окончание" ч5:8 (10:39)
17 : 18 "Линейные уравнения. Окончание" ч6:8 (10:40)
17 : 18 "Линейные уравнения. Окончание" ч7:8 (10:40)
17 : 18 "Линейные уравнения. Окончание" ч8:8 (08:20)
19 : 20 "д. ф.-м.н." ч. 1:8 (10:39)
19 : 20 "д. ф.-м.н." ч. 2:8 (10:40)
19 : 20 "д. ф.-м.н." ч. 3:8 (10:40)
19 : 20 "д. ф.-м.н." ч. 4:8 (02:41)
19 : 20 "д. ф.-м.н." ч. 5:8 (10:39)
19 : 20 "д. ф.-м.н." ч. 6:8 (10:40)
19 : 20 "д. ф.-м.н." ч. 7:8 (10:40)
19 : 20 "д. ф.-м.н." ч. 8:8 (07:26)
21:22 "Перестановки" (2015.11.30) ч. 2:4 (01:08)
21:22 "Перестановки" (2015.11.30) ч. 1:4 (37:15)
21:22 "Перестановки" (2015.11.30) ч. 3:4 (37:15)
21:22 "Перестановки" (2015.11.30) ч. 4:4 (03:27)
23, 24 лек. "Перестановки циклы, чётность, порядок" (2015 12 24) ч. 1:3 (37:14)
23, 24 лек. "Перестановки циклы, чётность, порядок" (2015 12 24) ч. 2:3 (25:25)
23, 24 лек. "Перестановки циклы, чётность, порядок" (2015 12 24) ч. 3:3 (13:42)
25 : 26 "задачки про перестановки" ч. 1:3 (37:15)
25 : 26 "задачки про перестановки" ч. 2:3 (00:48)
25 : 26 "задачки про перестановки" ч. 3:3 (36:55)
27, 28 лек. "Группа Клейна" (2016.02.16) ч. 1:3 (35:08)
27, 28 лек. "Группа Клейна" (2016.02.16) ч. 2:3 (37:15)
27, 28 лек. "Группа Клейна" (2016.02.16) ч. 3:3 (05:47)
29, 30 лек. "Перестановки. Дликатесы" (2016.03.22) ч. 1:2 (34:51)
29, 30 лек. "Перестановки. Дликатесы" (2016.03.22) ч. 2:2 (39:43)
31, 32 лек. 2016 04 12 Движения плоскости. Начало ч. 1:2 (36:14)
31, 32 лек. 2016 04 12 Движения плоскости. Начало ч. 2:2 (40:11)
33, 34 лек. 2016 04 19 Скользящая симметрия ч. 1:2 (37:51)
33, 34 лек. 2016 04 19 Скользящая симметрия ч. 2:2 (37:25)
35, 36 лек. 2016 04 26 Комплексные числа ч. 1:2 (30:09)
35, 36 лек. 2016 04 26 Комплексные числа ч. 2:2 (39:38)
37, 38 лек. "Геометрия комплексных чисел" ч. 1:4 (37:17)
37, 38 лек. "Геометрия комплексных чисел" ч. 2:4 (03:00)
37, 38 лек. "Геометрия комплексных чисел" ч. 3:4 (37:15)
37, 38 лек. "Геометрия комплексных чисел" ч. 4:4 (03:04)
39, 40 лек "Комплексные числа и их арифметика (повторение)" ч. 1:2 (36:57)
39, 40 лек "Комплексные числа и их арифметика (повторение)" ч. 2:2 (39:44)
41 лек. "Умножение комплексных чисел" (36:31)
42 лек. "Комплексные числа" (38:14)
43, 44 лек. "Знакомство с Гауссовыми числами" ч. 1:2 (34:40)
43, 44 лек. "Знакомство с Гауссовыми числами" ч. 2:2 (41:16)
45, 46 лек. "Основная теорема арифметики для Гауссовых чисел" ч. 1:2 (38:31)
45, 46 лек. "Основная теорема арифметики для Гауссовых чисел" ч. 2:2 (39:00)
47, 48 лек. "Основная теорема арифметики для Гауссовых чисел 2" ч. 1:2 (49:01)
47, 48 лек. "Основная теорема арифметики для Гауссовых чисел 2" ч. 2:2 (42:27)
49, 50 лек. "Пифагоровы тройки, общая формула. ч. 1:2 (34:09)
49, 50 лек. "Пифагоровы тройки, общая формула. ч. 2:2 (40:49)
51, 52 лек. "Конечная арифметика. Теорема Безу" ч. 1:2 (38:32)
51, 52 лек. "Конечная арифметика. Теорема Безу" ч. 2:2 (35:34)
53, 54 лек. "Теорема о корнях многочленов" ч. 1:2 (36:42)
53, 54 лек. "Теорема о корнях многочленов" ч. 2:2 (40:59)
55, 56 лек. "Теоремы Виета, Вильсона и Ферма (малая)" (83:36)
57, 58 лек. "Малая теорема Ферма осн. следствие. Теорема Вильсона" (88:23)
59, 60 лек. "Геометрия, арифметика и алгебра преобразований" ч. 1:2 (39:34)
59, 60 лек. "Геометрия, арифметика и алгебра преобразований" ч. 2 2 (2017.04.14) (38:36)
61, 62 лек. "Классификация подобий прямой. Подобия плоскости" (72:44)
63, 64 лек. "Высшая геометрия" (2017.09.26) (76:47)
65, 66 лек. "Векторы" (2017.09.26) (52:41)
67, 68 лек. "Линейные отображения прямой и плоскости" (2017.10.16) (50:41)
69, 70 лек. "Линейные отображения плоскости" (2017.11.21) (89:23)
71, 72 лек. "Координатная запись линейных отображений плоскости" (71:05)
73, 74 лек. "Арифметика матриц" (2017_12_05)

Интересное оформление первых 12 лекций
http://childrenscience.ru/courses/sav/14/
