О ПРОЕКТЕ "100 УРОКОВ МАТЕМАТИКИ

Математическая программа как золотая середина между
обычной школой и лучшими матшколами: проект и первые
наблюдения от воплощения 100 уроков математики

ПРОЛОГ

Когда я был маленьким, нас всех учили математике хорошо и бесплатно. 

Теперь хорошо и бесплатно учат только самых сильных школьников в нескольких 
знаменитых матшколах (а также на сменах целого ряда сезонных школ по стране). 

Чтобы частично восполнить этот пробел, я решил начитать на видео и выложить 
в интернет все основные темы школьной математики, так сказать, с точки зрения 
математики высшей. Запись лекций происходит на площадке Филипповской Школы 
в Москве. Впрочем, то и дело я читаю лекции и в других местах. Все уже записанные
видео проекта "100 уроков математики" находятся здесь:
https://www.youtube.com/playlist?list=PL8n_ZHoHDPESLDJN2NJivDYLNGtpJEBoy

(Так как записи в Филипповской Школе оставляют желать лучшего с точки зрения
качества (доску и всё, что на ней написано, очень плохо видно), то лицей Интеллектуал
взялся записать все 100 уроков на супертехнике. Однако деньги и техника там кончились.)

ОБОСНОВАНИЕ ПРОГРАММЫ 100 УРОКОВ

Одной из тенденций нашего времени служит растущий разрыв между фактическими
знаниями и умениями лучших учеников элитных матклассов, с одной стороны, и тем 
минимальным багажом понимания математики, который остаётся через месяц после 
сдачи всяческих ОГЭ и ЕГЭ у средних выпускников обычных школ, с другой.

Было бы неправильно кидаться огульными обвинениями в адрес нашего Министерства 
образования, потому что тенденция эта является общим отражением тенденции на 
сильнейшую дифференциацию всех сфер знания и умения в современном мире. 

Возможно, чиновники в чём-то и виноваты, но у их обвинителей (точно так же, как и 
у самих чиновников) нет никакой внятной "программы действий". То, что происходит, 
имеет множество причин, и вопрос их анализа мы оставим пока в стороне.

Важно только одно: признать, что всё это действительно происходит - и, не прячась
за замечательные и неизменно растущие индексы и показатели, честно сказать самим
себе: среднее образование - как массовый проект - провалилось в современной России.

Что же тогда делать?

Я предлагаю свой ответ на вызовы современности, нисколько не претендуя на его
универсальность (и даже не гарантируя никоим образом его успешной реализации).

Мой ответ состоит в формулировке и начитывании в видеолекциях КУЛЬТУРНОГО
МИНИМУМА МАТЕМАТИЧЕСКИХ ЗНАНИЙ для образованного человека. Минимума
математических знаний, вооружившись которым, человек сможет сказать, что он в 
общем и целом понимает "довузовскую" математику. 

(Также у меня есть минимум вузовской высшей математики, в форме уже прочитанных 
видеокурсов математического анализа и линейной алгебры в МФТИ на Национальной 
Платформе Открытого Образования.)

Школьный "культурный минимум математики" читался мною в течение последних 
4-х лет в Филипповской Школе в Москве для детей, сперва учившихся в четвёртом,
а затем постепенно переходивших в 5-й, 6-й и 7-й, а ныне обучающихся в восьмом
классе. На данный момент прочитано около 85 уроков из планируемых 100. Группа
сперва состояла из 15-20 человек, но вскоре подразбежалась, "и их осталось трое".

Следует особо отметить, что Филипповская Школа не является математической.
Это - школа, где учат доревние языки, историю, литературу на очень глубоком
уровне. Скорее надо было бы сказать, что Филипповская Школа - гуманитарная
школа с небольшим уклоном в православие, организующая много разных кружков
во внеурочное время и не задающая никаких рамок своим ученикам.

Иными словами, если у меня в группе из 20 человек отсеялись все, кроме троих, то
в целом по стране я вижу в качестве своей целевой группы 10 процентов наиболее
мотивированных к изучению математики учеников. Это гораздо больше, нежели 
все ученики всех лучших региональных математических школ, вместе взятые.

Всем, кто не согласится брать математику в объёме моих 100 уроков, я предлагаю
ограничиться простейшими правилами устного счёта, деления и умножения в столбик,
сложения дробей и заучивания наизусть обычной таблицы умножения. Что уже немало.

Перейдём к конкретике моей программы. Она стоит "на трёх китах":

I. Группы преобразований (в первую очередь, движений и подобий) простейших объектов
(прямой, плоскости) и группа перестановок на $n$ символах, а также общее понятие группы;

II. Арифметика остатков, вплоть до многочисленных следствий из Основной Теоремы
Арифметики (об однозначном разложении на простые множители в кольце целых чисел);

III. Комплексные числа и некоторые их подмножества, образующие кольцо, использование
свойств делимости в этих кольцах для решения классических диофантовых уравнений.

Так уж вышло, что при анализе всех свойств этих "китов" возникает необходимость
работать со всеми математическими понятиями и концепциями, изучаемыми в школе
(а также с гораздо более тонкими и продвинутыми построениями и правилами вывода).

Поэтому "со сдачей ЕГЭ и ОГЭ" у тех, кто проходит мою программу, не должно возникать
каких бы то ни было проблем - просто отвлекутся на пару дней, и сдадут всё, что нужно.

Также опыт моих троих слушателей показывает, что, не нацеленная на олимпиадные
успехи, моя программа значительно улучшает результаты выступлений на олимпиадах.

В качестве приложения см. ниже поэтапная программа уже отчитанных, а также
перспективная программа оставшихся пятнадцати уроков. Критика принимается!

ПРЕРЕКВИЗИТЫ ПРОГРАММЫ 100 УРОКОВ

Пояснение. Вовсе не нужно - и даже категорически не следует! -
всё это зубрить и штудировать до начала прохождения курса
математики. Просто это "неидейная часть" курса, поэтому к ней
нужно обращаться лишь по мере необходимости. "Неидейная" в 
том смысле, что это всё нам дано в ощущениях от рождения.

Из того, что я успел увидеть, могу сказать следующее: программа 
Якова Иосифовича Абрамсона - это вступление для моей программы.

1. Натуральные числа, счёт до 100 (а лучше до 1000). Системы счисления,
операции в них (сложение, умножение). Обратные операции (вычитание и
деление). Необходимость построения всех целых чисел (отрицательные
температуры воздуха!) и всех дробных (рациональных) чисел.

2. Геометрические фигуры: треугольник, квадрат, окружность, круг,
внутренность и граница на визуальном уровне. Визуальные длина и
площадь, расстояние между точками, углы. Площадь прямоугольника.
Прямая линия как кратчайшее расстояние между точками. Уравнение
прямой и окружности (но это будет обсуждаться по мере необходимости).

3. Понятие о порядке на прямой. Больше - меньше для дробей. Операции
с дробями (арифметические). Операция возведения в степень и обращение 
(операции извлечения корня и взятия логарифма). Правила работы с ними.

4. Принципы доказательств: от противного, математическая индукция,
логические операции и кванторы логики, метод бесконечного спуска
(по ходу дела эти и другие методы будут всплывать и усваиваться).

5. Множества и подмножества, всевозможные операции над подмножествами. 
Отображение множеств и операция композиции, ассоциативность композиции 
любых отображений. Образы, прообразы. Количество отображений из одного 
конечного множества в другое. Факторизация и отношение эквивалентности
(разумеется, последнее всплывёт само - я не предполагаю знакомства с этим!).

ПРОГРАММА УЖЕ СВЕРШИВШИХСЯ 84 УРОКОВ

1. Числа, символы и фигуры
2. Соизмеримость и несоизмеримость отрезков
3. Визуальное представление бинома Ньютона
4. Бесконечные суммы
5. Начальные представления о движении
6. Классификация движений прямой
7. Таблица умножения движений прямой
8. Движения окружности
9. Таблица умножения движений окружности
10. Конечные подгруппы движений прямой и окружности
11. Введение в арифметику остатков
12. Арифметика остатков
13-14. Основная теорема арифметики
15. Основная теорема арифметики: следствия
16-18. Линейные уравнения
19-20. Цепные дроби
21-22. Перестановки
23-24. Перестановки: циклы, чётность, порядок 
25-26. Задачи на перестановки
27-28. Группа Клейна
29-30. НАДО УТОЧНИТЬ !!!!!
31-32. Движение плоскости
33-34. Скользящая симметрия
35-36. Комплексные числа
37-38. Геометрия комплексных чисел
39-40. Комплексные числа и их арифметика (повторение)
41-42. Умножение комплексных чисел
43-44. Знакомство с Гауссовыми числами
45-48. Основная теорема арифметики для Гауссовых чисел
49-50. Пифагоровы тройки, общая формула
51-52. Конечная арифметика. Теорема Безу
53-54. Теорема о корнях многочленов
55-56. Теоремы Виета, Вильсона и Ферма (малая)
57-58. Малая теорема Ферма: осн. следствие. Теорема Вильсона
59-60. Геометрия, арифметика и алгебра преобразований
61-62. Классификация подобий прямой. Подобия плоскости
63-64. Геометрия: обзор того, что было
65-66. Векторы и действие группы подобий на них
67-68. Линейные отображения прямой и плоскости
69-70. Линейные отображения плоскости, окончание
71-72. Координатная запись линейных отображений плоскости
73-74. Матрицы: арифметика
75-76. Группа квадратных невырожденных матриц
77-78. Линейная алгебра, итоги
79-80. Алгебраические числа, первое знакомство
81-82. Алгебраические числа: минимальный многочлен
83-84. Всё про корень кубический из двух

ПЕРСПЕКТИВНАЯ ПРОГРАММА ОСТАВШИХСЯ УРОКОВ:

85-88. Теория построений циркулем и линейкой
89-90. Волшебство Гаусса: 17-угольник циркулем и линейкой
91-95. Канторовская теория множеств, масса примеров
96-100. Аксиоматика полноты действительных чисел
=====================================================
100 уроков кадетам !!!!
 Члены Совета Минобрнауки России по Кадетскому образованию - представители суворовских, нахимовских и кадетских общественных организаций РФ, подготовившие Концепцию кадетского образования, уже несколько лет пытаются доказать необходимость создания единой системы кадетского воспитания и образования, а «вариативность» и прочую «толерантную» шелуху убрать из военного образования...
======================================================
ПРОГРАММА СТА ЛЕКЦИЙ ПО МАТЕМАТИКЕ В МИНИАТЮРЕ

1-10. Геометрическая алгебра (в т.ч. алгоритм Евклида)
11-20. Преобразования прямой и плоскости, теорема Шаля
21-30. Перестановки, циклы, чётность. Комбинаторика
31-40. Виды чисел. Соизмеримость. Алгоритм Евклида вновь
41-50. ОТА, арифметика остатков, остатки по простому модулю
   Деление с остатком, Евклид, уравнение прямой на плоскости
51-60. Задачи на построение. Какие решаются, какие - нет.
61-70. Алгебра (многочлены, их деление с остатком и т.д.)
71-80. Топология прямой, аксиомы полноты и их эквивалентность
81-90. Комплексные числа и преобразования плоскости, тригонометрия
91-100. Теория множеств: операции, мощности, мера (длина и площадь)
==============================================================
Элементарная комбинаторика.

1. Правила суммы, произведения и дополнения, число подмножеств и перестановок. 
Число размещений. Примеры использования (математические и нет).

2. Число сочетаний, бином Ньютона и биноминальные коэффициенты. 
Треугольник Паскаля. Мультиноминальные коэффициенты.

3. "Четыре комбинаторные модели". 
Задача Муавра (метод точек и перегородок).

4. Формула включений и исключений.
==============================================================
Несколько первых уроков были посвящены "геометрической алгебре" (рисункам, из
которых следуют через сравнение площадей разные законы арифметики и алгебры,
несоизмеримость некоторых отрезков, теорема Пифагора и расходимость рядов).

Затем мы приступили к изучению группы движений прямой (сперва я не называл её
группой, говорил о множестве всех движений прямой и их классификации, сиречь
одномерной теореме Шаля). После прямой мы перешли к движениям окружности,
заодно отметив, что группа движений окружности совпадает с группой движений
плоскости, оставляющих на месте одну точку. Этот факт обобщается на бОльшие
размерности, и лежит в основе следующей программы построения геометрии.

Программа Клейна. В рамках неё начинается всё с исследования группы ВСЕХ 
преобразований конечного множества символов (и вообще, сбоку здесь проходит
линия взаимно-однозначных соответствий и три основных теоремы Кантора).

После внимательного анализа группы перестановок можно приступать к более
наглядным геометриям, начиная с самой наглядной: теорема Шаля на плоскости.

Побочным (важнейшим!) продуктом этого анализа служит

ПРИЛОЖЕНИЕ 2. ПЕРСПЕКТИВНАЯ ПРОГРАММА ОСТАВШИХСЯ 32 УРОКОВ

Алексей Савватеев, 25.10.2017

1-10. Геометрическая алгебра (в т.ч. алгоритм Евклида)

Про иррациональность корня из двух: http://avva.livejournal.com/2913841.html
Замощения пятиугольниками, укладки и упаковки. Масса открытых проблем тут!!
Задачка Эдика Лернера (Кнута) и открытая математическая проблема (и похожие !!)

Уроки 16-18: запишем линейное уравнение. Это - прямая линия. Частный случай 
c=0 всегда имеет решение (0,0), однако и ненулевое тоже. Как описать все? Нужно
нам воспользоваться тем, что если ax \del b, то при НОД(a,b)=1 должно быть x \del b.

Тогда, понятно, сокращая, имеет y \del a, и значит, все решения такого вида.

А если НОД не равен 1? Тогда сократим на НОД. И получим снова общую формулу.
(Нужно, чтобы Катерина Богданович записала кучу упражнений на эту тему!!!!!).

Заодно - прыжки кузнечиками, и представление НОД в форме ax+by. Переход к 
общей форме, алгоритм Евклида как самый кратчайший путь к цели. Основная
теорема арифметики снова, соизмеримость отрезков заодно. 

Математика - это решение уравнений с ограничениями. Иными словами, это
поиск точек специального вида на кривых и поверхностях. Целые точки на
любых прямых - это несколько сложнее; рациональные точки на прямой и 
на окружности - отличная мотивация занятий математикой в школе!
================================
Яндекс-контрольная:
https://yandex.ru/edu/tasks/55016b7659aa4048b3608eec/run/
================================
http://elementy.ru/lib/430915
(Успенский - Апология математики)
================================
ГРУППЫ:
остатки по модулю (операция сложения)
остатки (взаимно-простые с модулем, операция умножения)
перестановки (и куча важных подгрупп)
гомотопические группы
группа эллиптической кривой над любым полем
группа решений уравнения Пелля (относительно той операции в алгебре с делителями нуля)
дробно-линейных преобразований прямой (проектирований)
линейных преобразований с композицией (невырожденных матриц по умножению)
мультипликативная группа поля
элементов поля с нормой единица
обратимых элементов любого кольца
группа симметрии функции/многочлена
точная последовательность 0 --> T --> Aff-O --> O --> GL --> 0
================================

Графики квадратных уравнений включить в мои сюжеты

ГЕОМЕТРИЧЕСКАЯ АЛГЕБРА:

0. Законы арифметических действий
1. Теорема Пифагора
2. Сходимость и расходимость рядов (гармонического
    и суммы обратных квадратов, геометрической прогрессии
    - придумать, как геометрически увидеть её сходимость !!!)
3. ОПРЕДЕЛЕНИЕ: сумма бесконечного числа слагаемых 
    сходится, если её конечные куски могут быть больше 
    любого наперёд заданного количества. ТЕОРЕМА: если
    сумма сходится, то слагаемых - счётное число (!!!)
4. Соизмеримость и алгоритм Евклида через прямоугольники
5. Формулы сокращённого умножения (разность квадратов
    и разные дела с кубами), бином Ньютона, суммы нечётных 
    чисел и подряд идущих чисел - вспомнить всё про это !!!
6. Цепные дроби (вытягивание носов), Пелль и Минковский
7. Простые числа как суммы двух квадратов (Спивак)

11-20. Преобразования прямой и плоскости, теорема Шаля

21-30. Перестановки, циклы, чётность. Комбинаторика

31-40. Виды чисел. Соизмеримость. Алгоритм Евклида вновь

41-50. ОТА, арифметика остатков, остатки по простому модулю

51-60. Задачи на построение. Какие решаются, какие - нет.

61-70. Алгебра (многочлены, их деление с остатком и т.д.)

71-80. Топология прямой, аксиомы полноты и их эквивалентность

81-90. Комплексные числа и преобразования плоскости, тригонометрия

91-100. Теория множеств: операции, мощности, мера (длина и площадь)

"О взаимосвязях между алгеброй и анализом".

(От анализа к алгебре: максимизация функций и свойства квадратичных 
форм. От алгебры к анализу: числа Каталана, иррациональность числа
$\pi$ и распределение простых чисел, гипотеза Римана.)

ПРОГРАММА ЗАПЛАНИРОВАННОГО ИНТЕНСИВА:

24 ноября: "долг" перед неевклидовой геометрией - теорема Шаля,
движения сферы, несколько слов о группах и алгебрах Ли;
26 ноября: "По мотивам десанта" непосредственно - сложение точек
на эллиптической кривой и теорема Морделла-Вейля;
28 ноября: "Начала топологии" - кватернионы и маломерные группы
вращений, комплексная проективная плоскость и расслоение Хопфа.

ПРОГРАММА ОТДЕЛЬНЫХ ЛЕКЦИЙ (В Т.Ч. УЖЕ ПРОЧИТАННЫХ)

На картинках видно много чего. Формула 1+2+3+....+n = n(n+1)/2
видна из картинки с двумя лесенками, приставленными друг к
другу с поворотом на 180 градусов. Формула a^2 - b^2 тоже из
картинки замечательно читается, как и бином Ньютона (a+b)^2.
Сумма нечётных чисел равна квадрату - это тоже картинка!

Затем мы берём кубический кусок сыра. Режем его так, чтобы
получилось 8 разных кусков. Бином (a+b)^3 готов. А что будет
при степени 4? Можно догадаться? Можно !!!!!

Дальше - уже треугольник Паскаля. Число путей в прямоугольном
городе (путей одинаковой длины), число подмножеств данной 
мощности, бином в чистом виде, рекурсия, вычисление C_n^k.


Сходящиеся и расходящиеся ряды. Гармонический ряд, прогрессия
(причины, почему она сходится), обратные квадраты и формула Эйлера.
Обратные квадраты через площади и через оценку сверху очевидным.


1. О геометрии (от "теоремы о трёх гвоздях" через представление любого
движения как композиции отражений к классификации Шаля и арифметике
композиций движений. Ахуенная тема !!! И для прямой, и для Лобачевского,
и для обычной плоскости, и для сферы. Кстати, я буду это на Алкошколе
читать - может быть, кто-нибудь из студентов может записать за мной?
Тогда сразу оперативненько и добьём для Кванта !!!!

2. Вообще у меня сейчас "100 лекций в Филипповской Школе", и они все
записываются! Если бы кто-то из молодёжи тупо пролистал мои видео,
и все их записал, это был бы пиздец подарок нашему русскому миру !!!
Авось, там и для Кванта дохуища материала выгорело бы. Как думаешь?

3. Ну, и конечно "Уравнения Эйлера" - где подробно о решении y^2 = x^3 \pm 1.
Плюс отдельно можно Теорему Ферма, для n=3. Тут вообще мой текст готов !!!
============================================
"100 уроков математики" - где-то ещё выступить, с огромной рекламой (Коля Андреев?)

ПРЕРЕКВИЗИТЫ:

Понятие о числе как точке на прямой, операции над числами. Сложение
дробей, десятичная (двоичная и т.п.) запись числа, понятие о переменной.

"Геометрия и арифметика движений прямой и плоскости"
http://youtu.be/XMhP7KpHBlA

СИЛЕТСКАЯ и КВАНТ !!!!!

Смонтировали и выложили в закрытый доступ  лекцию 7. В аттаче к ней - конспект. 
 
http://youtu.be/xPhZx8hlVXE 

XV. Неевклидова геометрия

ПЕРЕВОД МОИХ ЛЕКЦИЙ НА ДРУГИЕ ЯЗЫКИ !!!

1. Сколько дней в году? (все известные поправки к високосности).
Почему поправок, скорее всего, бесконечно много?

2. На случайно выпущенном из нуля луче (на плоскости),
скорее всего, не встретится больше ни одной целой точки.

3. Определение: два отрезка соизмеримы, если прямоугольник,
построенный на них, можно разбить регулярной сеткой на квадраты.

4. Теорема: число рационально в тоММ случае, когда оно соизмеримо с 1.

5. Корень из двух (диагональ квадрата): два доказательства иррациональности
(арифметическое, и с помощью подобия и понятия соизмеримости, см. выше).

6. Построение цепной дроби для корня из двух (в лоб). Приближения.

7. Насколько близок может быть один квадрат к удвоенному другому?
(Введение в уравнение Пелля). Уравнение x^a - y^b = 1 (ознакомительно).

8. Решение через цепные дроби (анонс).
==============================================================
МОСКОВСКИЕ СУВОРОВЦЫ РОО
mccvu@mccvu.ru

 Московские суворовцы и офицеры России создают Совет по военному образованию

14 мая 2018 года в Москве состоялось подписание соглашения о сотрудничестве и взаимодействии Общероссийской общественной организацией «Офицеры России» и общественной организации «Московские Суворовцы». Документ подписали председатель Президиума Общероссийской организации «Офицеры России», Герой Российской Федерации Сергей Липовой и председатель общественной организации «Московские Суворовцы», член совета Министерства образования и науки РФ по кадетскому образованию Дмитрий Нестеров.

«Целью и предметом соглашения являются совместные согласованные действия, взаимная поддержка и сотрудничество по вопросам военно-патриотического воспитания молодежи в соответствии с постановлением Правительства РФ от 30 декабря 2015 г. №1493 «О государственной программе «Патриотическое воспитание граждан Российской Федерации на 2016-2020 годы»», - рассказал председатель Президиума Общероссийской организации «Офицеры России», Герой Российской Федерации Сергей Липовой. 

По его словам, соглашение нужно для того, чтобы можно было вместе работать по многим проектам, которые в ходе встречи уже определились и по которым, как ожидается, будет достигнут успех. 

Один из приоритетных совместных проектов - формирование Совета по военному образованию. Его участники займутся вопросами военного образования, довузовской военной подготовкой и доработкой концепции Кадетского образования в России.

«Представители общественных организаций сосредоточатся на системе военного образования в стране и продолжат работу по довузовскому военному образованию (кадетскому образованию) совместно со всеми заинтересованными министерствами и ведомствами», - подчеркнул председатель общественной организации «Московские Суворовцы», член совета Министерства образования и науки РФ по кадетскому образованию Дмитрий Нестеров. 

Он также пояснил, что Кадетское образование - это целенаправленный процесс воспитания и обучения учащихся в общеобразовательных организациях кадетского типа с учётом исторических традиций русских кадетских корпусов.

«Все наработки, предложения и замечания будут задокументированы и переданы до конца 2018 года на рассмотрение в совет Министерства образования и науки РФ по кадетскому образованию», - добавил председатель общественной организации «Московские Суворовцы» Дмитрий Нестеров. 

На сегодняшний день в России действует около 200 образовательных организаций кадетского типа. Кадетские и казачьи кадетские образовательные организации находятся в ведении различных федеральных министерств, ведомств, субъектов Российской Федерации (Минобороны, Росгвардии, МВД, ФСБ, МЧС, Следственного комитета, Минкультуры, а также Минобрнауки России и органов образования субъектов РФ). В них одновременно учатся более 45 000 человек, и ежегодно выпускается около 8000 суворовцев, нахимовцев и кадет. 
===========================================================
ШЕНЬ: ТЕОРИЯ ВЕРОЯТНОСТЕЙ

Ну, подробности ты можешь (если вдруг интересно) посмотреть на видео или 
расспросить школьников. Моё впечатление - что было вполне осмысленно, но 
я успел разобрать только самые простые вещи, подсчёт вариантов (даже для 
пар чисел от 1 до 6) был довольно трудным, а когда мы начали обсуждать 
четыре кубика, бросаемые одновременно, у народа было совсем много 
неожиданных идей, это уже было в самом конце и мы только пару слов про 
это успели сказать. Так что если ты будешь продолжать, то можно начинать 
с обсуждения вероятности того, что при бросании четырёх кубиков будет 
сумма 24, а затем 23, а затем уже и другие задачи. Про дополнительные 
события и про сумму вероятностей вскользь упоминалось, но не более, так 
что это надо подробно разбирать с самого начала. Условные вероятности и 
независимость вообще не затрагивались.


Финансист по кличке "Серый Гей Вроде" (=Сергей Мавроди) основал финансовую 
пирамиду. Каждый, кто принесёт ему 1000 рублей, через месяц заберёт 1500 рублей.

Считая, что Серый Гей не использует банковской системы (держит все деньги в жестяной
банке), оценить, когда рухнет эта пирамида в предположении, что ежедневно приходит на 
5 человек больше, чем вчера. В первый день, 1 января 2016 года, не пришёл никто.

http://ru-math.livejournal.com/797774.html
\sqrt{2+\sqrt{\sqrt{2+...}}}

Вспомнил таки, что после эфира выкладывал несколько невошедших
занятных фактов (что забылось - экспромт ведь, что не успелось, а
где решил не перегружать):
- Греческую буковку Пи предложил Эйлер в 1737 году, а вот иррациональность
таки Лежандр доказал чуть позднее (главный замеченный косяк на эфире)
- До сих пор не доказана иррациональность Пи+е, Пи*е и многих других выражений!
- Никто даже не проверил, не целое ли число Пи^Пи^Пи^Пи. Вряд ли, конечно, но вдруг. Слабо?! :)))
- Неизвестно, с одинаковой ли частотой в разложении числа Пи встречаются все цифры.
Неизвестно даже, все ли они встречаются бесконечное число раз.
- В 1975 году Брент и Саламин предложили формулы, на каждой итерации удваивающие
число верных знаков числа Пи. 40 итераций - и у вас на экране несколько триллионов
верных циферок!
- В 1997 году Саймон Плафф предложил формулу, позволяющую (вообще за гранью
моего понимания!!!) получать любую цифру числа Пи без вычисления предыдущих!
- в Штате Индиана Пи=3,2 (а в военное время и до 4!). Правда, закон не утвердили
(в отличие от гренландских китов с Пи=3)

НЕСКОЛЬКО ОПРЕДЕЛЕНИЙ:
- Пистолет – юбилей известной константы (нет, это примерно 314 лет!)
- Пижон – человек, с числом жен чуть большим трех.
- Питон – тритон-уродец.
- Пирог – вид единорога с чуть большим числом рогов.
- Пиастры – осенние цветы с числом лепестков от 3 до 4.

ВАЖНЫЕ РЕСУРСЫ:

Elena Kiseleva <hkiseleva@gmail.com> http://theoryandpractice.ru
===============================================
КОНЦЕПТУАЛЬНО. Математика в школе должна ставить целью освоение
стандартного языка - множества, мощности, группы, кольца, идеалы, поля, 
многочлены, делимость, предел, производная, интеграл, ряд, комплексные
числа, экспонента, алгебраические и трансцендентные числа и т.д. 

ТАКЖЕ знакомить с культовыми для всей математики результатами:
теорема Безу, формула Ньютона-Лейбница, формальное построение 
вещественных чисел и элементарных функций, работа с перестановками, 
теорема Шаля, все начальные факты про конечные поля, экспонента и
всякие штуки вокруг неё, начальная алгебраическая геометрия степеней
2 и 3 (включая эллиптические кривые), визуальная линейная алгебра.

ПРОЩЕ ГОВОРЯ, школа должна приводить к освоению первых глав
учебников Зорича, Зарисского и Самюэля, Кострикина, Колмогорова
и Фомина, Рудина, Шилова, Кострикина-Манина и Аэрленда-Роузена.

Не все пройдут дорогу целиком; но те, которые более медленные, хотя
бы получат представление о математике, а не будут полагать, что наша
наука - это стократное вычисление корней квадратного уравнения, или,
наоборот, всякие хитроногие смекалистые перекладывания спичек.

Математика - это ствол дерева; у этого дерева огромное количество
ветвей. Кто не ориентируется, где ствол, а где ветка (пускай и очень
красивая), тот будет пичкать школьников всякой разноцветной ерундой.
============================================================
КРАТКИЙ (ПЕССИМИСТИЧЕСКИЙ) СЦЕНАРИЙ:

21-30. Перестановки, циклы, чётность. Комбинаторика

31-40. Алгебра (многочлены, их деление с остатком и т.д.). Конечные поля.

41-50. Алгебраичские числа. Цепные дроби. Приближения. Числа e и \pi.

51-60. Задачи на построение. Какие решаются, какие - нет. Гаусс-17.

61-70. Теория множеств: операции, мощности, мера (длина и площадь).

71-80. Топология прямой, аксиомы полноты и их эквивалентность.

81-90. Комплексные числа и преобразования плоскости, тригонометрия

91-100. Знаменитые диофантовы уравнения. Эллиптические кривые.
=========================================================
ИДЕАЛЬНЫЙ (ОПТИМИСТИЧЕСКИЙ) СЦЕНАРИЙ:

I. Геометрическая алгебра. Теорема Пифагора. Апелляция к тому, что
сложение - это приставление отрезков друг к другу, а умножение - это 
площадь прямоугольника. Кроме того, здесь же открывается, что числа 
являются преобразованиями: сложение/вычитание - это когда прямая 
сдвигается вправо/влево, а умножение/деление - это если мы всю 
прямую растягиваем/сжимаем относительно фиксированной точки. 
Открытие того, что умножение на (-1) является переворачиванием.

II. Арифметика. Число - это ещё и повторение. 1 = "один раз", 2 = "два
раза" и так далее. Полшага, треть шага. Шаг "в другую сторону". Также
надо призвать на помощь "северную интуицию": что такое температура?
Она бывает и положительной, и отрицательной. Шкала, точка отсчёта.
Натуральные числа, целые числа и дробные (рациональные) числа, их
расположение на прямой. Заполняют ли они всю прямую?

III. Визуальная теория множеств. Подмножества прямой, плоскости,
немножко - пространства. Пересечение, объединение, даже прямое 
произведение (призма, квадрат, куб). Топрлогия: шары, открытость,
замкнутость, теоремы о предельных точках, непрерывность функций

IV. Комбинаторика и бином Ньютона. Биномиальные коэффициенты.
Треугольник Паскаля. Маршруты прямоугольного города. Конечные
подмножества конечного множества. Перестановки и отображения.
Образы и прообразы, визуализация (отображение перспективы).

V. Теорема Шаля на прямой и на окружности.

VI. Конечная арифметика.

VII. Перестановки "под микроскопом".

VIII. Теорема Шаля на плоскости. Запись преобразований в координатах
на плоскости. Матрица линейного отображения. Комплексные числа как 
преобразования поворотной гомотетии. Арифметический подход, связь
с геометрией через модуль и аргумент. Комплексные числа как матрицы.

IX. Функциональные зависимости: линейные, квадратичные и многочлены.

X. Арифметика "на новом уровне". Малая теорема Ферма, теорема Вильсона.

XI. Корни из единицы. Полезные соотношения. Построения циркулем и линейкой

XII. Гиперболическая математика и уравнение Пелля.

XIII. Плоды усилий: теорема Ферма при n=3, эллиптические кривые.

XIV. Математический анализ многочленов

XV. Неевклидова геометрия
===============================================
ЛИТЕРАТУРА:

Дьедонне. Линейная алгебра и элементарная геометрия.

Артин. Геометрическая алгебра.

Савватеев. Математика для гуманитариев.

Вейль. Симметрия.

Курант, Роббинс. Что такое математика.

Верещагин, Шень. Теория множеств.

Прасолов. Неевклидова геометрия

Иен Стюарт. Величайшие математические задачи