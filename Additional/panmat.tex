ВИЗУАЛЬНАЯ МАТЕМАТИКА

1. Числа, символы, фигуры: обзор. Площади, длины, числа со знаком, дроби. 
Геометрическое представление целых и дробных чисел. Числовая прямая.
Идея аффинной прямой - на которой есть точки, но нет начала координат.
Сложение точек с векторами (числами со знаком), векторов с векторами,
вычитание из точек векторов и точек из точек. Сдвиги и их композиция.

2. Умножение как площадь прямоугольника. Теорема Пифагора в картинках.
Формулы сокращённого умножения. Визуальное представление бинома
Ньютона, при n=2 и n=3. Попытка заглянуть в более высокие размерности.
(На самом уроке - Пифагор и вычисление площади с дробными сторонами;
в конце попытался разрезать куб, чтобы увидеть бином при n=3!)
===================================================
ГДЕ Я ОСТАНОВИЛСЯ??????

О курсе

Совершенно независимый курс введения в современную математику. Записывается Алексеем Савватеевым на планшете, он планирует изложить «всю математику, которую знает».

Алексей Савватеев — популяризатор математики среди детей и взрослых, доктор физико-математических наук, профессор МФТИ.
Для кого этот курс

Для широкой аудитории
Наши преподаватели
Аватар пользователя
Егор Кузьмичёв
Программа курса

    Начальная математическая подготовка
    Целые числа
    Что такое группа
    Целые числа как группа
    Коммутативность умножения. Подгруппы целых чисел
    Подгруппы целых чисел
    Деление с остатком
    На подступах к ОТА
    Формулировка ОТА
    ОТА — Лемма о кузнечике
    Доказательство леммы о кузнечике
    Определение группы
    Доказательство ОТА
    Вокруг ОТА
    План (анонс дальнейших видео)
    На подступах к остаткам
    Сумма Минковского
    Остатки
    НОД и сумма подгрупп
    Геометрическая алгебра
    Подготовка к пифагоровым тройкам
    Теория делимости
    Пифагоровы тройки
    Доказательство пифагоровых троек
    Геометрия целых чисел
    Отражения и дроби
    Дроби
    Все дроби разные
    Геометрия дробей
    Алгоритм Евклида
    Цепные дроби. Знакомство
    Кольца остатков
    Кольца и поля остатков
    Построение конечных полей
    План
    Кольцо многочленов
    Многочлены и системы счисления
    Многочлены над кольцами вычетов
    Многочлены как кольцо
    Простота многочленов
    ОТА для многочленов
    ОТА для многочленов с коэффициентами в поле
    Теорема Безу — формулировка
    Многочлены над полями
    Малая теорема Ферма
    Теорема Вильсона
    Движение. Лемма о двух гвоздях и теорема Шаля
    Арифметика движений

    Комплексные числа, мотивация и введение
    Комплексные числа как поле: начало. Глоссарий.
    Деление комплексных чисел. Основные соотношения.

ПЕРЕСТАНОВКИ

Сессия 28 марта 2019

+57. Группы перестановок.

Введение: группы симметрий правильных треугольника и тетраэдра. 
Перестановки вершин, абстрактная группа перестановок на n символах, 
число элементов в ней. Способ построения таблиц композиции таких групп.

ЛИТЕРАТУРА Кострикин. Введение в алгебру. Николай Вавилов Конкретная теория групп.

+58. Чётность перестановки.

Обоснование: чётность числа беспорядков меняется при домножении на произвольную 
транспозицию, справа либо слева. Чётность является гомоморфизмом групп!

ЛИТЕРАТУРА Кострикин. Введение в алгебру. Николай Вавилов Конкретная теория групп.

+59. Чётность: доказательство гомоморфизма.

Мы докажем сформулированную теорему о гомоморфизме.

ЛИТЕРАТУРА Кострикин. Введение в алгебру. Николай Вавилов Конкретная теория групп.

+60. Циклы.

Разложение любой перестановки в произведение не пересекающихся циклов.

ЛИТЕРАТУРА Кострикин. Введение в алгебру. Николай Вавилов Конкретная теория групп.

+61. Чётность циклов.

Арифметика циклов и композиций. Чётность любого цикла

ЛИТЕРАТУРА Кострикин. Введение в алгебру. Николай Вавилов Конкретная теория групп.

+ 62. Сопряжение.

Арифметика сопряжения.

ЛИТЕРАТУРА Кострикин. Введение в алгебру. Николай Вавилов Конкретная теория групп.
=======================================
ДАЛЬШЕ ПОКА НЕ БЫЛО:
63. Нормальные подгруппы.

Нормальные подгруппы: определение. Примеры нормальных подгрупп в группах 
перестановок: все чётные перестановки, а также группа Клейна.

ДВИЖЕНИЯ ПЛОСКОСТИ
КОМПЛЕКСНЫЕ ЧИСЛА
КОЛЬЦА ГАУССОВЫХ И ЭЙЗЕНШТЕЙНОВЫХ ЧИСЕЛ, И ДИОФАНТОВЫ УРАВНЕНИЯ

ЭРЛАНГЕНСКАЯ ПРОГРАММА КЛЕЙНА

ЛИНЕЙНЫЙ МИР, МАТРИЦЫ

ГЕОМЕТРИЯ ВСЕРЬЁЗ

АЛГЕБРАИЧЕСКИЕ ЧИСЛА

АКСИОМА ПОЛНОТЫ И ТОПОЛОГИЯ ПРЯМОЙ

ЭКСПОНЕНТА

К.Айерлэнд, М.Роузен. Классическое введение в современную теорию чисел. Верещагин, Шень. Начала теории множеств.
============================================================
============================================================
============================================================
Добрый день!
Сделал 1/5:
20 из 100 уроков
12 из 60 панкматематики

Жду корректирующих сигналов

Гуглдоки буду постепенно дополнять:
100 уроков
https://docs.google.com/document/d/1NWNTrIPx6fohcZ281W5uzQPdql4dUS4M5kVD2jbV1KE/

Панматематика
https://docs.google.com/document/d/17WmBpRObfjrD5hJ6VZ00OIWdzPB_5u24Lrav0aazAvk/
