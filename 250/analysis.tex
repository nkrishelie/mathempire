\newchapter[и алгебра]{Комплексная арифметика}

\vrezka{
В этой главы мы начинаем строить поле комплексных чисел, пока еще без участия вещественных. По сути мы здесь работаем только с комплексными рациональностями, что, однако, не мешает показать тесную геометрическую связь комплексных чисел и движений плоскости, а также изучить числа Гаусса.
}


\section{Алгебра комплексных чисел}

\lesson{Комплексные числа, мотивация: $x^2+1=0$, алгебраизация плоскости. Сложение, умножение, сопряжение, модуль, обратное число}

\begin{enumerate}
\item Когда мы строили поле $\Q[\sqrt 2]$, мы ввели в обращение новое число, которое позволяло решать уравнение $x^2=2$. Это число не является рациональным, но лежит где-то между рациональными числами. Тогда же мы задались вопросом, как быть с поиском корней других уравнений с целыми коэффициентами, неразрешимых в $\Q$.
\item Рассмотрим еще один пример уравнения: $x^2=-1$.
\item Ворде бы, все коэффициенты --- целые числа, и степень всего лишь вторая. Однако же, при детальном рассмотрении становится ясно, что у него нет решений не только в рациональных числах, но и где-то между ними, поскольку никакое известное нам число, возведенное в квадрат, и близко не подходит к -1.
\item Стало быть, если мы хотим ввести в обращение корень такого уравнения, то его наобходимо поместить где-то вне числовой оси, <<подвесить в воздухе>>.
\item Сделаем это из чисто эстетико-геометрических соображений. Как геометрически проявляют себя числа на прямой? Они обеспечивают сдвиг вдоль прямой: положительные --- вправо, отрицательные --- влево. Причем у всех этих сдвигов есть единица измерения --- число 1, которая заодно выступает и в роли мультипликативной единицы, когда мы определяем умножение чисел. Кроме того, сдвиг на 1 вправо и затем влево (или в обратном порядке) приводит нас обратно, т.е. является сдвигом на 0, или $\id$.
\item Новое же число мы хотим поместить так, чтобы оно обеспечивало сдвиг на плоскости, аналогичный сдвигу вдоль прямой.
\item Поскольку мы привыкли считать направление <<вверх>> положительным, поместим это число над числовой осью.
\item Заложим в этом числе сразу и единицу измерения: пусть оно отстоит от нуля на расстояние 1, тем самым мы согласуем масштаб сдвигов на плоскости со сдвигами на прямой. Наконец, сдвиг в направлении и на величину этой новой единицы не должен содержать в себе горизонтальных сдвигов, их проще добавить потом, взяв от сдвигов прямой, которые нам уже известны. Иначе говоря, числовая прямая при сдвиге на эту новую единицу должна сдвиуться вверх на расстояние 1 и таким образом, чтобы ее числовая разметка никуда не сдвинулась вправо или влево.
\item Так мы приходим к тому, что новую единицу сдвига следует отложить от нуля строго вверх на расстояние 1.
\item На координатной сетке она окажется в точке $(0,1)$.
\item Назовем это новое число-вектор буквой $i$,  которую принято называть \textbf{мнимой единицей} (от фр. \textit{imaginaire}).\index{Мнимая единица}
\item Теперь всякий сдвиг плоскости мы можем записать как композицию сдвига, выраженного в единицах (горизонтальный сдвиг), и сдвига, выраженного в мнимых единицах (вертикальный сдвиг). Просто по свойствам суммы векторов.
\item Иначе говоря, сдвиг на произвольный вектор $\vec z$ мы распишем как сдвиг на сумму векторов $x\vec 1+y\vec i$. См. рис.
\begin{center}
\includegraphics[scale=0.5]{complex.png}
\end{center}
\item Как и прежде, мы умеем отличать на плоскости векторы и точки. Векторы --- это направленные отрезки, которые можно откладывать от точек. Сложение векторов означает их последоватеьлное откладывание. В результате таких откладываний мы уходим от некоторой стартовой точки и приходим в какую-то финишную точку. Результирующий вектор соединяет стартовую и финишную точки. Договоримся для удобства считать стартовой точкой начало координат $O$, а финишную точку обозначать почти так же, как вектор, который в нее входит, только без векторной символики.
\item Итак, если вектор равен $x\vec 1+y\vec i$, то его финишная точка обозначается $x+yi$.
\item Пока все, что мы сделали --- это построили обычную арифметику векторов на плоскости. При чем же тут алгебраическая ипостась мнимой единицы, вытекающая из уравнения $x^2=-1$?
\item Алгебраическая ипостась $i$ нам нужна как раз для того, чтобы построить алгебру точек плоскости, т.е. научиться их не только складывать и умножать на число, но еще и умножать и делить друг на друга.
\item Примем за аксиому, что с числами вида $x+iy$ мы будем обращаться как с обычными числами, пользуясь аксиомами поля, и при этом пользоваться тем самым свойством мнимой единицы, которое ее определяет, т.е. равенством $i^2=1$.
\item Например,
$$
(a+bi)(x+yi) = ax + ayi + bxi + byi^2 = (ax-by) + (ay+bx)i.
$$
\item Числа вида $z=x+iy$ с заданными операциями сложения и умножения (сложение --- покоординатное, а умножение определено выше) называются \textbf{компл\'eксными числами}.\index{Числа!компл\'eксные} При этом $x$ называется \textbf{действительной} (или вещественной) частью комплексного числа $z$ и имеет также обозначение $\Re z$, а $y$ называется \textbf{мнимой} частью числа $z$ и имеет также обозначение $\Im z$.
\item Координатная ось $Ox$ на комплексной плоскости называется действительной осью, а координатная ось $Oy$ --- мнимой.
\item Дадим следующие определения. Число $\bar z=x-yi$ называется \textbf{комплексно сопряженным} к числу $z=x+iy$. Комплексное сопряжение --- это отражение относитеьлно действительной оси.
\item Модулем комплексного числа $z=x+yi$ называется число $$|z|=\sqrt{x^2+y^2}.$$
Нетрудно видеть, что модуль комплексного числа --- это длина соответствующего ему вектора (по теореме Пифагора). Кроме того, из геометрических соображений понятно, что $|z_1-z_2|$ --- это расстояние между точками $z_1$ и $z_2$ на плоскости.
\begin{center}
\includegraphics[scale=0.5]{rho.png}
\end{center}
\item Посмотрим, какие арифметические свойства комплексных чисел можно извлечь.
\begin{enumerate}[\bf C1)]
\item $z\bar z=|z|^2$. Действительно, $(x+yi)(x-yi)=x^2+y^2$.
\item $z=0$ (т.е. $z=0+0i$) тогда и только тогда, когда $|z|=0$.
\item Обратное по умножению число для $z\ne 0$ существует и равно
$$
z^{-1} = \frac{1}{x+yi}=\frac{x-yi}{(x+yi)(x-yi)}=\frac{\bar z}{|z|^2}
$$
Это можно получить и напрямую из свойства C1.
\item Мультипликативное свойство сопряжения: $\bar{zw}=\bar{z}\,\bar{w}$. Действительно,
$$ 
\bar{(x+yi)(a+bi)} = \bar{(ax-by)+(ay+bx)i} = (ax-by)-(ay+bx)i
$$
и
$$
\bar{(x+yi)}\,\bar{(a+bi)} = (x-yi)(a-bi) = (ax-by)-(ay+bx)i.
$$
\item Мультипликаивное свойство модуля: $|zw|=|z||w|$. Действительно,
$$
|zw|^2 = zw\bar{zw} = zw\bar{z}\,\bar{w} = z\bar{z}w\bar{w} = |z||w|.
$$
\end{enumerate}


\lesson{Сложение как параллельный перенос. Умножение на единичное число как поворот. Аргумент комплексного числа. Сопряжение как симметрия относительно вещественной оси}


\item Сложение с числом $z=x+iy$ --- это параллельный перенос $T_{\vec z}$ на вектор $\vec z=x\vec 1+y\vec i$. Это следует из геометрических свойств комплексных чисел, о которых мы говорили выше.

Кроме того, это легко проверить арифметически. Пусть даны две точки $z_1$ и $z_2$. Добавим к ним вектор $z$, получим новые точки $z_1'=z_1+z$ и $z_2'=z_2+z$. Во-первых, заметим, что расстояние сохранилось:
$$
|z_1'-z_2'| = |(z_1+z)-(z_2+z)| = |z_1-z_2|,
$$
т.е прибавление $z$ --- это движение. Во-вторых, если $z\ne 0$, то у этого движения нет неподвижных точек, иначе мы бы получили равенство $z_1+z=z_1$, откуда $z=0$. Следовательно, в силу теоремы Шаля прибавление $z$ есть параллельный перенос. Прибавление $z=0$ есть $\id$.

\item Умножение на комплексное число, по модулю равное 1, есть поворот с центром в нуле.

Пусть $|z|=1$. Возьмем точки $w_1=a_1+b_1i$ и $w_2=a_2+b_2i$,  умножим их на $z$, получим точки $w_1'=w_1z$ и $w_2'=w_2z$.

Найдем расстояние между $w_1'$ и $w_1'$:
$$
|w_1'-w_2'|=|(w_1-w_2)z|=|w_1-w_2|\cdot|z|=|w_1-w_2|,
$$
т.е. умножение на $z$ сохраняет расстояние. В то же время, очевидно, что при $z\ne 1$ единственной неподвижной точкой при умножении будет $w=0$, иначе мы бы получили $wz=w$, т.е. $z=w/w=1$. Умножение на $z=1$ есть $\id$.

Итак, умножение на число $z$, по модулю равное 1, является поворотом с центром в нуле. \textit{Каков при этом угол поворота?} 

Чтобы ответить на данный вопрос, рассморим для начала случай $|w|=1$, т.е. точку с единичой окружности будем умножать на другую точку с единичной окружности. По свойствам модуля имеем $|zw|=|z||w|=1$, т.е. в результате умножения мы вновь получим точку на единичной окружности! Иначе говоря, единичная окружность с операцией умножения комплексных чисел образует группу.

Теперь, заметим, что на окружности радиуса 1 хорда однозначно определяет опирающийся на нее угол. Рассмотрим углы, которые опираются на хорду $[z;1]$ и на хорду $[zw;w]$. На рисунке они выделены красным цветом.
\begin{center}
\includegraphics[scale=0.4]{complex-ring.png}
\end{center}

Легко видеть, что длины хорд равны: $|zw-w|=|z-1||w|=|z-1|$, так что и углы равны. Следовательно, точка $zw$ получается из точки $w$ поворотом на угол, соответствующий угу наклона вектора $z$ относительно положительного направления действитеьлной оси.

Что происходит в случае, когда $w$ не лежит на единичной окружности и отлична от нуля? Для этого представим произведение $zw$ следующим образом:
$$
zw = z\frac{w}{|w|}|w|,
$$
где отношение $w/|w|$ уже является комплексным числом единичной длины. Следовательно, число $zw/|w|$ получается из числа $w/|w|$ его поворотом на угол, заданный числом $z$. Осталось выяснить, как связаны $w$ с $w/|w|$ и $zw$ с $zw/|w|$.

В общем случае это означает, что мы имеем два комплексных числа, одно $v$, второе $\la v$, где действительное число $\la>0$. Пусть $v=a+bi$. Вспомним уравнение прямой, проходящей через начало координат и точку $(a,b)$. Это уравнение имеет вид $ay-bx=0$. А теперь умножим в этом уравнении обе части на $\la$, и получим $(\la a)y-(\la b)x=0$. То есть точка $\la v$ лежит на той же прямой, что и $v$.

Остался вопрос --- с одной ли стороны относительно нуля они лежат? Чтобы это проверить, нужно сравнить длину их разности с суммой длин:
$$
|\la v-v|=|v||\la-1|<(1+\la)|v|=|v|+|\la v|,
$$
т.е. да, они лежат на одной прямой по одну сторону от нуля.
\begin{center}
\includegraphics[scale=0.4]{complex-ring2.png}
\end{center}

Итак, число $zw$ получается следующим способом: сначала $w$ переводится на единичную окружность нормировкой, т.е. делением на модуль, получается $w/|w|$. Затем оно поворачивается на угол, заданный числом $z$, затем оно возвращается на свою орбиту, т.е. домножается на $|w|$. В итоге это есть не что иное, как поворот точки $w$ на угол, заданный числом $z$.
\item Кстати, угол, заданный числом $z$, а в общем случае, числом $z/|z|$ (если $z$ --- произвольное ненулевое комплексное число), называется \textbf{аргументом числа} $z$ и обозначается $\arg z$.
\item Основные тригонометрические функции определяются с помощью комплексного числа с единичной окружности так: пусть задан угол $\ph$. Повернем вектор $(1,0)$ на этот угол и найдем число $z$ на единичной окружности такое, что $\arg z=\ph$, тогда
$$
\cos\ph = \Re z,\quad \sin\ph=\Im z.
$$
\item Как уже отмечалось выше, операция комплексного сопряжения есть не что иное как отражение относительно действительной оси. Так что все базовые виды движений плоскости у нас представлены. Учитывая также, что поворот с произвольным центром можно представить как композицию сдвига, поворота с центром в нуле и обратного сдвига, а отражение относительно произвольной оси --- как композицию поворота или сдвига, отражения относительно действитеьлной оси и обратного поворота или сдвига, приходим к тому, что все движения плоскости можно выразить через три изученных нами действия с комплексными числами: сложение (произвольный сдвиг), умножение на число с единичной окружности (поворот с центром в нуле) и сопряжение (отражение относительно действитеьлной оси).
\item На будущее у нас остается вопрос: \textit{какое преобразование плоскости осуществляет умножение на произвольное ненулевое комплексное число?}
\item Поскольку мы пока знакомы только с рациональными дробями, комплексные числа у нас также являются рациональными, т.е. имеют вид $\frac{a}{b}+\frac{c}{d}i$, где $a,b,c,d\in\Z$ и $b,d\ne 0$. Но даже при таком существенном ограничении мы уже имеем дело с еще одним полем --- \textbf{полем комплексных рациональностей},\index{Поле!комплексных рациональностей} поскольку сложение, вычитание, умножение и деление не выводит нас за пределы этого множества (единственное исключение --- модуль числа может выпасть из $\Q$). Такое поле обоначается $\Q[i]$ и является расширением поля $\Q$, аналогично полю $\Q[\sqrt 2]$, рассмотренному ранее.
\end{enumerate}

\subsection*{Задачи}

\begin{enumerate}
\item Докажите, что если $\la>0$, то $|\la-1|<\la+1$.
\item Вычислить, нарисовать на плоскости и указать модуль и аргумент следующих комплексных чисел:
$$
i^2,\;i^3,\;i^4,\;1/i,\;(1+2i)(2-i),\;(1+i)(1+2i)(1+3i),\;\frac{1}{1+i},\;\frac{5}{2-i}.
$$
\end{enumerate}


\section{Гауссовы целые числа}

\lesson{Делимость гауссовых чисел. Норма. Группа корней 4 степени из 1. Ассоциированные числа. Свойства делимости}
\index{Гауссовы целые числа}\index{Числа!гауссовы}\index{Кольцо!гауссовых целых чисел}

\begin{enumerate}
\item В этом разделе мы ограничимся рассмотрением комплесных чисел с целыми координатами, т.е. чисел вида
$$
a+bi,\quad a,b\in\Z
$$
Легко видеть, что такие числа образуют коммутативное кольцо с единицей. Данное кольцо обозначается $\Z[i]$ и называется кольцом \textbf{гауссовых целых чисел}. На координатной плоскости точки $\Z[i]$ сосредоточены в узлах целочисленной решетки.
\item Число $a+bi$ \textbf{делится на} $c+di$, если существует число $a'+b'i$ такое, что $a+bi=(c+di)(a'+b'i)$. Обозначение аналогично обычному в натуральных числах: $(c+di)|(a+bi)$. Например, число 2 делится на $(1+i)$, т.\,к. $2=(1+i)(1-i)$.\index{Делимость чисел}

\item 
Нормой гауссова числа $a+bi$ называется величина $$N(a+bi)=(a+bi)(a-bi)=a^2+b^2,$$
т.е. норма числа $z$ равна $z\bar z$

 Несколько свойств нормы:
\begin{enumerate}[{\bf Norm1}]
\item $N(a+bi)=0$ тогда и только тогда, когда $a=b=0$.
\item Нормы комплексно сопряженных чисел совпадают.
\item Если норма нечётна, то она имеет вид $4k+1$, никакая норма не может быть равна $4n+3$.

Поскольку $N(a+bi)=a^2+b^2$, легко видеть, что она является нечетным числом только в том случае, когда $a$ четное, $b$ нечетное, либо наоборот. Пусть $a=2k$, $b=2j+1$, тогда $a^2+b^2=4k^2+4j^2+4j+1 = 1\pmod 4$.

\item $N(zw)=N(z)N(w)$, где $z,w$ --- гауссовы числа.

Пусть $z=a+bi$, $w=c+di$, тогда
\begin{align*}
N(zw)    & = N(ac-bd+i(ad+bc)) = a^2c^2+b^2d^2+a^2d^2+b^2c^2 \\
N(z)N(w) & = (a^2+b^2)(c^2+d^2) = a^2c^2+a^2d^2+b^2c^2+b^2d^2.
\end{align*}
\end{enumerate}
Последнее свойство означает, что  делителями единицы (обратимыми элементами) могут быть только числа с нормой 1, т.\,е. $\pm 1$ и $\pm i$. Других обратимых нет. Геометрически делителями единицы, являются те и только те гауссовы числа, которые лежат на единичной окружности. Отметим также, что все делители единицы образуют множество всех корней 4 степени из 1, т.е. корней уравнения $x^4=1$.

Наконец, множество $\{1,-1,i,-i\}$ является группой по умножению, причем уже хорошо знакомой нам группой, если не обращать внимание на символ операции и символы элементов группы. Сравните таблицы <<умножения двух групп>>: этой и группы сложения вычетов по модулю 4 $\Z_4$:
\begin{center}
\begin{tabular}{c||c|c|c|c|}
* & 1 & i & -1 & -i\\
\hline\hline
1 & 1 & i & -1 & -i\\ \hline
i & i & -1 & -i & 1 \\ \hline
-1 & -1 & -i & 1 & i \\ \hline
-i & -i & 1 & i & -1 \\ \hline
\end{tabular}
\qquad
\begin{tabular}{c||c|c|c|c|}
$+$ &0 & 1 & 2 & 3\\
\hline\hline
0 &0 & 1 & 2 & 3\\ \hline
1 &1 & 2 & 3 & 0 \\ \hline
2 &2 & 3 & 0 & 1 \\ \hline
3 &3 & 0 & 1 & 2\\ \hline
\end{tabular}
\end{center}

Если произвести соответствие $1\mapsto 0$, $i\mapsto 1$, $-1\mapsto 2$, $-i\mapsto 3$, а операции умножения поставить в соответствие операцию сложения по модулю 4, то мы получим и полное соответствие между результатами умножения в первой группе и сложения во второй: $i(-1)\mapsto 1+2$ и т.д.

В том случае, когда можно предъявить взаимно однозначное соответствие элементов двух групп так, чтобы операция в первой группе соответствовала операции во второй, говорят о том, что эти две группы \textbf{изоморфны}. Очень часто такие группы даже считают равными, хотя природа у них разная. Итак, группа по умножению обратимых гауссовых чисел изоморфна группе $\Z_4$.\index{Изоморфизм групп}


\item Все гауссовы числа делятся на делители единицы. Это легко понять из групповых свойств делителей единицы. Действительно, разделить на 1 означает умножить на нее, т.к. 1 сама себе обратна по умножению. Разделить на $i$ означает умножить на $-i$, т.к. эти числа взаимно обратны по умножени. Аналогично, разделить на $-1$ означает умножить на -1, и разделить на $-i$ означает умножить на $i$.

\item Делители единицы обладают еще одним замечательным свойством: умножение на них --- это поворот относительно начала координат, причем умножение на $i$ есть поворот на угол $\pi/2$, умножение на -1 --- поворот на угол $\pi$ (т.е. центральная симметрия), умножение на $-i$ --- поворот на угол $3\pi/2$ или $-\pi/2$. То есть делители единицы соответствуют еще и группе вращений квадрата.


\item Два гауссовых числа называют \textbf{ассоциированными},\index{Числа!ассоциированные гауссовы} если одно получается из другого умножением на делитель единицы. Ассоциированность является отношением эквивалентности, причем каждый класс эквивалентности включает ровно 4 числа, расположенных в углах квадрата с центром в 0. Например, $1+2i$, $-2+i$, $-1-2i$ и $2-i$ ассоциированы.
\begin{center}
\includegraphics[scale=0.35]{Gaussian.png}
\end{center}

\item Свойства делимости гауссовых чисел очень похожи на таковые свойства в арифметике натуральных чисел, но есть и отличия.
Приведем несколько свойств:
\begin{enumerate}[\bf\hbox{Div}1]
\item Если гауссово число $a+bi$ делится на обычное целое число $c+i0$, то $c|a$ и $c|b$ в целых числах.

Это легко видеть из равенства $a+bi=(x+yi)(c+i0)=xc+yci$, откуда $a=xc$ и $b=yc$.
\item Если $z|w$ и $w|z$, то $z$ и $w$ ассоциированы.

Пусть $w=z'z$ и $z=w'w$, откуда $N(w)=N(z')N(z)$ и $N(z)=N(w')N(w)$, откуда $N(z')N(w')=1$ и, следовательно, $N(z')=N(w')=1$, поскольку в натуральных числах это единственное решение. Стало быть, $z'$ и $w'$ --- делители единицы.

\item Ассоциированность сохраняет делимость: если $z$ и $w$ ассоциированы, $u$ и $v$ ассоциированы, то $(z|u)\to (w|v)$.

Действительно, пусть $z=z'w$ и $u=u'v$, где $z',u'$ --- делители единицы. Тогда
$$
\frac{u}{z} = \frac{u'}{z'}\frac{w}{v},
$$
так что если отношение $u/z$ является гауссовым числом, то отношение $v/w$ является ассоциированным с ним числом.
\item $z$ ($N(z)>1$) имеет как минимум 8 делителей: своих ассоциированных и ассоциированных с 1.
\item Делители $z$ являются делителями $N(z)$, если $N(z)$ рассматривать как гауссово число.

Пусть $w|z$, т.е. $z=uw$. Поскольку $N(z)=z\bar z=w(u\bar z)$, очевидно, что $N(z)$ делится на $w$.
\item Норма $z=a+bi$ четна тогда и только тогда, когда $(1+i)|z$, в частности, если $a$ и $b$ имеют разную четность, то $z$ не делится на $1+i$.

Для начала заметим, что норма $z=a+bi$ четна тогда и только тогда, когда $a$ и $b$ имеют одинаковую четность, т.е. сравнимы по модулю 2. Далее, поскольку $(1+i)|z$, существует $c+di$ такое, что
$$
a+bi = (1+i)(c+di) = c-d + i(c+d),
$$
т.е. $a=c-d$, $b=c+d$, что равносильно $a-b=-2d$, $a+b=2c$ при некоторых $с,d\in\Z$, а это равносильно тому, что $a\equiv b\pmod 2$.
\end{enumerate}


\lesson{Деление гауссовых чисел с остатком, алгоритм Евклида, представление НОД в виде линейной комбинации}


\item В кольце $\Z[i]$ можно любое число $u$ разделить на любое число $v\ne 0$ с остатком, так что получится
\begin{equation}\label{Ostat}
u=qv+r,\quad N(r)<N(v).
\end{equation}
При этом выбор чисел $q$ и $r$ можно строго ограничить, выбирая $q$ как ближайшее гауссово число к комплексному $u/v\in\Q[i]$, а $r$ как разность между $u$ и $qv$. В случае, когда выбор $q$ неоднозначен (может быть максимум 4 числа), можно договориться выбирать то, которое на координатной сетке находится левее и/или ниже.

Приведем один из вариантов вычисления $r$. Пусть $u=a+bi$ и $v=c+di$. Далее оперируем в поле $\Q[i]$:
$$
\frac{a+bi}{c+di}=\frac{(ac+bd)+(bc-ad)i}{c^2+d^2}=q_1+\frac{r_1}{c^2+d^2}+q_2i+\frac{r_2}{c^2+d^2}i,
$$
где $ac+bd=q_1(c^2+d^2)+r_1$ и $bc-ad=q_2(c^2+d^2)+r_2$. Здесь мы воспользовались делением с остатком в кольце $\Z$. При этом мы выбираем знаки $r_1$ и $r_2$ так, чтобы выполнялись неравенства:
$$
|r_1|,|r_2|\le (c^2+d^2)/2.
$$
Это всегда возможно, поскольку остатки от деления можно выбирать не только из ряда $0,1,2,\dots,c^2+d^2-1$, но также из ряда
$0,\pm 1,\pm 2,\dots,\pm k$, где $k$ --- целая часть от деления $c^2+d^2$ на 2, так что всегда $k\le(c^2+d^2)/2$. Здесь как раз и может оказаться вплоть до 4-х вариантов выбора.

Тогда
$$
u=(q_1+q_2i)v+\frac{(r_1+r_2i)(c+di)}{c^2+d^2}=(r_1+r_2i)\frac{v}{N(v)},
$$
эту последнюю дробь мы и выберем в качестве остатка $r$.

При этом заметим, что поскольку разность $u-(q_1+q_2i)v$ является гауссовым числом, то таковым же будет и число $(r_1/N(v)+r_2i/N(v))v$, хоть оно и выглядит нецелым.

Далее,
$$
N\left(\frac{r_1}{N(v)}+\frac{r_2}{N(v)}i\right)=\frac{r_1^2+r_2^2}{(c^2+d^2)^2}\le \frac 12,
$$
откуда $N(r)\le (1/2)N(v)<N(v)$.

\item На основе деления с остатком нетрудно получить выполнимость алгоритма Евклида (см. раздел \ref{EVKL}) для гауссовых чисел:
\begin{align*}
u & = q_1v + r_1, &  N(r_1)<N(v),\\
v & = q_2r_1 + r_2, &  N(r_2)<N(r_1),\\
r_1 & = q_3r_2 + r_3, &  N(r_3)<N(r_2),\\
\dots & \dots & \\
r_{n-1} = q_{n+1}r_n + r_{n+1}, &  N(r_{n+1})<N(r_n),\\
r_{n} = q_{n+2}r_{n+1}, &  N(r_{n+1})<N(r_n),\\
\end{align*}
поскольку норма является натуралным числом и не может убывать бесконечно.

Отсюда так же, как для целых чисел, выводится и представление $\gcd(u,v)$ в виде линейной комбинации исходных чисел $u,v$. Но мы докажем этот факт иным способом.
\begin{lem}\label{NOD}
Для любых гауссовых чисел $u,v\ne 0$ существует гауссово число $r$ такое, что:

\textup{1)} $r|u$ и $r|v$ (общий делитель);

\textup{2)} если $(q|u)\wedge(q|v)$, то $q|r$ (наибольший общий делитель);

\textup{3)} существуют гауссовы $x,y$ такие, что $r=xu+yv$.

\noindent Кроме того,

\textup{4)} число $r$, удовлетворяющее \textup{1)--3)}, единственное с точностью до ассоциированности.
\end{lem}
\pf
Рассмотрим множество $R(u,v)=\{xu+yv|\;x,y\in\Z[i]\}\setminus\{0\}$. В множестве норм $\{N(z)|\;z\in R(u,v)\}$ существует наименьшее положительное число (т.\,к. нуля там быть не может). Пусть $r\in R(u,v)$ такое число, у которого норма минимальная (оно может быть не единственное, выберем одно). Остается показать, что $r$ --- искомое.

Во-первых, $r$ имеет вид $xu+yv$ по построению, т.е. выполняется пункт 3). Во-вторых, если $(q|u)$ и $(q|v)$, то очевидно, что $q|r$ также по построению $r$, т.е. выполняется пункт 2).

Докажем пункт 1).
Из \eqref{Ostat} имеем: $u=rt+s$, где $N(s)<N(r)$. Подставляя представление $r$, имеем:
$u=xut+yvt+s$, откуда $s=(1-xt)u+(-yt)v$. Если $s\ne 0$, то $s\in R$ как линейная комбинация $u$ и $v$, но тогда $N(s)\ge N(r)$ в силу выбора $r$, а это не так в силу \eqref{Ostat}. Следовательно, $s=0$, откуда $r|u$. Аналогично, $r|v$.

Докажем пункт 4). Пусть $r'=x'u+y'v$ также удовлетворяет свойствам 1)--3). Тогда $r|r'$ и $r'|r$. Из первого следует, что $r'=rt$ и $N(r')=N(r)N(t)$, из второго следует, что $r=r't'$ и $N(r)=N(r')N(t')$. Таким образом, нормы $N(t)$ и $N(t')$ взаимно обратны в натуральных числах, откуда следует $N(t)=N(t')=1$, т.\,е. $t$ --- делитель 1 и, следовательно, $r$ и $r'$ ассоциированы.
\epf

\item Доказанная лемма позволяет определить понятие НОД для гауссовых чисел с точностью до ассоциированности. В качестве НОД мы будем выбирать какое-то одно из четырех (наиболее удобного вида).



\lesson{Гауссовы простые числа. Рождественская теорема Ферма. Критерий Гаусса. ОТА гауссовых чисел}

\item Гауссово число называется \textbf{простым}, если оно не имеет никаких делителей, кроме тривиальных (ассоциированных с 1 и самим собой), и не является делителем 1, т.\,е. простое гауссово число имеет ровно 8 делителей. Два гауссовых числа называются \textbf{взаимно простыми} (обозначается $u\perp v$), если их НОД --- обратимое число, т.\,е. 1 и ассоциированные с ней.

\item Верны следующие свойства простых гауссовых чисел:
\begin{enumerate}[\bf\hbox{Prim}1]
\item Если $a+bi$ простое, то $a-bi$ также простое.

Действательно, если $a-bi=uv$, где $u,v$ не ассоциированы с 1, то $a+bi=\bar u\bar v$, где $\bar u$ и $\bar v$ не ассоциированы  с 1 (т.к. для делителей нуля сопряжение не нарушает ассоциированности), нотогда $a+bi$ не является простым.
\item Если $z$ простое, то его ассоциированные также простые.
\item Если $z$ простое и $z|uv$, то $(z|u)\vee(z|v)$;

Действительно. Пусть простое $z|uv$. Предположим, что $\neg(z|u)$, тогда $z\perp u$, откуда по лемме \ref{NOD} получаем, что $1=xz+yu$. Умножаем на $v$: $v=xzv+yuv$. Справа оба слагаемых делятся на $z$, следовательно, $z|v$. Аналогично, если $\neg(z|v)$, то $z|u$.

\item Норма простого, неассоциированного с $1+i$, всегда нечетна, т.\,е. имеет вид $4k+1$.

Это следует из свойства Div6. Если простое число не ассоциировано с $1+i$, то оно и не делится на него, а значит, по свойству Div6 его норма нечетная. То, что она имеет вид $4k+1$, следут из $(2k)^2+(2n+1)^2=4m+1$.
\item Натуральное простое не всегда есть гауссово простое: $5=(2+i)(2-i)$.
\item Простое натуральное $4k+1$ можно представить как сумму квадратов $a^2+b^2$ (\textbf{рождественская теорема Ферма}).\index{Теорема!рождественская теорема Ферма}
\pf Рассмотрим факториал $(p-1)!$ в арифметике по модулю $p$. Поскольку $-1\equiv p-1$, $-2\equiv p-2$ и т.д., а всего множителей $p-1=4k$, то все они разбиваются на пары вида $1,-1$, $2,-2$, и т.д., $(p-1)/2,-(p-1)/2$, откуда
\begin{multline*}
(p-1)! \equiv 1(-1)2(-2)\dots\frac{p-1}{2}\frac{-p+1}{2} \equiv \\
 (-1)^{(p-1)/2}\left(\frac{p-1}{2}!\right)^2 \equiv \left(\frac{p-1}{2}!\right)^2\pmod p,
\end{multline*}
поскольку $(p-1)/2=2k$ --- четное число. С другой стороны, по теореме Вильсона \ref{Wilson} $(p-1)!\equiv p-1\pmod p$, так что
$$
p-1\equiv \left(\frac{p-1}{2}!\right)^2\pmod p,
$$
то есть, $p-1$ сравнимо с квадратом некоторого числа $c$, откуда следует, что $c^2+1$ делится на $p$.

Теперь переходим в числа Гаусса: $c^2+1=(c+i)(c-i)$. Если число $p$ --- простое в гауссовых числах, то в силу ОТА либо $c+i$ делится на $p$, либо $c-i$ делится на $p$, тогда по свойству Div1 число $1$ делится на $p$, что невозможно. Следовательно, $p$ --- не простое гауссово число, а значит,
$$
p = (a+bi)(x+yi),
$$
где оба множителя нетривиальны. В то же время $p$ --- не комплексное число, т.е. $p=\bar p$, т.е.
$$
p = (a-bi)(x-yi),
$$
наконец, норма $p$ будет равна
$$
p\bar p=p^2=(a^2+b^2)(x^2+y^2).
$$

А теперь возвращаемся в обычные натуральные числа, поскольку слева и справа именно они. Число $p$ --- простое, стало быть, его квадрат в силу ОТА раскладывается единственным образом на произведение $p$ и $p$, откуда
$$
p=(a^2+b^2)=(x^2+y^2),
$$
что и завершает доказательство.
\epf

\item Рождественская теорема Ферма еще проще выводится из критерия Гаусса того, что комплексное число является простым гауссовым числом. Доказательство этого критерия мы оставим за рамками курса.
\begin{thrm}[Критерий Гаусса]
$a+bi$ простое тогда и только тогда, когда\\
\textup{1)} либо одно из чисел $a,b$ нулевое, а второе --- простое целое число вида $\pm(4k+3)$ ($k>0$), \\
\textup{2)} либо $a,b$ ненулевые и норма $N(a+bi)=a^2+b^2$ --- простое натуральное число.
\end{thrm}
\item \textbf{Следствие}: простое натуральное вида $4k+1$ не может быть простым гауссовым, простые натуральные вида $4k+3$ являются простыми гауссовыми.

\item Из данного следствия рождественская теорема Ферма следует в один шаг (см. конец доказательства, приведенного выше).

\item Если $N(z)\perp N(w)$ в натуральных числах, то $z\perp w$ в гауссовых числах.

Пусть $u=\gcd(z,w)$ в гауссовых числах. Тогда $z=ut$, $w=ut'$ и $N(z)=N(u)N(t)$, $N(w)=N(u)N(t')$. Откуда $N(u)|N(z)$ и $N(u)|N(w)$. Тогда из условия $N(z)\perp N(w)$ следует, что $N(u)=1$, т.\,е. $u$ --- ассоциированное с 1 гауссово число. Откуда $z\perp w$.
\end{enumerate}
Примеры простых гауссовых чисел: $\pm 3, \pm 7, \pm 3i$; $1\pm i, 1\pm 2i, 1\pm 4i$.

\item Для гауссовых чисел существует аналог основной теоремы арифметики (см. раздел \ref{PrimeNumbers}):\index{Теорема!ОТА гауссовых чисел}
\begin{thrm}[Основная теорема арифметики гауссовых чисел]\label{OTAG}\quad\\
Каждое ненулевое неассоциированное с 1 гауссово число раскладывается на гауссовы простые множители, причем это разложение единственно с точностью до ассоциированных с этими множителями простых и порядка множителей, т.\,е. разложение имеет вид
$$
\al_1^{s_1}\dots \al_n^{s_n}=\be_1^{s_1}\dots \be_n^{s_n},
$$
где пары $\al_i,\be_i$ являются ассоциированными простыми числами, а степени $s_i$ --- натуральными числами.
\end{thrm}
Доказательство теоремы прямо следует из свойства Prim3.

Пример: $5=(2+i)(2-i)=(1+2i)(1-2i)$ (множители переводятся друг в друга умножением на $i$ и на $-i$).



\lesson{Диофантовы уравнения, уравнение $x^2+1=y^3$. Пифагоровы тройки. Теорема Ферма при $n=4$ --- метод бесконечного спуска}


\item Из ОТА легко выводится следующее утверждение
\begin{lem}\label{OLF}
Если $(u\perp v)\land(uv=c^n)$, то существуют $a\perp b$ такие, что $u=a^n$ и $v=b^n$ и $c=ab$.
\end{lem}


Заметим, что и в обычной арифметике целых чисел верна такая же лемма. Более того, как основная теорема арифметики \ref{OTAG}, так и лемма \ref{OLF} верны в любом \textbf{евклидовом кольце} (т.\,е. в таком кольце, где возможно деление с остатком в виде \eqref{Ostat} при некоторой натурально-значной норме и, как следствие, алгоритм Евклида). Этим свойством евклидовых колец мы еще воспользуемся в дальнейшем.\index{Алгоритм Евклида}

\item Рассмотрим парочку примеров, где числа Гаусса дают заметный выигрыш по скорости и простоте решения задач, связанных с уравнениями в целых числах, т.е. \textbf{диофантовыми уравнениями}.\index{Уравнение!диофантово}
\item Рассмотрим уравнение $$x^2+1=y^3,\quad x,y\in\Z.$$
В гауссовых числах оно эквивалентно уравнению $$(x+i)(x-i)=y^3.$$

\item Покажем, что $x+i\perp x-i$. Действительно, если это не так, т.\,е. $z|x+i$ и $z|x-i$, то $z|(x+i)-(x-i)=2i$, откуда $z=1+i$ или ему ассоциированное. Кроме того, $z|y^3$, причем, поскольку $1+i$ --- простое, оно должно входить в разложение $y^3$ трижды, т.\,е. $z^3|y^3$, но тогда в разложение $x+i$ или $x-i$ входит $z^2=2i$, чего быть не может, т.\,к. $x\pm i$ не делится на 2 (см. свойство Div1). Следовательно, $x+i\perp x-i$.

\item Из предыдущего и леммы \ref{OLF} следует, что существует число $a+bi$ такое, что $x+i=(a+bi)^3$. Возводя в куб и сравнивая коэффициенты при $i$, находим, что $1=b(a^2-b^2)$. Это --- уравнение в целых числах, поэтому $b=\pm 1$, откуда $a^2=0$ или 2. Но $a^2=2$ неразрешимо в целых числах, поэтому $a=0$, откуда $x=0$. Таким образом, единственно возможное решение в целых числах у исходного уравнения $x^2+1=y^3$ --- это $x=0, y=1$.

\item Рассмотрим \textbf{Теорему Ферма}\index{Теорема!Ферма} при $n=2$: $a^2+b^2=c^2$ (в натуральных числах). Ясно, что можно сразу считать, что все числа $a,b,c$ попарно взаимно простые натуральные числа (иначе можно было бы сократить уравнение на общий множитель). Отсюда также следует, что $a$ и $b$ имеют разную четность. Действительно, если $a$ и $b$ четные, то таково же и $c$, а значит, они не взаимно простые. Если $a$ и $b$ нечетные, то $a^2+b^2$ имеет остаток 2 при делении на 4, но $c^2$ может иметь остаток либо 0 (четное), либо 1 (нечетное). Таким образом, допускается только случай, когда $a$ и $b$ имеют различную четность. Тогда по свойству Div6 число $a+bi$ не делится на $1+i$.

\item Заметим, что $(a+bi)(a-bi)=a^2+b^2=c^2$. Предположим, что НОД чисел $a+bi$ и $a-bi$ равен $r$ и отличен от делителя 1. Тогда $r|2a$ и $r|2bi$. Но $a\perp b$ в натуральных числах, тогда $N(a)\perp N(b)$, откуда по свойству Prim9 $a\perp b$ в гауссовых числах. Это значит, что $r$ есть НОД 2 и $2i$, т.\,е. $r=1+i$ или его ассоциированным. Но такое число не может быть делителем $a+bi$ и $a-bi$ по доказанному выше. Следовательно, $(a+bi)\perp (a-bi)$.

\item Тогда по лемме \ref{OLF} существуют такие $z,w$, что $a+bi=z^2$, $a-bi=w^2$ и $c=zw$. Пусть $z=n+mi$, тогда $a+bi=n^2-m^2+2nmi$, откуда $a-bi=n^2-m^2-2nmi$, откуда $w=n-mi$ и $c=n^2+m^2$.

\item Таким образом, мы получаем формулу \textbf{пифагоровых троек}:
$$
a=n^2-m^2,\quad b=2nm,\quad c=n^2+m^2,
$$
где натуральные $n,m>0$.

\item Рассмотрим теперь уравнение $x^4+y^4=z^4$, неразрешимость которого доказал еще сам Ферма методом, который мы покажем ниже. 

\item Докажем более сильное утверждение: $x^4+y^4=z^2$ неразрешимо в целых положительных числах.
\item Как и прежде, считаем сразу же, что $x\perp y$. Посмотрим на это уравнение как на уравнение второй степени: $(x^2)^2+(y^2)^2=z^2$. Если оно разрешимо, то существуют ненулевые взаимно простые $n,m$ такие, что
$$
x^2=n^2-m^2,\quad y^2=2nm,\quad z=n^2+m^2,
$$
откуда вновь получаем уравнение второй степени $x^2+m^2=n^2$, а значит, его решение имеет вид:
$$
x=a^2+b^2,\quad m=2ab,\quad n=a^2+b^2,
$$
где ненулевые $a\perp b$. Тогда для $y$ имеет место равенство: $y^2=4nab$ и, поскольку число 2 простое (в обычных целых числах), $y=2y'$.

Тогда $(y')^2=nab$. Так как $n,a,b$ попарно взаимно просты (это следует из того, что $a\perp b$ и $n=a^2+b^2$), в силу леммы \ref{OLF} (для обычных целых чисел) существуют такие $s,t,k$, что $n=s^2$, $a=t^2$, $b=k^2$. Подставляем это в равенство $n=a^2+b^2$, получаем:
$$
t^4+k^4=s^2,
$$
где $t\perp k$ и $z>s>0$ (это следует из того, что $s=\sqrt n$, $n^2<z$). 

Таким образом, имея одно решение $(x,y,z)$ исходного уравнения, мы построили еще одно $(t,k,s)$, где $s<z$. Продолжая применять эти построения далее, мы получим бесконечную последовательность решений $(t_j,k_j,s_j)$ такую, что $z>s>s_1>s_2>\dots$ Но это невозможно, т.\,к. в натуральном ряде не существует бесконечная строго убывающая последовательность.

\item Полученное противоречие доказывает неразрешимость уравнения $x^4+y^4=z^2$ в целых положительных числах, а значит, и неразрешимость уравнения $x^4+y^4=z^4$. Заметим, что отсюда сразу же следует справедливость теоремы Ферма для всех степеней $n$, кратных 4.

\item Предъявленный здесь метод доказательства называется \textbf{методом бесконечного спуска}. Он напоминает индукцию, только не доказующую, а опровергающую, поскольку приводит к противоречию.\index{Метод бесконечного спуска}

\end{enumerate}



\subsection*{Задачи}
\begin{enumerate}
\item Найти все 4 остатка от деления $2+3i$ на $1+i$.
\item Найти минимальный по норме остаток от деления $3+7i$ на $1+2i$.
\end{enumerate}


\newchapter[линейную алгебру]{Введение в}

\vrezka{
Глава посвящена, в основном, анализу преобразований подобия прямой и плоскости. В то же время, она дает начальные сведения о линейных преобразованиях и матрицах, тем самым, открывая дверь в линейную алгебру.
}

\section{Преобразования}

\lesson{Отображения и преобразования. Композиция отображений. Ассоциативность композиции}

\begin{enumerate}
\item Ранее в главе \ref{Permutations} мы ввели несколько определений функций. Прежде всего, функция есть однозначное соответствие элементов одного множества элементам другого (или того же самого) множества. Во многих разделах математики функции принято называть \textbf{отображениями}, подразумевая при этом некоторую явную или неявную детерминированность при определении отображения, т.е. правило, по которому одним точкам ставятся в соответствие другие.\index{Отображение}
\item Например, рассмотрим множество всех треугольников и будем ставить им в соответствие меру наибольшего угла. Ясно, что это будет отображение из множества всех треугольников в множество чисел. Ясно также, что не все числа могут быть мерой наибольшего угла в треугольнике, поэтому данное отображение не является сюръективным. Однако, если сузить область значений такого отображения до интервала $[60^o;180^o)$, оно ставновится сюръективным, или \textit{отображением на} данный интервал.
\item Однако, данное отображение не является инъективным, поскольку разным треугольникам могут соответствовать одинаковые меры наибольших углов. И только в том случае, если сузить область определения данного отображения до множества всех равнобедренных треугольников, а подобные треугольники считать равными, то максимальный угол однозначно определит трееугольник, и отображение станет инъективным.
\item Таким образом, отображение, которое равнобедренному треугольнику ставит в соответствие его наибольший угол (в градусах) является взаимно однозначным, т.е. биекцией.
\item Биекция может быть не только между фигурами и числами, но вообще между любыми видами объектов. В частности, биекция может быть установлена между объектами одного рода, например, между теми же треугольниками или просто точками на плоскости или в пространстве.
\item В Геометрии обычно под словом \textbf{преобразование}\index{Преобразование} понимается биекция, которая отображает какое-то множество объектов в себя. Например, все изученные нами движения прямой, окружности, плоскости являются их преобразованиями, т.к. устанавливают взаимно однозначное соответствие между точками, соответственно, прямой, окружности и плоскости.
\item Если имеют место отображения $f:X\to Y$ и $g:Y\to Z$, то можно составить из них композицию $g\circ f$, которая также является отображением и действует из $X$ в $Z$ по правилу
$$
(g\circ f)(x)=g(f(x)),
$$
т.е. сначала применяется правый компонент, затем левый.
\item Композиция двух отображения легко обобщается на любое конечное число отображений, важно только, чтобы всякий раз каждое следующее отображение содеражло в своей области определения все те значения, которые были получены на предыдущей цепочке композиции отображений. То есть, если мы имеем композицию
$$
f_n\circ (f_{n-1}\dots (f_2\circ f_1)\dots),\mbox{ где }f_k:X_k\to Y_k,
$$
то должны выполняться следующие вложения образов:
$$
f_1[X_1]\subseteq X_2,(f_2\circ f_1)[X_1]\subseteq X_3,\dots,(f_{n-1}\dots f_2\circ f_1)[X_1]\subseteq X_n.
$$
\item Если композиция задана корректно, т.е. выполнены указанные вложения, то композиция подчиняется аксиоме ассоциативности, т.е. в этой цепочке можно как угодно расставлять скобки, например:
$$
p\circ (h\circ (g\circ f)) = (p\circ h)\circ (g\circ f) = (p\circ (h\circ g))\circ f
$$
и т.д.
\item Заметим, что ассоциативность операции --- ключевое требование для определения группы. И действительно, преобразования зачастую можно рассматривать как элементы группы с операцией композиции. Например, все движения плоскости с операцией композиции образуют группу, поскольку они подчиняются не только требованию ассоциативности, но также аксиомам нейтрального и обратного элемента.
\item Вообще, если мы рассматриваем преобразования множества в себя, то мы автоматически получаем группу, поскольку
\begin{enumerate}[a)]
\item операция композиции ассоциативна (требования о вложении выполняются, т.к. $X_1=\dots=X_n=X$);
\item существует нейтральный элемент --- это тождественное преобразование $\id$, которое <<ничего не делает>>;
\item преобразования по определению являются биекциями, а значит, обратимы: для вского $F$ есть $F^{-1}$ такой, что $F\circ F^{-1}=F^{-1}\circ F=\id$.
\end{enumerate}
\item А вот с требованием коммутативности у преобразований не все так хорошо. И мы помним примеры движений, которые не коммутируют, например, параллельный перенос и отражение.




\lesson{Определение подобия и гомотетии в общем виде}

\item Пусть у нас задано расстояние между точками множества $X$, обозначаемое $\rho(x,y)$. Например, это может быть обычная длина вектора, соединяющего две точки на плоскости. Если $P:X\to X$ такое преобразование, что выполняется тождество
$$
\rho(P(x),P(y)) = k\rho(x,y),
$$
т.е. когда расстояние между образами точек в $k$ раз больше исходного расстояния, то такое преобразование называется \textbf{подобием}.\index{Преобразование!подобие} Число $k$ называется коэффициентом подобия.
\item Заметим, во-первых, что $k\ge 0$, т.к. это число связывает два неотрицательных числа, поскольку расстояние всегда есть неотрицательное число. Кроме того, при $k=0$ отображение $P$ не будет биекцией, т.к. оно схлопывает все точки в одну. Действительно, пусть $y=P(x)$ при некотором $x$. Тогда для любього другого $x'\in X$ имеем $\rho(y,P(x'))=k\rho(x,x')=0$, следовательно, отображение $P$ все точки $X$ переводит в одну точку $y$. Это значит, не существует преобразования при $k=0$, так что у нас всегда $k>0$.
\item Еще один особый случай: $k=1$. При таком коэффициенте подобия мы получаем сохранение расстояний между точками при действии преобразования $P$, а значит, такое преобразование является движением.
\item Композиция подобий с коэффициентами $k$ и $s$ есть подобие с коэффициентом $ks$, что прямо следует из определения подобия.
\item Отсюда же следует, что если есть подобия с коэффициентами $k$ и $1/k$, то их композиция окажется движением. Это свойство поможет нам в дальнейшем свести изучение подобий к движениям, о которых мы уже почти все знаем.
\item Пусть теперь $X$ --- одно из известных нам множеств, а именно, прямая, плоскость или пространство. В таком случае мы можем задать на нем следующее отображение в себя. Зафиксируем точку $O$, и для каждой точки $A$ построим точку $H_O^k$ по правилу: на прямой $OA$ в направлении вектора $\vec{OA}$ отложим отрезок длины $k|OA|$, и полученную точку обозначим за $H_O^k(A)$. Иначе говоря, не меняя направления, мы переносим точку в новое место, удаленное от центра $O$ в $k$ раз дальше, чем исходное расстояние. При этом коэффициент $k$ мы можем выбрать и отрицательным, но в этом случае направление переноса следует сменить на противоположное. Так что в общем виде мы получаем, что 
$$
H_O^k(A) = O + k\vec{OA},
$$
где под суммой точки и вектора понимается точка, полученная откладыванием данного вектора от данной точки.
\item Отображение $H_O^k$ называется \textbf{гомотетией} (растяжением)\index{Преобразование!гомотетия}\index{Гомотетия} с центром $O$ и коэффициентом $k$. Число $k$ предполагается любым, однако, как и в случае подобия, при $k=0$ мы получим схлопывание всех точек в одну точку $O$, и такое отображение не только не будет преобразованием, но и вовсе неинтересно для изучения. Поэтому в дальнейшем мы считаем, что для гомотетий $|k|>0$.
\item Гомотетия является преобразованием подобия. Действительно, какова бы ни была точка $A$, отличная от $O$, можно применить к ней гомотетию $H_O^{1/k}$ и получить точку $A'$, которая под действтием гомотетии $H_O^k$ перейдет в точку $A$. Следовательно, гомотетия является сюръекцией. Также легко проверить, что разные точки под действием гомотетии переходят в разные точки: у них либо сразу же разные направления, либо. если направление общее, разное расстояние от центра. Следовательно, гомотетия есть биекция. Наконец, $|H_O^k(A)H_O^k(B)|=|k||AB|$ в силу подобия треугольников. Так что, гомотетия есть частны случай подобия.
\item При $k=1$ гомотетия является преобразованием $\id$.
\item При $k=1$ гомотетия является центральной симметрией, т.е. в случае прямой это --- отражение, а в случае плоскости --- поворот на $180^o$.
\item Если мы рассмотрим гомотетии с общим центром $O$, то для них легко увидеть мультипликативное тождество:
$$
H_O^k\circ H_O^s = H_O^{ks},
$$
т.е. композиция гомотетий --- это не что иное как произведение чисел --- коэффициентов гомотетий. Говоря языком Алгебры, множество всех гомотетий с общим центром с операцией композиции изоморфно числовой оси с операцией умножения.
\item Это нам в точности напоминает ситуацию со сдвигами на прямой, когда композиция $T_a\circ T_b=T_{a+b}$ соответствует операции сложения на числовой оси.
\item Таким образом, мы вновь замечаем, что числа --- это не только меры длин и площадей, но это еще и преобразования! Ранее мы уже видели, что целые числа отвечали за кратности и направления преобразований, затем мы наблюдали композицию сдвигов и поворотов, соответствующую сложению произвольных чисел, теперь мы видим, что еще один вид преобразований своей композицией экивалентен умножению чисел.

\end{enumerate}



\section{Подобия прямой и плоскости}



\lesson{Подобия прямой. Таблица композиций}

\begin{enumerate}

\item Пусть $P$ --- некоторое подобие прямой с коэффициентом подобия $k>0$. Основная идея анализа видов подобий заключается в том, чтобы разложить подобие в композицию гомотетии и сдвига, поскольку то и другое есть частный случай подобия.
\item Рассмотрим некоторую гомотетию $H_A^{1/k}$. Тогда композиция
\begin{equation}\label{GPH}
G = P\circ H_A^{1/k}
\end{equation}
является движением, т.е. либо сдвигом, либо отражением. Следовательно, исходное подобие $P$ является композицией либо сдвига и гомотетии, либо отражения и гомотетии. Но фишка в том, что отражение --- частный случай гомотетии, так что все подобия прямой описываются таблицей умножения сдвигов и гомотетий. Построим эту таблицу:\index{Таблица композиций}
\begin{center}
\begin{tabular}{c|c|c|}
$\id$    & $T_u$    &   $H_A^k$  \\ \hline
$T_v$    & $T_{u+v}$ & $\displaystyle H_{A+v/(1-k)}^k$ ($k\ne 1$) \\ \hline
$H_B^s$  &  $\displaystyle H_{B+sv/(1-s)}^s$ ($s\ne 1$)  & \specialcell{ $H_{A}^{ks}$ ($A=B$) \\[2pt]
$\displaystyle T_{(1-s)\vec{AB}}$ ($ks=1$) \\[4pt]  $\displaystyle H_{A+\vec{AB}\frac{1-s}{1-ks}}^{ks}$ ($ks\ne 1$) } \\ \hline
\end{tabular}
\end{center}
\item Итак, из формулы \eqref{GPH} мы получаем, что $P=G\circ H_A^k$, где $G$ --- это либо $T_v$, либо $H_B^{-1}$, т.е. отражение с центром в точке $B$. Тогда в первом случае получаем, что $P$ есть либо $T_v$ в случае $k=1$, либо $H_C^k$, т.е. гомотетия с некоторым центром $C$. Во втором случае получаем, что $P=H_D^{-k}$, т.е. гомотетия с некоторым центром $D$.
\begin{thrm}
Подобие на прямой является гомотетией, если коэффициент подобия отличен от $1$. Если коэффициен равен $1$, то это --- сдвиг.
\end{thrm}



\lesson{Подобия плоскости. Теорема об общем виде подобий плоскости}


\item Пусть $P^k$ --- некоторое подобие плоскости к коэффициентом подобия $k$. Домножим его справа на гомотетию $H_O^{1/k}$  с центром в точке $O$. Снова получим какое-то движение, которое может быть либо параллельным переносом $T_u$, либо поворотом $R_\al^A$  сцентром в точке $A$, либо скользящей симметрией относительно оси $l$ со сдвигом на вектор $u$. Следовательно, подобие $P^k$ есть композиция одного из этих движений с гомотетией $H_O^k$. Посмотрим, что получается в каждом из трех случаев.
\item Мы упростим себе задачу, положив все преобразования на комплексную плоскость и выбрав ноль и действительную ось наиболее удобным способом.
\item Чтобы найти $T_u\circ H_O^k$, поместим ноль комплексной плоскости в центр гомотетии, а вектор сдвига $u$ будем считать сонаправленным действительной оси $\Re$. Попробуем найти неподвижную точку такого преобразования. Проще всего ее искать на действительной оси, т.к. это сводит случай плоскости к прямой. Действительная точка $x$ переходит сначала в $kx$ под действием гомотетии, а затем сдвигается на число $u$, так что $P^k(x)=kx+u$. Эта точк абудет неподвижной, если $kx+u=x$, или $x=u/(1-k)$.

Итак, при $k\ne 1$ у подобия $P^k=T_u\circ H_O^k$ существует неподвижная точка $O+u/(1-k)$. При этом коэффициент подобия остается равным $k$, так что в итоге получаем
$$
P^k=T_u\circ H_O^k = \begin{cases} H_{A+u/(1-k)}^k, & k\ne 1 \\ T_u, & k=1.\end{cases}
$$

\item Чтобы найти $R_\al^A\circ H_O^k$, поместим ноль комплексной плоскости в центр гомотетии. Точка $z$ переходит сначала в точку $kz$, а затем ее нужно повернуть относительно $A$. Для этого находим разность $kz-A$, временно перемещая центр координат в точку $A$, поворачиваем ее, домножая на единичный вектор $w_\al$, соответствующий повороту на угол $\al$, затем возвращаем смещени на $A$, получаем $(kz-A)w_\al+A$. Находим неподвижную точку:
$$
z = \frac{A(1-w_\al)}{1-kw_\al}.
$$
Эта точка будет центром нового вида преобразований плоскости --- \textbf{поворотной гомотетии}, которая определяется центром, коэффициентом и углом.\index{Гомотетия!поворотная}

Здесь есть ограничение: $kw_\al\ne 1$, которое возможно лишь в двух случаях: $k=1,\al=0$ и $k=-1,\al=\pi$. Если ни одно из этих условий не выполняется, то неподвижная точка существует и вычисляется по указанной формуле. Если $k=1,\al=0$, то итоговая композиция есть $\id$.

Случай $k=-1$, вообще говоря, нас не интересует, т.к. $k$ --- коэффициент подобия, т.е. $k>0$. Но для пользы дела мы рассмотрим и композицию с гомотетией при $k=-1,\al=\pi$, тогда у нас появляется композиция двух поворотов: $H_O^{-1}=R_O^{\pi}$ и $R_\pi^A$, сумма углов которых равна нулю, стало быть это сдвиг на вектор $2\vec{OA}$. В итоге получаем
$$
P^k=R_\al^A\circ H_O^k = \begin{cases}
\id, & k=1,\al=0 \\
T_{2OA}, & k=-1,\al=\pi \\
R_\al H_{A(1-w_\al)/(1-kw_\al)}^k, & \mbox{ иначе}\end{cases}
$$

\item Чтобы найти $T_uS_l\circ H_O^k$, совместим действительную ось с осью симметрии $l$ так, чтобы направление сдвига $u$ совпадало с положительным направлением, а ноль совместим с проекцией точки $O$ на ось $l$. В этом случае получим, что $\Re O=0$. Снова ищем неподвижную точку. Произвольная точка комплексной плоскости может быть записана как $O+z$. Под действием гомотетии она переходит в $O+kz$, далее применяем отражение относительно действительной оси: $\bar O+k\bar z$, далее сдвигаем результат на действительное число $u$: $\bar O+k\bar z+u$. Находим неподвижную точку: $O+z=\bar O+k\bar z + u$. Решим действительную и мнимую часть этого уравнения по отдельности:
$$
\begin{cases}
\Re z = k\Re z+u,\\
\Im O+\Im z = -\Im O-k\Im z
\end{cases}
$$
откуда
$$
\begin{cases}
\Re z = u/(1-k),\\
\Im z = -2\Im O/(1+k),
\end{cases}
$$
и неподвижная точка имеет вид
$$
A=\frac{u}{1-k}+O\frac{k-1}{k+1}.
$$

И снова видим, что есть особые значения для $k$: $k=\pm 1$. При $k=1$ гомотетия $H_O^k=\id$.

Случай $k=-1$ здесь мы также рассматривам исключительно для полноты картины. В этом случае получим $H_O^k=R_O^{-\pi}$, и тогда воспользуемся таблицей композиций движений плоскости для $T_uS_l\circ R_O^\pi$, откуда следует, что это будет скользящая симметрия $T_{2\vec{OO_1}}S_{m+u/2}$, где $O_1=\Pr_lO$, прямая $m$ проходит через $OO_1$. Итого:
$$
P^k=T_uS_l\circ H_O^k=
\begin{cases}
T_uS_l, & k=1 \\
T_{2\vec{OO_1}}S_{m+u/2}, & k=-1 \\
H_A^k, & \mbox{ иначе}.
\end{cases}
$$

\item Подытожим.\index{Теорема!о подобиях плоскости}
\begin{thrm} Всякое подобие плоскости --- это либо параллельный перенос (при $k=1$), либо поворотная гомотетия (в частности, это может быть просто поворот или просто гомотетия), либо скользящая симметрия (в частности, просто симметрия).
\end{thrm}
\end{enumerate}

Кроме того, можно добавить, что любое подобие плоскости является композицией не более чем трех симметрий и не более чем одной гомотетии.

Заметим также, что гомотетию с отрицательным коэффициентом всегда можно заменить на поворотную гомотетию с положительным коэффициентом и поворотом на угол $\pi$.

\section{Линейное пространство}

\lesson{Свободный вектор, операции с точками и векторами}

\begin{enumerate}

\item Пространства, с которыми мы до сих пор работали, --- прямая и плоскость, --- это геометрические пространства точек. Несмотря на то, что при определении движений мы предполагали сдвиг всех точек на какой-то вектор, этот вектор был для нас некоторым внешним объектом, под которым мы подразумевали само действие переноса всех точек на одно и то же расстояние в одном и том же направлении.
\item На самом деле, векторы --- это такие же полноценные математические сущности, как точки, прямые, отрезки, плоскости и фигуры. Наша локальная задача --- выстроить понимание того, что такое вектор, как с ним работать, и какие новые возможности у нас при этом появятся.
\item Итак, первое, самое простое понятие вектора таково: вектор есть пара точек $(A,B)$, где первая точка называется началом вектора, вторая --- концом. Считается, что вектор соединяет эти две точки и имеет направление от первой ко второй. Заметим, что мы здесь не оговариваем специально, в каком пространстве мы находимся. В общем случае это может быть $n$-мерное пространство, но для простоты восприятия можно считать, что мы находимся на прямой или на плоскости. Обычно вектор, заданный парой точек обозначается $\vec{AB}$.\index{Вектор}
\item Такой вектор принято называть \textbf{фиксированным}, поскольку он жестко прибит двумя гвоздями--точками начала и конца к плоскости в определенном месте.\index{Вектор!фиксированный}
\item Если мы теперь обратимся к координатному методу, определим на плоскости начало координат $O$ и две перпендикулярные оси $Ox$ и $Oy$, на каждой из которых существует числовая сетка с нулем в начале координат, то мы можем рассматривать не пары точек, а пары чисел $(x,y)$, подразумевая, что откладываем их на соответствующих осях, чтобы получить конкретную точку на плоскости. В этом случае пару $(x,y)$ можно рассматривать как точку, заданную координатами (с учетом расположения $O$ и осей $Ox,Oy$), а можно --- как вектор, отложенный от начала координат и заканчивающийся в точке $(x,y)$. Тогда $x$ --- это величина проекции данного вектора на ось $Ox$, а $y$ --- величина проекции данного вектора на ось $Oy$. Мы используем термин <<величина>>, поскольку на может быть отрицательной.
\item Этот второй подход к определению вектора, с одной стороны, также фиксирует его в определенном месте плоскости, а с другой стороны, любой параллельный перенос начала координат приведет нас к другому фиксированному вектору на плоскости, который, однако, будет задан все той же парой чисел. Поэтому, если отвлечься от фиксации начала координат, то под парой $(x,y)$ можно понимать <<алгоритм>> построения вектора в любом мест плоскости: для этого достаточно выбрать его начало, а затем отложить проекции $x$ и $y$ вдоль соответствующих осей, чтобы вычислить конец вектора.
\item Эта идея алгоритма наводит нас на мысль о том, что мы можем ввести понятие вектора, не связанного с конкретными точками, а такого, который существует сразу во всех точках и может быть реализован в виде фиксированного вектора, если указать месо его приложения. Такой вектор называется \textbf{свободным}.\index{Вектор!свободный}
\item Наконец, заметим, что свободный вектор, заданный своим алгоритмом $(x,y)$, будучи приложен сразу во всех точках пространства, укажет единое направление и расстояние сдвига всех точек. А что это, если не параллельный перенос? Таким образом, под словом <<вектор>>, на самом деле, можно понимать паралельный перенос или сдвиг (если на прямой).
\item А теперь вспомним, что сдвиги и переносы образуют группу с операцией композиции, причем композиция переносов соответствует сумме векторов, на которые производится перенос всех точек. То есть, тот самый вектор $v$ в записи переноса $T_v$, есть не что иное как свободный вектор, не привязанный к конкретным точкам плоскости. Следовательно, свободный вектор и параллельный перенос --- суть одно и то же.
\item Существует еще один способ определить свободный вектор. С таким подходом мы уже сталкивались, когда говорили о классах вычетов и строили фактормножество $\Z/m\Z$. В том случае у нас числами новой арифметики стали классы точек, выбранных с равным шагом $m$ на оси $\Z$. Точно так же поступим и с векторами. Скажем, что два вектора $\vec{AB}$ и $\vec{CD}$ эквивалентны, если четырехугольник $ABDC$ есть параллелограмм (в том числе вырожденный, когда все точки лежат на одной прямой, но при этом отрезок $CD$ получается сдвигом отрезка $AB$ вдоль этой прямой). По сути это и значает, что данные векторы получаются один из другого некоторым параллельным переносом, поскольку если $g$ --- перенос, то четырехугольник $ABg(B)g(A)$ --- параллелограмм, и, следовательно, $g(A)g(B)\sin AB$. И обратно, если даны два эквивалентных вектора, то перенос на вектор $AC$ связывает их.
\item Далее, обозначим за $[\vec{AB}]$ класс фиксированных векторов, эквивалентных вектору $\vec{AB}$. И рассмотрим фактормножество, состоящщее из всех таких классов. Поскольку все векторы в одном классе получаются друг из друга переносами, то каждый такой класс и есть свободный вектор.
\item На свободных векторах мы можем определить операцию сложения так же, как мы это делали с другими классами эквивалентности, когда работали с $\Z/m\Z$ или построением множества $\Q$, а именно, будем складывать их представителей:
$$
[\vec{AB}] + [\vec{CD}] = [\vec{AB}+\vec{BB'}],
$$
где вектор $\vec{BB'}$ эквивалентен вектору $\vec{CD}$, но стартует в точке $B$. Пользуясь всё теми же параллелограммами, несложно доказать, что данное определение корректно, т.е. при смене представителей классов на эквивалентные получаем тот же самый класс в результате. 
\item В дальнейшем договоримся свободные векторы обозначть просто латинской буквой, без использования символики классов и отношения эквивалентности, понимая, что за этим стоит.
\item Итак, мы теперь имеем два сорта объектов: точки и свободные векторы. Векторы мы умеем складывать, и даже знаем, что они образуют абелеву группу по сложению (по сути, это группа переносов). Как быть с точками?
\item Скажем, что суммой точки и вектора $A+v$ является точка, полученная откладыванием вектора $v$ от точки $A$, т.е. мы находим вектор $AB$ из класса $v$, и в качестве результата берем точку $B$. Обратная операция: $B-v$ --- это такая точка $A$, что $A+v=B$. Кроме того, точки можно вычитать: $B-A$ --- это такой вектор $v$, что $\vec{AB}$ является элементом класса $v$. Отметим, что $B-A$ --- это не фикисрованный вектор, соединяющий точки $A,B$, а свободный вектор!
\item Тем самым мы подвели формальную базу под те записи движений, которые ранее использовали при изучении движений прямой и плоскости.


\lesson{Движения и подобия на векторах. Гомотетия как умножение. Определение векторного пространства}

\item Если ранее мы рассматривали движения и подобия применительно к точкам, то теперь посмотрим, как они работают на свободных векторах. Пусть $g$ --- некоторое подобие с коэффициентом $k>0$, в частности, при $k=1$ это движение.
Пусть, кроме того, имеются точки $A,B$ на прямой и вектор $v=B-A$. Каким будет вектор $g(B)-g(A)$? Очевидно, это будет вектор длины $k|v|$, при этом он будет либо сонаправлен с вектором $v$, и тогда это будет вектор $kv$, либо он будет ему противоположен по направлению, и тогда это будет вектор $-kv$.

Из строения подобий прямой мы знаем, что подобие есть либо сдвиг, либо гомотетия, а последняя может быть двух видов: с положительным коэффициентом $k$ или с отрицательным $-k$, переворачивающая прямую в обратную сторону. Отсюда легко заключить, что все эквивалентные фиксированные векторы под действием подобия $g$ переходят в эквивалентные же векторы, т.е. подобие сохраняет отношение эквивалентности.

Но тогда корректным будет следуюшее определение:
$$
g(v)=\pm kv,
$$
где знак перед $k$ зависит от знака соответствующей гомотетии. При этом если $g$ является сдвигом, то оно не меняет класс вектора, т.е. в случае свободных векторов сдвигов, можно сказать, вовсе не существует.

Совершенно точно так же действуем в случае плоскости и любого более многомерного пространства. В случае плоскости мы знаем также, что любое подобие есть либо параллельный перенос, либо поворотная гомотетия на угол $\al$ с коэффициентом $k$. В таком случае положим
$$
g(v)=kv',
$$
где $v'$ есть вектор, повернутый на угол $\al$ относительно вектора $v$. Перенос в данном случае будет соответствовать параметрам $\al=0$ и $k=1$, т.е. не будет ничего делать с вектором $v$.

\item Уместно задаться вопросом: как действует движение на векторах? Оказывается, что поскольку на векторах перенос действует как $\id$, то существенным движением для векторов будут только повороты и симметрии. Это значит, что группа движений на векторах действует ровно так же, как группа движений окружности действует на самой окружности --- либо вращает, либо отражает.
\item И здесь мы можем вспомнить тот прием, которым мы пользовались при анализе движений плоскости, когда любое движение сводили к движению окружности.
\item Далее, посмотрим действие гомотетии на векторах. Пусть $g=H_O^k$. Нетрудно видеть, что $g(v)=kv$, т.к. гомотетия не совершает поворот векторов, а только меняет их длину и, возможно, направление на противоположное (при отрицательном $k$).
\item Таким образом, гомотетия обеспечиват нам операцию умножения вектора на число.
\item В частности, $1*v=v$, поскольку гомотетия с коэффициентом 1 есть $\id$ на векторах.
\item Далее, $k(u+v)=ku+kv$, поскольку любое подобие переводит сумму векторов в сумму.
\item Кроме того, $k_1(k_2v)=(k_1k_2)v$, поскольку произведение $k_1k_2$ соответствует композиции гомотетий.
\item Наконец, $(k_1+k_2)v=k_1v+k_2v$. Это следует из того, что вектор $(k_1+k_2)v$ определяется как вектор, имеющий длину $|k_1+k_2||v|$ и сонаправленный с $v$, если $k_1+k_2>0$, и противоположно направленный, если $k_1+k_2<0$. То есть, арифметика векторов сводится к арифметике чисел на прямой, только с поправкой на направление, которое соответствует вычитанию чисел.
\item Отметим также особый случай --- гомотетия с коэффициентом 0, которая схлопывает все векторы в точку, или, в нулевой вектор. Нулевой вектор в арифметике векторов играет такую же роль, что и обычный ноль в кольце чисел: $\la\vec 0=\vec 0$ и $v+\vec 0=v$.
\item Итак, мы определили понятие свободного вектора, научились их складывать и умножать на числа, получили основные свойства этих операций. 
\item Такая структура, составленная из векторов и чисел, и подчиняющаяся нижеперечисленным свойствам, в Алгебре называется \textbf{модулем} (над соответствующей системой чисел, например, $\Z$ или $\Q$, или $\R$).\index{Модуль над кольцом} Приведем все требования к модулю:
\begin{enumerate}[\bf Mod1]
\item Векторы образуют абелеву группу по сложению;
\item Числа образуют коммутативное кольцо с единицей;
\item Умножение * числа на вектор и сложение векторов подчиняются следующим правилам:

\begin{center}
\includegraphics[scale=0.3]{ModuleOverRing.png}
\end{center}
\end{enumerate}

В том случае, если числовая структура является полем, т.е. числа можно делить друг на друга, модуль называется \textbf{векторным пространством}.\index{Пространство!векторное} Векторное пространство еще называется \textbf{линейным пространством}\index{Пространство!линейное} и является объектом изучения Линейной алгебры.



\lesson{Линейная комбинация, оболочка. Базис векторного пространства, теорема о размерности}


\item Линейной комбинацией векторов $\e_1,\dots,\e_n$ из векторного пространства $V$ называется всякое выражение вида
$$
k_1\e_1+\dots+k_n\e_n,
$$
где коэффициенты $k_1,\dots, k_n$ принадлежат той числовой структуре, над которой задано векторное пространство.

Возьмем все возможные такие комбинации и соберем в множество $V'$:
$$
V'=\{k_1\e_1+\dots+k_n\e_n\}.
$$
Тогда $V'$ называется \textbf{линейной оболочкой} системы векторов $\{\e_1,\dots,\e_n\}$.\index{Линейная оболочка}

Нетрудно доказать, что линейная оболочка любой системы векторов, даже одного нулевого вектора, является сама по себе векторным пространством. Понятно, что $V'\subseteq V$, т.е. является подпространством пространства $V$.

Ну, а в том случае, когда $V'=V$, говорят, что система векторов $\{\e_1,\dots,\e_n\}$ порождает пространство $V$.

\item Например, на плоскости можно выбрать два перпендикулярных вектора, и тогда все возмодные линейные комбинации этих векторов заметают всю плоскость. Если к ним добавить третий вектор из той же плоскости, то ничего не изменится --- они по-прежнему будут заметать всю плоскость, и не более того. Однако если взять только один вектор, то его линейные комбинации будут накрывать всего лишь ту прямую, на которой лежит данный вектор, и эта прямая будет собственным подпространством плоскости. Поэтому системы векторов могут быть избыточными или недостаточными для чтого, чтобы получить равенство $V'=V$.


\item Система векторов $\{\e_1,\dots,\e_n\}$ называется \textbf{линейно независимой}, если никакая нетривиальная линейная комбинация этих векторов не равна нулю. Для примера, на плоскости никакая линейная комбинация, кроме как когда все коэффициенты нули, не обратит в ноль систему из двух перпендикулярных векторов. В то же время, если векторы коллинеарны, можно так подобрать коэффициенты, что их линейная комбинация обратится в ноль. 

Проще говоря, в том случае, когда векторы линейно зависимы, один из них можно выразить через остальные в виде линейной комбинации (в случае модуля над кольцом это не совсем так, но у нас-то поле!).

\item Система векторов $\{\e_1,\dots,\e_n\}$ называется \textbf{базисом пространства} $V$\index{Базис пространства}, если она линейно независима, а ее линейная оболочка равна $V$, т.е. любой вектор пространства можно представить в виде ланейной комбинации базисных векторов.

Отметим, что поскольку мы сразу же предположили, что система чисел, из которой берутся коэффициенты перед векторами, является полем, то количество элементов базиса одинаково для всех базисов (если существует конечный базис).

\begin{lem} В коммутативном кольце линейная система уравнений
$$
\begin{cases}
a_{11}x_1+\dots+a_{1n}x_n=0 \\
\dots \\
a_{m1}x_1+\dots+a_{mn}x_n=0
\end{cases}
$$
имеет нетривиальное решение $x_1^*,\dots,x_n^*$, если $m<n$.
\end{lem}
\pf Проведем доказательство индукцией по $n$. Для $n=2$ имеем единственный вариант системы: $a_{11}x_1+a_{12}x_2=0$.

Случай 1. $a_{11}=a_{12}=0$. Тогда решение $x_1^*=1, x_2^*=0$ является нетривиальным решением (как и вообще любое другое).

Случай 2. $a_{11}\ne 0$ или $a_{12}\ne 0$. Тогда решение $x_1^*=a_{12}, x_2^*=-a_{11}$ является нетривиальным решением (мы воспользовались коммутативностью кольца $K$).

Предположим, что для $n$ лемма доказана и рассмотрим случай $n+1$ переменной и $m$ уравнений ($m<n+1$):
\begin{equation}\label{systema2}
\begin{cases}
a_{11}x_1+\dots+a_{1,n+1}x_{n+1} &= 0 \\
\hdotsfor{2} \\
a_{m1}x_1+\dots+a_{m,n+1}x_{n+1} &= 0
\end{cases}
\end{equation}

Случай 1. $a_{11}=\dots=a_{m1}=0$. Тогда решение $x_1^*=1, x_2^*=\dots=x_{n+1}^*=0$ является нетривиальным решением.

Случай 2. $a_{11}\ne 0$ (не ограничивая общности, можно считать, что именно самый первый коэффициент обладает таким свойством, в противном случае можно просто иначе перенумеровать уравнения). Тогда из $m$-го и первого уравнения получим новое (снова используя коммутативность):
\begin{center}
\begin{tabular}{rr|l}
& $-a_{m1}\cdot$  & $a_{11}x_1+\dots+a_{1,n+1}x_{n+1}=0$ \\[-5pt]
+ & & \\[-5pt]
& $a_{11}\cdot$ & $a_{m1}x_1+\dots+a_{m,n+1}x_{n+1}=0$\\ \hline
= & \multicolumn{2}{l}{$0\cdot x_1+(a_{11}a_{m2}-a_{m1}a_{12})x_2+\dots+(a_{m,n+1}a_{11}-a_{1,n+1}a_{m1})x_{n+1}=0$}
\end{tabular}
\end{center}
И так проделаем с каждым из уравнений со 2-го по $m$-ое. В итоге мы придем к системе уравнений с переменными $x_2,\dots,x_{n+1}$ и $m-1$ уравнением (так как $m<n+1$, то $m-1<n$, и мы оказываемся в рамках индуктивного предположения). По предположению такая редуцированная система имеет нетривиальное решение $x_2^*,\dots,x_{n+1}^*$. Но тогда нетривиальным решением этой же системы будет и $a_{11}x_2^*,\dots,a_{11}x_{n+1}^*$, т.\,к. $a_{11}\ne 0$.

Положим $x_1^*=-(a_{12}x_2^*+\dots+a_{1,n+1}x_{n+1}^*)$. Нетрудно видеть, что 
$$
x_1^*,a_{11}x_2^*,\dots,a_{11}x_{n+1}^*
$$ является нетривиальным решением первого уравнения исходной системы \eqref{systema2}.

Проверим, что это решение всей исходной системы в целом. Для этого подставим данное решение в $m$-ое уравнение:
\begin{gather*}
a_{m1}x_1^*+a_{m2}a_{11}x_2^*+\dots+a_{m,n+1}a_{11}x_{n+1}^*=\\
=-a_{m1}(a_{12}x_2^*+\dots+a_{1,n+1}x_{n+1}^*)+a_{m2}a_{11}x_2^*+\dots+a_{m,n+1}a_{11}x_{n+1}^*=\\
=(a_{11}a_{m2}-a_{m1}a_{12})x_2^*+\dots+(a_{m,n+1}a_{11}-a_{1,n+1}a_{m1})x_{n+1}^*=0
\end{gather*}
Аналогично --- для остальных уравнений системы. Таким образом, нетривиальное решение найдено, индукция завершена.
\epf

\begin{thrm} Если в линейном пространстве существует конечный базис, то все базисы имеют такую же размерность.\index{Теорема!о базисе ЛП}
\end{thrm}
\pf Пусть $\e_1,\dots,\e_m$ и $\e'_1,\dots,\e'_n$ --- базисы и $m<n$. Тогда в силу определения базиса имеем:
$$
\begin{matrix}
\e_1'=a_{11}\e_1+\dots+a_{m1}\e_m \\
\hdotsfor{1}\\
\e_n'=a_{1n}\e_1+\dots+a_{mn}\e_m
\end{matrix}
$$
и рассмотрим линейное уравнение $x_1\e_1'+\dots+x_n\e_n'=0$. Если $\e_1',\dots,\e_n'$ --- базис, то это уравнение может иметь только тривиальное решение $x_1=\dots=x_n=0$. Подставим сюда разложения базисных векторов и получим:
$$
\begin{matrix}
0 = & (a_{11}x_1+\dots+a_{1n}x_n)\e_1+\\
\hdotsfor{2}\\
 & (a_{m1}x_1+\dots+a_{mn}x_n)\e_m
\end{matrix}
$$
откуда в силу того, что $\e_1,\dots,\e_m$ --- базис, все коэффициенты должны быть равны нулю, т.\,е.
$$
\begin{cases}
a_{11}x_1+\dots+a_{1n}x_{n} &= 0 \\
\hdotsfor{2} \\
a_{m1}x_1+\dots+a_{mn}x_{n} &= 0
\end{cases}
$$
Но в силу предыдущей леммы эта система имеет нетривиальное решение $x_1^*,\dots,x_n^*$, а это означает, что и уравнение $x_1\e_1'+\dots+x_n\e_n'=0$ имеет нетривиальное решение. Следовательно, $\e_1',\dots,\e_n'$ не может быть базисом при $n>m$.

Таким образом, все конечные базисы модуля (над коммутативным кольцом) равномощны, если они существуют.
\epf

Заметим, что ровно такое же доказательство проходит и для модуля над коммутативным кольцом.

\item \textbf{Размерностью} векторного пространства называется мощность его базиса. \index{Размерность пространства}
\end{enumerate}


\section{Линейные операторы}


\lesson{Подобия как линейные операторы. Определение линейного оператора и линейного преобразования}

\begin{enumerate}
\item Рассмотрим арифметические свойства подобий с точки зрения операций над векторами.
\item Всякое подобие сумму векторов переводит в сумму: $g(u+v)=g(u)+g(v)$. Действительно, подобие на векторах (плоскости) --- это поворотная гомотетия. Поворот и гомотетия не меняют углов между векторами, так что по свойствам подобных трегуольников данное равенство выполняется. Такое свойство называется аддитивностью.
\item Всякое подобие однородно: $g(kv)=kg(v)$. Это легко проверить из определения умножения вектора на число и свойств такого умножения: $g(v)=k'v'$, поэтому $g(kv)=k'kv'=kg(v)$.
\item Обозначим за $V$ множество всех векторов (для простоты считаем, что это векторы на плоскости). Отображение $L:V\to V$ называется \textbf{линейным оператором}, если выполнены условия:\index{Линейный оператор}
\begin{enumerate}[{\bf Lin}1]
\item Аддитивность: $L(u+v)=L(u)+L(v)$.
\item Однородность: $L(kv)=kL(v)$.
\end{enumerate}
Термин <<оператор>> используется здесь для выделения частного случая функции, когда она действует из какого-то пространства с  <<хорошей>> алгебраической структурой (в нашем случае это аксиомы модуля над кольцом или полем) в это же или ему подобное (не в смысле подобия фигур) пространство.
\begin{thrm}
Всякое подобие является линейным оператором на множестве векторов.
\end{thrm}
Эта теорема следует из наших предыдущих построений.

\item Бывают ли, и если да, то какие еще бывают линейные операторы над пространством векторов? Изучению этого вопроса мы посвятим оставшуюся часть главы.
\item Существенным признаком, разделяющим линейные операторы, действующие на одном пространстве, на два больших класса, является биекивность, т.е. свойство быть преобразованием пространства векторов. Если линейный оператор является биекцией, то он называется \textbf{обратимым линейным оператором} или линейным преобразованием.\index{Линейный оператор!обратимый}

Обратимые линейные операторы с операцией композиции образуют группу (ведь мы всегда можем взять обратный оператор, а кроме того, у нас имеется оператор $\id$, ну а требование ассоциативности отображений выполняется автоматически). Эта группа, которая обозначается $\GL(V)$, называется \textbf{полной линейной группой} пространства $V$.\index{Группа!полная линейная}

\item Для линейных операторов (не обязательно однородных) можно естественным образом задать операции сложения и умножения на число из того же самого поля, над которым заданы векторы пространства $V$. Для линейных операторов $L$ и $M$ положим
$$
(L+M)(v) = L(v) + M(v),\quad (kL)(v) = kL(v),
$$
т.е. мы переносим операции с векторов на операторы. Нетрудно видеть, что при такой арифметике сами операторы ведут себя ровно так же как векторы, т.е. заданные операции над ними подчиняются аксиомам модуля. Это значит, что множество всех линейных операторов над векторным пространством $V$ само по себе образует новое линейное пространство.
\item Но и это еще не все. На линейных операторах задана операция композиции, которую в случае операторов мы будем называть умножением и обозначть соответствующим образом. Умножение линейных операторов подчинаяется правилу:
\begin{equation}\label{algebra}
k(LM)=(kL)M=L(kM).
\end{equation}
Действительно, $k(LM)=L(kM)$ в силу однородности $L$. В то же время, $(kL)M(v)=kL(M(v))=k(LM)(v)$. То есть, число $k$ можно безнаказанно проносить сквозь символ линейного оператора и любые композиции линейных операторов.

\item Если модуль над кольцом/полем (т.е. структура с аксиомами Mod1--Mod3) подчиняется еще и свойству \eqref{algebra}, то он называется \textbf{алгеброй над кольцом/полем}.\index{Алгебра над кольцом}
\begin{center}
\includegraphics[scale=0.3]{AlgebraOverRing.png}
\end{center}
\item Как видим, множество линейных операторов, действующих на векторном пространстве, является алгеброй с операциями сложения и умножения (композиции).
\item Отметим, что при записи линейных операторов часто опускаются скобки, так что выражение $kL(v)$ приобретает вид $kLv$ и может рассматриваться как умножение соответствующих сущностей. Мы также будем следовать этой традиции в тех случаях, когда это не вызовет двойного толкования записи.



\lesson{Общий вид линейного отображения. Матрица оператора в заданном базисе. Матрица поворота и растяжения.}

\item Рассмотрим на плоскости два свободных вектора $\e_1$ и $\e_2$. Мы будем предполагать, что эти векторы не лежат на одной прямой, т.е. неколлинеарны. В этом случае любой другой вектор $v$ можно спроецировать на оси векторов $\e_1$ и $\e_2$, и данные две проекции выразить числами, считая векторы $\e_1$ и $\e_2$ условными единицами каждый на своей оси. Таким образом, вектору $v$ будет сопоставлена пара чисел $(x,y)$, где $x=\Pr_{\e_1}v$ и $y=\Pr_{\e_2}v$. Поскольку векторы $\e_1$ и $\e_2$ не лежат на одной прямой, т.е. линейно независимы, такое сопоставление вектора $v\leftrightarrow(x,y)$ является взаимно однозначным: вектор $v$ легко восстанавливается с помощью этих чисел по правилу параллелограмма
$$
v= x\e_1+y\e_2.
$$
\item Набор векторов $\{\e_1,\e_2\}$ на плоскости образует базис векторного пространства $V$, поскольку любой вектор однозначно представляется в виде такой линейной комбинации векторов $\e_1$ и $\e_2$, и в то же время векторы $\e_1$ и $\e_2$ независимы.
\item Если мы теперь вспомним про один из способов задания вектора, а именно, через пару координат на координатной плоскости, то поймем, что это частный случай разложения вектора по базису. Именно, пусть вектор задан парой $(x,y)$, где числа $x$ и $y$ суть его координаты, отложенные по осям $Ox$ и $Oy$. Тогда
$$
(x,y) = x(1,0) + y(0,1),
$$
т.е. в данном случае векторы $(1,0)$ и $(0,1)$ являются базисом, в котором наш вектор имеет представление $(x,y)$.
\item Рассмотрим произвольный линейный оператор $L:V\to V$. В силу свойств линейности для произвольного вектора $v=x\e_1+y\e_2$ имеем
$$
L(v)=x w_1+yw_2,\mbox{ где }w_1=L(\e_1), w_2=L(\e_2),
$$
т.е. образ вектора $v$ точно так же раскладывается по векторам $w_1$ и $w_2$, которые являются образами базисных векторов $\e_1$ и $\e_2$.
\item Заметим, что векторы $w_1$ и $w_2$ могут оказаться коллинеарными, а значит, и все линейные комбинации $xw_1+yw_2$ будут коллинеарны этим двум векторам. В этом случае отображение $L$ схлопнет исходную плоскость в одну прямую, на которой лежат эти векторы, и окажется, что отображение $L$ не является биективным, т.е. не будет преобразованием плоскости. Собственно, в этом и кроется отличие линейных преобразований от произвольных линейных отображений.
\item Поскольку векторы $w_1$ и $w_2$ лежат в том же самом пространстве $V$, их тоже можно разложить по базису $\e_1,\e_2$. Пусть
$$
w_1=w_{11}\e_1+w_{12}\e_2,\quad w_2=w_{21}\e_1+w_{22}\e_2.
$$
Составим из этих векторов квадратную матрицу $2\times 2$:
$$
W = \begin{pmatrix}
w_{11} & w_{21} \\ w_{21} & w_{22}
\end{pmatrix}
$$

\item Матрица $W$ называется \textbf{матрицей линейного оператора} $L$.\index{Матрица!линейного оператора}\index{Линейный оператор!матрица} Эта матрица составлена из векторов-образов базиса путем выписывания в столбик координат каждого вектора, в которые перешли базисные векторы под действием оператора $L$.
\item Посмотрим, как будут выглядеть координаты вектора $L(v)$ в исходном базисе:
\begin{align*}
L(v)= & xw_1+yw_2 = x(w_{11}\e_1+w_{12}\e_2)+y(w_{21}\e_1+w_{22}\e_2)= \\
= & (xw_{11}+yw_{21})\e_1+(xw_{21}+yw_{22})\e_2
\end{align*}
Эти равенства задают умножение матрицы на вектор:
$$
\begin{pmatrix}
w_{11} & w_{21} \\ w_{21} & w_{22}
\end{pmatrix}
\begin{pmatrix}
x  \\ y
\end{pmatrix} =
\begin{pmatrix}
xw_{11}+yw_{21}  \\ xw_{21}+yw_{22}
\end{pmatrix}
$$

\item Приведем пример. Пусть $\e_1$ и $\e_2$ --- стандратный базис, т.е. векторы $(1,0)$ и $(0,1)$ на ортогональной координатной сетке. Пусть нам нужно сделать поворот плоскости на угол $\ph$ относительно начала координат. В этом случае векторы $w_1$ и $w_2$ будут образами базисных векторов при таком повороте. Но тогда проекцией $w_1$ на $\e_1$ будет $\cos\ph$, проекцией $w_1$ на $\e_2$ будет $\sin\ph$, проекцией $w_2$ на $\e_1$ будет уже $\sin(\pi/2-\ph)=-\sin\ph$ и проекцией $w_2$ на $\e_2$ будет $\cos\ph$. Таким образом, матрицей поворота $R_O^\ph$ будет матрица
$$
R_\ph = \begin{pmatrix}
\cos\ph & -\sin\ph \\ \sin\ph & \cos\ph
\end{pmatrix}
$$
\item Матрица
$$
\begin{pmatrix}
\la & 0 \\ 0 & 1
\end{pmatrix}
$$
задает, как легко видеть, растяжение вдоль оси $Ox$, поскольку
$$
\begin{pmatrix}
\la & 0 \\ 0 & 1
\end{pmatrix}
\begin{pmatrix}
x  \\ y
\end{pmatrix} =
\begin{pmatrix}
\la x  \\ y
\end{pmatrix}
$$

\textbf{ВНИМАНИЕ!} Это линейное преобразование, которое не является ни движением, ни гомотетией! Это пример такого линейного преобразования, которые мы до сих пор не встречали!


\item Немного модифицируем предыдущую матрицу
$$
H_\la = \begin{pmatrix}
\la & 0 \\ 0 & \la
\end{pmatrix}
$$
и уже видим, что она вектор $(x,y)$ переводит в вектор $(\la x,\la y)$, т.е. осуществляет гомотетию с центром в начале коорддинат и коэффициентом $\la$.

\item Наконец, еще один пример:
$$
S_x = \begin{pmatrix}
1 & 0 \\ 0 & -1
\end{pmatrix}
$$
Данная матрица соответствует оператору, который переводит вектор $(x,y)$ в вектор $(x,-y)$, т.е. осуществляет симметрию относительно оси $Ox$.

\item Таким образом, мы нашли матрицы линейных преобразований, которые являются кирпичиками, из которых складывается любое подобие. Осталось научиться строить их композиции.

\end{enumerate}


\section{Арифметика матриц}

\lesson{Свойства сложения и умножения матриц $2\times 2$ и $3\times 3$}

\begin{enumerate}
\item Выше было установлено, как умножать матрицу на вектор. На самом деле, вектор --- это тоже матрица, только c одним столбцом, поэтому отчасти мы уже знаем, как умножать матрицу на матрицу. Выведем свойства арифметики матриц из свойств соответствующих им линейных операторов. Для простоты мы рассмотрим случай, когда базисные векторы $\e_1$ и $\e_2$ перпендикулярны и задают обычную координатную сетку, т.е. $\e_1=(1,0)$ и $\e_2=(0,1)$. Но на самом деле арифметика матриц основана исключительно на арифметике операторов, и потому не зависит от того, в каком базисе мы их изучаем.
\item Сложение матриц производится покомпонентно (так же, как векторов). Пусть даны два оператора $L$ и $M$, и им соответствующие матрицы $W$ и $U$. Как мы уже выяснили, эти матрицы составлены из векторов-столбиков, причем эти векторы есть представление образов базисных векторов в исходном базисе. То есть, если $L\e_1=w_1$ и $L\e_2=w_2$, то матрица $W=[w_1;w_2]$. Такая запись означает, что мы берем столбики $w_1=\binom{w_{11}}{w_{12}}$ и $w_2=\binom{w_{21}}{w_{22}}$ и ставим их слева направо, образуя квадратную матрицу $W$.

Пусть также $M\e_1=u_1$ и $M\e_2=u_2$, тогда $U=[u_1;u_2]$.

Пользуясь тем, что сумма операторов $L+M$ определяется как сумма образов $(L+M)v=Lv+Mv$, заключаем, что 
$$
(L+M)\e_1=w_1+u_1,\quad (L+M)\e_2=w_2+u_2,
$$
откуда следует, что матрицей оператора $L+M$ будет
$$
W+U = [w_1+u_1;w_2+u_2],
$$
т.е. сложение матриц производится покомпонентно, как и векторов.

\item Заметим, что если мы перейдем в трехмерное пространство, то вся механика останется ровно такой же. Просто вместо двух базовых векторов будет 3, а матрица будет иметь размер $3\times 3$.
\item Посмотрим теперь на матрицу композиции операторов $LM$. В этом случае мы должны разложить по исходному базису векторы $L(M\e_1)$ и $L(M\e_2)$ и составить из них матрицу. Но также мы можем выразить это разложение через промежуточные векторы $u_1$ и $u_2$:
\begin{align*}
L(M\e_1) & = L(u_{11}\e_1+u_{12}\e_2) = u_{11}L(\e_1)+u_{12}L(\e_2) = \\
& = u_{11}(w_{11}\e_1+w_{12}\e_2) + u_{12}(w_{21}\e_1+w_{22}\e_2) = \\
& = (u_{11}w_{11}+u_{12}w_{21})\e_1 + (u_{11}w_{12}+u_{12}w_{22})\e_2,
\end{align*}
\begin{align*}
L(M\e_2) & = L(u_{21}\e_1+u_{22}\e_2) = u_{21}L(\e_1)+u_{22}L(\e_2) = \\
& = u_{21}(w_{11}\e_1+w_{12}\e_2) + u_{22}(w_{21}\e_1+w_{22}\e_2) = \\
& = (u_{21}w_{11}+u_{22}w_{21})\e_1 + (u_{21}w_{12}+u_{22}w_{22})\e_2,
\end{align*}
откуда
$$
\begin{pmatrix}
w_{11} & w_{21} \\ w_{12} & w_{22}
\end{pmatrix}
\begin{pmatrix}
u_{11} & u_{21} \\ u_{12} & u_{22}
\end{pmatrix}
=
\begin{pmatrix}
u_{11}w_{11} + u_{12}w_{21} & u_{21}w_{11} + u_{22}w_{21} \\ 
u_{11}w_{12} + u_{12}w_{22} & u_{21}w_{12} + u_{22}w_{22}
\end{pmatrix}
$$


\item Определим скалярное произведение векторов $x=(x_1,x_2)$ и $y=(y_1,y_2)$, заданных координатами в базисе $\e_1,\e_2$ по правилу
\begin{equation}\label{scal}
x\cdot y = x_1y_1+x_2y_2.
\end{equation}
Подчеркнем, что такое представление скалярного произведения векторов справедливо только в стандартном базисе $(1,0), (0,1)$ или базисе, который получается из стандартного движением. В общем случае скалярное произведение вводится как функция от двух векторов со следующими свойствами:
\begin{enumerate}[SP1]
\item $x\cdot y=y\cdot x$;
\item $(x+y)\cdot z = x\cdot z + y\cdot z$;
\item $(\la x)\cdot y = \la(x\cdot y)$;
\item $x\cdot x\ge 0$, $x\cdot x=0\Leftrightarrow x=0$.
\end{enumerate}
Можно проверить, что скалярное произведение, заданное по формуле \eqref{scal} через координаты векторов в стандартном базисе, удовлетворяет данным требованиям.
\item Для наших целей координатная запись скалярного произведения понадобится, чтобы упростить запись произведения матриц: ровно по такому правилу получаются элементы произведения, если представить, что мы скалярно унможаем строки первой матрицы на столбцы второй
$$
\begin{pmatrix}
w_{11} & w_{21} \\ w_{12} & w_{22}
\end{pmatrix}
\begin{pmatrix}
u_{11} & u_{21} \\ u_{12} & u_{22}
\end{pmatrix}
=
\begin{pmatrix}
w_{\bullet 1}\cdot u_{1\bullet} & w_{\bullet 1}\cdot u_{2\bullet} \\ 
w_{\bullet 2}\cdot u_{1\bullet} & w_{\bullet 2}\cdot u_{2\bullet}
\end{pmatrix},
$$
где значок $\bullet$ вместо индекса означает, что мы производим суммирование по всем значениям данного индекса.

\item Наконец, умножение матрицы на число соответствует умножению оператора на то же число. Но если оператору $L$ соответствует матрица $W$, то оператору $kL$, очевидно, соответствует матрица $kW$, где все элементы матрицы необходимо умножить на число $k$, тогда и результирующий вектор $kLv$ будет в $k$ раз больше вектора $Lv$.

\item В трехмерном пространстве все происходит точно так же. Базисом будут три вектора $\e_1,\e_2,\e_3$, каждый вектор можно разложить по базису и представить в координатной форме тройкой чисел, например, $u=u_1\e_1+u_2\e_2+u_3\e_3$. Соответственно, матрица линейного оператора в трехмерном пространстве будет составлена из трех векторов-столбиков, у каждого по три координаты. Сложение и умножение матриц будет подчинено ровно тем же самым правилам: при сложении нужно складывать соответствующие элементы матриц, а при умножении --- скалярно умножать строки первой матрицы на столбцы второй. Умножение на число также осуществляется покомпонентно.

\item Ранее мы нашли матрицы поворота, гомотетии и отражения. Однако существует еще один виде подобий плоскости, сохраняющий точку. Это --- поворотная гомотетия. Поскольку такое подобие является композицией поворота и гомотетии, для получения его матрицы нужно умножить матрицы поворота и гомотетии:
$$
R_\ph H_\la = 
\begin{pmatrix}
\cos\ph & -\sin\ph \\ \sin\ph & \cos\ph
\end{pmatrix}
\begin{pmatrix}
\la & 0 \\ 0 & \la
\end{pmatrix} =
\begin{pmatrix}
\la\cos\ph & -\la\sin\ph \\ \la\sin\ph & \la\cos\ph
\end{pmatrix}
$$


\lesson{Определитель. Группа обратимых матриц, ее изоморфность группе обратимых линейных операторов}

\item Итак, мы видим, что на множестве всех квадратных матриц можно задать точно такие же операции, как и на операторах, причем между матрицами и операторами существует взаимно однозначное соответствие, если мы зафиксировали неокторый базис векторного пространства. В другом базисе матрицы операторов будут, вообще говоря, иметь другой набор чисел в своих ячейках. Но такое соответствие между операторами и матрицами говорит нам о том, что вся арифметика операторов полностью воспроизводится в матрицах, что мы и видим в предыдущих вычислениях. Более того, поскольку композиция операторов ассоциативна, а композиции соответствует произведени матриц, то и произведение матриц также ассоциативно.

Кроме того, существует единичная матрица, соответствующая оператору $\id$. В двумерном случае она выглядит так:
$$
\E_2=\begin{pmatrix}
1 & 0 \\ 0 & 1
\end{pmatrix}.
$$
На самом деле, для любого числа измерений пространства матрица оператора $\id$ будет иметь такой вид: на главной диагонали матрицы стоят единицы (того поля или кольца, над которым заданы векторы), а на всех остальных местах нули (того же поля или кольца). Причем, вид матрицы оператора $\id$ не зависит от выбора базиса!

\item Таким образом, квадратные матрицы образуют моноид по умножению. Существуют ли обратные элементы к матрицам? ответ напрямую связан с обратимостью соответствующих линейных операторов.

\item Если задан обратимый линейный оператор $L$, то существует обратная функция $L^{-1}$, которая также является линеныйм оператором. А значит, существует матрица обратного линейного оператора, которая при умножении на матрицу исходного оператора даст единичную матрицу.

Пусть дана матрица $W$, найдем ей обратную. Обратная матрица $U$ должна удовлетворять уравнению $WU=UW=\E$:
$$
\begin{pmatrix}
w_{11} & w_{21} \\ w_{12} & w_{22}
\end{pmatrix}
\begin{pmatrix}
u_{11} & u_{21} \\ u_{12} & u_{22}
\end{pmatrix}
=
\begin{pmatrix}
u_{11}w_{11} + u_{12}w_{21} & u_{21}w_{11} + u_{22}w_{21} \\ 
u_{11}w_{12} + u_{12}w_{22} & u_{21}w_{12} + u_{22}w_{22}
\end{pmatrix}
=\begin{pmatrix}
1 & 0 \\ 0 & 1
\end{pmatrix}.
$$

Можно решать эти 4 уравнения, а можно заметить, что
$$
\begin{pmatrix}
w_{11} & w_{21} \\ w_{12} & w_{22}
\end{pmatrix}
\begin{pmatrix}
w_{22} & -w_{21} \\ -w_{12} & w_{11}
\end{pmatrix}
=
\begin{pmatrix}
w_{11}w_{22} - w_{12}w_{21} & 0 \\ 
0 & w_{11}w_{22} - w_{12}w_{21},
\end{pmatrix}.
$$
В любом случае, находим, что для получения единицы требуется разделить вторую матрицу на число $w_{11}w_{22} - w_{12}w_{21}$, и мы получим, что
$$
W^{-1} = 
\begin{pmatrix}
\displaystyle\frac{w_{22}}{w_{11}w_{22} - w_{12}w_{21}} & \displaystyle\frac{-w_{21}}{w_{11}w_{22} - w_{12}w_{21}} \\[7pt]
\displaystyle\frac{-w_{12}}{w_{11}w_{22} - w_{12}w_{21}} & \displaystyle\frac{w_{11}}{w_{11}w_{22} - w_{12}w_{21}}
\end{pmatrix}
$$

Отсюда видно, что матрица $W$ обратима тогда и только тогда, когда число $w_{11}w_{22} - w_{12}w_{21}$ отлично от нуля.

\item С числом $w_{11}w_{22} - w_{12}w_{21}$ очень многое связано в линейной алгебре! Это число называется \textbf{определителем матрицы} $W$ и обозначается обычно $|W|$ или $\det W$.\index{Определитель матрицы}

При изучении перестановок мы уже приводили в пример вычисления определителя второго и третьего порядка с помощью перестановок, а также заметили, что определитель является ориентированной площадью параллелограмма (в двумерном случае) или ориентированным объемом параллелепипеда (в трехмерном случае), построенного на векторах-столбиках матрицы.

\begin{center}
\includegraphics[scale=0.3]{paralel.png}
\end{center}

Ориентированная площадь параллелограмма равна его геометрической площади, взятой со знаком плюс или минус в зависимости от расположения векторов. Так, если мы ищем площадь параллелограмма, построенного на паре векторов $(u,v)$ (см. картинку), то если данная площадь заметается при повороте от вектора $u$ к вектору $v$ против часовой стрелки, т.е. в положительном направлении, то и знак у площади положительный. Если же от первого ко второму вектору через площадь данного параллелограмма нужно идти в обратном направлении, то знак у площади будет отрицательный. Поэтому знак площади зависит от порядка, в котором указаны векторы:
$$
S(u,v) = -S(v,u).
$$
Кроме того, если вектор $u$ будет представлен как сумма векторов $u_1+u_2$, то ориентированная площадь будет равна сумме соответствующих площадей:
$$
S(u_1+u_2,v)=S(u_1,v)+S(u_2,v).
$$
В силу предвдущего свойства это верно и для суммы по второму вектору $v$.

Наконец, если вектор $u$ домножить на число $k$, то и ориентированная площадь умножится на него же. При этом, если $k<0$, то площадь сменит ориентацию. Таким образом,
$$
S(ku,v) = kS(u,v).
$$

Замечаем интересную вещь: площадь, как функция от двух аргементов-векторов по каждому из своих аргументов ведет себя линейно, т.к. выполняется свойство аддитивности и однородности. Точно так же ведет себя другая функция от двух векторов: скалярное произведение. Но на этом их сходство и заканчивается.

Свойство смены знака при перестановке аргументов местами называется \textit{кососимметричностью}. Скалярного произведения, как мы помним, симметрично (на самом деле, это не совсем так, ибо в комплексном случае от него требуется комплексное сопряжение при перестановке векторов местами, но все равно это не появление минуса)

Можно проверить, что и определитель ведет себя точно так же.

На самом деле, все функции от двух векторов, обладащие свойствами кососимметричности и линейности, отличаются друг от друга лишь коэффициентом, а если потребовать, чтобы такая функция на определенной паре векторов принимала какое-то определенное значение, то такая фнукция и вовсе будет единственной!
\begin{thrm}
Функция $f(W)$ от квадратной матрицы $W=[u,v]$, удовлетворяющая свойствам косисмметричности и линейности, и такая, что
$$
f\begin{pmatrix}
1 & 0 \\ 0 & 1
\end{pmatrix}=1,
$$
существует и единственна.
\end{thrm}

Из этой теоремы, в частности, следует равенство функций
$$
\det W = w_{11}w_{22} - w_{12}w_{21} = S(w_1,w_2),
$$
т.е. определитель и ориентированная площадь --- это одна и та же функция. Такой же результат имеет место в трехмерном случае.

\item Приведем без доказательства следующие общие свойства определителя:
\begin{enumerate}[{\bf Det}1]
\item $\det E=1$;
\item $\det(WU)=\det(W)\det(U)$;
\item $\det(W^{-1}) = (\det W)^{-1}$;
\item определитель равен нулю тогда и только тогда, когда столбцы (и строки) матрицы линейно зависимы;
\item определитель матрицы линейного оператора не зависит от выбора базиса пространства.
\end{enumerate}

Последнее свойство говорит о том, что определитель является характеристикой самого линейного оператора. Это свойство легко понять, если исходить из концепции определителя как ориентированной площади. Действительно, если линейный оператор переводит исходный квадрат $1\times 1$ в какой-то параллелограмм, то неважно, в каких координатах мы рассматриваем действие такого оператора - параллелограмм останется тем же самым, и его площадь вместе с ориентацией --- тоже.

\item Итак, подведем итог. При выборе базиса пространства (плоскости) каждому линейному оператору соответствует единственная квадратная матрица, и каждой квадратной матрице (при данном базисе) соответствует единственный линейный оператор. Операции на матрицах в точности соответствуют операциям на операторах, т.е. мы имеем изоморфизм между операторами и квадратными матрицами. 
И мы приходим к тому, что алгебра линейных операторов изоморфна алгебре квадратных матриц.

Более того, оператор обратим тогда и только тогда, когда его матрица имеет определитель, отличный от нуля. Оператору $\id$ соответствует единичная матрица, обратному оператору соответствует обратная матрица. Соответственно, полной линейной группе $\GL(V)$ обратимых линейных операторов над пространством векторов $V$ изоморфна группа обратимых квадратных матриц, размерность которых равна размерности пространства $V$. Такая группа матриц обозначается $\GL(n)$ для $n$-мерного пространства.
\end{enumerate}


\section{Матрицы и комплексные числа}


\lesson{Представление комплексных чисел матрицами $2\times 2$. Связь подобий с арифметикой комплексных чисел}


\begin{enumerate}
\item Ранее мы уже показали, что комплексные числа позволяют выразить движения плоскости. А именно, функция $T(z)=z+z_0$ есть параллельный перенос на плоскости $\C$, функция $R(z) = zz_0/|z_0|$ есть поворот относительно нуля, функция $S(z)=\bar z$ есть отражение относительно вещественной оси. За кадром остался вопрос о том, что же получится, если домножить на произвольное комплексное число. Теперь уже должно быть очевидно, что это будет поворотная гомотетия с центром в нуле. Чуть ниже мы это докажем.
\item Мы также нашли, что тригонометрические функции появляются как компоненты комплексного числа с единичной окружности, а именно:
$$
\Re z = \cos\ph,\quad \Im z=\sin\ph.
$$
\item С другой стороны, мы знаем вид матрицы поворота:
$$
R_\ph = \begin{pmatrix}
\cos\ph & -\sin\ph \\ \sin\ph & \cos\ph
\end{pmatrix},
$$
что очень похоже на запись комплексного числа, олицетворяющего поворот плоскости относительно нуля.
\item На самом деле, умножение на всякое комплексное число $z_0$ можно интерпретировать как действие линейного оператора, поскольку
$$
(\la z+\mu z')z_0 = \la zz_0 + \mu z'z_0,\quad \la,\mu\in\R,
$$
т.е. подчиняется аксиомам линейности.
\item Но если $z_0=x+iy$ понимать как линейный оператор, то легко найти его матрицу в стандартном базисе, проследив за тем, куда переходят едичниные базисные векторы: $\bar 1=(1,0)$ и $\bar i=(0,1)$. Первый переходит в $(x,y)$, а второй, в соответствии с арифметикой комплексных чисел, --- в вектор $(-y,x)$. Следовательно, матрицей линейного оператора, соответствующего умножению на число $z_0=x+iy$, является двухпараметрическая матрица
$$
\begin{pmatrix}
x & -y \\ y & x
\end{pmatrix}
$$
\item Легко проверить, что алгебра таких матриц в точности соответствует алгебре комплексных чисел: сложение чисел переходит в сложение матриц, умножение чисел --- в умножение матриц, кроме того, комплексное сопряжение соответствует транспонированию матрицы
$$
\begin{pmatrix}
x & -y \\ y & x
\end{pmatrix}^T =
\begin{pmatrix}
x & y \\ -y & x
\end{pmatrix},
$$
а определитель матрицы соответствует норме комплексного числа:
$$
\det \begin{pmatrix}
x & -y \\ y & x
\end{pmatrix} = 
x^2+y^2.
$$

\item Осталось вспомнить, что матрицей поворотной гомотетии с центром в нуле и коэффициентом $\la$ является
$$
\begin{pmatrix}
\la\cos\ph & -\la\sin\ph \\ \la\sin\ph & \la\cos\ph
\end{pmatrix},
$$
что соответствует умножению на комплексное число $\la\cos\ph + i\la\sin\ph$. Через некоторое время мы найдем иную запись такого выражения: $\la e^{i\ph}$.
\item Итак, в терминах комплексных функций получается, что всякое подобие есть линейная функция от $z$, либо ее сопряжение. Обратное также верно.\index{Теорема!о подобиях в $\C$}
\begin{thrm} Преобразование $z\mapsto w$ является преобразованием подобия тогда и только тогда, когда оно имеет вид:
$$
w = az+b,\mbox{ либо }w = a\bar z+b,
$$
где $a,b$ --- произвольные комплексные числа, $|a|>0$.
\end{thrm}

\item Стоит отметить также, что преобразование вида $w=az+b\bar z+c$, где $|a|\ne|b|$, доставляет так называемое \textbf{аффинное преобразование}, при котором параллельные прямые переходят в параллельные прямые, пересекающиеся --- в пересекающиеся.

\item Наконец, пробразование вида $w=(az+b)/(cz+d)$, где $ad\ne bc$, доставляет дробно-линейное преобразование, основной инструмент проективной геометрии. При дробно-линейном преобразовании окружности и прямые переходят в окружности и прямые (в том числе, окружность в прямую, и наоборот).

\end{enumerate}





\section{Действие линейных отображений на векторном пространстве}

\lesson{Ядро и образ оператора, связь с определителем и рангом матрицы}

\begin{enumerate}
\item Поскольку линейное отображение сохраняет операцию сложения, оно является гомоморфизмом векторного пространства $V$ на себя. А поскольку $V$ есть группа по сложению, то естественно вспомнить о том, что ядро гомоморфизма является нормальной подгруппой. Это значит, что для любого линейного оператора $L:V\to V$ множество
$$
\Ker L = \{v\in V\mid Lv=0\}
$$
есть замкнутое по операции сложения векторов, а кроме того, пространство $V$ можно представить как сумму попарно непересекающихся множеств вида $\Ker(L)+w$, где $w$ --- произвольный вектор пространства $V$.

Кроме того, ядро линейного оператора замкнуто и относительно операции умножения на число, т.к. если $Lv=0$, то $L(kv)=0$, и наоборот. Иначе говоря, ядро линейного оператора само по себе является векторным пространством внутри пространства $V$. Соответственно, у него есть базис, по количеству векторов не превосходящий базис всего пространства $V$. Таким образом, $\Ker L$ является векторным подпространством пространства $V$, и его размерность не превосходит размерность $V$. Если размерность ядра равна размерности $V$, то $\Ker L=V$.

Смещенное на вектор $w$ подпространство называется \textbf{линейным многообразием}.\index{Линейное многообразие}


\item Интересно, что в случае линейного оператора не только его ядро, но и образ $L[V]$ является линейным подпространством пространства $V$, т.к. если $u,v\in L[V]$, то любая их линейная комбинация также принадлежит $L[V]$. Кроме того, если векторы $u,v\notin\Ker L$ и при этом линейно независимы, то и векторы $Lu$ и $Lv$ линейно независимы. Это значит, что размерность $L[V]$ такова, сколько линейно независимых векторов найдется вне подпространства $\Ker L$.
\item Обозначим за $\dim V$ размерность пространства $V$. Тогда имеет место равенство
$$
\dim \Ker L + \dim L[V] = \dim V.
$$
В частности, отсюда следует, что $L[V]=V$ тогда и только тогда, когда ядро линейного оператора тривиально, т.е. равно $\{0\}$, а это эквивалентно тому, что $L$ является обратимым линейным оператором.

\item Например, если линейный оператор действует в трехмерном пространстве и переводит плоскость в ноль, то его образом будет прямая. Простейший пример: проекция на прямую $\Pr(x,y,z) = x$. 
\item Заметим далее, что поскольку матрица линейного оператора строится из векторов-столбцов, которые являются образами базисных векторов, то эти столбцы линейно независимы тогда и только тогда, когда размерность образа равна $\dim V$, т.е. когда оператор $L$ обратим. С другой стороны, независимость столбцов матрицы эквивалентна тому, что ее определитель отличен от нуля.
\begin{thrm} Следующие утверждения о линейном операторе $L:V\to V$ эквиваленты:\index{Теорема!критерий обратимости оператора}
\begin{enumerate}[\textup{(1)}]
\item[\textup{(1)}] линейный оператор $L$ обратим;
\item[\textup{(2)}] матрица линейного оператора обратима;
\item[\textup{(3)}] определитель матрицы линейного оператора отличен от нуля;
\item[\textup{(4)}] размерность ядря $L$ равна 0;
\item[\textup{(5)}] размерность образа $L[V]$ равна $\dim V$;
\item[\textup{(6)}] $L[V]=V$.
\end{enumerate}
\end{thrm}
\item Число линейно независимых столбцов матрицы (оно же --- размерность образа соответствующего линейного оператора) называется \textbf{рангом матрицы}.\index{Матрица!ранг} Если $W$ --- матрица линейного оператора $L$, то $\rank W=\dim L[V]$. Определитель матрицы равен нулю тогда и только тогда, когда ее ранг меньше ее размерности.

Если оператор вырожденный, т.е. $Lv=0$ для любого вектора $v$, то ранг матрицы такого оператора равен 0, а значит, даже никакой один столбец матрицы не является линейно независимым, а такое возможно только в том случае, когда этот вектор нулевой. Следовательно, матрица вырожденного оператора состоит из одних нулей.
\end{enumerate}




\newchapter{Алгебраические числа}

\vrezka{
В этой главе мы заглянем за пределы поля рациональных чисел при помощи многочленов с рациональными коэффициентами, построим поле алгебраических чисел, разберем некоторые теоремы о многочленах над кольцами и полями.
}

\section{*Упорядоченные множества}\label{Ordering}

\lesson{Определение линейного порядка. Упорядочение $\N,\Z,\Q$. Интервал}

\begin{enumerate}
\hard{
\item Ранее мы определяли \textbf{отношение на множестве} $A$ как произвольное подмножество $R\subseteq A\times A$.\index{Отношение}
\item Рассмотрим здесь частный случай отношений: \textbf{отношения порядка}.
\item Вспоминая изучение движений прямой, скажем, что точка $X$ на прямой \textbf{больше}, чем точка $Y$, если $X$ находится правее, чем $Y$. Иначе это можно сформулировать так: если $X=T_a(Y)$, где $a$ --- вектор движения вправо, т.е. положительный вектор. То есть, $X>Y$ (или $Y<X$), если $X$ получается смещением $Y$ в положительном направлении.
\item Симметрично рассуждая, получаем, что $X<Y$, если $X$ находится левее $Y$, либо если из точки $Y$ можно попасть в $X$ сдвигом влево.
\item Пользуясь этим наглядным представлением, легко получить следующие свойства сравнения:
\begin{enumerate}[{\bf Rel1}]
\item если $X<Y$ и $Y<Z$, то $X<Z$ (транзитивность отношения $<$);\index{Отношение!транзитивное}

Это достаточно очевидно, поскольку сдвиг $T_{XZ}$ есть композиция положительных сдвигов $T_{XY}$ и $T_{YZ}$.
\item для любой точки $X$ не верно, что $X<X$ (антирефлексивность);\index{Отношение!антирефлексивное}

Это также довольно-таки очевидно, т.к. сдвиг, сотавляющий на месте $X$, является $\id$, а не положительным сдвигом.
\item для любых точек $X,Y$ имеет место одно из трех отношений: $X<Y$ или $X=Y$ или $X>Y$ (связность).\index{Отношение!связное}

Это следует из того, что если $X\ne Y$, то можно построить вектор $\vec{XY}$, а он может смотреть либо влево, либо в право, что и будет соответствовать оставшимся двум сравнениям. Заметим сразу, что для любой пары точек всегда выполняется только одно из трех отношений, поскольку равенство $X=Y$ исключает неравенства $X<Y$ и $Y<X$ в силу свойства антирефлексивности, а неравенство $X<Y$ исключает неравенство $Y<X$, т.к. иначе по свойству транзитивности мы бы получили $X<X$, что противоречит антирефлексивности.
\end{enumerate}
\item Транзитивное антирефлексивное связное отношение на множестве называется \textbf{линейным порядком}, а само множество, на котором задан линейный порядок, называется \textbf{линейно упорядоченным}.\index{Отношение!линейного порядка}\index{Множество!линейно упорядоченное}

\item Ранее мы уже определяли отношение эквивалентности как рефлексивное транзитивное симметричное отношение (см. раздел \ref{Rels}). Ниже представлена графическая схема того, из каких понятий формируются отношение эквивалентности и линейного порядка.\index{Отношение!эквивалентности}
\begin{center}
\includegraphics[scale=0.4]{RelOrder.png}
\end{center}

Обратим внимание на то, что отношение линейного порядка здесь представленов двух ипостасях: как антирефлексивное транзитивное и связное отношение, а также как рефлексивное антисимметричное транзитивное и связное оношение. Первый случай соответствует отношению строгого порядка ($<$), второй --- нестрогого ($\le$). Можно доказать, что это на самом деле одно и то же с точностью до исключения равенства.

\begin{lem} Отношение $<$ является отношением строгого линейного порядка тогда и только тогда, когда отношение $\le$ ($(x\le y)\Leftrightarrow (x<y)\lor (x=y)$) является отношением нестрогого линейного порядка.
\end{lem}

В дальнейшем под термином <<линейный порядок>> мы будем понимать именно строгий линейный порядок, а нестрогим будем делать его, используя знак $\le$ и ему подобные.}

\item Множества $\N$ и $\Z$ упорядочены естественным образом: числа, стоящие правее, больше, чем их левые собратья.
\item Множество $\Q$ мы, на самом деле, упорядочили еще во время его построения, когда говорили о наклоне прямой, задающей рациональное число: чем круче наклон, тем больше число. Но это верно только для положительных дробей. С отрицательными числами все наоборот (в полной аналогии с целыми числами!) --- чем круче наклон, тем меньше число.
\item Это же можно записать и более формально: если $b>0$ и $d>0$, то
$$
\frac{a}{b}<\frac{c}{d}\Leftrightarrow ad<bc
$$
\begin{center}
\includegraphics[scale=0.3]{ratio.png}
\end{center}
\item Если на множестве $L$ задан линейный порядок $<$, то \textbf{интервалом} $(x;y)$ называется множество всех точек множества $L$, лежащих между $x$ и $y$:\index{Интервал л.у.м.}
$$
(x;y) = \{z\in A\mid x<z<y\}
$$
В частности, если $x>y$, то интервал $(x;y)$ пуст. На множестве $\Z$ имеем следующие примеры:
$$
(0;1) = \emptyset,\quad (0;2) = \{1\}, \quad (0;n+1) = \{1,2,\dots, n\}.
$$


\lesson{Виды порядков: плотный линейный порядок, частичный порядок, непрерывный линейный порядок. Согласование линейного порядка с операциями в группе/кольце: линейно упорядоченная группа, упорядоченное кольцо/поле}


\item Линейный порядок $<$ называется \textbf{плотным}, если между любыми двумя элементами всегда есть третий: \index{Отношение!линейного порядка!плотное}
$$
\forall x,y\in L\;(x<y)\to\exists z\; (x<z<y),
$$
т.е. когда любой интервал $(x;y)$ непуст при условии, что $x<y$.
\item Линейно упорядоченное множество с плотным линейным порядком называется \textbf{плотным линейно упорядоченным множеством}.\index{Множество!линейно упорядоченное!плотное}
\item Подмножества линейно упорядоченного множества можно сравнивать. Пусть множество $L$ линейно упорядочено отношением $<$, тогда для его подмножеств $X,Y\subseteq L$ положим
$$
X<Y\Leftrightarrow \forall x\forall y\;(x\in X)\land(y\in Y)\to (x<y),
$$
т.е. когда все точки множества $X$ меньше всех точек множества $Y$.
\item Нетрудно проверить, что отношение $<$ на подмножествах $L$ транзитивно и антирефлексивно. Однако же легко привести пример, когда множества могут быть несравнимы между собой, например, в качестве  $X$ взять все четные числа, а в качестве $Y$ --- все нечетные числа. несмотря на то, что эти множества не пересекаются, невозможно сказать, что $X<Y$ или что $Y<X$, и уж тем более неверно $X=Y$. Таким образом, отношение $<$, определенное на подмножествах $L$ не является линейным порядком.
\item Про отношение, которое является транзитивным и антирефлексивным, говорят, что оно является \textbf{частичным порядком}.\index{Отношение!частичного порядка}
\item Если множество снабжено частичным порядком, то оно называется \textbf{частично упорядоченным множеством}.\index{Множество!частично упорядоченное}
\item Сравнения множеств, как и операции Минковского над ними, удобны тем, что сокращают длинные формальные выкладки, а кроме того, образно очень хорошо воспринимаются: одно множество меньше другого, если оно целиком лежит левее другого.

\item Пусть $(L,<)$ --- линейно упорядоченное множество и $A,B\subset L$, причем $A\ne\emptyset$, $B\ne\emptyset$, $A\cap B=\emptyset$, $A\cup B=L$, $A\le B$\footnote{Здесь и далее сравнение множеств означает сравнение их элементов с квантором всеобщности: $X\le Y$ ($X<Y$) означает, что $\forall x \in X\;\forall y\in Y:\; x\le y$ ($x<y$). То же относится к сравнению множества и элемента: $c\le Y$.}. Тогда пара $(A,B)$ называется \textbf{сечением}\index{Сечение л.у.м.} множества $L$, $A$ --- нижним классом сечения, $B$ --- верхним классом сечения.

\item Линейный порядок $<$ на множестве $L$ называется \textbf{непрерывным}\index{Отношение!линейного порядка!непрерывное} (множество $L$ с таким порядком непрерывно), если каково бы ни было его сечение, либо в нижнем классе сечения существует наибольший элемент, а в верхнем нет наименьшего, либо в верхнем классе существует наименьший элемент, а в нижнем нет наибольшего (такие сечения называются \textbf{дедекиндовыми}).\index{Сечение л.у.м.!дедекиндово} Такое свойство линейного порядка еще называется \textbf{аксиомой непрерывности} или аксиомой полноты.\index{Аксиома!непрерывности (полноты)}

\item Отметим, что плотное линейно упорядоченное множество не всегда непрерывно. Например, $\Q$ плотно, но не непрерывно (сечение для $\sqrt 2$ не является дедекиндовым).\footnote{ВНИМАНИЕ! Термин <<дедекиндово сечение>> в разных источниках определяется по-разному, хотя все эти определения эквивалентны. Часто в качестве сечения берется не пара множеств, а только нижний класс сечения.}

\item Завершим этот раздел соединением двух разнородных понятий: алгебраической структуры и отношения.
\item Пусть имеется числовая структура с операциями $+$ (сложение) и $\cdot$ (умножение), причем символ 0 обозначает в ней нейтральный элемент по сложению. Пусть также на этой структуре задано отношение $<$ линейного порядка. Говорят, что \textbf{отношение $<$ согласовано с операцией $+$}, если выполнено условие:
\begin{enumerate}[resume*]
\item Если $a\le b$, то $a+c\le b+c$ (и $c+a\le c+b$) для любых чисел $a,b,c$ из данной структуры.
\end{enumerate}
Говорят, что \textbf{отношение $<$ согласовано с операцией $\cdot$}, если выполнено условие:
\begin{enumerate}[resume*]
\item Если $a\ge 0$ и $b\ge 0$, то $ab\ge 0$.
\end{enumerate}

Соответственно, если на элементах группы $(G,+)$ задан линейный порядок $<$, согласованный с групповой операцией, то структура $(G,+,<)$ называется \textbf{линейно упорядоченной группой}.\index{Группа!линейно упорядоченная} Пример такой группы --- группа целых чисел по сложению с обычным порядком.

Если на элементах кольца $(K,+,\cdot)$ задан линейный порядок, согласованный с операциями кольца, то структура $(K,+,\cdot,<)$ называется \textbf{упорядоченным кольцом}.\index{Кольцо!упорядоченное} Пример такого кольца --- кольцо целых чисел с обычным порядком.

Наконец, если на элементах поля задан линейный порядок, согласованный с его операциями, то такая структура называется \textbf{упорядоченным полем}. Пример такого поля --- поле рациональных чисел с обычным порядком.\index{Поле!упорядоченное!непрерывное}

Безусловно, самым <<продвинутым>> для нас на текущий момент понятием является \textit{непрерывное упорядоченное поле}, т.е. поле с непрерывным линейным порядком, согласованным с операциями сложения и умножения. Именно построение такого уникального поля является основной целью нашего курса.
\end{enumerate}

\subsection*{Задачи}

\begin{enumerate}
\item Пусть множество $X$ не пусто. Верно ли, что $\emptyset<X$? Верно ли, что $X<\emptyset$? Верно ли, что $\emptyset<\emptyset$?
\item Каким отношением (антирефлексивным, транзитивным, связным) является отношение несобственного вложения множеств? $X$ есть несобственное подмножество $Y$ (обозначение: $X\subset Y$), если $X\subseteq Y$ и $X\ne Y$.
\item Каким отношением является отношение делимости $x|y$ на положительных целых числах?
\item Является ли всюду плотным множество всех десятично рациональных чисел, т.е. чисел вида $k/10^n$, где $k\in \Z$ и $n\in\N$?
\item Выпишите полный список аксиом упорядоченного поля.
\end{enumerate}


\section{Плотные множества}

\lesson{Определение плотного множества, всюду плотного множества. Множество двоично-рациональных чисел $\B$ всюду плотно в $\Q$ и вообще на числовой прямой}

\begin{enumerate}
\item Вернемся к анализу поля рациональных чисел $\Q$. Данное поле интересно тем, что какие бы два различных числа мы ни взяли, между ними всегда найдется третье рациональное число. Действительно, пусть есть две дроби $r=n/m$ и $q=t/s$, тогда их среднее арифметическое $(r+q)/2$ является рациональным числом и лежит строго между ними.
\item Таким образом, множество рациональных чисел с обычным отношением сравнения является плотным линейно упорядоченным множеством.
\item К определению понятия плотного множества существует и другой, топологический подход. Пусть $A\subseteq B$, где $B$ --- линейно упорядоченное множество. Говорят, что множество $A$ \textbf{плотно в} $B$, если любой непустой интервал множества $B$ содержит точки $A$. Иначе говоря, как бы мы ни старались выбрать как можно более маленький (но непустой) интервал множества $B$, в нем всегда будут сидеть и точки множества $A$.\index{Множество!плотное}
\item Можно взять, например, множество $\B\subseteq\Q$ всех двоичных рациональностей, т.е. дробей вида $k/2^n$, где $k\in\Z$,\index{Числа!двоично-рациональные} $n\in\N$. Такое множество, во-первых, является плотным, поскольку среднее арифметическое любых двух его представителей
$$
\left(\frac{k}{2^n}+\frac{l}{2^m}\right)/2 = \frac{k\cdot 2^{\max(n,m)-n}+l\cdot 2^{\max(n,m)-m}}{2^{\max(n,m)+1}}
$$
является двоично-рациональным числом.
\item А во-вторых, множество $\B$ плотно в $\Q$, поскольку каковы бы ни были два рациональных числа $r\ne q$, между ними найдется двоично-рациональное. Предлагаем доказать это самостоятельно в качестве упражнения.
\item На самом деле, это легко понять, если представить себе, как нужно наносить на числовую ось двоично-рациональные дроби. Сначала мы берем все целые числа, затем ровно между ними ставим все полуцелые (с шагом $1/2$), затем в оставшихся полуцелых интервалах отмечаем середины (получаем числа с шагом $1/4$), затем снова делим эти интервалы пополам (получаем шаг $1/8$), и т.д. Ясно, что чем больше шагов мы пройдем, тем мельче будет сетка двоичных рациональностей, тем точнее с их помощью можно приблизить произвольное рациональное число. А это и есть свойство быть плотным в $\Q$.
\item Более того, какую бы точку на прямой мы ни выбрали, ее можно сколь угодно точно приблизить с помощью точек множества $\B$. Действительно, пусть имеется точка $A$ на прямой. Снова начнем наносить сетку двоично-рациональных чисел. Сначала на расстоянии 1, и выберем тот отрезок, в котором эта точка сидит. Если она совпада с одной из его границ, то мы уже нашли ее приближение (абсолютно точное) точками множества $\B$. Если нет, разделим отрезок пополам и снова выберем ту его часть, в которой находится точка $A$. Снова, если она совпадает с границей отрезка, то мы нашли точное приближение, иначе продолжим процесс деления отрезков. С каждым шагом точность приближения точки $A$ будет удваиваться. Сначла она будет лучше чем 1, затем лучше, чем $1/2$, и т.д., на $n$-м шаге мы найдем точки множества $\B$ на расстонии менее $1/2^n$ от точки $A$. Так что, рано или поздно мы достигнем заданной нам точности.
\item Множество $\B$, обладающее способностью подбираться сколь угодно близко к произвольным точкам прямой, называется \textbf{всюду плотным множеством}. Ясно, что и множество $\Q$ всюду плотно, поскольку оно содержит в себе $\B$.\index{Множество!всюду плотное}
\item Более того, всякое множество, плотное в $\Q$, всюду плотно на прямой.
\end{enumerate}


\section{Зазоры между рациональными числами}

\lesson{Дырки в поле $\Q$ в смысле порядка. Сечение $\sqrt 2$. Другие дырки вида $\sqrt[k]{p^l}$, $p$ --- простое}


\begin{enumerate}
\item Ранее мы уже приводили пример числа, которое определяется уравнением в целых числах, но притом не является рациональным. Это число $\sqrt 2$, которое разрешает уравнение $x^2-2=0$.
\item Такое число как бы вставляет клин между рациональными числами, рассекая их на две части. Действительно, если мы посмотрим на два множества
$$
X = \{r\in\Q\mid r^2<2\mbox{ или }r<0\},\quad Y=\{r\in\Q\mid r^2>2\mbox{ и }r>0\},
$$
то можно заметить, что, во-первых, их объединение $X\cup Y$ равно $\Q$, т.к. они включают в себя все рациональные числа, во-вторых, что они не пересекаются: $X\cap Y=\emptyset$. Наконец, в-третьих, $X<Y$ в смысле сравнения множеств. Такие пары множеств называются дедекиндовыми сечениями, и к ним мы еще вернемся чуть позже.
\item Заметим еще одну особенность такого разбиения $\Q$: какой бы интервал $(x;y)$ с концами $x\in X$, $y\in Y$ мы ни взяли, он всегда не пуст, что является следствием плотности $\Q$. То есть мы можем сколь угодно близки подбираться к условной границе двух множеств $X$ и $Y$, но никогда не найдем крайние, т.е. соседствующие точки! В множестве $X$ нет максимума, а в множестве $Y$ нет минимума. А в интервале $(x;y)$ всегда найдется бесконечно много точек как из множества $X$, так и из мноежества $Y$.
\item Получается, что множество $\Q$ удается распилить на два луча, причем таким способом, что у них нет граничных точек!
\item И единственно возможным кандидатом на роль границы будет именно число $\sqrt 2$, которое, как мы уже выяснили, не является рациональным, т.е. не принадлежит $\Q$.
\item Стало быть между точками множества рациональных чисел есть дырки. Причем, их довольно много.
\item Возьмем, например, произвольное простое число $p$ и два целых положительных числа $k,l$, взаимно простых ($k\perp l$). И запишем уравнение 
\begin{equation}\label{xkpl}
x^k-p^l=0.
\end{equation}

Это --- уравнение в целых числах. Но может ли оно иметь рациональный корень?
\item Предположим, что существует рациональное число $x=n/m$ ($n\perp m$), которое разрешает данное уравнение.
Тогда
$$
n^k=p^lm^k,
$$
откуда в силу основной теоремы арифметики следует, что простое число $p$ присутствует в разложении $n$ по степеням простых. Пусть оно входит в разложение $n$ со степенью $t\ge 1$, т.е. $n=p^ts$, где $s$ не делится на $p$.

Отсюда следует, что $n^k=p^{kt}s^k=p^lm^k$. но при этом, поскольку $n\perp m$, $m$ не делится на $p$, значит, вся степень $p^{kt}$ совпадает со степенью $p^l$, т.е. $kt=l$, т.е. $l$ делится на $k$.

По условию $\gcd(k,l)=1$, значит, $k=1$. Отсюда следует, что при указанных условиях корень уравнения \eqref{xkpl} будет рациональным числом тогда и только тогда, когда $k=1$, т.е. уравнение имеет вид $x-p^l=0$.
\item Как только мы берем $k=2,3,\dots$, уравнение \eqref{xkpl} становится неразрешимым в рациональных числах.
\item Между тем, как и в случае $\sqrt 2$, мы можем сколь угодно близко подбираться к положительному решению $x$, которое мы обозначим $\sqrt[k]{p^l}$, при помощи рациональных чисел. Это прямо следует из наших рассуждений о всюду плотности множества $\B$ и, как следствие, множества $\Q$.
\item Итак, мы видим, что <<дырки>> между рациональными числами --- явление нередкое. И точно так же, как мы расширяли $\Q$ до поля $\Q[\sqrt 2]$, мы можем строить любые расширения $\Q[\sqrt[k]{p^l}]$, почти всегда получая новые поля.
\index{Поле!конечное расширение}
\end{enumerate}



\section{О построениях циркулем и линейкой}

\lesson{Построение чисел циркулем и линейкой. Связь с расширениями полей. Размерности полей над $\Q$}


\begin{enumerate}
\item Одной из классических задач геометрии является изучение вопроса о том, какие геометрические построения можно произвести, имея только циркуль и линейку. Первый позволяет строить окружности заданного радиуса, вторая --- соединять любые две заданные точки и неограниченно продлевать отрезок за его границы. Все в точности с первыми тремя постулатами Евклида. При этом предполагается, что у нас есть некий мерный отрезок, задающий масштаб (единицу длины), а следовательно, вопрос о построениях циркулем и лнейкой сводится к умению строить отрезки различной длины или, по-просту, строить числа.

Числа при этом получаются как длины отрезков, соединяющих получаемые при построениях точки пересечения линий --- прямых и окружностей. Ясно, что такие пересечения могут давать только числа, являющиеся решениями линейных и квадратных уравнений. Иначе говоря, имея единицу, мы можем строить все рациональные числа (по теореме Фал\'eса), затем все рациональные комбинации различных корней из рациональных чисел, затем корней из этих корней и т.д. Речь идет, конечно же, о квадратных корнях.

В качестве упражнения предлагаем читателю самостоятельно построить циркулем и линейкой отрезки длины $\sqrt 2$ и $\sqrt 3$. А чтобы задача не казалась сложной, скажем, что юный Гаусс в начале XIX века построил таким способом правильный 17-угольник впервые в истории математики. Это построение любопытно посмотреть в динамике, например, \href{https://en.wikipedia.org/wiki/Constructible_polygon}{тут} (\verb|https://en.wikipedia.org| \verb|/wiki/Constructible_polygon|). Там же смотрите построение 15-, 257- и 65537-угольников.

\item Возникает вопрос: можно ли построить таким способом ребро куба, объем которого равен 2, т.\,е. число $\sqrt[3]{2}$ (это  античная задача об удвоении куба, известная наравне с задачей о квадратуре круга)?

\item Чтобы ответить на него, вспомним о линейных пространствах. Дело в том, что всякое расширение поля $\Q$ (и не только этого поля) с помощью присоединения каких-либо новых чисел (природа которых, как правило алгебраическая, т.е. они являются корнями алгебраических уравнений) можно рассматривать как линейное пространство над $\Q$. У этого пространства будет некоторая размерность. Например, $\Q[\sqrt 2]$ является пространством размерности 2 над полем $\Q$, базисными векторами в нем являются числа 1 и $\sqrt 2$ (они линейно независимы, т.к. $\sqrt 2$ --- иррациональное число).

\item Присоединяя к $\Q$ квадратные корни натуральных чисел, мы либо не увеличиваем поле, либо удваивем его размерность. Например, $\Q[\sqrt 2,\sqrt 3]$ имеет размерность 4 над $\Q$, а базисными векторами являются $1,\sqrt 2,\sqrt 3,\sqrt 6$. Доказательство этого факта мы оставим за рамкками курса. Главное для нас состоит в том, что присоединение корней дает только такие размерности расширений поля $\Q$, которые являются степенями двойки.

\item Следующий факт из высшей алгебры состоит в том, что последовательные расширения поля можно рассматривать как расширения друг над другом. Например, поле $\Q[\sqrt 2,\sqrt 3]$ является расширением вида $\Q[\sqrt 2][\sqrt 3]$, т.е. его можно рассматривать как расширения поля $\Q[\sqrt 2]$ путем присоединения числа $\sqrt 3$, или наоброт. В любом случае это будут расширения размерности 2.

\item Отсюда мы можем сделать следующий вывод: если какое-то число $\al$ содержится в каком-то поле $\Q[\be_1,\dots,\be_n]$, то это поле является конечным расширением над $\Q[\al]$, а значит, размерность поля $\Q[\be_1,\dots,\be_n]$ делится на размерность поля $\Q[\al]$. Напоминает делимость порядка группы на порядок подгруппы, не правда ли?

\item Теперь о $\sqrt[3]{2}$. Это число дает поле $\Q[\sqrt[3]{2}]=\{a+b\sqrt[3]{2}+c\sqrt[3]{4}\mid a,b,c\in\Q\}$, которое имеет размерность 3 над $\Q$. Проблема в том, что никакое чилсо $2^n$ не делится на 3 в силу основной теоремы арифметики. Это значит, что как бы мы ни расширяли $\Q$ с помощью квадратных корней из ранее полученных чисел, мы никогда не сможем построить число $\sqrt[3]{2}$. Следовательно, удвоить куб циркулем и линейкой невозможно!
\end{enumerate}


\subsection*{Задачи}

\begin{enumerate}
\item Построить циркулем и линейкой отрезки длины $\sqrt 2$ и $\sqrt 3$.
\item Доказать, что $\Q[\sqrt[3]{2}]=\{a+b\sqrt[3]{2}+c\sqrt[3]{4}\;|\;a,b,c\in\Q\}$ и имеет размерность 3 над $\Q$.
\end{enumerate}





\section{Многочлены и алгебраические числа}

\lesson{Определение алгебраического числа. Лемма Гаусса и неприводимость многочлена над $\Z$ и $\Q$}


\begin{enumerate}
\item Заметим, что уравнение \eqref{xkpl} --- это алгебраическое уравнение, т.е. уравнение, записанное с помощью суммы степеней переменной $x$ с некоторыми коэффициентами из данной нам алгебраической структуры, в нашем случае --- кольца целых чисел.
\item Перейдем теперь к изучению многочленов над полем $\Q$.
\item Корни многочленов над $\Q$ называются \textbf{алгебраическими числами}.\index{Алгебраические числа}\index{Числа!алгебраические} Множество всех алгебраических чисел обозначается $\A$. Заметим, что алгебраические числа, вообще говоря, лежат в комплексной плоскости, т.е. могут иметь мнимую часть.
\item Все корни многочленов из $\Q[x]$ и все корни многочленов из $\Z[x]$ --- это одно и то же множество $\A$. Действительно, $\Z[x]\subseteq \Q[x]$, поскольку $\Z\subseteq\Q$, так что корни многочленов с целыми коэффициентами принадлежат $\A$. С другой стороны, если какое-то $x$ зануляет многочлен с рациональными коэффициентами, то он же зануляет и многочлен с целыми коэффициентами. Этот многочлен получается из исходного домножением на все знаменатели всех его коэффициентов, так что вместо дробей мы получаем целые числа, а равенство нулю при этом сохраняется.
\item Поэтому часто при анализе корней многочленов используется один из следующих подходов: либо рассматриваются многочлены с произвольными целыми коэффициентами, либо рассматриваются многочлены с рациональными коэффициентами, у которых старший коэффициент $k_n=1$:
$$
k_nx^n+k_{n-1}x^{n-1}+\dots+k_1x+k_0,\;k_s\in\Z,\mbox{ либо }x^n+q_{n-1}x^{n-1}+\dots+q_1x+q_0,\;q_s\in\Q.
$$
\item Есть, правда, один объединяющий эти два кейса вариант многочленов --- многочлены с целыми коэффициентами и старшим коэффициентом $k_n=1$. Корни таких многочленов называются \textbf{целыми алгебраическими числами}. Таких числа образуют собственное подмножество в $\A$.
\item Введем следующее понятие. Пусть $f\in\Z[x]$. \textbf{Содержанием многочлена} $f$ называется наибольший общий делитель всех его коэффициентов:
$$
\content(k_0+k_1x+\dots+k_nx^n) = \gcd(k_0,k_1,\dots,k_n)
$$
\begin{lem}[Гаусса] Пусть $f,g\in\Z[x]$, тогда\index{Лемма!Гаусса о неприводимости}
$$
\content(fg)=\content(f)\content(g)
$$
\end{lem}
\pf Ясно, что достаточно рассмотреть случай $\content(f)=\content(g)=1$. Остальные случаи приводятся к этому делением коэффициентов многочленов на их содержание. При этом, очевидно, они не перестают быть многочленами над $\Z$.

Итак, предполагая $\content(f)=\content(g)=1$, покажем, что $\content(fg)=1$. Пусть, кроме того,
$$
f(x) = f_0+f_1x+\dots+f_nx^n,\quad g(x) = g_0+g_1x+\dots+g_mx^m
$$

Предположим, что $\content(fg)=d>1$. Пусть $p$ --- простое число, делящее $d$. Так как $\content(f)=1$, существует хотя бы один коэффициент многочлена $f$, который не делится на $p$. Пусть $f_k$ --- коэффициент с минимальным номером, не делящийся на $p$.
Аналогично обозначим за $g_s$ коэффициент многочлена $g$ с минмальным номером, не делящийся на $p$.

Найдем коэффициент многочлена $fg$ при степени $x^{k+s}$. Она равен
$$
[x^{k+s}]f(x) = f_0g_{k+s}+f_1g_{k+s-1}+\dots+f_kg_s+f_{k+1}g_{s-1}+\dots+f_{k+s}g_0\equiv f_kg_s\pmod p,
$$
поскольку
$f_0,\dots,f_{k-1}\equiv 0\pmod p$ и $g_{s-1},\dots,g_0\equiv 0\pmod p$.

Но $f_kg_s\not\equiv 0\pmod p$ в силу их выбора, а значит, $[x^{k+s}]f(x)\not\equiv 0\pmod p$, т.е. среди коэффициентов многочлена $fg$ есть такой, который не делится на $p$, откуда следует, что $p$ не может быть общим делителем коэффициентов $fg$, а значит, не делит и наибольшой общий делитель $\content(fg)=d$. Противоречие.
\epf
\begin{sled}
Многочлен с целыми коэффициентами неприводим над $\Z$ тогда и только тогда, когда он не приводим над $\Q$.
\end{sled}
\pf
Ясно, что если многочлен неприводим над $\Q$, то он неприводим и над $\Z$, поэтому докажем обратное. А точнее, покажем, что если многочлен раскладывается в произведение над $\Q$, то его можно разложить и над $\Z$.

Пусть $f\in\Z[x]$ и $f=gh$, где $g,h\in\Q[x]$. Можно считать, что $\content(f)=1$ (если это не так, то разделим равенство $f=gh$ на $\content(f)$ и получим то, что требуется, $f$ при этом не выпадет из $\Z[x]$).

Найдем такое натуральное число $n>0$, что $ng\in\Z[x]$ (например, произведение всех знаменателей коэффициентов $g$). И пусть $m=\content(ng)$. Тогда рациональное число $r=n/m$ таково, что $rg\in\Z[x]$ и $\content(rg)=1$. Аналогично найдем рациональное $s$ такое, что $sh\in\Z[x]$ и $\content(sh)=1$.

По лемме Гаусса получаем
$$
1=\content(rg)\content(sh)=\content(rsgh)=\content(rsf)=rs\content(f)=rs,
$$

Но тогда имеем слеующее разложение
$$
f=gh=rsgh=(rg)(sh),
$$
где $rg,sh\in\Z[x]$, т.е. $f$ разложим над $\Z$.
\epf



\lesson{Степень алгебраического числа. Минимальный многочлен. Неразложимость $x^3-2$. Замечание про степень числа $\sqrt[k]{p^l}$}


\item Если $\al$ --- алгебраическое число, то минимальная степень многочлена, корнем которого является $\al$, называется \textbf{степенью алгебраического числа} $\al$.\index{Степень алгебраического числа}
\item Например, мы ранее показали, что $\sqrt 2$ не является рациональным числом, значит, никакой многочлен первой степени, т.е. $x+q$, не обращается в ноль при $x=\sqrt 2$, значит, $\sqrt 2$ не является алгебраическим числом первой степени. 
\item На самом деле, все рациональные числа, и только они, являются алгебраическими числами первой степени. То есть, $\Q\subset\A$ (вложение собственное!)
\item С другой стороны, $x^2-2$ зануляется числом $\sqrt 2$, так что это число является алгебраическим числом степени 2. То же самое можно сказать про любой квадратный корень из любого простого числа или его нечетной степени (ибо нечетные числа взаимно просты с двойкой, см. выше про корни $\sqrt[k]{p^l}$).
\item Далее мы установим, что число $\sqrt[3]{2}$ является алгебраическим числом степени 3. Для этого нам нужно показать, что никакой многочлен степени 2 не может его занулить.
\begin{lem}\label{gfh}
Пусть $f(x)$ --- многочлен минимальной степени, зануляющий число $\al\in\A$, и пусть многочлен $g(\al)=0$, тогда существует многочлен $h(x)$ такой, что
$$
g=fh.
$$
\end{lem}
\pf
Пусть степень $f$ равна $m$, а степень $g$ равна $n$. Ясно, что $n\ge m$ в силу минимальности $f$.
Разделим $g$ на $f$ с остатком: $g=fh+r$. Здесь степень $h$ равна $n-m$, а степень $r$ строго меньше $m$.

Но поскольку $g(\al)=0$ и $f(\al)=0$, получаем, что и $r(\al)=0$. Таким образом, $r$ является многочленом, зануляющим $\al$, и притом его степень меньше $m$ в противоречии с определением числа $m$. Следовательно, $r$ есть тождественный ноль, а значит, $g$ делится на $f$ без остатка.
\epf
\item Перейдем к $\sqrt[3]{2}$, точнее, к определяющему его многочлену.
\begin{lem}
Многочлен $x^3-2$ неразложим на множители с целыми коэфициентами степени $\ge 1$.
\end{lem}
\pf Предположим, что это не так, тогда $x^3-2$ делится на линейный многочлен вида $(ax+b)$ (если он делится на многочлен второй степени, то делится и на многочлен первой степени):
$$
x^3-2 = (ax+b)(kx^2+mx+n),
$$
где $a,b,k,m,n\in\Z$ и $a,k\ne 0$. Сравним коэффициенты при одинаковых степенях:

\begin{align*}
1 & = ak \\
0 & = am+bk \\
0 & = an+bm \\
-2 & = bn
\end{align*}

Из первого равенства следует, что либо $a=k=1$, либо $a=k=-1$. Учитывая это, из второго равенства получаем, что $m=-b$, откуда с помощью третьего равенства получаем, что $an=b^2$. Наконец, четвертое равенство предлагает варианты:
$$
b=2,n=-1\mbox{ или }b=-2,n=1\mbox{ или }b=1,n=-2\mbox{ или }b=-1,n=2.
$$

В первом и втором случае получаем, что $an=4$, но это невозможно, поскольку $a,n\in\{1,-1\}$.

В третьем и четвертом случае $an=1$, но и это невозможно при $a\in\{1,-1\}$, $b\in\{2,-2\}$.
\epf

\item В силу следствия из леммы Гаусса и леммы о неразложимости $x^3-2$ получаем, что $x^3-2$ неразложим на множители с рациональными коэффициентами. Во-вторых, если бы минимальный многочлен для $\sqrt[3]{2}$ имел степень 1 или 2, то в силу леммы \ref{gfh} $x^3-2$ делился бы на на него. А это невозможно в силу неразложимости $x^3-2$. Следовательно, минимальным многочленом для $\sqrt[3]{2}$ является многочлен третьей степени, значит, $\sqrt[3]{2}$ --- алгебраическое число степени 3.
\item Существует подробно разработанная теория алгебраических чисел, из которой, в частности, следует, что число $\sqrt[k]{p}$ является алгебраическим числом степени $k$. То есть, по крайней мере, для каждого натурального $k$ и каждого простого $p$ можно найти свое алгебраическое число.


\lesson{Теорема про $\A$ (без док-ва). Числа $k\al+t$, где $\al\in\A$, $k,t\in\Z$, всюда плотны}


\item Про множество алгебраических чисел известна следующая теорема, доказательство которой опирается на теорию расширения полей.
\begin{thrm}
\textup{(1)} $\A$ является полем, \textup{(2)} $\A$ алгебраически замкнуто.\index{Поле!алгебраических чисел}
\end{thrm}
Первое утверждение говорит нам о том, что алгебраические числа можно складывать, вычитать, умножать и делить, а результат все равно останется алгебраическим числом, т.е. корнем некоторого многочлена с целыми коэфициентами. Второе --- о том, что если даже мы рассмотрим кольцо многочленов $\A[x]$, т.е. всех многочленов с коэффициентами из поля $\A$, то корни таких многочленов все равно будут алгебраическими числами. Иначе говоря, замкнутость $\A$ означает, что его невозможно расширить алгебраическими методами, как это мы проделывали с полем $\Q$. Для дальнейшего расширения $\A$ понадобятся многочлены бесконечной степени, т.е. ряды.

\item Отметим, что если число $\al$ --- алгебраическое степени $n>1$, то числа $\al,\al^2,\dots,\al^{n-1}$ также являются алгебраическими степени не выше $n$ и притом иррациональными (если бы какое-то из них было рациональным, то степень $\al$ оказалсь бы ниже $n$). Например, $(\sqrt[3]{2})^2=\sqrt[3]{4}$ --- алгебраическое число степени 3.
\item Более того, если $\al$ --- алгебраическое число степени выше 1, то любая комбинация $k+\al t$ с целыми коэффициентами $k,t$ ($t\ne 0$) также будет алгебраическим числом той же степени. И еще более того, любая комбинация вида
$$
k_0+k_1\al+k_2\al^2+\dots+k_m\al^m,
$$
где $\al$ --- алгебраическое степени $n$, $k_m$ --- целое ненулевое число, $m<n$, также будет алгебраическим числом степени выше 1 (иначе перед нами был бы многочлен степени $<n$, зануляющий $\al$), причем все такие комбинации различны (если бы нашлось две равные комбинации с разными наборами коэффициентов, то их разность была бы многочленом степени $<n$, зануляющим $\al$).
\item Таким образом, мы уже существенно пополнили множество иррациональных чисел всего лишь с помощью алгебраических корней и целочисленных линейных комбинаций их степеней. Точнее, мы надстроили над множеством $\Q$ бескончный слоеный пирог, в каждом слое которого сидят алгебраические числа какой-то одной степени (а первый слой пирога --- это само $\Q$), причем каждый слой содержит беконечно много чисел. И все эти слои каким-то образом умещаются в <<дырках>> между рациональными числами, не смотря на то, что $\Q$ всюду плотно на прямой.
\item \textit{Насколько же плотны алгебраические числа заданной степени на числовой прямой?}
\item Возьмем произвольное алгебраическое число $\al$ (например, $\sqrt 2$) какой-то алгебраической степени $>1$. Это число иррациональное, т.е. не соизмеримо с целыми числами. И пусть у нас кузнечик одной ногой прыгает на 1, а второй --- на $\al$. \textit{Вопрос}: какими свойствами обладает множество всех тех точек, куда может допрыгнуть кузнечик?
\item Ясно, что все точки, в которые попадает кузнечик, описываются в общем виде формулой
$$
k + \al t,\quad k,t\in\Z,
$$
т.е. это числа-собратья исходного числа $\al$ по степени своей алгебраичности. Выбрав число определенной алгебраической степени, кузнечик прыгает только по числам такой же степени.
\item Вспомним наш пример с наматыванием прямой на окружность (колесо на дороге), и выберем радиус окружности так, чтобы один полный виток по ней составлял как раз 1 единицу длины ($R=1/2\pi$). Тогда однократное наматывание прямой на окружность будет соответствовать прыжку кузнечика на 1, а значит, с точки зрения жителя окружности кузнечик будет топтаться на месте всякий раз, когда он прыгает на любое целое число.
\item В то же время, если кузнечик начинает прыгать с шагом $\al$, он, очевидно, начинает встречаться с жителем окружности в каких-то других точках, отличных от нуля. Посмотрим, как расположены эти точки на окружности.
\item Для начала заметим, что кузнечик при этом никогда не повторяется, т.к. точки $k\al$ и $l\al$ при $k\ne l$, намотанные на окружность, соответствуют длинам дуг за вычетом каких-то целых оборотов:
$$
k\al = n + \be,\quad l\al=m+\ga,
$$
и если бы мы получили совпадение дуг $\be$ и $\ga$, то получилось бы уравнение
$$
k\al-n=l\al-m,
$$
откуда видно, что $\al=(n-m)/(k-l)$ --- рациональное число, а это противоречит иррациональности $\al$.
\item Итак, все шаги кузнечика вида $0,\pm\al,\pm 2\al,\dots$ дают попарно различные точки на окружности.
\item Далее. Покажем, что какое бы маленькое $\ep$ мы ни выбрали, найдется такое целое $t$, что число $\al t$ будет отстоять от некоторого целого числа на расстояние меньше этого $\ep$.
\item Действительно, выберем целое число $N$ заведомо большее, чем $1/\ep$, и поделим окружность на $N$ равных секторов. В каждом секторе длина дуги будет равна $1/N$, что меньше $\ep$. Но всех различных точек, кратных $\al$, как мы доказали чуть выше, бесконечно много, а значит, хотя бы на одной дуге из этих $N$ штук окажется хотя бы 2 точки, и мы получим, что
$$
k\al = n + \be,\quad l\al=m+\ga,\quad|\be-\ga|<\ep.
$$
Для удобства можем считать, что $\be>\ga$ и, следовательно, $\ga<\be<\ga+\ep$. Тогда
$$
l\al-m<k\al-n<l\al-m+\ep,
$$
откуда 
$$
n-m<(k-l)\al < n-m+\ep,
$$
т.е. при $t=k-l$ число $\al t$ отстоит от целого $n-m$ на расстояние меньше $\ep$.
\item А это значит, что кузнечик попадает в точки, сколь угодно близкие к нулю (ведь до нуля он может допрыгать своей целочисленной ногой, сделав $m-n$ шагов).
\item Но, умея сдвигаться на какое-то число $\de$ от нуля хоть в какую-то сторону, кузнечк может повторять этот алгоритм раз за разом, и уходить от 0 на расстояние $\pm\de,\pm 2\de,\pm 3\de$ и т.д.  То есть, числа вида $k+\al t$, достигаемые кузнечиком, находятся сколь угодно близко к любой точке на прямой!
\item Но в таком случае множество $\{k+\al t\mid k,t\in\Z\}$ всюду плотно на прямой, как и множество $\Q$. Хотя, строго говоря, оно не содержит в себе $\Q$ (ведь коэффициенты $k$ --- целые, а при $t\ne 0$ комбинация $k+\al t$ и вовсе иррациональна). Получается, что множество алгебраических чисел одной выбранной степени всюду плотно, причем не за счет $\Q$, а за счет точек, лежащих вне $\Q$!
\item По сути, мы получаем, что между рациональными числами столь много <<дырок>>, что там умещается бесконечно много всюду плотных попарно различных множеств.
\item Вопрос: а все ли <<дырки>> могут быть ими заполнены? Сколько вообще существует алгебраических чисел и сколько существует <<дыр>> на рациональной прямой?
\item Для ответа на этот вопрос нам снова следует обратиться к теории множеств.
\end{enumerate}







\newchapter{Континуум}

\vrezka{
В этой главе мы полностью завершим наполнение геометрической прямой числами, обсудим разные версии понятия полноты вещественной прямой, достроим до конца комплексную плоскость.
}


\section{Мощности множеств}\label{powers}

\lesson{Равномощные множества. Разные биекции между счетными множествами. Теорема Кантора}


\begin{enumerate}
\item Если в множестве $n\in\N$ элементов, то число $n$ называется мощностью этого множества. Множества, имеющие мощность, равную натуральному числу, называются \textbf{конечными}.\index{Множество!конечное}
\item Очевидно, что само множество натуральных чисел не является конечным. Вопрос: как сравнивать мощности любых множеств?
\item Если мы возьмем конечное множество $A=\{a_1,\dots,a_n\}$, то мы сразу же подразумеваем наличие нумерации его элементов: номеру $k$ ставится в соответствие элемент $a_k$. И если все $a_k$ попарно различны, то в этом множестве столько же элементов, сколько чисел $1,\dots,n$, т.е. $n$ штук.
\item В этом рассуждении мы неявно указали на то, что существует взаимно однозначное соответствие между элементами множества $A$ и множеством-эталоном $\{1,\dots, n\}$.
\item Теперь, если задано некоторое множество $B=\{b_1,\dots,b_n\}$, у которого также все $b_k$ попарно различны, то мы внось имеем дело со взаимно однозначм соответствием между $B$ и множеством-эталоном $\{1,\dots, n\}$. Иначе говоря, мы имеем биекции 
$$
f:\{1,\dots, n\}\leftrightarrow A\mbox{ и }g:\{1,\dots, n\}\leftrightarrow B.
$$
\item Но тогда сложная функция $h(a)=g(f^{-1}(a))$ устанавливает взаимно однозначное соответствие между $A$ и $B$. Это можно проиллюстрировать на диаграмме:
\[ \begin{diagram}
\node{A} \arrow[2]{e,t}{g\circ f^{-1}}
\node[2]{B} \\
\node[2]{\{1,\dots,n\}} \arrow{ne,r}{g}
\arrow{nw,b}{f}
\end{diagram}\]

\item Таким образом, мы находим, что конечные множества $A$ и $B$ обладают одинаковой мощностью (количеством элементов), если между ними существует биекция.
\item Именно такой подход и выбирается при определении равномощности произвольных множеств!
\item Множества $X$ и $Y$ \textbf{равномощны} (будем это записывать так: $X\leftrightarrow Y$), если существует биекция $f:X\leftrightarrow Y$.\index{Мощность множества}
\item Заметим, что при этом мы не требуем наличия какого-то эталонного множества, хотя для конечных множеств, как мы уже видели, таковыми могут считаться множества $\{1,\dots,n\}$ или, как принято в теории множеств, множества $\{0,\dots,n-1\}$.
\item Отношение $\leftrightarrow$ рефлексивно (всякое множество само себе равномощно), симметрично (если $X\leftrightarrow Y$, то $Y\leftrightarrow X$) и транзитивно (если $X\leftrightarrow Y$ и $Y\leftrightarrow Z$, то $X\leftrightarrow Z$), т.е. является отношением эквивалентности. Это значит, что, вообще говоря, все множества можно разделить на непересекающиеся классы, так что внутри каждого класса будут находиться только равномощные множества. Такой класс и принято называть \textbf{мощностью множества}.
\item Другой подход к определению самого понятия мощности заключается в том, чтобы в каждом таком классе найти некоторого типичного представителя и его объявить мерой мощности всех множеств данного класса. В случае конечных множеств такими представителями как раз и являются отрезки натурального ряда $\{0,\dots,n-1\}$.
\item Множество, не содержащее элементов, т.е. пустое множество $\emptyset$, имеет мощность 0, оно биективно не сопоставляется ни с каким другим множеством, т.е. является уникальным представителем своего класса, и само себе является мерой мощности.
\item Еще одна разновидность множеств, имеющих эталон мощности --- \textbf{счетные множества}. К ним относятся все множества, равномощные $\N$.\index{Множество!счетное}
\item Например, счетными являются такие множества:
$$
2\N,\Z,2\Z,\{p\in\N\mid p \mbox{--- простое}\},\Z[i],\Q,\A.
$$
Все множества, которые не конечны, и элементы которых можно перенумеровать (пересчитать) натуральными числами, являются счетными.
\item Нумерацию $2\N$ представить очень просто: каждому номеру $k$ поставим в соответствие элемент $2k$, тем самым мы перенумеруем все четные числа, а значит, множество четных чисел счетное. Здесь мы впервые сталкиваемся с тем, что бесконечное множество равномощно какой-то своей части (которая может казаться очень маленькой, ведь точно так же устанавливается биекция между $\N$ и $10^9\N$ и т.п.). Иногда такое свойство бесконечных множеств берется за их определение:\textit{ множество бесконечное, если оно равномощно некоторому своему собственному подмножеству}.\index{Множество!бесконечное}
\item Биекцию $\N\leftrightarrow\Z$ установить также относительно просто: будем нумеровать целые числа по мере их удаления от нуля: $0, 1, -1, 2, -2, 3, -3$, и т.д. Ясно, что каждое целое число будет пронумеровано, и притом только один раз. Следовательно, $\Z$ --- счетное.
\item Для нумерации $\Z[i]$ нужно придумать алгоритм нумерации по <<квадратным орбитам>> вокруг нуля. Например, это можно сделать следующим способом:
$$
f(n+mi) = \begin{cases}
n+(|n|+|m|)^2, & (m>0)\lor(m=0\land n\ge 0) \\
-n-1-(|n|+|m|)^2, & (m<0)\lor(m=0\land n< 0) \\
\end{cases}
$$
Графически это выглядит так:
\begin{center}
\includegraphics[scale=0.3]{ZZnumero.png}
\end{center}
Такая нумерация дает взаимно однозначное соответствие между $\Z\times\Z$ и $\Z$, и мы получаем замечательный факт, который называется \textbf{теоремой о квадрате}. То есть, счетное множество равномощно своему квадрату. Конечным множествам такое и не снилось!\index{Теорема!о мощности квадрата множества}
\item Можно развить это достижение и дальше. Имея нумерацию $\Z\times\Z$ и $\Z$, мы можем пронумеровать $\Z\times\Z\times\Z$, т.е. куб множества $\Z$, и так далее. Любая конечная степень $\Z$ равномощна $\Z$.
\item Рациональные числа --- это пары целых чисел, но не всех, а только взаимно простых. Это значит, что они тоже получают номера в процессе нумерования $\Z\times\Z$, только с пропусками. Если затем эти номера перенумеровать заново, уже без пропусков, то мы получим взаимно однозначное соответствие $\Q$ и $\N$. То есть множество $\Q$ также счетно.
\item Для нумерации $\A$ воспользуемся следующим приемом. Вспомним, что $\A$ --- это все корни всех многочленов с целыми коэффициентами. Возьмем тогда некоторое целое положительное число $L$ и соберем в множество $\A_L$ те и только те алгебраические числа, которые определяются многочленами вида $P(x)=k_0+k_1x+\dots+k_nx^n$ ($k_n\ne 0$), удовлетворяющими условию
$$
|k_0|+|k_1|+\dots+|k_n|+n=L.
$$
Ясно, что таких многочленов существует лишь конечный набор, т.к. выбор коэффициентов $k_s$ и степени $n$ ограничен числом $L$. Но и корней у многочлена --- тоже конечное количество, не превышающее его степень (см. выше теорему о корнях над полем). Таким образом, множество $\A_L$ конечно.

С другой стороны, множество $\A$ есть объединение всех множеств $\A_L$ при $L=1,2,\dots$ Поэтому, нумеруя последовательно, сначала числа из $\A_1$, затем числа из $\A_2$, и т.д. мы пронумеруем все множество $\A$, а значит, это множество счетное!

\item Что же получается в итоге? Множество алгебраических чисел, которыми мы старались заткнуть все дыры между рациональными числами, и которое состоит из бесконечного числа бесконечных слоев, равномощно множеству $\Q$! С точки зрения мощности множества мы так ничего и не добавили к рациональным числам, хотя и позатыкали много дыр.
\item Возникает вопрос: \textit{а бывают ли вообще какие-то другие мощности, кроме счетной?} Ответ на этот вопрос дает
\begin{thrm}[Кантора]\index{Теорема!Кантора}
Никакое множество не равномощно множеству всех своих подмножеств.
\end{thrm}
\pf
Пусть имеется множество $X$. Можем сразу считать, что оно непустое, т.к. для пустого множества теорема, очевидно, верна (в пустом множестве 0 элементов, а в множестве $\{\emptyset\}$ --- один элемент). Через $\Pcal(X)$ обозначим множество всех подмножеств множества $X$.

Предположим, что существует биекция $f:X\leftrightarrow\Pcal(X)$. Ясно, что поскольку для всякого $x\in X$ значение $f(x)$ есть какое-то подмножество $X$, то возможны две ситуации: либо $x\in f(x)$, либо $x\notin f(x)$. Соберем тогда в множество $Y$ все такие элементы $x$, которые удовлетворяют второму соотношению:
$$
Y=\{x\in X\mid x\notin f(x)\}.
$$
Понятно, что $Y\subseteq X$, а значит, $Y\in\Pcal(X)$, а значит, существует единственный элемент $y\in X$ такой, что $f(y)=Y$ (поскольку $f$ --- биекция по предположению).

Вопрос: $y\in Y$ или нет?

Если $y\in Y$, то по определению множества $Y$ получаем, что $y\notin f(y)$, но тогда $y\notin Y$. Противоречие.

Если $y\notin Y$, то по определению множества $Y$ получаем, что \textbf{неверно} $y\notin f(y)$, т.е. $y\in Y$. Противоречие.

Любой вариант приводит к противоречию, следовательно, предположение о существовании биекции $f:X\leftrightarrow\Pcal(X)$ неверно, т.е. множество $X$ и множество всех его подмножеств неравномощны.
\epf


\lesson{Обсуждение теоремы Кантора, примеры. Теорема Кантора--Бернштейна. Объединение счетного множества счетных множеств}


Теорему Кантора можно дополнить тем, что множество всех подмножеств множества $X$ мощнее исходного множества $X$, т.к. $X$ в него инъективно вкладывается, т.е. равномощно некоторой части $\Pcal(X)$. Действительно, в $\Pcal(X)$ существует подмножество следующего вида:
$$
\{\{x\}\mid x\in X\},
$$
это --- множество всех синглетов (т.е. одноточечных подмножеств), образованных точками множества $X$. Таким образом, $\Pcal(X)$ получается более мощным множеством, чем $X$.

\item Приведем пример. Пусть $X=\{1,2,3\}$. Тогда
$$
\Pcal(X) = \{\emptyset,\{1\},\{2\},\{3\},\{1,2\},\{1,3\},\{2,3\},\{1,2,3\}\}.
$$
Подмножество синглетов здесь --- это множество $\{\{1\},\{2\},\{3\}\}$.

\item Для конечного множества $X$ мощности $n$ мощность множества его подмножеств $\Pcal(X)$ равна $2^n$. Это легко проверить, поскольку каждое подмножество $X$ задается состояниями его элементов: каждый из них может входить в данное подмножество или не входить. Поскольку у каждого элемента ровно 2 состояния, а всего элементов $n$, то общее количество состояний всех элементов равно $2\cdot 2\cdot\ldots\cdot 2$ ($n$ раз), т.е. $2^n$.
\item Отсюда берет начало второе обозначение для множества всех подмножеств множества $X$ --- $2^X$.
\item Но если с конечными множествми все укладывается в рамки обычно арифметики, то с множеством $2^\N$ возникают проблемы. Его мощность не только не выражается натуральным числом, но она также не равна и мощности $\N$, как мы только что доказали. Эта мощность называется \textbf{мощностью континуума} и обозначается $\cgot$.\index{Множество!континуальное}\index{Континуум}
\item Итак, мы теперь знаем, что бесконечные множества отличаются по мощности. Как же можно устанавливать их равномощность, если сложно или невозможно в явном виде указать биекцию между множествами? ответ на этот вопрос дает
\begin{thrm}[Кантора--Бернштейна]\index{Теорема!Кантора--Бернштейна}
Если существуют инъекции $f:A\to B$ и $g:B\to A$, то множества $A$ и $B$ равномощны.
\end{thrm}
\pf Рассмотрим функцию $H:2^A\to 2^B$, определяемую следующей формулой:
\[H(X)=A\setminus g[B\setminus fX]\mbox{ для всех }X\subseteq A.\]
\textit{Примечание}: здесь под записью $fX$ и $f[X]$ понимается \textbf{образ множества} $X$, т.е.
$$
f[X] = \{f(x)\mid x\in X\}.
$$

Предположим, что существует корень уравнения $H(X)=X$ --- некое множество $X_0$.
Посмотрим, какими свойствами оно обладает. Обозначим через $Y_0$ множество
$B\setminus fX_0$. Поскольку $H(X_0)=X_0$, то $gY_0=A\setminus X_0$. Таким
образом, сужения функций $f|_{X_0}$ и $g|_{Y_0}$ действуют на непересекающихся
подмножествах множеств $A$ и $B$ и <<покрывают>> эти множества
целиком.

Точнее, рассмотрим обратную к $g|_{Y_0}$ функцию, определенную на множестве
$A\setminus X_0$, обозначив ее $h$. Имеем: $h:A\setminus X_0\to Y_0$ ---
биекция, $f|_{X_0}:X_0\to B\setminus Y_0$ --- тоже биекция. Тогда объединение
этих функций $h\cup f|_{X_0}$ есть биекция между $A$ и $B$.

\begin{center}
\includegraphics[scale=0.2]{CBS1.png}
\end{center}

Итак, доказательство теоремы свелось к поиску корня уравнения $H(X)=X$. Назовем
множество $X$ {\itshape хорошим}, если $X\subseteq H(X)$ (почти как в теореме Кантора!), и через $Z$ обозначим
объединение всех хороших множеств.

Нетрудно проверить, что функция $H$ монотонна по вложению множеств, т.\,е.
если $X\subseteq Y$, то $H(X)\subseteq H(Y)$. Пусть $z\in Z$. Тогда существует
хорошее множество $X$ такое, что $z\in X$. Кроме того, имеем $H(X)\subseteq
H(Z)$ и $X\subseteq H(X)$. Отсюда заключаем, что $z\in H(Z)$, и это верно
для любого $z\in Z$. Таким образом, $Z$ --- хорошее множество.

Чтобы показать, что $Z$ и есть корень нашего уравнения, надо проверить обратное
вложение, т.\,е. что $H(Z)\subseteq Z$. Допустим, что это не так. Тогда
существует $x\in H(Z)\setminus Z$. Рассмотрим множество $Z'=Z\cup\{x\}$.
Это множество не может быть хорошим, т.\,к. иначе оно бы содержалось в $Z$.
Поэтому $Z'\not\subseteq H(Z')$. Ясно, что $H(Z)\subseteq H(Z')$, поэтому
$Z'\not\subseteq H(Z)$. С другой стороны, $Z\subseteq H(Z)$. Следовательно,
точка $x$ --- единственная точка множества $Z'$, не попадающая в $H(Z)$.
Получено противоречие с тем, что $x\in H(Z)$.

Итак, $H(Z)\subseteq Z$, откуда $H(Z)=Z$, т.\,е. $Z$ --- искомое множество.

\begin{center}
\includegraphics[scale=0.2]{CBS2.png}
\end{center}


\epf

\item Теорема Кантора--Бернштейна дает достаточно простой инструмент сравнения мощностей. Достаточно показать, что первое множество равномощно какой-то части второго, а второе --- какой-то части первого, т.е. что они взаимно друг в друга вкладываются. Это будет означать, что между ними существует биекция, т.е. что они равномощны. При этом данная теорема не прдъявляет нам какого-то простого алогритма построения этой биекции, т.к. явно найти и описать все хорошие множества --- далеко не всегда разрешимая задача.
\item С помощью этой теоремы легко показать равномощность $\Q$ и натурального ряда. Действительно, $\Q$ легко вкладывается в часть множества $\Z\times \Z$ (пары взаимно простых составляют его часть). Но $\Z\times\Z$ мы умеем явно нумеровать целыми числами, т.е. умеем строить инъекцию из $\Z\times\Z$ в $\Z$, а значит, мы имеем вложение $\Q$ в $\Z$. Ну, а вложение в обратную сторону тривиально. Таким способом получается много результатов о равномощности.
\item Назовем множество \textbf{не более чем счетным}, если оно счетно или конечно (или пустое).
\begin{thrm}
Объединение не более чем счетного множества не более чем счетных множеств не более чем счетно.
\end{thrm}
\pf Мы можем считать, что нам дан счетный набор не более счетных множеств. При необходимости мы просто дополним этот набор до счетного пустыми множествами (ведь формулировка не требует, что они были разные). Раз их счетный набор, значит, они как-то уже пронумерованы. Пусть они обозначаются символами $A_n$. Тогда требуемое объединение равно
$$
A = \bigcup_{n=1}^\infty A_n,
$$
где все $A_n$ --- не более чем счетные множества.

Поскольку нам известно, что $A_n$ --- не более чем счетное, значит, существуют биекции $f_n$ между $A_n$ и либо $\N$, либо каким-то-конечным отрезком $\N$. В любом случае $f_n$ --- это инъекция из $A_n$ в $\N$.

Теперь построим инъекцию из $A$ в $\Z\times\Z$.

Пусть $a\in A$. Тогда существует $n$ такой, что $a\in A_n$. Может оказаться так, что $a$ лежит сразу во многих множествах $A_n$, в таком случае выберем нименьший из их номеров и обозначим его за $n_a$. Поскольку $a\in A_{n_a}$, мы можем применить к нему функцию $f_{n_a}$. Положим далее
$$
g(a) = (n_a,f_{n_a}(a)).
$$
Ясно, что $g$ --- инъекция, т.к. для разных точек $a$ и $a'$ либо отличаются номера $n_a$ и $n_{a'}$, и следовательно $g(a)\ne g(a')$ по свойствам упорядоченной пары, либо у них общий номер $n_a$, но тогда отличаются значения $f_{n_a}(a)$ и $f_{n_a}(a')$, поскольку $f_{n_a}$ является инъекцией, и, стало быть, $g(a)\ne g(a')$.

Итак, у нас есть инъекция $g:A\to\Z\times\Z$. К ней можно применить биекцию $f:\Z\times\Z\leftrightarrow\Z$, которую мы ранее выписывали в явном виде, а затем применить биекцию $h:\Z\leftrightarrow\N$, полагая $h(m)=2m$, если $m\ge 0$, и $h(m)=-2m-1$, если $m<0$. Тогда композиция $P=h(f(g(a)))$ будет инъекцией из $A$ в $\N$.

Область значений $P$ есть подмножество в $\N$. Она либо имеет максимум, и тогда это конечное множество, либо не имеет максимума, и тогда это бесконечное множество. В любом случае, область значений $P$ можно перенумеровать так, чтобы номера шли подряд от нуля без пропусков. И тогда с помощью $P^{-1}$ мы построим биекцию между $A$ и либо конечным отрезком натурального ряда, либо самим $\N$.\epf

\end{enumerate}


\subsection*{Задачи}

\begin{enumerate}
\item Найти мощность множества всех функций из $X=\{1,2,3\}$ в $Y=\{1,2,3,4\}$.
\item Какова мощность множества $\{(n,m)\mid n,m\in\Z, n<m\}$?
\item Пусть множество $C$ счетное, а множество $X$ бесконечное.
\begin{enumerate}[a)]
\item Доказать, что $X\cup C\leftrightarrow X$ (равномощны).
\item Доказать, что если $X\setminus C$ --- бесконечное, то $X\setminus C\leftrightarrow X$.
\end{enumerate}
\end{enumerate}

\section{Изоморфизмы}

\lesson{Все об изоморфизмах. Примеры с группами $\Z_m,\Z_m^*$. Порядковый изоморфизм. Теорема о счетных плотных неограниченных множествах}

\begin{enumerate}
\item Установление биекции между множествами позволяет нам судить об их количественном сходстве, но ничего не говорит о том, насколько похожи могут быть структуры, заданные на них. Поэтому на базе понятия биекции строятся более сильные критерии <<похожести>> двух множеств.
\item Центральным термином здесь является \textbf{<<изоморфизм>>}.\index{Изоморфизм} Это --- биекция, сохраняющая операции (например, сложение или умножение) и/или отношения (например, линейный порядок или отношение эквивалентности) и/или функционалы (например, норма или длина), заданные на этих двух множествах.
\item Поясним. Пусть у нас задана группа $\Z_4$ со сложением по модулю 4, и группа корней 4 степени из 1 с операцией умножения (делители 1 в кольце гауссовых чисел). В обоих множествах 4 элемента, следовательно, существует биекция между ними. Причем, таких биекций ровно столько, сколько перестановок в группе $\Sb_4$, т.е. 24 штуки. Однако, если дополнительно потребовать, чтобы результат сложения двух элементов в первой группе переходил в результат умножения образов этих элементов во второй группе, то таких биекций окажется всего две. Они-то и будут изоморфизмами этих групп.

\item Рассмотрим таблицы умножения группы $\Z_9^*$ и сложения группы $\Z_6$:

\begin{center}
\begin{tabular}{c|cccccc}
$\Z_9^*$ & 1 & 2 & \cellcolor{lightRed} 4 & 5 & 7 & 8\\  \hline
1 & 1 & 2 & 4 & 5 & 7 & 8\\
2 & 2 & 4 & 8 & 1 & 5 & 7\\
4 & 4 & 8 & 7 & 2 & 1 & 5\\
5 & 5 & 1 & 2 & 7 & 8 & 4\\
7 & 7 & 5 & 1 & 8 & \cellcolor{lightRed} 4 & 2\\
8 & 8 & 7 & 5 & 4 & 2 & 1
\end{tabular}
\qquad
\begin{tabular}{c|cccccc}
$\Z_6$ & 0 & 1 & \cellcolor{lightRed} 2 & 5 & 4 & 3\\  \hline
0 & 0 & 1 & 2 & 5 & 4 & 3\\
1 & 1 & 2 & 3 & 0 & 5 & 4\\
2 & 2 & 3 & 4 & 1 & 0 & 5\\
5 & 5 & 0 & 1 & 4 & 3 & 2\\
4 & 4 & 5 & 0 & 3 & \cellcolor{lightRed} 2 & 1\\
3 & 3 & 4 & 5 & 2 & 1 & 0
\end{tabular}
\end{center}

Во второй таблице мы специально перемешали порядок элементов таким образом, чтобы показать изоморфизм групп, при котором умножение в $\Z_9^*$ соответствует сложению в $\Z_6$, а соответствие элементов можно установить по правилу: $2^a\equiv b\mod 9$, где $a\in\Z_6$, $b\in\Z_9^*$, поскольку $Z_9^*=\langle 2\rangle$. Аналогичное соответствие можно посторить, опираясь на степени элементов 5 и 7 группы $\Z_9^*$.

\item Заметим, что не любая группа $\Z_m^*$ изоморфна некоторой группе $\Z_n$. Например, в группе $\Z_8^*$ содержится 4 элемента, но ни один из них не является образующим, группа $\Z_8^*$ не является циклической, а значит, она не может быть изоморфна $\Z_4$.

\item Еще пример. Рассмотрим множества $\Z$ и $2\Z$ с обычныи операциями сложения и умножения, и обычным линейным порядком на них. Мы уже знаем, что эти множества равномощны. Но посмотрим повнимательнее на биекцию $f(n)=2n$, действующую из $\Z$ в $2\Z$. Оказывается, что:
$$
f(n+m)=f(n)+f(m),\quad n<m\Leftrightarrow f(n)<f(m),
$$
т.е. $f$ сохраняет сложение и порядок. А это значит, что $f$ является изморфизмом упорядоченных групп $(\Z,<)$ и $(2\Z,<)$.

Однако, $f$ не сохраняет умножение, поскольку $f(nm)=2nm\ne f(n)f(m)$. Следовательно, $f$ не является изоморфизмом колец $(\Z,+,\cdot)$ и $(2\Z,+,\cdot)$. Более того, эти два кольца вовсе неизоморфны. Дело в том, что изоморфизм должен сохранять единицу, т.е. если какое-то чисало $\e$ является единицей по умножению в первом кольце, то $f(\e)$ будет единицей во втором кольце. Просто потому, что $n\e=n$ соответствует $f(n)=f(n)f(\e)$. Но чему бы ни было равно $f(1)$ в кольце $2\Z$, оно не обладает свойствами единицы, а значит, эти кольца не изоморфны.
\item Бывает и еще хуже. Изоморфизм работает только по отношению порядка, но не работает по алгебраическим операциям. Для этого достаточно вспомнить два изученных нами поля: $\Q$ и $\A$.
\item Ясно, что эти поля не могут быть изоморфны по операциям, т.к. иначе в обоих полях одинаково бы разрешалось или не разрешалось уравнение $x^2=2$. Но мы знаем, что оно разрешается в $\A$ и не разрешается в $\Q$. Тем не менее, с порядковым изоморфизмом у них все в порядке.
\begin{thrm} Все счетные неограниченные сверху и снизу плотные линейно упорядоченные множества порядково изоморфны друг другу.\end{thrm}\index{Изоморфизм!порядковый}
Иначе гвооря, пусть у нас имеется два множества $A$ и $B$, которые счетны (т.\,е. все их элементы можно перенумеровать натуральными числами), на них заданы линейные порядки $<_A$ и $<_B$ такие, что в обоих множествах нет ни наибольшего, ни наименьшего элемента, и эти множества плотны в своем порядке, тогда существует изоморфизм $f:A\leftrightarrow B$, сохраняющий порядок, т.е. $f(x)<_Bf(y)\Leftrightarrow x<_Ay$.
\pf
Будем строить соответствие пошагово. Пусть мы сделали некоторое соответствие для подмножеств $A_n\subset A$ и $B_n\subset B$ из $n$ элементов. Возьмем любой элемент одного из множеств (для определенности $A$), который не вошел в $A_n$. Посмотрим, в каком отношении он находится со всеми элементами из $A_n$. Он оказался либо наибольшим элементом, либо наименьшим элементом, либо стоящим между некоторыми элементами $a_i$ и $a_{i+1}$. Найдем элемент в $B$, находящийся в таком же отношении со всеми элементами $B_n$. Мы можем это сделать, так как $B$ --- плотное множество без наибольшего и наименьшего элементов. Будем считать эти два элемента сопоставленными. Таким образом, мы научились получать из соответствия для $n$ элементов соответствие для $n+1$ элемента. Чтобы в пределе получить соответствие для всех элементов, воспользуемся счетностью множеств $A$ и $B$. Пронумеруем все элементы и на каждом четном шаге будем выбирать еще не взятый элемент из множества $A$ с наименьшим номером, а на нечетном --- из $B$.
\epf

Из этой теоремы следует, например, что множество $\Q$ с обычным линейным порядком и множество $\A$ всех алгебраических чисел с обычным линейным порядком порядково изоморфны! Больше того, рациональный интервал $(a;b)$ порядково изоморфен всему множеству  $\Q$.
\end{enumerate}



\section{Действительные числа}

\lesson{Двоично-рациональные числа, алгоритм вылавливания произвольной точки прямой с помощью сетки двоичных рациональностей}


\begin{enumerate}
\item Вспомним снова про множество $\B$, которое состоит из рациональных чисел вида $k/2^n$, где $k\in\Z$, $n\in\N$.
\item Это множество есть собственное подмножество $\Q$. Оно счетно и всюду плотно. Но главное --- оно очень удобно устроено.
При $n=0$ мы имеем целочиселнную решетку на числовой прямой, при $n=1$ мы получаем все целые и полуцелые числа, при $n=2$ --- все числа с шагом $1/4$. Обозначим
$$
\B_n=\{k/2^n\mid k\in\Z\}.
$$
Тем самым мы определяем некоторый слой в множестве $\B$ с фиксированным шагом, равным $1/2^n$.
\item Множества $\B_n$ хороши тем, что образуют возрастающую последовательность по вложению, которая стартует с $\Z$ и заканчивается $\B$:\index{Цепь множеств}
$$
\Z=\B_0\subset\B_1\subset\B_2\subset\dots\subset\B
$$
В таком случае принято называть множество $\B$ \textbf{пределом возрастающей цепи множеств}.\index{Предел!цепи множеств}
\item Поскольку расстояние между соседними точками $\B_n$ очень быстро сокращается с ростом $n$, то любая точка на прямой может быть сколь угодно точно приближена точками множества $\B$.
\item Процесс последовательного приближения произвольной точки $A$ на числовой прямой можно осуществить следующим образом.
\begin{enumerate}[{\bf Step1}]
\item Находим целое число $k$ такое, что точка $A$ лежит в полуинтервале $[k;k+1)$, т.е. либо между соседними целыми числами, либо совпадает с целым числом. Очевидно, что такое $k$ определяется однозначно и всегда существует. Обозначим отрезок $[k;k+1]$ за $\De_0$. Ясно, что его границы $k$ и $k+1$ есть элементы множества $\B_0$.
\item Пусть далее $n$ --- это номер текущего интервала (начиная с нуля).
\item Находясь в отрезке $\De_n$, делим его на 2 части посередине так, чтобы получилось два одинаковых полуинтервала: если $\De_n=[a;b]$, то новые полуинтервалы будут $[a;(a+b)/2)$ и $[(a+b)/2;b)$. Точка $A$ лежит в одном из этих полуинтервалов: либо в левом, либо в правом, третьего не дано.

Заметим, что и границы $\De_n$, и его середина --- это точки множества $\B$, причем границы $\De_n$ находятся в множестве $\B_n$, а его середина --- в множестве $\B_{n+1}$. Таким образом, это подразбиение является переходом к следующему уровню разбиения в множестве $\B$.
\item Тогда через $\De_{n+1}$ обозначим отрезок $[a;(a+b)/2]$, если $A\in[a;(a+b)/2)$, и отрезок $[(a+b)/2;b]$, если $A\in[(a+b)/2;b)$. После чего перейдем на предыдущий шаг, увеличивая номер $n$ на 1.
\end{enumerate}

\begin{center}
\includegraphics[scale=0.3]{Step1-3.png}
\end{center}

В результате мы получим последовательность вложенных отрезков\index{Вложенные отрезки}
$$
[k;k+1]=\De_0\supset\De_1\supset\De_2\supset\dots
$$
Эта последовательность монотонно убывает, причем на каждом шаге отрезок становится вдвое короче, а концы отрезков прыгают по точкам множества $\B$, постепенно переходя ко все более мелкой сетке --- от $\B_n$ к $\B_{n+1}$.

\item Где же в итоге окажется точка $A$?
\item Поскольку $A\in\De_n$ для всех $n$, то она также лежит в пересечении всех этих отрезков:
$$
A\in\bigcup_n\De_n.
$$
Такое пересечение называется пределом убывающей цепи множеств.
\item Может ли в этом пересечении лежать еще какая-то точка? Ответ: нет. Если мы возьмем любую другую точку $B\ne A$, то, очевидно, что между ними есть какое-то расстояние $\ep>0$. Возьмем тогда такое $n$, что $\ep>1/2^n$, и посмотрим на отрезок $\De_n$. Еого длина равна $1/2^n$ и он содержит точку $A$. Но тогда он не содержит точку $B$, а значит, и пересечение всех отрезков $\De_n$ не содержит точку $B$.
\item Итак, точка $A$ --- единственный представитель пересечения отрезков $\De_n$:
$$
\bigcap_n\De_n = \{A\}.
$$
\item По сути, мы уже сформулировали главный принцип непрерывности (полноты) числовой прямой --- принцип вложенных отрезков. Однако, здесь нужно проявить осмотрительность. Дело в том, что мы не доказали существование точки $A$, а сразу же выбрали ее из существующих точек прямой. Но в настоящий момент нам известны только рациональные числа и алгебраические, поэтому разумно ожидать, что точка $A$ есть одно из таких чисел. Но рассмотренный принцип <<ловли>> произвольной точки на прямой с помощью стягивающейся сетки двоично-рациональных чисел сам по себе является мощным инструментом для определения новых чисел и, что самое главное, для ликвидации всех возможных дыр на числовой оси, состоящей из алгебраических чисел.



\lesson{Потребность в аксиоме непрерывности. A1:Принцип вложенных отрезков}


\item На самом деле, вовсе не очевидно, что если мы выберем произвольную последовательность вложенных отрезков, длина которых стремится к нулю с ростом номера, то в пределе получим множество, состоящее из одной точки. Быть может, никакой точки там вовсе не окажется. Поэтому приходится вводить принцип непрерывности при помощи \textbf{аксиомы непрерывности} (она же --- аксиома полноты).\index{Аксиома!непрерывности (полноты)}
\item У этой аксиомы существует много равносильных формулировок, и мы начнем с той, к которой нас подготовил сюжет по поимке точки $A$ в сеть множества $\B$.

\subsection*{A1. Принцип вложенных отрезков}



\noindent
\textit{Пусть дана последовательность вложенных отрезков на прямой:\index{Вложенные отрезки}
$$
[a_0;b_0]\supseteq[a_1;b_1]\supseteq[a_2;b_2]\supseteq\dots[a_n;b_n]\supseteq\dots,
$$
где $a_n<b_n$ для всех $n$. Тогда множество точек, принадлежащих всем отрезкам одновременно, не пусто.}
\item В терминах, которые мы упоминали выше, принцип A1 можно переформулировать так: любая убывающая по вложению бесконечная цепь отрезков имеет непустой предел.\index{Принцип!вложенных отрезков}
\item Отметим, что в формулировке мы не требовали, чтобы концы отрезков были двоично-рациональными числами, а также не требовали, чтобы их длина стремилась к нулю. Очевидно, что случай с двоично-рациональными отрезками является частным случаем последовательности вложенных отрезков, а значит, в силу принципа \textbf{A1} имеет непустой предел.

На самом деле, если сформулировать принцип вложенных отрезков, используя только двоично-рациональные концы отрезков и деление отрезка пополам на каждом шаге, мы получим эквивалентную формулировку принципа вложенных отрезков. Но доказательство этого факта мы оставим за рамками курса.
\item Принцип вложенных отрезков уже позволяет нам доказать, что на числовой прямой существуют не только алгебраические числа, более того, что точек на прямой существует несчетное множество.

Предположим, что это не так, и пусть на прямой есть только счетный набор точек. В соответствии с определением счетности мы можем перенумеровать все эти точки натуальными числами $x_0,x_1,x_2,\dots$

Построим цепь вложенных отрезков следующим способом. Выберем любой отрезок $\De_0$ (можно считать, что он имеет концы в множестве $\B$, но это не обязательно) так, чтобы $x_0\notin\De_0$. Точка $x_1$ может лежать или не лежать в отрезке $\De_0$, но в любом случае мы можем выбрать отрезок $\De_1$ так, чтобы выполнялись условия: $\De_1\subseteq\De_0$ и $x_1\notin\De_1$. Заметим, что при этом также $x_0\notin\De_1$. Далее, точка $x_2$ может лежать или не лежать в отрезке $\De_1$, но мы всегда можем выбрать отрезок $\De_2\subseteq\De_1$ так, чтобы $x_2\notin\De_2$. Продолжаем эту процедуру до бесконечности, поддерживая следующую ситуацию:
$$
\De_{n+1}\subseteq\De_n,\quad x_0,\dots,x_{n+1}\notin\De_{n+1}.
$$

Но тогда в пересечении $\cap\De_n$ нет ни одной точки $x_n$, т.е. нет вообще ни одной точки числовой прямой. Но в силу принципа \textbf{A1} там должна быть хотя бы одна точка. Противоречие.

Таким образом, принцип вложенных отрезков гарантирует нам несчетность множества чисел на прямой. Насколько велико это множество, мы сможем оценить чуть позже.


\lesson{Монотонные последовательности. Ограниченные множества. Предел. A2:Предел монотонной ограниченной последователности}


\item Следующее, что можно отметить, опираясь на наш пример с поимкой точки $A$ в сеть множества $\B$, это что последовательность левых границ вложенных отрезков не убывает. На каждом шаге левая граница $\De_n$ либо остается такой же, как у предыдущего отрезка, либо перескакивает в его середину. Но при этом все левые границы отрезков $\De_n$ остаются ограниченными сверху правой границей начального отрезка $\De_0$. То же самое мы видим и в ситуации с произвольной последовательностью вложенных отрезков, которую мы описываоли при формулировке принципа \textbf{A1}.
\item Аналогичное наблюдение можно вывести и для правых концов вложенных отрезков. Мы имеем некую бесконечную монотонную последовательность точек, и притом ограниченную, т.е. находящуюся в некотором заранее известном отрезке.
\item Введем определения. Последовательность $\{x_n\}_{n=0}^\infty$ называется \textbf{убывающей} (или \textbf{возрастающей}), если для всех $n$ выполняется неравенство $x_n\ge x_{n+1}$ ($x_n\le x_{n+1}$). Убывающие и возрастающие последователности называются \textbf{монтонными}. Последовательность \textbf{строго} монотонна (строго убывающая или строго возрастающая), если указанное неравенство всегда строгое.\index{Последовательность}\index{Последовательность!монотонная}

Множество $X$ на прямой называется \textbf{ограниченным сверху}, если существует точка $a$ такая, что $X\le a$.\footnote{Мы ранее уже вводили сравнение множеств и множеств с точкой. $X\le Y$, если для всех $x\in X,y\in Y$ имеем $x\le y$. $X\le a$, если $X\le\{a\}$. } Если ситуация противоположная, т.е. $X\ge a$, то множество $X$ называется \textbf{ограниченым снизу}. Если множество ограничено сверху и снизу, то но называется \textbf{ограниченным}.\index{Множество!ограниченное}

Последовательность ограничена (сверху и/или снизу), если ограничено ее множество значений (сверху и/или снизу). отметим, что последовательность мы рассматриваем не просто как множество точек на прямой, а как функцию из $\N$ в множество точек прямой или любое другое множество. Это позволяет рассматривать, например, стационарные последовательности, когда $x_n=\const$, или циклические последовательности, когда $x_n$ принимает конечный набор значений, последовательно повторяя их. Например, $x_n=n\pmod m$ повторяет значения $0,1,\dots,m-1$.

\item Имея любую монотонную ограниченную последовательность, мы легко можем выстроить цепь вложенных отрезков. Если эта последовательность неубывающая, то в качестве левых границ отрезков берем ее элементы, а правую границу зафиксируем в какой-то одной точке, которая точно больше всех членов последователности (ее существование следует из ограниченности последовательности).
\item Можем усилить эффект, если в качестве правых границ вложенных отрезков на $n$-ом шаге выбирать наименьшее из чисел множества $\B_n$, превосходящих все оставшиеся члены последовательности.
$$
a_n=x_n,\quad b_n=\min\{q\in\B_n\mid \forall j\ge n\;x_j\le q\}
$$
В этом случае последовательность $b_n$ будет убывающей, причем она будет очень быстро сближаться с <<хвостом>> последовательности $\{x_n\}$, т.к. шаг между соседними числами в множетсве $\B_n$ равен $1/2^n$. И тогда длина отрезка $[a_n;b_n]$ с каждым шагом будет становиться все меньше, так что в предельном множестве, существование которого нам гарантирует принцип \textbf{A1}, не сможет находиться две и более точек. А та единственная, которая останется внутри пересечения всех $[a_n;b_n]$, будет пределом последовательности $\{x_n\}$, т.е. такой точкой, к которой данная последовательность приближается вплотную, не оставляя никакого зазора.
\item Введем одно из основных понятий математического анализа. Число $a$ называется \textbf{пределом последовательности} $\{x_n\}$ и обозначается\index{Предел!последовательности}
$$
a = \lim_{n\to\infty}x_n,
$$
если для любого $\ep>0$ существует такой номер $N$, что для всех $n>N$ имеет место неравенство $|x_n-a|<\ep$. Если последовательность $\{x_n\}$ имеет предел, то она называется \textbf{сходящейся} (к данному пределу), в противном случае --- \textbf{расходящейся}.\index{Последовательность!сходящаяся}

Назовем интервал $(a-\ep;a+\ep)$ $\ep$--окрестностью точки $a$. Ясно, что утверждение $|x_n-a|<\ep$ означает, что $x_n$ лежит в $\ep$--окретности точки $a$. Кроме того, очень часто говорят <<почти все члены последовательности>>, когда имеют ввиду некоторый ее хвост, т.е. ту же последовательность, но за исключением, быть может, какого-то ее конечного начального отрезка. Поэтому тот факт, что $a$ является пределом последовательности $\{x_n\}$, можно записать следующими словами: \textit{в любой сколь угодно малой окрестности точки $a$ лежат почти все члены последовательности} $\{x_n\}$.

Стоит отметить, что символика пределов в обязательном порядке предписывает указывать, при каком именно изменении параметра совершается предельный переход. В нашем случае параметром последовательности является индекс (или номер) ее членов, т.е. число $n$. И если еще раз внимательно прочитать определение предела, а также смысл фразы <<почти все члены последовательности>>, то мы увидим, что требование малого отклонения от предела выполняется при больших $n$, т.е. при $n>N$ при некотром номере $N$, который, вообще говоря, зависит от выбранного $\ep$. И весь смысл данного предела в том, что при забегании индекса $n$ в сторону бесконечности величина $x_n$ становится близкой к $a$.

Кроме того, параметров может быть несколько, и поэтому всегда следует указывать, относительно какого из них осуществляется предельный переход.

Приведем пример. Пусть $x_{n,m}=m/(1+(n-m)^2)$. Найдем ее предел при $n\to\infty$. Мы можем построить экспериментальный график и убедиться в том, что предел равен нулю.
\begin{center}
\includegraphics[scale=0.5]{limits.png}

График $\displaystyle x_{n,m}=\frac{m}{(1+(n-m)^2)}$.
\end{center}

Докажем, что так оно и есть на самом деле. Выберем произвольный достаточно малый $\ep>0$. Поскольку $x_{n,m}>0$, нам достаточно установить, при каких $n$ выполняется неравенство $x_{n,m}<\ep$, что и будет означать близость к нулю. Решая это неравенство, находим, что
$$
n>m+\sqrt{\frac{m}{\ep}-1},
$$
откуда видно, что начиная с некоторого $N$ (например, можно округлить вверх корень и добавить $m$) для всех последующих $n$ требуемое неравенство выполняется. Стало быть, по определениею получаем, что
$$
\lim_{n\to\infty}\frac{m}{(1+(n-m)^2)}=0.
$$

Однако, даже по графику видно, что если мы будем менять параметр $m$ вместе с $n$, оставаясь на гребне волны графика, то мы увидим рост величины $x_{n,m}$.
Можно так подобрать зависимость параметров $n$ и $m$ (например, положить $m=kt^2$ и $n=kt^2+t$, тогда получим $x_{n,m}=kt^2/(1+t^2)\to k$ при $t\to\infty$), что $x_{n,m}$ будет сходиться к любому наперед заданному положительному числу или уходить на бесконечность.

Поэтому всегда очень важно указывать, при каком изменении какого параметра ищется предел.

\item Из рассуждений, проведенных выше относительно монотонной ограниченной последовательности, ясно, что она должна иметь предел в силу принципа \textbf{A1}. На самом деле, верно как это, так и обратное: если монотонные ограниченные последовательности имеют предел, то выполняется принцип вложенных отрезков. Это совсем легко установить, поскольку границы вложенных отрезков образуют две монотнные ограниченные последовательности, а значит, имеют пределы. Причем как эти пределы, так и точки, лежащие между ними, будут принадлежать пределу цепи вложенных отрезков.

\subsection*{A2. Предел монотонной последовательности}

\noindent
\textit{Всякая монотонная ограниченная последовательность имеет предел}.\footnote{Если быть точным, то еще требуется архимедовость системы чисел, в котрой этот принцип постулируется, но для числовой прямой, в которой множество $\Z$ не ораничено ни сверху, ни снизу, это выполняется автоматически.}

\item Принцип \textbf{A2} равносилен принципу \textbf{A1}.



\lesson{А3:Фундаментальная последовательность. Точные грани множеств. Принцип А4:Существование точных граней множеств. Принцип А5: Дедекиндовые сечения. Обозначение $\R$}


\item Следующее, что мы можем заметить из сюжета с поимкой точки $A$ сетью точек множества $\B$, это то, что границы отрезков не просто стремятся к точке $A$, но и как бы стремятся друг к другу, т.е. расстояния между всеми членами последовательности, начиная с некоторого номера, становятся сколь угодно малыми. Поэтому, даже ничего не зная о существовании предела, мы можем сделать некоторые выводы о последовательности.
\item Последовательность $\{x_n\}$ называется \textbf{фундаментальной}, если для любого $\ep>0$ существует такой номер $N$, что для всех номеров $n,m>N$ выполняется $|x_n-x_m|<\ep$. Иными словами, \textit{почти все члены фундаментальной последовательности находятся сколь угодно близко друг к другу}.\index{Последовательность!фундаментальная}



\subsection*{A3. Предел фундаментальной последовательности}

\noindent
\textit{Всякая фундаментальная последовательность имеет предел}.

\item Принцип \textbf{A3} равносилен двум предыдущим принципам.
\item Мы уже говорили о том, что множество может быть ограничено сверху или снизу. Пусть $X$ ограничено сверху числом $a$. Тогда число $a$ называется его \textbf{верхней гранью}. Ясно, что если $a$ --- верхняя грань $X$, то верхними гранями будут также $a+1, a+10000, a+0.0001$  и т.д. Все числа, большие $a$, будут верхними гранями $X$.
\item Вспомним уравнение $x^2=2$. Пусть $X=\{x\mid x^2<2\land x>0\}$. Это есть интервал $(0;\sqrt 2)$. Мы помним, что в $\Q$ нет числа $\sqrt 2$, поэтому в рамках множества $\Q$ верхними гранями $X$ будут положительные рациональные числа $r$ такие, что $r^2>2$. И среди этих чисел нет наименьшего, т.к. он недостижимо в $\Q$. Однако, стоит нам выйти в поле алгебраических чисел, как мы уже можем использрвать $\sqrt 2$, и он будет не просто верхней гранью $X$, а наименьшей их всех верхних граней.
\item Наименьшая из верхних граней $X$ называется \textbf{супремумом} $X$ или \textbf{точной верхней гранью} $X$.
\item Предлагаем читателю самостоятельно определить понятие точной нижней грани.\index{Грань множества верхняя и нижняя!точная}
\item К точной верхней грани $X$ мы можем подбираться, находясь внутри $X$, выбирая каждый раз все большее число из $X$, находящееся как можно ближе к его верхней грани. Например, у цепи вложенных отрезков есть множество левых границ, образующее возрастающую последовательность. При этом все правые границы отрезков будут верхними гранями для этой последовательности. И если в пределе получится множество, состоящее из одной точки (как в сюже с ловлей точки $A$ точками множества $\B$), то эта точка и будет точной верхней гранью для последовательности левых границ вложенных отрезков.
\item Мы снова видим некоторую связь между существованием супремума и аксиомой непрерывности в ее трех предыдущих формулировках. На самом деле, эта связь абсолютная.

\subsection*{A4. Существование точных граней}

\noindent
{\bf A4} \textit{Всякое ограниченное сверху множество имеет точную верхнюю грань.}

\noindent
{\bf A4'} \textit{Всякое ограниченное снизу множество имеет точную нижнюю грань.}
\item Эти два принципа непрерывности эквивалентны друг другу и трем предыдущим принципам.
\item Наконец, заметим, что существует некоторая двойственность между верхними и нижними гранями и точныи гранями. Так, если взять некоторое множество $X$, ограниченное сверху, то оно само будет множеством нижних граней (не обязательно всех) для множества $Y$ своих верхних граней. При этом окажется, что $\sup X=\inf Y$. Аналогичная ситуация и с ограниченным снизу множеством. Возникает желание разбить всю числовую прямую на два луча --- левый и правый, --- так, чтобы левый был множеством нижних граней для верхнего и наоборот.
\item В соответствии с определением, данным в разделе \ref{Ordering},
пара $(X,Y)$ непустых подмножеств, таких, что их объединение $X\cup Y=\Q$, $X\cap Y=\emptyset$ и $X<Y$, называется \textbf{сечением}.\index{Сечение Дедекинда}\footnote{Иногда определяется только нижний класс, и он называется сечением. Оба определения эквивалентны.}
\item Ранее мы уже видели такое разбиение. Оно представляло собой два интервала рациональных чисел: $(-\infty;\sqrt 2)$ и $(\sqrt 2;+\infty)$. Действительно, их объединение равно $\Q$, пересечение пусто, и левый интервал меньше правого.
\item В случае $\Q$ сечение может состоять их двух интервалов, т.е. таких множеств, что верхнее не имеет минимума, а нижнее не имеет максимума, и при этом между ними ничего нет. И это говорит нам о том, что в $\Q$ имеются дырки. В случае $\A$ дырки найти сложнее, но, например, разбиение на интервалы $(-\infty;\pi)$ и $(\pi;+\infty)$ доставляет такой пример, поскольку число $\pi$ не является алгебраическим.\footnote{Этот факт был доказан только в XX веке!}
\item Так вот, еще один подход к определению непрерывности числовой прямой заключаетсы в том, чтобы исключить такие дырки.

\subsection*{A5. Дедекиндовы сечения}

\noindent
\textit{Если $(X,Y)$ --- сечение числовой прямой, то существует точка $z$ такая, что $X\le z\le Y$.}
\item При этом точка $z$ обязана попасть либо в верхний, либо в нижний класс разбиения (ей просто деваться некуда), т.е. всякое сечение числовой прямой должно быть дедекиндовым (в соответствии с данным нами определением в разделе \ref{Ordering}). Иначе говоря, принцип \textbf{A5} утверждает, что линейный порядок на числовой прямой должен быть непрерывным.
\begin{figure}
\begin{center}
\includegraphics[scale=0.7]{dedekind.png}
\end{center}
\caption{Цитата из работы О.Дедекинда <<Непрерывность и иррациональные числа>> в пер. Шатуновского, 1923.}
\end{figure}
\item Формулировка аксиомы непрерывности в виде принципа \textbf{A5} эквивалентна всем предыдущим формулировкам \textbf{A1--A4}.
\item Наконец, самое главное определение данной главы. Числовая прямая, удовлетворяющая аксиоме непрерывности, называется \textbf{вещественной (действительной) прямой} и обозначается $\R$.

\item Если мы соберем вместе все накопленные свойства $\R$, то мы увидим, что $\R$ --- это \textit{непрерывное линейно упорядоченное поле}. Известно, что такое поле единственное с точностью до изоморфизма, сохраняющего  операции поля и согласованный с ними линейный порядок.\index{Действительные числа}\index{Числа!действительные}

\end{enumerate}


\subsection*{Задачи}
\begin{enumerate}
\item Доказать, что между любыми двумя рациональными числами $r\ne q$ лежит какое-то двоично-рациональное.
\end{enumerate}


\section{Модели действительных чисел}

\lesson{Построение $\R$ с помощью дедекиндовых сечений: сложение и умножение, порядок}

\begin{enumerate}
\item В предыдущем разделе было сформулировано пять вариантов аксиомы непрерывности, которая необходима для того, чтобы узаконить действительные числа как непрерывную числовую структуру. Аксиома непрерывности не оставляет дыр на числовой прямой, поскольку вводит в обращение все числа, к которым можно обратиться с помощью счетной последователньости рациональных чисел.
\item Тем не менее, одной лишь аксиомы недостаточно, чтобы действительные числа имели право на существование. Необходимо убедиться в том, что их можно непротиворечиво сконструировать. Поэтому здесь мы рассмотрим несколько подходов к построению действительных чисел.

\subsection*{Дедекиндовы сечения}

\item Первый подход связан непосредственно с тем, чем мы закончили предыдущий раздел --- с дедекиндовыми сечениями. А именно, рассмотрим все сечения множества рациональных чисел, причем только такие, у которых нижний класс не содержит наибольшего элемента (т.е. если есть между классами граница, то она отнесена к верхнему классу). И соберем в множество $R$ нижние классы всех таких сечений. Таким образом, $R$ состоит из подмножеств $X\subset\Q$ таких, что
$$
X\ne\emptyset,\quad X\ne\Q,\quad\forall r,q\in\Q\;(q\in X)\land(r<q)\to(r\in X),\quad\not\exists\max X.
$$
Таким образом, $R$ --- это множество лучей из рациональных чисел, направленных в $-\infty$.
\item На множестве $R$ введем операцию сложения: пусть $\al,\be\in R$, тогда положим
$$
\al+\be = \{x+y\mid x\in \al,y\in \be\},
$$
т.е. просто сложим их по Минковскому.
\item Можно проверить, что $R$ с такой операцией сложения является абелевой группой, т.е. сложение ассоциативно, коммутативно, имеется нейтральный элемент $0=\{x\in\Q\mid x<0_\Q\}$, где $0_\Q$ --- ноль в системе рациональных чисел, и для каждого $\al$ имеется противоположный элемент $-\al=\{x\in\Q\mid (-x>\al)\land (-x\ne\min\Q\setminus\al)\}$.

Здесь появляется первая тонкость определения. По сути, в качестве $\al$ мы берем соответствующий ему верхний класс сечения и умножаем на -1 (в поле рациональных чисел). Однако верхний класс может иметь минимум, а элемент $R$ не должен иметь максимума, поэтому мы подстраховываемся и выбрасываем из верхнего класса $\Q\setminus\al$ его минимум (если такой существует).

\item Достаточно легко определяется и порядок на $R$. Скажем, что $\al<\be$, если $\al\subset\be$ (как собственое подмножество).
\item Отсюда же следует согласованность сложения и порядка, поскольку сдвиг вверх или вниз интервалов $\al<\be$ не меняет их вложенности.
\item Сложнее дело обстоит с умножением. Если мы попытаемся умножить $\al\cdot\be$ по Минковскому, то произведение будет содержать сколь угодно большие числа (при перемножении двух чисел, сильно меньших 0 в поле $\Q$). Поэтому сначала определяется произведение положительных чисел: пусть $\al>0$ и $\be>0$, тогда
$$
\al\cdot\be = \{xy\mid (x\in\al)\land(x\ge 0)\land(y\in\be)\land(y\ge 0)\}\cup\Q^-,
$$
где $\Q^-$ --- все отрицательные рациональные числа.

Далее, просто полагаем, что
$$
\al\cdot\be = 
\begin{cases}
0           ,&\mbox{ если }(\al=0)\lor(\be=0) \\
(-\al)\cdot\be,&\mbox{ если }(\al<0)\land(\be>0) \\
\al\cdot(-\be),&\mbox{ если }(\al>0)\land(\be<0) \\
(-\al)\cdot(-\be),&\mbox{ если }(\al<0)\land(\be<0)
\end{cases}
$$
\item Можно проверить, что такая операция умножения на $R$ ассоциативна, коммутативна, имеет единицу $1=\{x\in\Q\mid x<1_\Q\}$, где $1_\Q$ --- единица в системе рациональных чисел, и для каждого положительного $\al\in R$ имеется обратный по умножению
$$
1/\al = \{x\in\Q\mid \exists y>\al\;(x<1/y)\},
$$
а обратный к отрицательному числу определяется сменой знака: $1/\al=-(1/(-\al))$, если $\al<0$.
\item Кроме того, можно доказать, что операции сложения и умножения удовлетворяют дистрибутивному закону, а также что умножение согласовано с порядком.
\item Таким образом, $R$ с указанными операциями и порядком есть упорядоченное поле. Остается показать его непрерывность.
\item Для этого воспользуемся формулировкой аксиомы непрерывности в виде \textbf{A4}. Пусть $X\subset R$ непусто и ограничено сверху. Положим $\al=\cup X$. Т.е. мы включаем в множество $\al$ все рациональные числа входящие во все элементы множества $X$. Легко видеть, что $X\le\al$ (в смысле сравнения множества и числа в лнейном порядке на $R$). В то же время любая верхняя грань $X$ окажется не меньше $\al$, иначе она бы отсекла какое-то рациональное число одного из элементов $X$. Таким образом,
$$
\sup X=\cup X.
$$
\item Итак, множество $R$, построенное из подмножеств $\Q$ специального вида, с заданными на нем операциями и порядком, является непрерывным упорядоченным полем, т.е. полем действительных чисел $\R$.


\lesson{Модель бесконечных двоичных дробей без хвоста единиц: порядок определяется легко, а сложение и умножение сложно, ркурсивно. Замечание о десятичных дробях}


\subsection*{Двоичные дроби}



\item Следующий подход к моделированию $\R$ прямо связан с нашим множеством $\B$. Вспомним сюжет о поимке точки $A$ в сеть точек множества $\B$, т.е. рациональных точек со знаменателями вида $2^n$. Мы выстраивали убывающую цепь отрезков, концы которых находятся в множестве $\B_n$, т.е. имеют вид $[k/2^n;(k+1)/2^n]$. Длина этих отрезков быстро стремится к нулю, а по аксиоме непрерывности в форме \textbf{A1} существует непустой предел этой цепи отрезков, который состоит из единственной точки $A$.
\item Почему бы тогда не обращаться к точке $A$ с помощью этой последовательности? Точнее, мы определим некоторый условный код, который позволит нам однозначно поймать точку $A$ и никакую другую.
\item Вспомним алгоритм построения этих отрезков. Сначала мы выбираем целочисленный полуинтервал $[k;k+1)$, в котором лежит $A$. Ок --- запишем число $k$ как стартовое число кода.
\item Затем мы делим этот интервал ровно пополам: $[k;k+1)=[k+1/2)\cup[k+1/2;k+1)$. Точка $A$ лежит либо в левом полуинтервале, либо в правом. Ок, следующим числом кода запишем 0, если $A$ лежит в левом интервале, и 1 --- если в правом. Перйдем к соответстующему интервалу.
\item Снова поделим его пополам и произведем аналогичную процедуру записи следующей цифры кода. И так будем продолжать до бесконечности.
\item В итоге у нас получится код, стартующий с некоторого целого числа, после которого идет бесконечная (счетная) цепочка нулей и единиц. Этот код однозначно формируется по заданной точке $A$ (поскольку мы всегда работаем с полуинтервалами, а они всякий раз выбираются единственным способом).
\item Особенностью данного кода является то, что в нем нет хвоста единиц, т.е. когда начиная с некотрой позиции все цифры равны 1. Это объясняется очень просто: если есть хвост единиц, то начиная с какого-то шага алгоритм всегда выбирал правый интервал, в результате чего хвост последовательности вложенных отрезков имел бы вид
$$
[r/2^m-1/2;r/2^m]\supset[r/2^m-1/4;r/2^m]\supset[r/2^m-1/8;r/2^m]\supset\dots,
$$
где $r$ и $m$ не зависят от $n$. Но пределом такой цепи будет, очевидно, множество $\{r/2^m\}$, т.е. такая цепь вложенных отрезков должна сходиться к числу $r/2^m$. Проблема в том, что алгоритм еще на предыдущем $m-1$ шаге выберет полуинтервал, лежащий справа от этой точки, в резульате чего отрезками, сходящимися к точке $r/2^m$, будут такие
$$
[r/2^m;r/2^m+1/2]\supset[r/2^m;r/2^m+1/4]\supset[k/2^m;k/2^m+1/8]\supset\dots,
$$
и мы увидим не хвост единиц, а хвост нулей! Правда, перед ним будет стоять единица.

То есть, кодовая последовательность вида $k,[01]*0111111\dots$ невозможна, а вместо нее будет последовательность $k,[01]*1000000\dots$. Здесь символ $[01]*$ представлят собой \textit{регулярное выражение}, означающее цепочку произвольной конечной длины (в том числе нулевой длины), состоящую только из символов 0 и 1.
\item Верно и обратное. По такому коду (без хвоста единиц) можно однозначно восстановить закодированную им точку $A$.
\item Поэтому между точками вещественной прямой $\R$ и кодами указанного вида существует взаимно однозначное соответствие (биекция), и это значит, что моделью $\R$ может быть множество всех таких цепочек.
\item Точнее, положим
$$
R=\{(k,f)\mid k\in\Z, f:\N\to\{0,1\}, \forall n\exists m>n\;(f(m)=0)\}.
$$
Здесь условие $\forall n\exists m>n\;(f(m)=0)$ как раз и означает, что в последовательности $f$ нет хвоста единиц (ноль встречается бесконечно часто).

\item Сложности в такой модели $\R$ начинаются, когда мы хотим определить операции сложения и умножения. Порядок же оределяется предельно просто. Пусть даны две последователности $(k,f)$ и $(k',f')$. Отношение порядка между ними основано на сравнении первого расхождения кодов. Если $k<k'$, то $(k,f)<(k',f')$. Если $k=k'$, смотрим $f(0)$ и $f'(0)$. Если $f(0)<f'(0)$, то $(k,f)<(k',f')$. Если они равны, то переходим к следующей цифре кода, и т.д. Если не нашлось ни одного расхжодения в коде, то числа , то $(k,f)$ и $(k',f')$ равны.
\item Мы не будем здесь заниматься рекурсивным определением операций сложения и умножения. Скажем только, что его можно задать, используя арифметику двоично-рациональных чисел множества $\B$, и во многом он напоминает определение операций в следующей модели $\R$, основанноя на классах экивалентных последовательностей. Действительно, ведь двочиный код задает не только алгоритм вычисления вложенных отрезков, он задает последовательность их левых границ, которая сходится к адресуемому числу $A$. А это --- последовательность двоично-рациональных чисел. которые мы умеем складывать и умножать, не выходя за рамки множества $\B$.
\item Больше того, число, которое закодировано парой $(k,f)$, можно записать в виде бесконечной суммы
$$
A = k+\sum_{n=0}^\infty\frac{f(n)}{2^n},
$$
поскольку переход к правому отрезку на $n$-ом шаге в описанном алгоритме означает добавление $1/2^n$ к левой границе предыдущего отрезка, а переход к левому отрезку означает добавление $0/2^n$. Так что любое действительное число можно записать в виде разложения по степеням 2, а это и есть не что иное как записать произвольного числа в двоичной системе счисления. Число $k$ при этом можно тоже записать в двочином коде, и тогда код произвольного числа будет иметь вид: конечный набор нулей и единиц, затем стоит точка, затем идет бесконечный набор нулей и единиц (без хвоста единиц).
\item Завершая описание двоичной модели, скажем, что в качестве основания можно выбрать любое натуральное число $d>1$. Например, если мы хотим получить троичные последовательности, нам следует модифицировать алгоритм разбиения на отрезки следующим образом: интервал $[k;k+1)$ делить на три части $[k;k+1/3)$, $[k+1/3;k+2/3)$ и $[k+2/3;k+1)$, и далее к каждому следующему интервалу применять аналогичное деление на 3 части. В результате для записи кода будем выбирать 0, если $A$ оказалась в левом интервале, 1 --- если в среднем, 2 --- если в правом. И получим код из цифр 0,1,2, причем без хвоста двоек. Все рассуждения здесь полностью аналогичны предудщему.
\item Мы можем использовать число $d=10$ в качестве основания, и каждое действительное число записывать кодом из цифр $0\dots 9$ без хвоста девяток (хвост девяток всегда можно заменить хвостом нулей, увеличив стоящую перед девятками цифру на 1).


\lesson{Мощность множества $\R$. Двоичные дроби интервала $[0;1)$ и $2^\N$. Биекция из $\R$ в $[0;1)$}


\item Двоичное представление вещественных чисел открывает нам возможность оценить мощность множества $\R$, а точнее, полуинтервала $[0;1)$. Всякое число $\al\in[0;1)$ имеет код, заданный функцией $f:\N\to\{0,1\}$, т.е. мощность множества вещественных чисел в полуинтервале $[0;1)$ равна мощности множества таких функций без хвоста единиц.

С другой стороны, всякая функция вида $f:\N\to\{0,1\}$ взаимно однозначно задает некоторое подмножество в $\N$. Нужно в этом подмножестве собрать только те элементы, на которых $f=1$.

Это значит, что мы можем построить инъекцию $F:[0;1)\to\Pcal(\N)$.

Теперь по произвольному подмножеству $\N$ построим функцию $f:\N\to\{0,1\}$. Такая функция может содержать хвост единиц. Но теперь мы этот код будем рассматривать не как двоичный, а как троичный! У нас гарантированно не будет хвоста двоек, а значит, мы инъективно построим какие-то числа в $[0;1)$ (точнее, даже в $[0;2/3)$, причем с очень многими дырами). Тем самым, мы имеем инъекцию $G:\Pcal(\N)\to[0;1)$.

Окончательно, по теореме Кантора--Берштейна мы получаем равномощность множеств $[0;1)$ и $\Pcal(\N)$. То есть, интервал $[0;1)$ имеет мощность континуума!
\item Чтобы перейти к $\R$, нужно сначала научиться строить биекцию между интервалом и полуинтервалом.

Легко видеть, что функция
$$
f(x) = \begin{cases}
1/2, & x=0 \\
x/2, & x=1/2^n,\; n=1,2,\dots \\
x, & \mbox{иначе}
\end{cases}
$$
биективно переводит $[0;1)$ в $(0;1)$. Все точки вида $1/2^n$ сдвигаются вниз на 1 шаг, а ноль переходит в точку $1/2$.

Далее, функция $g(x)=2x-1$, очевидно, биективно переводит $(0;1)$ в $(-1;1)$.

Наконец, функция
$$
h(x)=\begin{cases}
\frac{x}{1-x}, & 0\le y<1 \\
\frac{x}{1+x}, & -1<y<0 
\end{cases}
$$
биективно переводит интервал $(-1;1)$ в $\R$. График функции, обратной к $h(x)$, представлен на рисунке ниже:
\begin{center}
\includegraphics[scale=0.5]{interval.png}
\end{center}

Таким образом, композиция $h(g(f(x)))$ биективно переводит $[0;1)$ в $\R$. 
Следовательно, множество вещественных чисел имеет мощность континуума.


\lesson{Еще одна модель $\R$: классы эквивалентных последовательностей: сложение, умножение и сравнение определить легко. Замечание про континуум-гипотезу: Гёдель доказал, что отсутствие промежуточных мощностей непротиворечиво, Коэн --- что наличие промежуточных мощностей непротиворечиво. Сравнение с Пятым постулатом Евклида}


\subsection*{Эквивалентные последовательности}

\item Наконец, рассмотрим еще один способ конструирования множества $\R$.
\item Обозначим за $Q$ множество всех фундаментальных последовательностей рациональных чисел. Напомним, что последовательность фундаментальная, если почти все ее члены лежат в сколь угодно малой окрестности. На множестве $Q$ введем отношение следующим образом:
$$
q\sim r\Leftrightarrow \lim_{n\to\infty}(q_n-r_n) = 0.
$$

Вспоминая аксиому непрерывности в форме \textbf{A3}, мы понимаем, что если $q\sim r$, то эти последовательности имеют одинаковый предел. Отсюда легко видеть, что отношение $\sim$ является отношением эквивалентности, а значит, мы можем разбить $Q$ на классы эквивалентных последовательностей, т.е. построить фактормножество $R=Q/\sim$.

Вот это множество мы и объявляем множеством действительных чисел.

\item После чего мы должны ввести соответствующие операции и отношение сравнения.
\item Сложение классов эквивалентности вводится c помощью их представителей:
$$
[q]+[r] = [q+r].
$$
Необходимо лишь доказать, что если $q\sim q'$ и $r\sim r'$, то $q+r\sim q'+r'$. Это легко заметить из следующего неравенства
$$
|(q+r)_n-(q'+r')_n| \le |q_n-q'_n| + |r_n-r'_n|,
$$
посколку два модуля справа стремятся к нулю.
\item Аналогично вводится умножение:
$$
[q]\cdot[r] = [qr]
$$
Для доказательства корректности определения заметим, что если $q\sim q'$ и $r\sim r'$, то
$$
|q_nr_n-q'_nr'_n| \le |q_n||r_n-r'_n| + |r'_n||q_n-q'_n|.
$$
Здесь справа стоят слагаемые, в каждом из которых ограниченная величина (в силу фундаментальности) умножается на величину, стремящуюся к нулю. Так что и все вместе стремится к нулю.
\item Наконец, о сравнении:
$$
[q]<[r], \mbox{ если }\exists\ep>0\;\exists N\;(\forall n>N) (q_n<r_n-\ep),
$$
т.е. для почти всех индексов разность $r_n-q_n$ отделена от нуля положительным числом $\ep$ (оно может быть очень маленьким, но не нулевым).
\item Итак, мы видим, что при определении $\R$ через эквивалентные классы фундаментальных последовательностей и операции, и отншение чисел просто переносятся один-в-один с рациональных чисел. Главная задача тут --- показать корректносьт такого определения. Кроме того, здесь мы активно пользуемся понятием предела.
\item Выше мы рассмотрели три модели $\R$:
\begin{enumerate}[M1]
\item Модель дедеиндовых сечений.
\item Модель двоичных (в общем случае $d$-ичных) дробей.
\item Модель классов фундаментальных последовательностей.
\end{enumerate}
\item Попутно мы установили раномощность $\R$ и множества всех подмножеств $\N$.

\hard{
\item Возникает вопрос: если $\R$ такое большое множество, а $\Q$ такое маленькое (по мощности), есть ли какие-то множества, имеющие промежуточные мощности между счетной и континуумом? Ответ на этот вопрос дали два человека: К.Гёдель и П.Коэн. Первый доказал, что отсутствие промежуточных мощностей не противоречит аксиоматике теории множеств, второй --- что существование таких мощностей также не противоерчит аксиоматике теории множеств. Таким образом, мы оказываемся в ситуации пятого постулата Евклида, когда можем принимать или отвергать континуум-гипотезу\index{Континуум-гипотеза} (именно так называется утверждение о том, что между счетной мощностью и континуумом нет промежуточных мощностей), не опасаясь получить противоречие.}
\end{enumerate}


\subsection*{Задачи}

\begin{enumerate}
\item Дать определение вещественного числа $\sqrt[3]{-5}$ с помощью дедекиндового сечения, т.е. построить соответствующую пару подмножеств множества $\Q$.
\end{enumerate}

\newchapter[математического анализа]{Элементы}


\vrezka{Здесь мы познакомимся с базовой терминологией и методиками анализа функций вещественных чисел.}


\section{Оценки и пределы}

\lesson{Определение предела, $1/2^n\to 0$, арифметические свойства пределов. Лемма о двух милиционерах.}


\begin{enumerate}
\item Выше мы ввели общее понятие предела числовой последовательности: число $a$ называется пределом последовательности $\{x_n\}\subset\R$, если для любого $\ep>0$ найдется такой номер $N$, что для всех $n>N$ имеем $|x_n-a|<\ep$. Предел $a$ в этом случае обозначается\index{Предел!последовательности}
$$
a=\lim_{n\to\infty}x_n.
$$
Очень часто используется более простая нотация в виде $x_n\to a$, и если ясно из контекста, по какому параметру берется предел (в данном случае при $n\to\infty$), то больше ничего не дописывают. О подводных камнях такой нотации мы уже поговорили чуть выше.

\item Для отыскания пределов существует масса прямых и косвенных методов. Прямой, т.е. по определению, предполагает в явном виде выписать номер $N$, зависящий от переменной $\ep$, либо же доказать существование такого номера, отправляясь от каких-то известных фактов о строении множества $\R$. Приведем простой пример. Как доказать, что $1/2^n\to 0$?

Возьмем произвольный $\ep>0$. Необходимо показать, что выбором достаточно большого $n$ величину $1/2^n$ можно сделать меньше $\ep$. Для начала заметим, что число $1/\ep$ хоть и огромное (при маленьких $\ep$), но все-таки конечное. А это значит, что существует $n>1/\ep$ (здесь неявно работает такое свойство вещественных чисел, которое называется \textit{архимедовостью}). А далее, как легко видеть, $2^n>n>1/\ep$ (то, что $2^n>n$, доказывается индукцией), откуда уже по арифметическим правилам следует, что $\ep>1/2^n$. Следовательно, $1/2^n\to 0$.

\item К косвенным методам можно отнести арифметику пределов.
\begin{enumerate}[\bf {Lim}1]
\item Если $x_n\to a$ и $y_n\to b$, то $x_n+y_n\to a+b$.

Действительно, для любого $\ep>0$ найдутся номера $N_1$ и $N_2$ такие, что при $n>N_1$ имеем $|x_n-a|<\ep/2$ (поскольку $\ep$ произвольный, почему бы не взять его половинку?) и при $n>N_2$ имеем $|y_n-b|<\ep/2$. Далее, пользуясь неравенством треугольника для модуля, получаем, что
$$
|x_n+y_n-(a+b)|\le|x_n-a|+|y_n-b|<\ep/2+\ep/2=\ep
$$
при $n>N=\max(N_1,N_2)$.

\item Если $x_n\to a$ и $y_n\to b$, то $x_ny_n\to ab$.

Здесь --- аналогичные рассуждения. Выберем $n$ такие, чтобы было
$$
|x_n-a|,|y_n-b|<\frac{\ep}{|a|+|b|+1}\mbox{ и одновременно }|x_n-a|,|y_n-b|<1,
$$
где, опять же, деление $\ep$ на константу никак не ограничивает его произвольного выбора, а дальше
\begin{align*}
|x_ny_n-ab|=|x_ny_n-x_nb+x_nb-ab| \le |x_n||y_n-b|+|b||x_n-a|< \\
<\frac{(|a|+1)\ep}{|a|+|b|+1}+\frac{|b|\ep}{|a|+|b|+1}=\ep.
\end{align*}

Тут весь фокус заключаетсч в том, что $\ep$ для результирующей последовательности $x_ny_n$ выбирается произвольный, а $\ep$ из определения сходимости $x_n$ и $y_n$ выбирается уже на основе исходного $\ep$ так, чтобы потом в итоге получилось то, что требует определение предела.
\item Если $x_n\to a$ и $y_n\to b\ne 0$, то $x_n/y_n\to a/b$.
\item Если $x_n\to a$ и то $kx_n\to ka$ при любом фиксированном $k$.

Заметим, что слово <<фиксированный>> означает, что $k$ не меняется при изменении параметра $n$, по которому берется предел.
\end{enumerate}

\item Другие косвенные методы нахождения пределов связаны с различными оценками последовательности.
\begin{lem}[о двух милиционерах]\index{Лемма!о двух милиционерах}
Если последовательности $\{x_n\}$, $\{a_n\}$ и $\{b_n\}$ таковы, что $a_n\to x_0$ и $b_n\to x_0$, и, кроме того, $a_n\le x_n\le b_n$, то $x_n\to 0$
\end{lem}
Это легко доказать, пользуясь неравенствами для модулей, но заметим, что этот факт уже вполне очевиден из тех построений, которые мы проводили при определении вещественных чисел и описании аксиомы непрерывности.

Таким образом, если нам удается зажать последовательнось между двумя сходящимися к одному и тому же числу последовательностями, то мы легко находим ее предел.



\lesson{$x^n/n!\to 0$. Порядок малости, $o()$. Примеры с графиками}


\item Например, рассмотрим величину $x^n/n!$ при фиксированном $x>0$. Очевидно, что снизу она оценивается последовательностью, тождественно равной нулю ($a_n=0$), а для оценки сверху заметим, что начиная с некоторого номера $N$ будет верно неравенство $n>2x$, так что
$$
\frac{x^n}{n!}=\frac{x\cdot x\dots x}{1\cdot 2\dots n}=\frac{x^N}{N!}\left(\frac{x}{N+1}\dots\frac{x}{n}\right)<
\frac{x^N}{N!}\left(\frac 12\right)^{n}/(1/2)^N.
$$
В итоге мы имеем некое постоянное число $k=x^N/n!/(1/2)^N$, а также последовательность $1/2^n$. Их произведение стремится к нулю, так что
$$
b_n=\frac{x^N}{N!}\left(\frac 12\right)^{n}/(1/2)^N\to 0.
$$
Но поскольку 
$$
0\le\frac{x^n}{n!}\le b_n,
$$
заключаем, что и $x^n/n!\to 0$. На графике можно проследить, как быстро это происходит.
\begin{center}
\includegraphics[scale=0.5]{fact.png}
\end{center}


\item Наконец, очень важен такой метод, как оценка порядка малости. Если мы складываем две положительные последовательности $x_n+y_n$, причем $x_n\to a$, $y_n/x_n\to 0$, то сумма $x_n+y_n$ ведет себя ровно так же, как $x_n$, поскольку $y_n$ вносит бесконечно малый вклад в сумму в сростом $n$.

Действительно,
$$
x_n+y_n=x_n(1+y_n/x_n),
$$
где выражение в скобках стремится к 1, т.к. $y_n/x_n\to 0$, а тогда по правилу умножения пределов плоучаем, что
$$
\lim_{n\to\infty} (x_n+y_n)=\lim_{n\to\infty} x_n.
$$

Например, если у нас имеется конечная сумма вида $ax_n+bx_n^2+cx_n^3$ и при этом $x_n\to 0$, то слагаемые, содержащие $x_n^2$, $x_n^3$ и т.д., можно отбросить при нахождении предела, т.к. $x_n^k/x_n=x_n^{k-1}\to 0$ при $k\ge 2$, и получаем, что
$$
\lim_{n\to\infty}(ax_n+bx_n^2+cx_n^3) = \lim_{n\to\infty} (ax_n).
$$

Этот прием (отбрасывание бесконечно малых слагаемых) характерен при получении пределов, связанных со сложными комбинаторными оценками, где количество слагаемых постоянное, а их порядок поддается порядковой оценке.


\lesson{$n^k/a^n\to 0$. Пример с графиками}


\item Покажем, например, что $n^k/a^n\to 0$ при любом фкисированном $k\in\N$ и любом фиксированном $a>1$.

Рассмотрим отношение следующего члена последовательности к предыдущему:
$$
\frac{(n+1)^k/a^{n+1}}{n^k/a^n} = \frac 1a\left(1+\frac 1n\right)^k.
$$
Рассмотрим второй множитель. Если раскрыть скобки, то мы получим выражение
$$
\left(1+\frac 1n\right)^k = 1 + \frac kn + \frac{k(k-1)}{2n^2} + \dots + \frac{1}{n^k}.
$$
Здесь, как мы видим, имеет конечное постоянное число слагаемых, у которых порядок бесконечно малый по сравнению с первым слагаемым, т.е. с единицей: они все имеют вид $\al/n^j$ и стремятся к нулю с ростом $n$. Следовательно,
$$
\left(1+\frac 1n\right)^k \to 1.
$$
Но тогда начиная с какого-то номера $N$ этот множитель будет меньше, чем $1+(a-1)/2$. И тогда для $n>N$ будем иметь
$$
\frac{(n+1)^k/a^{n+1}}{n^k/a^n} < \frac{a+1}{2a} < 1.
$$
Перемножая такие отношения, начиная с $n=N$ и заканчивая произвольным $n$, получаем
$$
\frac{(N+1)^k/a^{N+1}}{N^k/a^N}\frac{(N+2)^k/a^{N+2}}{(N+1)^k/a^{N+1}}\dots
\frac{n^k/a^n}{(n-1)^k/a^{n-1}}<\left(\frac{a+1}{2a}\right)^{n-N},
$$
откуда следует, что
$$
\frac{n^k}{a^n} < \frac{N^k}{a^N}\left(\frac{a+1}{2a}\right)^{n-N}.
$$
То есть имеем произведение константы на некоторое число <1 в растущей степени. И по аналогии с $(1/2)^n\to 0$ заключаем, что
$$
\frac{n^k}{a^n}\to 0.
$$
На графике ниже представлено 4 варианта параметров $a$ и $k$, а по оси $Ox$ отложен параметр $n$. Как видим, несмотря на большой скачок в начале, эти последовательности довольно быстро уходят в ноль с ростом $n$.
\begin{center}
\includegraphics[scale=0.5]{power.png}
\end{center}



\lesson{$\sqrt[n]{x}\to 1$. Графики}


\item Рассмотрим теперь такую функцию, как корень натуральной степени $\sqrt[n]{x}$. Ее определение таково: $\sqrt[n]{x}$ --- это такое положительное $y$, что $y^n=x$. Обоснование существования корня любого положительного $x$ мы отложим до следующего раздела, а пока решим такую задачку: чему равен 
$$
\lim_{n\to \infty}\sqrt[n]{x}
$$
при любом $x>0$?
\item Пусть для начала $x>1$. Сначала покажем такое неравенство:
$$
(1+c)^n\ge 1+cn\quad(c>-1).
$$
Для этого воспользуемся индукцией. При $n=1$ неравенство очевидно. Пусть оно верно при $n$, тогда покажем при $n+1$.

Действительно, из предположения имеем $(1+c)^n>cn$, тогда
$$
(1+c)^{n+1}=(1+c)^n(1+c)>(1+cn)(1+c)=1+c+cn+c^2n\ge 1+c(n+1),
$$
что и требовалось.

Из доказанного, в частности, следует, что $(1+c)^n>cn$. Теперь подставим $c=\sqrt[n]{x}-1$. Здесь $c>0$, т.к. $x>1$. Получим
$$
x>(\sqrt[n]{x}-1)n\mbox{ или }\sqrt[n]{x}<1+\frac xn.
$$
С другой стороны, $\sqrt[n]{x}>1$. Тогда по лемме о двух милицонерах получаем, что
$$
\lim_{n\to \infty}\sqrt[n]{x}=1.
$$

\item Если $x<1$, то перейдем к величине $y=1/x$. Для нее получим, что $\sqrt[n]{y}\to 1$. Но
$$
\sqrt[n]{x}=1/\sqrt[n]{y},
$$
так что и в этом случае получаем тот же самый предел.

На картинке представлено несколько графиков корня различной степени. Хорошо видно, что с ростом $n$ кривая графика все плотнее прижимается к прямой $y=1$ как слева, так и справа от точки $x=1$.
\begin{center}
\includegraphics[scale=0.5]{root.png}
\end{center}


\lesson{Первый замечательный предел: $(\sin x)/x\to 1$ при $x\to 0$. Следствие: $(1-\cos x)/x^2$.}


\item Рассмотрим еще один важный предел, который называется \textbf{первым замечательным пределом}:\index{Предел!первый замечательный}
$$
x_n\to 0\Rightarrow \lim_{n\to\infty}\frac{\sin x_n}{x_n}=1.
$$
Позже мы покажем, что наличие такого предела для любой последовательности $\{x_n\}$, сходящейся к нулю, равносильно существованию предела по вещественной переменной
$$
\lim_{x\to 0}\frac{\sin(x)}{x}=1.
$$

Далее для простоты будем считать, что $x_n>0$ (к отрицательным $x$ перейти очень просто, т.к. $\sin(-x)=-\sin(x)$ из геометрических построений).

Вспомним, что угол, измеренный в радианах, на единичной окружности равен длине дуги, соответствующей данному углу. Так что возьмем единичную окружность с центром в точке $O$, отложим от направления $Ox$ угол $x_n$.
\begin{center}
\includegraphics[scale=0.3]{sinx.png}
\end{center}
На картинке этот угол называется $AOK$, где $A=(1,0)$. Кроме того, построим две нормали $KH$ и $LA$ (см. рис.)

Очевидно, что
\begin{equation}\label{SSS}
S_{\triangle OAK} < S_{sect KOA} < S_{\triangle OAL}.
\end{equation}
(где $S_{sect KOA}$ --- площадь сектора $KOA$)

Поскольку $|KH| = \sin x_n$, $|LA| = \tg x_n$:
\begin{align*}
S_{\triangle OAK} = & \frac{1}{2} \cdot |OA| \cdot|KH| = \frac{1}{2} \cdot 1 \cdot \sin x_n = \frac{\sin x_n}{2}, \\
S_{sect KOA} = & \frac{1}{2} \cdot |OA|^2 \cdot x_n = \frac{x_n}{2}, \\
S_{\triangle OAL} = & \frac{1}{2} \cdot |OA| \cdot |LA| = \frac{\tg x_n}{2}.
\end{align*}

Подставляя в \eqref{SSS}, получим:
$$
\frac{\sin x_n}{2} < \frac{x_n}{2} < \frac{\tg x_n}{2}.
$$

Так как угол $x_n$ близок к нулю и положителен, можно считать, что он находится в первой четверти плоскости, поэтому
$\sin x_n > 0, \; x_n > 0, \; \tg x_n > 0$, откуда
$$
\frac{1}{\tg x_n} < \frac{1}{x_n} < \frac{1}{\sin x_n}.
$$

Умножаем на $\sin x_n$:
$$
\cos x_n < \frac{\sin x_n}{x_n} < 1.
$$

Отсюда, поскольку $\cos x_n\to 1$, получаем требуемый предел.

\item Следствием первого замечательного предела является такой предел:
$$
\lim_{x\to 0}\frac{1-\cos(x)}{x^2}=\frac 12.
$$
Действительно,
\begin{multline*}
\frac{1-\cos(x)}{x^2}=\frac{1-\sqrt{1-\sin^2 x}}{x^2}\frac{1+\sqrt{1-\sin^2 x}}{1+\sqrt{1-\sin^2 x}}=\\
=\frac{1-(1-\sin^2 x)}{x^2(1+\sqrt{1-\sin^2 x})}=\left(\frac{\sin x}{x}\right)\frac{1}{1+\sqrt{1-\sin^2x}}\to \frac12,
\end{multline*}
поскольку $\sin(x)/x\to 1$ и $\sin x\to 0$.

\end{enumerate}



\section{Экспонента}

\lesson{Подход к экспоненте: мотивация через поиск изоморфизмов операций: переводим сложение в умножение и сохраняем порядок. Конструктивное построение для рациональных чисел}


\begin{enumerate}
\item Итак, $\R$, или вещественная прямая, --- это непрерывное линейно упорядоченное поле. То есть в $\R$ можно складывать, вычитать, умножать и делить, а также сравнивать, причем сравнение согласовано с операциями сложения и умножения. Кроме того, в нем <<нет дыр>>, т.е. каждую точку прямой можно адресовать пределом счетной последовательности вложенных отрезков и, что самое главное, каждый адрес, заданный пределом цепи вложенных отрезков, заселен как минимум одной точкой.
\item Мы также получили доказательство того, что вся прямая равномощна интервалу $(-1;1)$, а значит, и любому открытому интервалу $(a;b)$, где $a<b$. Для этого достаточно было предъявить биекцию в явном виде. Проще всего в таких случаях строить строго монотонную биекцию, чтобы быть уверенным в ее инъективности. Кроме того, монотонная инъекция сохраняет порядок, т.е. устанавливает порядковый изоморфизм между всей прямой и ее частью.
\item В этом плане было бы интересно задаться вопросом --- нет ли биекций, которые сохраняли бы хоть как-то еще и операции сложения и умножения, заданные на $\R$? Как уже отмечалось выше, поле $\R$ уникально по своей природе, т.е. не существует других полей (скажем, не содержащих в себе $\Z$ целиком), полностью изоморфных ему, т.е. с сохранением всех операций и отношения линейного порядка. Однако при некотором ослаблении требований к изоморфизму кое-что интересное мы можем отыскать.
\item Посмотрим на операции сложения и умножения как на две операции из разных полей. Точнее, рассмотрим группу по сложению $(\R,+)$ и группу по умножению $(\R^+,\cdot)$, где под $\R^+$ мы понимаем множество всех положительных действительных чисел. Эти группы, кроме того, линейно упорядочены стандартным отношением $<$.

Мы хотим найти взаимно однозначное соответствие $f:\R\leftrightarrow\R^+$ между этими группами такое, чтобы выполнялось функциональное тождество:
\begin{equation}\label{plusmult}
f(x+y)=f(x)f(y).
\end{equation}
Иначе говоря, $f$ должно переводить сложение в умножение. Кроме того, мы хотим, чтобы $f$ сохраняло и порядок:
\begin{equation}\label{porad}
x<y\Rightarrow f(x)<f(y).
\end{equation}

Требования \eqref{plusmult} и \eqref{porad} называются функциональными уравнением и неравенством, поскольку ограничивают не выбор переменной, а выбор функции. Переменные $x$ и $y$ предполагаются произвольными из области определения $f$.

Вместо условия \eqref{porad} можно требовать сохранение обратного порядка, т.е. $f(x)>f(y)$ при $x<y$. К этому случаю мы вернемся чуть позже.

\item Итак, мы ищем изоморфизм групп $(\R,+,<)$ и $(\R^+,\cdot,<)$, сохраняющий операцию и порядок.
\item Для начала заметим, что, как и положено изоморфизму, $f$ переводит нейтральный элемент в нейтральный: $f(0)=1$. Действительно, $f(x)=f(x+0)=f(x)f(0)$, откуда $f(0)=1$ (сокращать на $f(x)$ мы можем, т.к. $f(x)>0$ по определению).
\item Далее обозначим за $a$ число $f(1)$. В силу требования \eqref{porad} имеем $a>1$.
\item Легко видеть, что
$$
f(2)=f(1)f(1)=a^2,\quad f(3)=f(2)f(1)=a^3,\dots,\quad f(n)=a^n,
$$
причем $n\in\Z$ может быть и отрицательным числом, поскольку $f(-n)=1/f(n)$. Таким образом, уже на целых числах мы видим, что $f$ является \textbf{степенн\'oй функцией}.\index{Степенн\'aя функция}
\item Пусть теперь $x=p/q$. Сложим $x$ сам с собой $q$ раз, и получим
$$
f\left(q\frac pq\right) = f(p/q)^q = f(p) = a^p.
$$
Так что, $f(p/q)=\sqrt[q]{a^p}$ или $f(p/q)=a^{p/q}$.

\item Отметим, что на рациональных точках наша функция монотонно возрастает, как того и требует условие \eqref{porad}. Действительно, пусть $p/q<t/s$. Сравним $a^{p/q}$ и $a^{t/s}$.

Поскольку порядок на $\R$ согласован с операцией умножения, легко получить, что для положительных $x,y$ и натурального $m$ неравенство $x^m<y^m$ верно тогда и только тогда, когда $x<y$. Поэтому, полагая $m=qs$, получаем, что
$$
a^{p/q}<a^{t/s}\Leftrightarrow (a^{p/q})^{qs}<(a^{t/s})^{qs}\Leftrightarrow a^{ps}<a^{qt},
$$
а это уже легко выводится из определения натуральной степени, поскольку при $ps<qt$ имеем
$$
a^{qt}=a^{ps}\cdot\underbrace{a\cdot\dots\cdot a}_{qt-ps\mbox{ раз}},
$$
и, так как $a>1$, неравенство выполняется. Для отрицательных дробей все сводится к положительным, если рассмотреть обратные числа, а для разнознаковых дробей достаточно отметить, что $a^{-p/q}a^{p/q}=1$, так что если одно число больше 1, то второе меньше, и монотонность снова имеет место.



\lesson{Достраивание $a^x$ по непрерывности на все $\R$. Корректность определения через вложенные отрезки. Доказательство биективности $\R\leftrightarrow\R^+$}


\item Итак, мы научились рассчитывать функцию $f$ для рациональных чисел, причем она задается единственным способом в зависимости только от параметра $a=f(1)$. Кроме того, она оказалась строго возрастающей при $a>1$. Как осуществить переход к иррациональным числам?

\item Для этого у нас есть аксиома полноты, которая позволяет осуществлять предельные переходы. Действительно, предположим, что $x$ есть предел вложенных отрезков с двоично-рациональными концами:
$$
[r_0;s_0]\supset [r_1;s_1]\supset [r_2;s_2]\supset \dots \supset\{x\},
$$
где все $r_k,s_k\in\B_k$, и каждый следующий отрезок вдвое короче предыдущего.

Построим точки $R_k=f(r_k)=a^{r_k}$, $S_k=f(s_k)=a^{s_k}$. По доказанному ранее получаем, что если $r_k<s_k$, то $R_k<S_k$ и, кроме того, вложенность отрезков также сохранится, т.е.
$$
[R_0;S_0]\supset [R_1;S_1]\supset [R_2;S_2]\supset \dots
$$
По аксиоме полноты пределом такой цепи будет непустое множество
$$
X=\bigcap_{k=0}^\infty [R_k;S_k].
$$
Покажем, что это множество состоит из одного элемента. Предположим, что это не так, и в $X$ есть хотя бы два элемента $c<d$. Но тогда $R_k\le c<d\le S_k$ для всех $k$, а значит, разность $S_k-R_k$ не может быть меньше, чем $d-c$.


Рассмотрим отношение $S_k/R_k=a^{s_k}/a^{r_k}$. По уже доказанным свойствам функции $f$ легко получить, что это отношение равно
$a^{s_k-r_k}$. В силу того, что $r_k,s_k$ выбирались как двоично-рациональные числа последовательным делением предыдущего отрезка пополам, очевидно, что $s_k-r_k=1/2^k$.

Далее, в силу монотонности функции $f$ для рациональных чисел, получаем, что
$$
1<a^{s_k-r_k}=a^{\frac{1}{2^k}}<1+\frac{a}{2^k}.
$$
Последнее неравенство следует из того, что
$$
a=(a^{1/m})^m < \left(1+\frac{a}{m}\right)^m,\mbox{ откуда }a^{1/m}<1+\frac am,
$$
т.к. для любого положительного $b$ выполняется неравенство $(1+b)^m>bm$. Это можно доказать по индукции или по формуле для бинома Ньютона.

Итак, $a^{s_k-r_k}$ оказывается зажатым между 1 и $1+a/2^k$, а какое бы большое $a$ ни было, с ростом $k$ отношение $a/2^k$ приближается к нулю. В частности, можно найти такое $k$, начиная с которого $a/2^k<(d-c)/2d$.

Тогда для тех же $k$ получим
$$
S_k-R_k = R_k\left(\frac{S_k}{R_k}-1\right)<d(a^{s_k-r_k}-1)<d\frac{a}{2^k}<\frac{d-c}2,
$$
а это противоречит тому, что $S_k-R_k\ge d-c$.

Следовательно, в $X$ нет двух различных точек, т.е. $X=\{y\}$. Больше того, заметим, что так как $r_k\le x\le s_k$ для всех $k$, то и $R_k\le f(x)\le S_k$ для всех $k$ в силу требования монотонности функции $f$. Но тогда $f(x)$ больше некуда деваться, кроме как быть равным числу $y$ --- единственному, удовлетворяющему таким же неравенствам по доказаннмоу выше.

Итак,
$$
a^{r_k}\le f(x)\le a^{s_k}.
$$
В этом случае вместо $f(x)$ мы также пишем $a^x$. Тем самым, мы продлили определение $f(x)$, как говорят, \textit{по непрерывности} на все иррациональные числа. Можно показать, что данное определение корректно, т.е. не зависит от выбора рациональных последовательностей $\{r_k\}$ и $\{s_k\}$. Просто потому, что предельное множество $X$ всегда будет содержать одну и ту же точку.

\item Чтобы убедиться в том, что построенная функция $f(x)$ является искомым изоморфзимом групп $(\R,+,<)$ и $(\R^+,\cdot,<)$, нужно проверить также, что она является биекцией.

Прежде всего, заметим, что это инъекция, т.к. из того, что $f(x)=f(y)$ следует $x=y$ в силу требования \eqref{porad}.

Чтобы показать сюръективность $f(x)$, снова вспомним о неравенстве $(1+b)^m>bm$. Полагая $1+b=a$, находим, что $a^m>(a-1)m$, так что выбирая $m$, мы можем плучить сколь угодно большое число $a^m$, т.е. $f(x)$ не ограничена сверху.

На самом деле, $f(x)$ принимает и все промежуточные значения между 1 и $\infty$. Докажем это. Пусть $f(x)<f(y)$ (соответственно, $x<y$) и пусть $C\in(f(x);f(y))$ --- промежуточная точка в области значений $f$. Необходимо найти такое $c$, что $f(c)=C$. 

Поскольку $f$ определена в точках $x,y$, она также определена и в точке $(x+y)/2$, причем $\sqrt{f(x)f(y)}$ (к слову, это есть среднее геометрическое $f(x)$ и $f(y)$).

При этом, либо $C=\sqrt{f(x)f(y)}$, и тогда мы нашли точку $c$, либо $C\in(x;c)$, либо $C\in(c;y)$. Далее, в зависимости от того, в какой интервал попала точка $C$, мы его снова делим пополам, и продолжаем процедуру либо пока $C$ не совпадет с очередным корнем, либо до бесконечности. Во втором случае мы получим цепь вложенных отрезков, дина которых стремится к нулю как $(y-x)/2^n$. Следовательно, в пределе будет одна точка, и это будет искомая точка $c$ в силу аксиомы непрерывности.

Итак, мы показали, что $f$ взаимно однозначно отображает положительные числа во все числа $>1$. Для того, чтобы показать, что $f$ взаимно однозначно отображает отрицательные числа в числа из интервала $(0;1)$, достаточно знать, что $f(-x)=1/f(x)$.

Таким образом, требования \eqref{plusmult} и \eqref{porad} и условие $f(1)>0$ приводят нас к построению взаимно однозначного соответствия между линейно упорядоченными группами $(\R,+)$ и $(\R^+,\cdot)$.

\item Функция $f(x)$, обозначаемая $a^x$ и удовлетворяющая условиям \eqref{plusmult} и \eqref{porad}, единственная с точностью до выбора числа $a$. Такая функция называется \textbf{показательной} с основанием $a$.


\lesson{Случай $a<1$ и $a=1$. Графики. Определения непрерывности функции в точке через $\ep-\de$ и пределы последовательностей. Доказательство эквивалентности определений}


\item Если число $a$ выбрать меньше 1, то мы можем повторить все те же рассуждения, заменяя всюду знак $<$ на $>$, т.е. мы построим изоморфизм между линейно упорядоченными группами $(\R,+,<)$ и $(\R^+,\cdot,>)$, инвертировав порядок в мультипликативной группе вещественных чисел.
\item На самом деле, можно поступить еще проще, и вместо $a^x$ рассмотреть функцию $(1/a)^x$, которая будет изоморфизмом, сохраняющим прямой порядок на $\R$, а уже функция $a^x$ будет выражаться через нее как $1/(1/a)^x$.
\item Единственный случай, который выбивается из требований сохранения порядка, но при этом сохраняет операции, --- это случай $a=1$. Поскольку тогда мы получим $a^x=1$ для всех $x$. Очевидно, что требование \eqref{plusmult} выполняется, а требование \eqref{porad} --- нет. Однако все три случая $(a>1,a=1,a<1)$ относятся к показательной функции без ограничения общности.

На графике изображено несколько случаев показательной функции при различных $a$.
\begin{center}
\includegraphics[scale=0.5]{exp.png}
\end{center}

\item Выше мы упомянули о том, что функция $a^x$ продлена по непрерывности на иррациональные числа. В данном случае мы, конечно, имели ввиду непрерывность в смысле аксиомы непрерывности действительных чисел, поскольку именно ею пользовались для построения показательной функции. Однако термин \textbf{непрерывность} имеет намного более широкий спектр значений. И прежде всего он связан с непрерывностью функций. Грубо говоря, непрерывная функция действует таким образом, что сохраняет непрерывность своей области определения, переводит непрерывное в непрерывное (обратное не обязательно верно).
\item В математике существует несколько определений непрерывности функции. Приведем два из них. Пусть $f:X\to\R$, где $X$ --- некоторое непустое подмножество в $\R$. Функция $f$ \textbf{непрерывна в точке} $x_0\in X$, если\index{Функция!непрерывная}
\begin{equation}\label{ep-de}
\forall\ep>0 \; \exists\de>0\;\forall x\in X\;|x-x_0|<\de\Rightarrow |f(x)-f(x_0)|<\ep.
\end{equation}
Иначе это записывается так:
$$
\lim_{x\to x_0}f(x)=f(x_0).
$$

\item Иначе говоря, значения $f(x)$ становятся сколь угодно близки к значению $f(x_0)$ при $x$, достаточно близких к $x_0$.
\item Заметим, что стандартным словарем в анализе являются выражения: <<сколь угодно близкий>> или <<сколь угодно малый>> (что означает произвольно малое отклонение от какой-то величины или от нуля и формально сопровождается квантором $\forall\ep>0$), а также <<достаточно близкий>> или <<достаточно малый>> (что означает возможность найти некоторую малость отклонения от какой-то величины или от нуля, достаточную для выполнения некоего условия, и формально сопровождается квантором $\exists\de>0$)
\item Второе определение связано с последовательностями. Функция $f:X\to\R$ непрерывна а точке $x_0$, если для любой последовательности $\{x_n\}$ 
\begin{equation}\label{seq}
\lim_{n\to\infty}x_n=x_0\Rightarrow \lim_{n\to\infty}f(x_n)=f(x_0),
\end{equation}
т.е. $f$ непрерывна, когда она \textit{сохраняет предельные переходы}.
\item Верна следующая
\begin{thrm}Эти два определения эквивалентны.\end{thrm}
\pf
Действительно, пусть выполняется условие \eqref{ep-de} и пусть $x_n\to x_0$ при $n\to\infty$. Необходимо показать, что $f(x_n)\to f(x_0)$, т.е. что для любого $\ep>0$ существует такой номер $N$, что для всех $n>N$ имеет место неравенство $|f(x_n)-f(x_0)|<\ep$.

Возьмем любое $\ep>0$. Для него по определению \eqref{ep-de} найдется такое $\de>0$, что $|f(x)-f(x_0)|<\ep$ для всех $x$ из $\de$-окрестности $x_0$. А по определению предела $x_n\to x_0$ для всякого $\de$ (в том числе вышеназванного) найдется номер $N$ такой, что для всех $n>N$ имеет место $|x_n-x_0|<\de$. Комбинируя эти два вывода, получаем, что для тех же $n>N$ будем иметь неравенство $|f(x_n)-f(x_0)|<\ep$. Что и требовалось.

Обратно. Пусть известно, что для любой последовательности $\{x_n\}$ верна импликация \eqref{seq}. Предположим, что \eqref{ep-de} ложно. Это означает следующее:
$$
\exists\ep>0\;\forall\de>0\;\exists x\in X\;(|x-x_0|<\de)\land(|f(x)-f(x_0)|>\ep),
$$
т.е. есть некая малая величина $\ep$, за которую разность $|f(x)-f(x_0)|$ уходит бесконечно часто. И здесь нам придется построить рекурсию, аналогичную построению вложенных двоично-рациональных отрезков.

Поскольку можно брать любое $\de$, возьмем $\de=1/2$. Для него найдется $x_1\in X$ такой, что, с одной стороны $|x_1-x_0|<1/2$, а с другой стороны, $|f(x_1)-f(x_0)|>\ep$.

Далее, в качестве $\de$ возьмем $1/4$. Для него найдется $x_2\in X$ такой, что, с одной стороны $|x_2-x_0|<1/4$, а с другой стороны, $|f(x_2)-f(x_0)|>\ep$.

И так далее. Для $\de=1/2^n$ найдется $x_n\in X$ такой, что, с одной стороны $|x_n-x_0|<1/2^n$, а с другой стороны, $|f(x_n)-f(x_0)|>\ep$.

В итоге мы построили последовательность $\{x_n\}$, которая сходится к $x_0$, т.к. $|x_n-x_0|<1/2^n\to 0$. Но при этом $|f(x_n)-f(x_0)|>\ep$, т.е. $f(x_n)\not\to f(x_0)$. А это противоречит \eqref{seq}.

Следовательно, наше предположение неверно, а значит, \eqref{ep-de} выполняется для $f(x)$.\epf

\item Если функция $f$ непрерывна в каждой точке множества $M$ (не обязательно всей области определения), то говорят, что $f$ \textbf{непрерывна на} $M$.

\item Из построения функции $a^x$, в принципе, видено, что она непрерывна на $\R$, но проведем строгие рассуждения для доказательства этого факта. Требуется показать, что для любого $x$ и для любого $\ep>0$ найдется такое $\de>0$, что для любого $y$, если $|x-y|<\de$, то $|a^x-a^y|<\ep$.

Выберем произвольные $x$ и $\ep$, а в качестве $\de$ возьмем число $1/m$ такое, что $\ep>a^{1+x}/m$ (какие бы $a$ и $x$ ни были, такое число $m$ найдется). Пусть теперь $|x-y|<1/m$, причем $y>x$. В силу монотонности показательной функции $a^x>a^y$. Заметим, что в этом случае $|a^x-a^y|=a^x(a^{y-x}-1)$.

Далее, поскольку $y-x<1/m$, оценим $a^{y-x}-1<a^{1/m}-1<a/m$ (это мы уже показывали ранее), откуда
$$
|a^x-a^y|<a^{x+1}/m<\ep.
$$

Если же $y<x$, то модуль раскрывается иначе, поскольку $a^y<a^x$:
$$
|a^x-a^y|=a^y(a^{x-y}-1)<a^x(a^{x-y}-1)<\ep.
$$

Таким образом, по определению \eqref{ep-de} функция $a^x$ непрерывна.

\hard{
\item Отметим, что в приведенном доказательстве выбор $\de$ неустранимым образом звисит от $x$, и чем больше $x$, тем больше нужно выбирать число $m$ (при $a>1$). Поэтому в каждой конкретной точке $x$ функция $a^x$ непрерывна, но ее сходимость к своему пределу тем медленнее, чем больше $x$. Если бы мы могли выбрать $\de$ одинаковым для всех $x$, то непрерывность функции $a^x$ стала бы \textbf{равномерной}. Равномерная непрерывность играет важную роль в математическим анализе при доказательстве сходимости функциональных рядов и теорем о смене порядка предельного перехода. Для обычной непрерывывной функции, зависящей от параметра, сходимость может существенно зависеть от изменения этого апраметра вместе с аргументом функции.}


\lesson{Свойства показательной функции. Определение производной. Касательная. Теорема о непрерывности в точке существования производной. Число Эйлера, экспонента}


\item Приведем несколько свойств показательной функции:
\begin{enumerate}[\bf Pow1]
\item $a^xa^y = a^{x+y}$;
\item $a^x/a^y=a^{x-y}$;
\item $(ab)^x = a^x b^x$;
\item $(a/b)^x = a^x/b^x$;
\item $\sqrt[n]{a^x}=a^{x/n}$;
\item $a^x<a^y$ тогда и только тогда, когда $x<y$;
\item $(a^x)^y = a^{xy}$.
\end{enumerate}
Все эти свойства сначала доказываются для натуральных и рациональных чисел, а затем переносятся на иррациональные числа по непрерывности.

\item Следующее замечательное применение теории пределов находит в определении производной. Если существует предел\index{Производная функции}
$$
\lim_{\De\to 0}\frac{f(x+\De)-f(x)}{\De},
$$
то такой предел называется \textbf{производной функции} $f$ в точке $x$ и обозначается $f'(x)$. Геометрически производная в конкретной точке $x$ --- это тангенс угла наклона касательной к графику функции в точке $(x,f(x))$.

\item Глядя на следующий рисунок, можно заметить, что чем меньший шаг мы выбираем (сначала $\De=1$ (зеленая линия), затем $\De=0.7$ (розовая линия), затем он практически равен нулю (оранжевая линия)), тем ближе секущая графика функции $y=a^x$ подбирается к касательной линии $y=x+1$. А отношение $(f(x+\De)-f(x))/\De$ есть не что иное, как отношение катетов треугольников под этими секущими линиями, т.е. тангенс угла наклона секущих, а в пределе --- касательной.
\begin{center}
\includegraphics[scale=0.5]{deriv.png}
\end{center}
\item Конечно, для того, чтобы в пределе получить касательную, функция должна быть достаточно хорошей или, как говорят, \textbf{гладкой}. Собственно, функция и называется гладкой, если у нее есть производная. Гладких функций не больше, чем непрерывных, точнее, справедива
\begin{thrm} Если у функции $f(x)$ есть производная в точке $x$, то она непрерывна в этой точке.
\end{thrm}
\pf
В силу существования предела
$$
f'(x)=\lim_{\De\to 0}\frac{f(x+\De)-f(x)}{\De}
$$
получаем, что
$$
\frac{f(x+\De)-f(x)}{\De}=f'(x)+\al(\De),
$$
где $\al(\De)\to 0$ при $\De\to 0$. Находим разность значений функции:
$$
f(x+\De)-f(x) = f'(x)\De+\al(\De)\De.
$$
Как видим, справа стоит сумма двух бесконечно малых, так что $f(x+\De)\to f(x)$ при $\De\to 0$. Следовательно, $f$ непрерывна в точке $x$.
\epf

\item Каковы могут быть значения производной для показательной функции? Если посмотреть на график различных показательных функций с основанием $a=2,1.25,1,0.8,0.5$, то легко заметить, что даже в точке $x=0, y=1$ наклон касательной может быть каким угодно, кроме абсолютно вертикального. В частности, касательная может быть горизонтальной, если $a=1$.
\item Действительно, в случае $a=1$ получаем, что $y=a^x=1$ для всех $x$, т.е. является константой, а для константы производная вычисляется очень легко:
$$
\lim_{\De\to 0}\frac{C-C}{\De}=\lim_{\De\to 0}0=0.
$$
\item Среди всех показательных функций принято особо выделять одну (как некоторый образующий элемент в классе показательных функций), а именно такую, у которой наклон касательной в точке $x=0$ равен $45^o$ или, иначе говоря, производная равна 1. Основание $a$ в этом случае обозначается буквой $e$ и называется \textbf{числом Эйлера}.\index{Число Эйлера}
$$
e\approx 2.718281828459045
$$
\item Итак, по определению число $e$ таково, что $(e^x)'=1$ в точке $x=0$, т.е.
$$
\lim_{x\to 0}\frac{e^x-1}{x}=1,\mbox{ или }e^x=1+x+o(x),
$$
где $o(x)$ --- бесконечно малая в сравнении с $x$ величина при $x\to 0$. Из этого равенства понятно также, почему $y=1+x$ является касательной к графику функции $e^x$ в точке $(0;1)$.
\item Функция $e^x$ называется \textbf{экспонентой}.\index{Экспонента} Иногда также используется обозначение $\exp(x)$.
\item Найдем производную экспоненты в произвольной точке $x$:
$$
\frac{e^{x+\De}-e^x}{\De}=e^x\frac{e^\De-1}{\De}\to e^x,
$$
так что $(e^x)'=e^x$. Итак, производная экспоненты есть сама же экспонента! В терминах высшей математики это можно выразить так: экспонента является неподвижной точкой оператора дифференцирования.


\lesson{Производная показательной функции. Логарифм. Примеры экспоненты: радиоактивный распад, диффур $x'=kx$.}


\item Найдем теперь производную $a^x$ в общем случае. Поскольку число $a$ --- некоторое положительное число, а функция $e^x$ принимает все положительные значения, существует такое $x_0$, что $e^{x_0}=a$. В таком случае,
$$
a^x = (e^{x_0})^x = e^{xx_0},
$$
и далее получаем, что
$$
\frac{a^{x+\De}-a^x}{\De} = x_0\frac{e^{xx_0+x_0\De}-e^{xx_0}}{x_0\De} \to x_0 e^{xx_0} = x_0a^x.
$$
\item Таким образом, число $x_0$ --- это тангенс угла наклона касательной к графику функции $a^x$ в точке $(0;1)$. Число $x_0$, определяемое равенством $e^{x_0}=a$, называется \textbf{натуральным логарифмом}\index{Логарифм} числа $a$ и обозначается $\ln a$. Натуральный логарифм --- это функция, обратная к $e^x$.
\item Во многих науках, в том числе в физике, часто встречается экспоненциальный (т.е. показательный) закон роста или убывания какой-либо величины. Например, такое понятие как период полураспада связано с показательной функцией. Точнее, для каждого радиоактивного вещества существует время $T$, за которое распадается половина ядер атомов этого вещества. Так что, если мы обозначим количество оставшися атомов за $\tau_n$, то
$$
\tau_{n+1}=\tau_n/2.
$$
Отсюда легко получить, что $\tau_n=\tau_0/2^n$. Конечно, это закон статистический, и его точность тем хуже, чем меньше осталось атомов, но на гиганских количествах он работает очень точно.

И здесь мы видим показтельную функцию с основанием $0.5$.

Во многих задачах встречается дифференциальное уравнение
$$
f'(x)=kf(x),
$$
которое говорит о том, что скорость роста величины $f(x)$ (то есть тангенс угла наклона ее графика) пропорциональна самой этой величине. Это --- обобщение предыдущего уравнения (в конечных разностях) для непрерывной функции. Решением такого уравнения является показательная функция
$$
f(x) = Ae^{kx},
$$
где константа $A>0$ обозначает начальное значение при $x=0$.
Например, рост популяции при неограниченном ресурсе происходит по такому закону.

\item Если производная функции положительна на некотором интервале, то сама функция на жанном интервале строго возрастает, а если производная отрицательна, то сама функция строго убывает. Этот факт вполне очевиден из того же графика с касательными и секущими, который мы видели выше. Кроме того, из определения производной мы видим, что
$$
f(x)-f(x_0)=f'(x_0)(x-x_0)+o(x-x_0),
$$
т.е. при малых отклонения аргумента от $x_0$ разность значений $f(x)-f(x_0)$ имеет такой же знак, как производная в точке $x_0$.

Проще всего, конечно, монотонность видна при интегрировании производной, но это мы оставим за рамками курса.



\lesson{Представление экспоненты в виде ряда Тейлора и второго замечательного предела $(1+x/n)^n$}


\item Наконец, покажем, что экспонента имеет следующие два представления.
\begin{thrm}\label{exp-series-lim} Для любого вещественного $x$\index{Теорема!о представлении экспоненты}
$$
e^x = 1+x+\frac{x^2}{2!}+\frac{x^3}{3!}+\dots,\qquad e^x=\lim_{n\to\infty}\left(1+\frac xn\right)^n
$$
\end{thrm}
\pf
Пусть
$$
f(x) = 1+x+\frac{x^2}{2!}+\frac{x^3}{3!}+\dots
$$
Необходимо проверить, что $f(x)$ удовлетворяет условиям \eqref{plusmult} и \eqref{porad}, а кроме того, $f'(0)=1$.

Равенство $f(x+y)=f(x)f(y)$ проверяется следующим образом. Поскольку мы должны сравнить один бесконечный степенной ряд с произведением двух других рядов от двух переменных $x$ и $y$, мы должны действовать так же, как при сравнении многочленов, а именно: сравнить коэффициенты при одинаковых степенях $x^ny^m$.

Ясно, что при перемножении $f(x)f(y)$ существует только один способ получить $x^ny^m$:
$$
[x^nx^m]f(x)f(y) = \frac{1}{n!m!}.
$$
В случае $f(x+y)$ требуемая степень возникает только в слагаемом $(x+y)^{n+m}/(n+m)!$, которое по формуле бинома Ньютона раскладывается следующим образом:
$$
\frac{(x+y)^{n+m}}{(n+m)!} = \frac{1}{(n+m)!}\sum_{k+j=n+m}x^ky^j\frac{(n+m)!}{k!j!},
$$
где степень $x^ny^m$ можно получить единственным способом при $k=n, j=m$. Так что
$$
[x^ny^m]f(x+y) = \frac{1}{n!m!}.
$$
Итак, $f(x+y)=f(x)f(y)$.

Покажем ее монотонность. Для случая $0\le x<y$ все очевидно, т.к. $x^n<y^n$, откуда и сумма ряда $f(x)$ меньше, чем сумма ряда $f(y)$. Кроме того, ясно, что $f(x)>1$, когда $x>0$.

Далее, из свойства сохранения операции мы видим, что $f(-x)=1/f(x)$. Это значит, что для отрицательных $x$ значение суммы ряда $f(x)<1$, т.е. $f(x)<f(y)$, если $x<0<y$. Наконец, в случае $x<y<0$ мы просто переходим к обратным величинам:
$$
f(x)=\frac{1}{f(-x)}<\frac{1}{f(-y)}=f(y),
$$
поскольку $f(-x)>f(-y)$ по доказанному выше (т.к. $0<-y<-x$).

Итак, $f(x)$ монотонно возрастает. Следовательно, $f(x)=a^x$ при каком-то положительном $a$. Не вдаваясь в анализ степенных рядов, скажем, что ряд с факториальными коэффициентами настолько хорош, что с ним можно работать как с обычной суммой. В частности, $f(x)=1+x+o(x)$, т.к. сумма всех членов спени выше 1 имеет порядок малости сильнее, чем $x$. Но $y=1+x$ есть уравнение касательной к экспоненте, т.к. имеет наклон $\pi/4$. А это и означает, что $a=e$, т.е. $f(x)=e^x$.

Перейдем к доказательству соотношения, именуемого также \textbf{вторым замечательным пределом}\index{Предел!второй замечательный}
$$
e^x=\lim_{n\to\infty}\left(1+\frac xn\right)^n.
$$

Мы проведем его способом, типовым для получения предельных соотношений в математическим анализе. Он заключается в том, чтобы исследуемый ряд так удачно разделить на части, что про одну можно сказать, что она стремится к нужному пределу, а про остальные --- либо они сокращаются с аналогичной частью предельного выражения, либо стремтся к нулю. Итак, снова бином Ньютона:
\begin{align*}
\left(1+\frac xn\right)^n = & 1 + x + \frac{n(n-1)}{2}\frac{x^2}{n^2} + \dots +
\frac{n!}{(n-k)!k!}\frac{x^k}{n^k}+\dots = \\
& 1 + x + \frac{n-1}{n}\frac{x^2}{2!} + \dots + \frac{(n-1)\dots(n-k+1)}{n^{k-1}}\frac{x^k}{k!}+\dots \\
& 1 + x + \frac{n-1}{n}\frac{x^2}{2!} + \dots + \left(1-\frac 1n\right)\dots\left(1-\frac{k-1}{n}\right)\frac{x^k}{k!}+\dots
\end{align*}
Видим, что получается ряд, очень похожий на ряд экспоненты, который мы получили выше. Только перед каждым слагаемым появляется коэффициент вида
$$
\left(1-\frac 1n\right)\dots\left(1-\frac{k-1}{n}\right).
$$

Если мы фиксируем $k$, то перед нами обычное конечное произведение некоторых величин, которые с ростом $n$ стремятся к 1. Следовательно, по свойствам пределов и всё произведение стремится к 1. Это значит, что любой сколь угодно большой отрезок фиксированной длины $k$ из разложения $\left(1+\frac xn\right)^n$ стремится к такому же отрезку ряда экспоненты:
$$
1 + x + \frac{n-1}{n}\frac{x^2}{2!} + \dots + \left(1-\frac 1n\right)\dots\left(1-\frac{k-1}{n}\right)\frac{x^k}{k!} \to
1 + x + \frac{x^2}{2!} + \dots + \frac{x^k}{k!}
$$
при $n\to\infty$.

Тогда получается, что при любом фиксированном $k$ разность между этими суммами можно сделать сколь угодно малой выбором достаточно большого $n$. Посмотрим тогда на итоговую разность:
\begin{multline*}
\left|\left(1+\frac xn\right)^n-e^x\right| \le 
\left|\sum_{j=0}^k\left(1-\frac 1n\right)\dots\left(1-\frac{j-1}{n}\right)\frac{x^j}{j!}-\sum_{j=0}^k\frac{x^j}{j!}\right|+ \\
+ \left|\frac{x^{k+1}}{(k+1)!}-\frac{(n-1)\dots(n-k)}{n^{k}}\frac{x^{k+1}}{(k+1)!}
+ \dots + \frac{x^n}{n!} - \frac{x^n}{n!} + \frac{x^{n+1}}{(n+1)!+\dots}\right|
\end{multline*}
теперь заметим, что первый модуль можно сделать меньше произвольного $\ep>0$ (при любом каком угодно большом $k$), а во втором модуле все слагаемые со знаком <<минус>> по модулю строго меньше слагаемых со знаком <<плюс>>, поскольку
$$
\frac{(n-1)\dots(n-k)}{n^{k}}\le 1.
$$
А это значит, что второй модуль можно оценить <<хвостом>> ряда экспоненты, т.е. выражением
\begin{multline*}
\frac{|x|^{k+1}}{(k+1)!} + \dots + \frac{|x|^n}{n!} + \dots = \\
= |x|^{k+1}\left(\frac{|x|^0}{(k+1)!} + \dots + \frac{|x|^{n-k-1}}{(n-k-1)!(n-1)\dots n}+\dots\right)\le \\
\le |x|^{k+1}\left(\frac{|x|^0}{(k+1)!} + \dots + \frac{|x|^{n-k-1}}{(n-k-1)!(k+1)!}+\dots\right),
\end{multline*}
где мы воспользовались тем, что $(n-k)\dots n\ge (k+1)!$.

Таким образом,
\begin{equation}\label{exp-chvost}
\frac{|x|^{k+1}}{(k+1)!} + \dots + \frac{|x|^n}{n!} + \dots \le \frac{|x|^{k+1}}{(k+1)!}e^{|x|},
\end{equation}
а эта величина, как мы уже ранее доказывали, стремится к нулю с ростом $k$. Следовательно, хвост экспоненты можно сделать меньше $\ep$ выбором достаточно большого $k$.

Таким образом, разность $\left|\left(1+\frac xn\right)^n-e^x\right|$ оценивается величиной $2\ep$, причем сначала выбирается достаточно большое $k$, при котором хвост ряда экспоненты меньше $\ep$, а затем уже для этого $k$ выбирается достаточно большое $n$, при ктором начальный отрезок ряда экспоненты отличается от аналогичного отрезка ряда бинома Ньютона меньше, чем на $\ep$. В итоге вся разность может быть сделана сколь угодно малой, а значит, имеет место предел
$$
\left(1+\frac xn\right)^n\to e^x
$$
при $n\to\infty$.
\epf
\end{enumerate}


\subsection*{Задачи}
\begin{enumerate}
\item Пусть дана последовательность
$$
x_n = \left(1+\frac xn\right)^n.
$$
доказать, что она монотонно возрастает и ограничена сверху.
\item А как ведет себя последовательность
$$
y_n = \left(1+\frac xn\right)^{n+1}?
$$
\item Определим логарифм числа $x>0$ по основанию $a>0$ как такое число $y$, что $a^y=x$. Обозначение: $y=\log_a x$. Вывести формулу:
$$
\log_a x = \frac{\ln x}{\ln a} = \frac{\log_b x}{\log_b a}
$$
при любом положительном основании $b$.
\item Сравнить два числа:
$$
5^{\log_7 3}\quad vs\quad   3^{\log_7 5}.
$$

\end{enumerate}



\section{Комплексная экспонента}

\lesson{Комплексные числа: повторение арифметики, определение предела по норме. Полнота $\C$. Основная теорема Алгебры (без док-ва)}

\begin{enumerate}
\item Комплексные числа мы теперь рассматриваем в их полном объеме, т.е. как множество векторов $z=(x,y)=x+yi$ с вещественными координатами, подчиненных арифметическим операциям\index{Комплексные числа}\index{Числа!комплексные}
$$
(x+iy)+(x'+y'i)=(x+x')+(y+y')i,\quad (x+iy)(x'+y'i)=(xx'-yy')+(xy'+x'y)i.
$$
Множество комплексных чисел обозначается $\C$.
\item Модулем комплексного числа $x+yi$ называется $|x+yi|=\sqrt{x^2+y^2}$.
\item Так же, как в $\R$, мы можем рассматривать последовательности комплексных чисел, которые представляют собой две упакованные в одну последовательности чисел действительных:
$$
\{z_n\} = \{x_n+y_ni\}
$$
\item Соответственно, точно так же, с использование модуля комплексного числа, определяется предел последовательности:
$$
z=\lim_{n\to\infty}z_n\Leftrightarrow\forall\ep>0\;\exists N\;\forall n>N\;|z-z_n|<\ep.
$$
\item Термин $\ep$-\textit{окрестность} на плоскости приобретает уже привычное бытовое понимание. $\ep$-окрестностью точки $z_0$ называется круг радиуса $\ep$ с центром в точке $z_0$. т.е. множество
$$
\{z\mid |z-z_0|<\ep\}.
$$
таким образом, сходимость $z_n\to z$ означает, что в любой сколь угодно малой окрестности точки $z$ находятся почти все члены последовательности $\{z_n\}$.
\item Сходимость последовательности $\{z_n\}$ к какому-то пределу $z=x+yi$ равносильна сходимости последовательностей $\{x_n\}$ и $\{y_n\}$, соответственно, к точкам $x$ и $y$. Поэтому все свойства полноты (непрерывности) $\R$ наследуются множеством комплексных чисел. За исключением свойства упорядоченности.
\item Так, всякая фундаментальная последовательность, т.е. такая $\{z_n\}$, что
$$
\forall\ep>0\;\exists N\;\forall n,m>N\;|z_n-z_m|<\ep,
$$
имеет предел.
\item Полнота (непрерывность) $\C$ означает, что в нем не появляется никаких новых <<дыр>>, которые можно было бы задать уравнениями с целыми коэффициентами. Больше того, даже если коэффициенты многочлена являются комплексными числами, никаких новых расширений с помощью корней таких многочлено мы не получим.
\item Для поля $\C$ справедлива
\begin{thrm}[основная теорема Алгебры]\index{Теорема!основная теорема Алгебры}
Всякий многочлен над $\C$ раскладывается в произведение линейных множителей однозначно. То есть для всякого многочлена $f(z)=a_0+a_1z+a_2z^2+\dots+a_nz^n$ имеет место разложение
$$
f(z)=a_n(z-z_1)(z-z_2)\dots(z-z_n),
$$
где корни $z_1,\dots,z_n$ могут быть кратными, т.е. повторяться.
\end{thrm}
\item Из этой теоремы следует также, что любой многочлен с вещественными коэффициентами (который можно рассматривать как частный случай многочлена с комплексными коэффициентами) раскладывается в произведение многочленов первой и второй степени. Дело тут в том, что если $f(z)=0$, то и $f(\bar z)=0$, т.е. комплексные корни (не являющиеся вещественными) всегда идут парой вместе со своим сопряжением. Ну а произведение $(z-z_1)(z-\bar z_1)$ имеет уже строго вещественные коэффициенты. и потому такой квадратный трехчлен входит в разложение исходного многочлена. Если же корень $z_1$ сам себе сопряжен, то он является вещественным числом, и потому двучлен $(z-z_1)$ входит в разложение исходного многочлена.
\item Итак, $\C$ --- это алгебраически замкнутое полное поле. Оно не является упорядоченным полем ни при каком линейном порядке (т.к. в упорядоченном поле квадрат отрицательного числа всегда есть положительное число, т.е. невозможна мнимая единица). Поле $\C$ есть конечное расширение $\R[i]$.


\lesson{Мощность $\C$ есть континуум, т.е. равномощно $\R$. Регулярные функции. Теорема о продлении}


\item  Мощность множества $\C$ есть континуум.

Докажем этот факт в упрощенной форме, а именно, покажем, что квадрат $[0;1)\times[0;1)$ равномощен отрезку $[0;1)$. Возьмем произвольную точку $(x,y)$ из квадрата и запишем ее координаты двоичной последовательностью без хвоста единиц:
$$
x = 0.x_1x_2x_3\dots x_n\dots,\quad y=0.y_1y_2y_3\dots y_n\dots,
$$
где $x_n,y_n\in\{0,1\}$. Построим точку $z$, чередуя двоичные цифры из исходных представлений:
$$
z = 0.x_1y_1x_2y_2x_3y_3\dots x_ny_n\dots
$$
Такое соответствие точек квадрата и оточек отрезка является инъекцией, поскольку $(x,y)$ однозначно восстанавливается по $z$, но не является биекцией, т.к. точки вида $z=0.0101010101010101$ не имеют соответствия в квадрате.

С другой стороны, очевидное вложение $z\mapsto (z,0)$ точек отрезка в квадрат также является инъекцией. Следовательно, по теореме Кантора--Бернштейна квадрат и отрезок равномощны.

Расширить доказательство на случай $\C$ и $\R$ можно с помощью преобразований, которые мы ранее уже использовали при установлении равномощности $[0;1)$ и $\R$. В случае $\C$ их нужно будет применить к каждой координате по отдельности.

\item Богатый инструментарий дает понятие комплексной производной. Определяется она точно так же:\index{Производная!комплексная}
$$
f'(z) = \lim_{\De\to 0}\frac{f(z+\De)-f(z)}{\De},
$$
только приращение $\De$ здесь --- комплексное, и это приводит ко многим хорошим свойствам комплексного дифференцирования.
\item Функция $f(z)$ называется \textbf{регулярной} в области $D\subseteq\C$, если она дифференцируема в каждой ее точке.\index{Функция!регулярная} При этом словом <<область>> обозначается такое подмножество плоскости, что всякая точка этого множества входит в него вместе с некоторой своей окрестностью.
\item В отличие от вещественного случая, если комплексная функция дифференцируема в некоторой области, то у нее существуют все производные высших порядков в этой области. Это обеспечивает возможность представить регулярную функцию (в области регулярности) в виде степенного ряда:
$$
f(z) = f(z_0)+f'(z_0)(z-z_0)+f''(z_0)\frac{(z-z_0)^2}{2}+f'''(z_0)\frac{(z-z_0)^3}{3!}+\dots
$$
Такой ряд именуется рядом Тейлора (как в комплексном, так и вещественном случае).
\item И здесь мы подходим к кульминационному моменту: если на интервале вещественной оси задана некоторая вещественная функция $f(x)$, которая на этом интервале раскладывается ряд Тейлора (т.е. она бесконечно дифференцируема), то на любую область комплексной плоскости, содержащую данный интервал, функция $f(x)$ единственным способом продолжается до регулярной в этой области функции $f(z)$.
\item В частности, это означает, что если мы умеем какую-то функцию $f(x)$ раскладывать в ряд Тейлора на $\R$, то существует комплексная функция $f(z)$, заданная на всей плоскости $\C$, регулярная на $\C$ (такие функции называются целыми) и такая, что при вещественных аргументах она тождественна исходной $f(x)$. Более того, разложение $f(z)$ в ряд Тейлора такой функции в вещественной точке тождественно разложению исходной $f(x)$ в той же точке.


\lesson{Продление экспоненты с $\R$ на $\C$. Замечание о матричной экспоненте. Сходимость ряда экспоненты в $\C$. Получение формулы $e^{i\pi}-1=0$. Ряды Тейлора для $\cos$ и $\sin$}

\item Для нас это означает следующее. Возьмем функцию $e^x$, у которой ряд Тейлора в точке $x=0$ имеет вид
$$
e^x = 1+x+\frac{x^2}{2}+\frac{x^3}{3!}+\dots
$$
Тогда существует единственная продолжающая ее на всю комплексную плоскость регулярная функция с тем же рядом Тейлора:
$$
e^z = 1+z+\frac{z^2}{2}+\frac{z^3}{3!}+\dots
$$
Ее мы тоже будем называть экспонентой и обозначать аналогично вещественному случаю.
\item На самом деле, с помощью такого ряда экспонента может быть определена для многих числовых структур, однако для сохранения свойства $e^{x+y}=e^xe^y$ требуется коммутативность умножения, что не всегда выполняется. Например, матричная экспонента, когда вместо $z$ мы подставляем квадратную матрицу, тоже определена, но сохранение операций не всегда происходит.\index{Экспонента!матричная}

Для примера достаточно взять $X=\begin{pmatrix}1 & 0\\ 0 & 0\end{pmatrix}$ и $Y=\begin{pmatrix}0 & 1\\ 0 & 0\end{pmatrix}$.
В этом случае будем иметь:
\begin{gather*}
e^X = E+(e-1)X,\quad e^Y = E+Y,\\
e^Xe^Y = E+(e-1)X+eY,\quad e^{X+Y}=E+(e-1)X+(e-1)Y.
\end{gather*}

\item Нетрудно видеть, что комплексная экспонента сходится в каждой точке. Для этого достаточно оценить <<хвост>> ряда Тейлора с помощью полученной ранее оценки \eqref{exp-chvost}, поскольку $|z|$ уже будет вещественным положительным числом. И тогда очень просто доказать, что последовательность частичных сумм
$$
z_n=1+z+\frac{z^2}{2!}+\dots+\frac{z^n}{n!}
$$
является фундаментальной, т.к. мы будем иметь оценку при $n<m$
$$
|z_n-z_m|\le \frac{|z|^n}{n!}+\dots+\frac{|z|^{m-1}}{(m-1)!}\le \frac{|z|^n}{n!}e^{|z|}\to 0
$$
при $n\to\infty$. И далее, в силу полноты $\C$ заключаем, что существует предел $z_n$, который по определению и есть $e^z$.
\begin{center}
\includegraphics[scale=0.5]{exp-series.png}
\end{center}

\item Наконец, полностью повторяя рассуждения теоремы \ref{exp-series-lim}, нетрудно показать, что в комплексном случае также имеет место представление
$$
e^z = \lim_{n\to\infty}\left(1+\frac zn\right)^n.
$$
\item Рассмотрим частный случай этого предела при $z=i\ph$. С одной стороны,
$$
e^{i\ph} = \lim_{n\to\infty}\left(1+\frac{i\ph}{n}\right)^n.
$$
С другой стороны, в силу найденных ранее пределов
$$
\cos(\ph/n) = 1-\frac{(\ph/n)^2}{2}+\ep_n/n^2,\quad\sin(\ph/n)=\ph/n+\de_n/n,
$$
где $\ep_n\to 0$ и $\de_n\to 0$, откуда
$$
\cos(\ph/n)+i\sin(\ph/n) = 1 + i\ph/n + \ga_n/n,
$$
где $\ga_n\to 0$.

Наконец, пользуясь биномом Ньютона, получаем, что:
\begin{gather*}
\left|\frac{(\cos(\ph/n)+i\sin(\ph/n))^n}{(1+i\ph/n)^n}-1\right| = \left|\left(1+\frac{\ga_n}{n+i\ph}\right)^n-1\right| = \\
= \left|\frac{\ga_n}{1+i\ph/n} + \frac{n(n-1)\ga_n^2}{2(n+i\ph)^2} + \dots + 
\frac{n(n-1)\dots(n-k+1)\ga_n^k}{k!(n+i\ph)^k} + \dots \right| \le \\
\le |\ga_n| + |\ga_n|^2 + \dots + |\ga_n|^k + \dots = \frac{|\ga_n|}{1+|\ga_n|} \to 0,
\end{gather*}
где мы воспользовались тем, что $|1+i\ph/n|>1$, а также формулой для суммы геометрической прогрессии.

Итак, в конечном счете получаем, что
$$
e^{i\ph} = \lim_{n\to\infty}(\cos(\pi/n)+i\sin(\ph/n))^n.
$$

Однако справа под знаком предела стоит посотянная величина. Ранее мы уже видели, что при умножении комплексных чисел с единичной окружности их аргументы складываются. В данном случае мы умножаем $n$ одинаковых чисел $z=\cos(\ph/n)+i\sin(\ph/n)$. Число $z$ лежит на единичной окружности, а его аргумент, т.е. угол наклона относительно оси $Ox$, равен $\ph/n$. Следовательно, у произведения $z^n$ этот угол равен $\ph$, а модуль равен 1, поэтому $z^n=\cos(\ph)+i\sin(\ph)$.

Окончательно получаем равенство, которое именуется \textbf{формулой Эйлера}:\index{Формула Эйлера}
$$
e^{i\ph} = \cos(\ph)+i\sin(\ph).
$$

В частности, отсюда следует знаменитое тождество Эйлера, связывающее сразу 5 фундаментальных математических констант:
$$
e^{i\pi}+1=0.
$$

\item Из формулы Эйлера легко получить разложения в ряд Тйлора для $\sin$ и $\cos$. Действительно, посмотрим на ряд
$$
e^{ix} = 1 + ix - \frac{x^2}{2} - i\frac{x^3}{3!} + \frac{x^4}{4!} + i\frac{x^5}{5!} - \frac{x^6}{6!} - i\frac{x^7}{7!}+\dots
$$
Поскольку $e^{ix}=\cos x+i\sin x$, соберем для косинуса все вещественные слагаемые ряда, а для синуса --- мнимые, и получим:
\begin{align*}
\cos x = & \;1 - \frac{x^2}{2!} + \frac{x^4}{4!} - \frac{x^6}{6!} + \dots \\
\sin x = &\; x - \frac{x^3}{2!} + \frac{x^5}{5!} - \frac{x^7}{7!} + \dots
\end{align*}
\end{enumerate}



\subsection*{Задачи}
\begin{enumerate}
\item Найти решения уравнения $\sin z=4$.
\item Чему равны выражения
$$
\frac{e^{ix}+e^{-ix}}{2},\quad \frac{e^{ix}-e^{-ix}}{2i}?
$$
\end{enumerate}


