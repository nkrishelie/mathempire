{\thispagestyle{empty}
%\clearpage
%\setcounter{page}{0}


\begin{flushright}
\quad

\vspace{2cm}

{\fontsize{100pt}{0pt}\bfseries\sffamily ВВЕДЕНИЕ В\\[20pt] МАТЕМАТИКУ}


\vspace{2cm}

{\fontsize{20pt}{22pt}\sffamily листки с задачами\\
<<100 урокам математики>> Алексея Савватеева\\[5pt]
}

\vspace{2cm}

{\Large\bfseries\sffamily составители: Н.~Казимиров, П.~Иванов, М.~Бочкарев}

\vfill

{\Large\bfseries\sffamily 	Москва, \number\year}
\end{flushright}
}



%\shorttableofcontents{Оглавление}{0}
%\markboth{}{}

%\clearpage
%\renewcommand*\contentsname{\vspace{-20mm}\quad\hfill\Large\bfseries\sffamily\MakeUppercase{Содержание}\vspace{2mm}\textcolor{darkred}{\hrule}\thispagestyle{empty}}
%\tableofcontents
 

\nochapter{Аннотация к сборнику}

Данный сборник задач может использоваться как приложение к конспекту \href{https://github.com/nkrishelie/mathempire/raw/master/250/250le\%C3\%A7ons.pdf}{<<Введение в математику>>} Алексея Савватеева, а также как самостоятельный практикум для изучения основ математики. Нумерация глав-уроков в сборнике соответствует урокам \href{https://childrenscience.ru/courses/sav/}{онлайн-курса}, подготовленным проектом <<Дети и наука>>.

В каждом уроке даны ссылки на соответствующие видеоуроки и главы и разделы конспекта. Перед блоком задач даны краткие сведения из курса, содержащие необходимые определения и обозначения.

При составлении задачника было использовано несколько источников, в частности, задачи видеокурса по <<100 урокам математики>> проекта \href{http://childrenscience.ru/}{<<Дети и наука>>}, листки задач кружка \href{https://www.shashkovs.ru/vmsh/}{Вечерней математической школы} в 179 школе г.~Москвы, листки задач для матшкольников из \href{https://dev.mccme.ru/~merzon/listki.html}{подборки Григория Мерзона}.

Составители настоящего сборника: Николай Казимиров, Павел Иванов, Михаил Бочкарев.

Сайт проекта: \href{https://savvateev.xyz/100urokov}{savvateev.xyz}

\quad

\quad \hfill \today



