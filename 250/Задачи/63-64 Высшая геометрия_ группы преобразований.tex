\documentclass[12pt]{article}
\usepackage[utf8]{inputenc}
\usepackage[T2A]{fontenc}
\usepackage[russian, english]{babel}
\usepackage{amsmath}
\usepackage{amssymb}
\usepackage{setspace}
\usepackage{mathtext}
\pagestyle{empty}

\begin{document}
\section*{63-64 Высшая геометрия: группы преобразований}
\textbf{Задача 1.} Какие из указанных множеств с числовыми операциями являются группами?\\
а) (G, +), где G = $\mathbb{N,Z,Q,R,C}$;\\
б) (G, $\cdot$), где G = $\mathbb{N,Z,Q,R,C}$;\\
в) ($G^*$, $\cdot$), где G = $\mathbb{N,Z,Q,R,C}$ (здесь $G^*$ = $G\backslash\{0\}$);\\
г) ($n\mathbb Z$, +), где N $\in \mathbb{N}$;\\
д) ({-1, 1}, $\cdot$);\\
е) ({$a^n$| $n \in \mathbb{Z}$}, $\cdot$), где $a \in R$ и $a \neq 0$\\
\\
2. Доказать, что $[0,1)$ с операцией $\oplus$, где $a \oplus b = \{a+b\}$ - дробная часть числа $a+b$, является группой\\
\\
3. Доказать, что группа преобразований любого множества содержит тождественное преобразование.\\
\\
4. Пусть множество $X$ - это квадрат $ABCD$. Обозначит через $s_{ac},s_{bd},s_{H},s_{V}$ симметрии относительно диагонали $AC$, диагонали $BD$, горизонтали и вертикали квадрата соответственно. Далее, обозначим через $r_0, r_1, r_2, r_3$ повороты вокруг центра на $0^\circ$, $90^\circ$, $180^\circ$, $270^\circ$ соответственно.\\
а) Доказать, что $G = \{s_{ac}, s_{bd}, s_H, s_V, r_0, r_1, r_2, r_3\}$ образуют группу преобразований квадрата.\\
б) Составить таблицу умножения этой группы.\\
в) Придумать группу преобразований квадрата, состоящую из четырёх преобразований.\\
\\
5. а) Описать все преобразования правильного треугольника, сохраняющие расстояния между любыми двумя его точками\\
б) Доказать, что эти преобразования образуют группу.\\
\\
\textbf{Определение 1. }Порядком элемента $g$ группы преобразований $G$ назвается наименьшее натуральное число k, такое, что $g^k = \underbrace{g \circ ... \circ g}_{k} = id$. Обозначение: $ord(g)$.\\
\textbf{Определение 2. }Порядком группы $G$ называется количество элементов в $G$. Обозначение: $|G|$ или $\#G$.\\
\\
6. Перечислить все элементы и их порядки в группах движений следующих множеств:\\
а) прямоугольник\\
б) правильный m-угольник\\
в) правильный тетраэдр\\
г) куб\\
д) октаэдр\\
е) икосаэдр\\
ж) додекаэдр\\
\textbf{Замечание:} Группа из пункта \textbf{б} называется группой диэдра и обозначается $D_m$.
\\
7. Доказать, что $ord(xy) = ord(yx)$ и $ord(x) = ord(yxy^{-1})$.\\
\\
8. Сколько элементов порядка 2 содержится в группе:\\
а) $\mathbb{C}^*$;\\
б) $S_5$;\\
в) $A_5$.\\
\\

\textbf{Определение 3. } Движением или изометрией евклидовой плоскости $\mathbb{R}^2$ называется биективное отображение $\mathbb{R}^2 \to \mathbb{R}^2$, сохраняющее расстояние между точками. Изометрии удобно описывать, используя комплексные числа. Действительно, расстояние между точками $(x_1, y_1)$ и $(x_2, y_2)$ равно $|z_1 - z_2|$, где $z_k = x_k + iy_k$. Значит, изометрии можно рассматривать как функции $f: \mathbb{C} \to \mathbb{C}$, для которых $|f(z_1) - f(z_2)| = |z_1 - z_2|$ для любых $z_1, z_2 \in \mathbb{C}$\\
\\
9. Доказать, что изометрии плоскости образуют группу (иногда она обозначается Isom($\mathbb{R}^2$)).\\
\\
10. Доказать, что любое движение плоскости является композицией не более чем трех симметрий относительно прямых. \\
\\
11. Пусть движение плоскости переводит фигуру $F$ в фигуру $F'$. Для каждой пары соответственных точек $A$ и $A'$ рассмотрим середину $X$ отрезка $AA'$. Докажите, что либо все точки $X$ совпадают, либо все они лежат на одной прямой, либо образуют фигуру, подобную $F$.\\
\\
12. Докажите, что композицию чётного числа симметрий относительно прямых нельзя представить в виде композиции нечётного числа симметрий относительно прямых. 
\end{document}