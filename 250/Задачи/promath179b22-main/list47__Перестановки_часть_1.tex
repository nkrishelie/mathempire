% !TeX encoding = windows-1251
\documentclass[a4paper,11pt]{article}
\usepackage{newlistok}
\usepackage[matrix,arrow]{xy}
%\documentstyle[11pt, russcorr, listok]{article}
\renewcommand{\spacer}{\vspace{1pt}}
\newcommand{\Zm}[1]{\Z/ {#1}\Z}


\УвеличитьШирину{1.1truecm}
\УвеличитьВысоту{2.5truecm}
%\hoffset=-2.5truecm
%\voffset=-27.3truemm
%\documentstyle[11pt, russcorr, ll]{article}

\pagestyle{empty}

\Заголовок{Перестановки. Часть 1.}
\Подзаголовок{}
\НомерЛистка{47}
\ДатаЛистка{06.04.2020 -- 15.04.2020}
\Оценки{24/19/14}

\begin{document}

%\scalebox{1}{\vbox{
%\ncopy{1}{

\СоздатьЗаголовок

% \опр {\it Перестановкой} $n$ элементов называется биекция $\{1,2,\ldots,n\}\rightarrow \{1,2,\ldots,n\}.$
% \копр

\опр
\emph{Перестановка} чисел $1,\dots,n$ --- это взаимно
однозначное отображение множества $\{1,\dots,n\}$ на себя.
Множество перестановок чисел $1,\dots,n$ обозначается $S_n$ и
называется \emph{симметрической группой}.
\копр

\пзадача
Сколько элементов в симметрической группе $S_n$?
\кзадача

Перестановки записывают таблицами вида
$\displaystyle\begin{pmatrix}1&2&3&4\\2&4&1&3\end{pmatrix}$; такая
таблица означает перестановку $1\mapsto2$ (то есть $1$ переходит в
$2$), $2\mapsto4$, $3\mapsto1$, $4\mapsto3$. Вообще, если
$\sigma\in S_n$, то
$\displaystyle\sigma=\begin{pmatrix}1&2&\dots&n\\
\sigma(1)&\sigma(2)&\dots&\sigma(n)\end{pmatrix}$.

\пзадача
Сколько разных таблиц размера $2\times n$ задают одну и ту же перестановку?
\кзадача

\опр
\emph{Произведение} перестановок $\sigma,\tau\in S_n$ определяется
так: $\sigma\tau(i)=\sigma(\tau(i))$ (для произвольных
отображений $\sigma$ и $\tau$ такое произведение обычно называется
\emph{композицией отображений}). Например, если
$$
\llap{$\tau$}=\begin{pmatrix}1&2&3&4\\3&2&1&4\end{pmatrix}
%$$
%$$
,\qquad\quad
\llap{$\sigma$}=\begin{pmatrix}1&2&3&4\\2&4&1&3\end{pmatrix}
,
\qquad
\text{то}
\qquad\quad
\llap{$\sigma\tau$}=\begin{pmatrix}1&2&3&4\\1&4&2&3\end{pmatrix}.

Отметим, что сначала применяется второй сомножитель, а потом
первый.
\копр


%Далее в листке все перестановки переставляют $n$ элементов, если не указано противное.

% \опр
% Композицию двух перестановок $\alpha$ и $\beta$ обозначим $\alpha\circ\beta$: $(\alpha\circ\beta) (k) = \alpha(\beta(k))$.
% \копр

\пзадача
Перед Петей на столе лежат в ряд $n$ шариков, пронумерованные по порядку числами от $1$ до $n$. Петя переставил местами шарики. Пусть $\alpha$ сопоставляет числу $k$ число $\alpha(k)$ --- номер места в ряду, на котором оказался шарик под номером $k$. \пункт Покажите, что $\alpha$ --- перестановка из $S_n$.
\пункт Затем Петя повторил движения рук (опять переставил шарики, даже не глядя на них). На этот раз шарик под номером $k$ оказался на месте под номером $\beta(k)$. Выразите перестановку $\beta$ через перестановку $\alpha$.
\кзадача

\пзадача
\вСтрочку
\пункт
Всегда ли $\sigma\tau=\tau\sigma$?
\пункт
Пусть
$\sigma=\displaystyle\begin{pmatrix}1&2&3&4\\2&4&1&3\end{pmatrix}$,
$\tau=\displaystyle\begin{pmatrix}1&2&3&4\\3&1&4&2\end{pmatrix}$.
Найти $\sigma\tau$ и $\tau\sigma$.
\кзадача

% \задача
% Всегда ли $\alpha\beta=\beta\alpha$, где $\alpha$ и $\beta$ --- перестановки?
% \кзадача

\пзадача
Найдите такую перестановку $e$, что $e\alpha=\alpha e=\alpha$ при всех $\alpha$ (она называется {\it тождественной}). Докажите её единственность.
\кзадача

% \опр Перестановка $e$ из прошлой задачи называется {\it тождественной} перестановкой.
% \копр

\опр
Перестановка $\alpha^{-1}$, такая что $\alpha\alpha^{-1}=e$, называется {\it обратной} к перестановке $\alpha$.
\копр

\пзадача \пункт Докажите, что $\alpha^{-1}$ существует и единственна.
\пункт Найдите $\alpha^{-1}$ для каждой $\alpha$~из~$S_3$.
\кзадача

\пзадача
Какой шарик стоит на месте $k$ после применения перестановки~$\alpha$ из задачи 3?
\кзадача


\пзадача
Пусть $p$ --- простое число, $\Zm{p}$ --- классы вычетов по модулю $p$. Докажите, что
умножение на ненулевой остаток $a\in \Zm{p}$ является перестановкой ненулевых остатков $\{1,2,\ldots,p-1\}$, причём $a=1$ соответствует тождественной перестановке, обратный элемент --- обратной, а произведение --- композиции.
\кзадача

\пзадача
$P(x)$ и $Q(x)$ --- многочлены с целыми коэффициентами.
Пусть $P(Q(x))-x$ делится на $100$
при любом целом $x$. Докажите, что тогда $Q(P(x))-x$ делится на $100$ при любом
целом $x$.
\кзадача


% \задача
% Коробки пронумерованы числами от $1$ до $n$, и в каждой коробке лежит шар с таким же номером. Петя переложил шары в коробках перестановкой $\alpha$, затем Вася незаметно переклеил номера на \пункт коробках \пункт шарах \пункт коробках и шарах перестановкой $\beta$. Таня видела как Петя перекладывал шары, но не заметила как Вася переклеил номера. Тогда Таня подумала, что Петя перекладывал перестановкой $\gamma$. Найдите $\gamma$.
% \кзадача

\пзадача\label{balls}
Во дворе стоят \пункт 17 \пункт 18 мальчиков. У каждого в руках мяч. Вдруг они одновременно кинули свои мячи друг другу. Петя и Вася наблюдали за ними. Петя утверждает, что может мысленно расположить мальчиков в круг так, что каждый кинул стоящему через одного по часовой стрелке. Аналогично Вася, но в кругу Васи каждый кидает стоящему через двух по часовой стрелке. Не врут ли Петя и Вася?
\кзадача

% \опр
% Перестановку, при которой $a_1$ переходит в $a_2$, $a_2$ переходит в $a_3$, \dots, $a_k$ -- в $a_1$ (все остальные элементы остаются на месте, и числа $a_1,a_2,\ldots,a_k$ предполагаются различными), назовём циклом длины $k$ и обозначим $(a_1\ a_2\ \ldots\ a_k)$. Цикл длины $2$ назовём {транспозицией}.
% \копр

\опр
Если элементы $a_1,a_2,\dots,a_k$ различны, то перестановка,
при которой $a_1\mapsto a_2$, $a_2\mapsto a_3$, \dots, $a_k\mapsto a_1$,
а все остальные элементы множества $\{1,\dots,n\}$ переходят в себя, называется
\emph{циклом} и обозначается $(a_1\ a_2\ \ldots\ a_k)$.
% называется перестановка,
%циклически переставляющая элементы\break $a_1,a_2,\dots,a_k$ (то есть
%при которой $a_1\mapsto a_2$, $a_2\mapsto a_3$, \dots, $a_k\mapsto a_1$;
%имеется в виду, что все
%элементы $a_1,a_2,\dots,a_k$ раз\-ли\-ч\-ны; все
%остальные элементы множества $\{1,\dots,n\}$ переходят в себя).
Число $k$ называют \выд{длиной} цикла. Цикл длины $2$ называется {\it транспозицией}.
\копр


\пзадача
Сколько всего различных циклов длины $k$ в $S_n$?
\кзадача

\пзадача
Докажите, что любая перестановка из $S_n$ однозначно, с точностью до порядка множителей, разлагается в произведение <<непересекающихся>> ({\em независимых}) циклов (циклы длины 1 обычно пропускают).
\кзадача

\пзадача
Какие перестановки из $S_4$ --- не циклы? Разложите их в произведение независимых циклов.
\кзадача

\пзадача
Текст на русском языке зашифрован программой, заменяющей взаимно однозначно каждую
букву на некоторую другую.
\вСтрочку
\пункт
Докажите, что существует такое число $k$, что текст
расшифровывается применением $k$ раз шифрующей программы.
\пункт
Найдите хотя бы одно такое $k$.
\кзадача

\опр Минимальное натуральное $k$ такое, что $\alpha^k$ --- тождественная перестановка, называется {\it порядком} перестановки $\alpha$ и обозначается $\ord\alpha$.
\копр

\пзадача
Найдите порядки: \пункт перестановок из $S_3$; \пункт цикла длины $k$; \пункт перестановок задачи \ref{balls}.
\кзадача

% \задача
% Докажите, что порядки сопряжённых перестановок совпадают. Верно ли обратное?
% \кзадача

\пзадача Найдите все $\alpha$ из $S_n$, для которых $\alpha=\alpha^{-1}$.
\кзадача

\пзадача
Пусть $\alpha$ --- это $(1\ 2 \dots n)^k$.
На сколько независимых циклов раскладывается~$\alpha$, каковы их длины?
\кзадача

\пзадача
Найдите максимальный возможный порядок перестановки
\вСтрочку
\пункт из $S_5$;
\пункт из $S_{13}$.
%из $S_n$.
\кзадача

\пзадача
Докажите, что порядок перестановки из $S_n$ делит $n!$.
\кзадача

%\ЛичныйКондуит{0mm}{4mm}

% \GenXML

\end{document}


\опр
Назовём две перестановки $\alpha$ и $\beta$  сопряжёнными, если существует перестановка $\gamma$ такая, что $\alpha=\gamma\beta\gamma^{-1}$.
\копр

\задача
Докажите, что две перестановки сопряжены, если и только если наборы длин циклов в их разложении на непересекающиеся циклы совпадают.
\кзадача

\задача
Найдите хотя бы одну перестановку $x$, такую что $x^2=\alpha$, где $\alpha$ -- перестановка.
\кзадача

\задача
Рассмотрим дерево на $n$ пронумерованных вершинах. Ребру, соединяющему вершины $i$ и $j$ сопоставим транспозицию $(i\ j)$. Возьмём композицию этих транспозиций в некотором порядке. Покажите, что получится цикл длины $n$.
\кзадача


\begin{center}{\bf ***}
\end{center}

\опр Непорядком перестановки $\alpha$ назовём количество пар $i<j$, таких что $\alpha(i)>\alpha(j)$.
\копр

\задача
Верно ли, что перестановки одинакового непорядка сопряжены?
\кзадача

\задача
Найдите непорядок перестановки $\alpha^{-1}$, зная непорядок перестановки $\alpha$.
\кзадача

\задача
Найдите перестановку с максимальным непорядком.
\кзадача

\сзадача
Сколько всего циклов длины $n$ с минимальным непорядком?
\кзадача

\опр
Назовём чётностью перестановки чётность её непорядка.
\копр

\задача
Найдите чётность композиции двух перестановок, если даны их чётности.
\кзадача

\задача
Найдите чётность цикла, если дана чётность его длины.
\кзадача

\задача
Сколько чётных перестановок?
\кзадача

\задача
В игре "пятнашки" поменяли квадраты с числами $1$ и $2$ местами. Можно ли из этой позиции по правилам игры получить исходную?
\кзадача

\опр Обозначим за $\sigma_i$ транспозицию $(i\ i{+}1)$.
\копр

\задача
\пункт Докажите, что любая перестановка является композицией транспозиций $\sigma_i$.\\
\пункт Как связаны минимальное количество транспозиций $\sigma_i$ в разложении перестановки и её непорядок?
\кзадача

\задача Дан граф с пронумерованными вершинами числами от $1$ до $n$. Докажите, что любую перестановку можно представить в виде композиции транспозиций $(i\ j)$, где вершины $i$ и $j$ соединены ребром, если и только если граф связен.
\кзадача

\задача Какие перестановки являются композицией \пункт транспозиций $(1\ 2),(1\ 3),\ldots,(1\ n)$;\\
\пункт транспозиции $(1\ 2)$ и цикла $(1\ 2\ \ldots\ n)$ \пункт циклов длины $3$?
\кзадача

\задача
\пункт Перестановке $\alpha$ сопоставим набор чисел $a_1,a_2,\ldots,a_{n-1}$, где $a_i$ -- число пар $i<j$, таких что $\alpha(i)>\alpha(j)$. Очевидно, что $0\leqslant a_i\leqslant n{-}i$. Докажите, что набор чисел $a_1,a_2,\ldots,a_{n-1}$, полученный таким образом, однозначно определяет перестановку, а также любой набор, удовлетворяющий выписанным неравенствам, получается из некоторой перестановки указанным способом.\\
\пункт Покажите, что любая перестановка однозначно представляется в виде $$(1{+}a_1\ \ldots\ 2\ 1)\circ(2+a_2\ \ldots\ 3\ 2)\circ(n{-}1+a_{n-1}\ \ldots\ n\ n{-}1),$$ где $0\leqslant a_i\leqslant n{-}i$.\\
\пункт Пусть $s_l$ -- количество перестановок непорядка $l$. Покажите, что
$$1+s_1x+s_2x^2+s_3x^3+\ldots=(1+x)(1+x+x^2)\ldots(1+x+\ldots+x^{n-1}).$$
\кзадача

\задача
Рассмотрим всевозможные композиции перестановок $\sigma_1,\sigma_2,\ldots,\sigma_{n-1}$ (в зависимости от порядка). Сколько получится различных перестановок?
\кзадача

\begin{center} {\bf ***}
\end{center}

\задача
\пункт Постройте соответствие между перестановками трёх элементов и движениями плоскости, переводящими равносторонний треугольник в себя такое, что композиции перестановок будет соответствовать композиция соответствующих движений.\\
\пункт Аналогично постройте соответствие между чётными перестановками четырёх элементов и вращениями пространства, переводящих правильный тетраэдр в себя.\\
\пункт Аналогично постройте соответствие между перестановками четырёх элементов и вращениями пространства, переводящих куб в себя.
\кзадача

%\СделатьКондуит{4mm}{9mm}

\end{document}
\vspace*{2pt}

\noindent
\rule{10cm}{.3pt}

%}}}

