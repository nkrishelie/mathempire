\documentclass[12pt]{article}
\usepackage[utf8]{inputenc}
\usepackage[T2A]{fontenc}
\usepackage[russian]{babel}
\usepackage{amsmath}
\usepackage{amssymb}
\usepackage{setspace}
\usepackage{indentfirst}
\usepackage{mathtext}

\newenvironment{psmallmatrix}
  {\left(\begin{smallmatrix}}
  {\end{smallmatrix}\right)}


\begin{document}
\section*{Группа квадратных невырожденных матриц}
Ниже представлены основные определения и задачи к лекции «Группа квадратных невырожденных матриц» к курсу «100 уроков математики» Алексея Владимировича Саватеева.
Задачи разделены на 2 вида: типовые и нетиповые.

{\bf Определение 1.} Матрица $B$ называется обратной по отношению к матрице $A$, если $AB = BA = E$. При этом используется обозначение $B = A^{-1}$. Обратная матрица находится по формуле:
\[
A^{-1} = \frac{1}{det A}\begin{pmatrix}
A_{11} & A_{21} & \dots & A_{n1}\\
A_{12} & A_{22} & \dots & A_{n2}\\
\vdots & \vdots & \ddots& \vdots\\
A_{1n} & A_{2n} & \dots & A_{nn}\\
\end{pmatrix}, \text{где}
\] $A_{mn}$ - алгебраическое дополнение элемента матрицы $a_{mn}$.

\textbf{Определение 2.} Алгебраическим дополнением к элементу $a_{ij}$ определителя n-ого порядка называется число $A_{ij}$ = $(-1)^{i+j}$ $\cdot$ $M_{ij}$, где $M_{ij}$ - определитель порядка (n-1), полученный из исходной матрицы вычёркиванием i-ой строки и j-ого столбца.

\subsection*{Типовые задачи}
1. Вычислить определители:
\[
\text{a)}
\begin{vmatrix}
4 & -5\\
3 & 8
\end{vmatrix},
\text{б)}
\begin{vmatrix}
5 & -2\\
-9 & 6
\end{vmatrix},
\text{в)}
\begin{vmatrix}
-10 & 2\\
-6 & 7
\end{vmatrix},
\text{г)}
\begin{vmatrix}
1,5 & -0,2\\
0,3 & -4
\end{vmatrix}
\]

2. Вычислить определители
\[
\text{а)}
\begin{vmatrix}
7 & -4\\
5 & 2
\end{vmatrix},
\text{б)}
\begin{vmatrix}
3 & -7 & 1\\
-2 & 4 & -3\\
9 & -1 & -5
\end{vmatrix},
\text{в)}
\begin{vmatrix}
3 & -5 & 2\\
4 & 1 & 5\\
2 & 7 & -3
\end{vmatrix},
\text{г)}
\begin{vmatrix}
4 & -3 & 5 & 6\\
-7 & 2 & 1 & 8\\
 0 & 5 & 4 & 0\\
 9 & 7 & 0 & -2
\end{vmatrix}
\]

3. Вычислить опеределитель Вандермонда:
\[
\Delta = \begin{vmatrix}
1 & 1 & 1\\
x & y & z\\
x^2 & y^2 & z^2
\end{vmatrix}
\]

Выяснить, при каких значениях $x,y,z$ определитель равен нулю.

4. Вычислить следующие опеределители
\[
\text{а)}\begin{vmatrix}
\sin^2{\alpha} & 1 & \cos^2{\alpha}\\
\sin^2{\beta} & 1 & \cos^2{\beta}\\
\sin^2{\gamma} & 1 & \cos^2{\gamma}\\
\end{vmatrix},
\text{б)}\begin{vmatrix}
1 & 1 & 1\\
x & y & z\\
x^3 & y^3 & z^3
\end{vmatrix}
\text{в)}\begin{vmatrix}
\cos{2\alpha} & \cos^2{\alpha} & \sin^2{\alpha}\\
\cos{2\beta} & \cos^2{\beta} & \sin^2{\beta}\\
\cos{2\gamma} & \cos^2{\gamma} & \sin^2{\gamma}\\
\end{vmatrix},
\]

5. Вычислить определители четвёртого порядка
\[
\text{а)}\begin{vmatrix}
2 & 1 & 0 & 2\\
3 & 2 & 1 & 0\\
-1 & 0 & 1 & 3\\
-1 & 2 & 1 & 3
\end{vmatrix},
\text{б)}\begin{vmatrix}
a & b & b & a\\
1 & 2 & 1 & 0\\
3 & 2 & 0 & 1\\
2 & 0 & 1 & 2
\end{vmatrix},
\text{в)}\begin{vmatrix}
0 & a & b & c\\
1 & x & 0 & 0\\
1 & 0 & y & 0\\
1 & 0 & 0 & z
\end{vmatrix},
\]

6. Найти матрицы, обратные для следующих:
\[
\text{a)} \begin{pmatrix}
0 & 1\\
1 & 0
\end{pmatrix},
\text{б)} \begin{pmatrix}
4 & -2 & 0\\
1 & 1 & 2\\
3 & -2 & 0
\end{pmatrix},
\text{в)}
\begin{pmatrix}
3 & 0 & 0\\
0 & 1 & 0\\
0 & 0 & 1
\end{pmatrix}
\]

7. Найти матрицы, обратные для следующих:
\[
\text{а) }\begin{pmatrix}
1 & -2\\
3 & 4
\end{pmatrix},
\text{б) }\begin{pmatrix}
\sin x  & \cos x \\
-\cos x & \sin x
\end{pmatrix},
\text{в) }\begin{pmatrix}
1 & 3 & 4\\
2 & 0 & 3\\
-2 & 1 & -3
\end{pmatrix},
\text{г) }\begin{pmatrix}
a & 0 & 0\\
0 & b & 0\\
0 & 0 & c
\end{pmatrix}
\]

8. Доказать, что определитель диагональной матрицы равен произведению её диагональных элементов.

9. Доказать, что определитель треугольной матрицы равен произведению её диагональных элементов.

10. Как изменится определитель матрицы, если все элементы матрицы заменить комплексно-сопряженными числами?

11. Вычислить определители матриц:
\[
\text{а)}\begin{vmatrix}
\alpha & 1\\
-1 & \alpha
\end{vmatrix},
\text{б)}\begin{vmatrix}
1 & 1 & 1\\
1 & \alpha & \alpha^2\\
1 & \alpha^2 & \alpha
\end{vmatrix}
\] при: 1. $\alpha = e^{\pi i/3}$, 2. $\alpha = e^{2\pi i/3}$


12. Доказать, что квадратную матрицу с помощью элементарных преобразований строк можно перевести в единичную тогда и только тогда, когда она невырождена. Сформулировать и доказать аналогичное свойство для элементарных преобразований столбцов матрицы.

13. Матрица $A$ коммутирует с $B$. Доказать, что тогда $A_{-1}$ коммутирует с $B_{-1}$ (предполагается, что матрицы обратимы).


\subsection*{Нетиповые задачи}	

1. Придумайте две матрицы C и D так, чтобы: а) $CD = D$; б) $CD = DC$; в) $CD = DC$.

2. Докажите, что матрицу с положительным определителем можно представить в виде произведения матрицы с единичным определителем и константы.

3. Вычислить определители n-ого порядка
\[
\delta_{n} = \begin{vmatrix}
1 & 2 & 2 & \dots & 2\\
2 & 2 & 2 & \dots & 2\\
2 & 2 & 3 & \dots & 2\\
\vdots & \vdots & \vdots & \ddots & \vdots\\ 
2 & 2 & 2 & \dots & n
\end{vmatrix}; 
\Delta_{n} = 
\begin{vmatrix}
2 & 1 & 0 & 0 & \dots & 2\\
1 & 2 & 1 & 0 & \dots & 2\\
0 & 1 & 2 & 1 & \dots & 2\\
\vdots & \vdots & \vdots & \vdots & \ddots & \vdots\\
0 & 0 & 0 & 0 & \dots & 2
\end{vmatrix}
\]
{\small Указание: Показать, что $\Delta_{n} = 2\Delta_{n-1} - \Delta_{n-2}$. Найти $\Delta_{2}$ и $\Delta_{3}$.}

4. Как изменится определитель, если все элементы матрицы заменить комплексно-сопряжёнными числами?

5. Пусть det A $\neq$ 0. Доказать, что применяя к строкам матрицы элементарные преобразования, сохраняющие определитель, можно получить:
а) треугольную матрицу; б) диагональную матрицу

6. Вычислить $det A$, зная, что в матрице $A$ сумма строк с чётными номерами равна сумме строк с нечётными номерами.

7. Доказать, что для любой вещественной матрицы $A$ выполнено условие $det(AA^T) \ge 0$.

8. Доказать, что определитель кососимметрической матрицы нечётного порядка равен 0.

9. Показать, что определитель матрицы $A$ порядка n равен нулю, если в ней имеется нулевая подматрицы размеров $k \times l$, и $k + l > n$.

10. Числа 1081, 1403, 2093, 1541 делятся нацело на 23. Не производя вычислений, объяснить, почему det A также делится на 23.
\[
A = \begin{vmatrix}
1 & 0 & 8 & 1\\
1 & 4 & 0 & 3\\
2 & 0 & 9 & 3\\
1 & 5 & 4 & 1
\end{vmatrix}
\]

11. Показать, что вещественная матрица $A$ размера $2011 \times 2011$ вырождена тогда и только тогда, когда её можно превратить в $-A$ элементарными преобразованиями вида прибавление к одной строке другой строки, умноженной на число.

12. В квадратной матрице $A$ столбцы являются попарно ортогональными векторами. Докажите, что абсолютная величина определителя матрицы $A$ равна произведению длин векторов-столбцов.

\end{document} 