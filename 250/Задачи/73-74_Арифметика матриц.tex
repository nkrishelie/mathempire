\documentclass[12pt]{article}
\usepackage[utf8]{inputenc}
\usepackage[T2A]{fontenc}
\usepackage[russian]{babel}
\usepackage{amsmath}
\usepackage{amssymb}
\usepackage{setspace}
\usepackage{indentfirst}
\usepackage{mathtext}

\newenvironment{psmallmatrix}
  {\left(\begin{smallmatrix}}
  {\end{smallmatrix}\right)}
  

\begin{document}
\section*{Арифметика матриц}
Ниже представлены основные определения и задачи к лекции «Арифметика матриц» к курсу «100 уроков математики» Алексея Владимировича Саватеева.
Задачи разделены на 2 вида: типовые и нетиповые.
\subsection*{Типовые задачи}
1. Найти $3A-B-4C$, если:
\[
A = \begin{pmatrix}1&2\\1&2\end{pmatrix}, 
B = \begin{pmatrix}3&2\\3&2\end{pmatrix}, 
C = \begin{pmatrix}0&1\\0&1\end{pmatrix}
\]

2. Найти $2A-B$, если:
\[
A = \begin{pmatrix}
1 & 8 & 7 & -15\\
1 & -5 & -6 & 11
\end{pmatrix},
B = \begin{pmatrix}
5 & 24 & -7 & -1\\
-1 & -2 & 7 & 3
\end{pmatrix}
\]

3. Найти $AB - BA$ если
\[
A = \begin{pmatrix}
2 & 1 & 0\\
1 & 1 & 2\\
-1 & 2 & 1
\end{pmatrix},
B = \begin{pmatrix}
3 & 1 & -2\\
3 & -2 & 4\\
-3 & 5 & -1
\end{pmatrix}
\]

4. Найти $(A-B)\cdot A+2B$, если
\[
A = \begin{pmatrix}
1 & 0 & 2\\
0 & 1 & 1
\end{pmatrix},
B = \begin{pmatrix}
0 & 2 & 1\\
1 & 1 & 1
\end{pmatrix}
\]

5. Показать, что $D(2,\mathbb{Q})\cap SL(2,\mathbb{Q}) \cong \mathbb{Q}^*$

6. Найти $A^3$ для следующих матриц:
\[
\text{a)}
A = \begin{pmatrix}
1 & 0\\
3 & 4
\end{pmatrix},
\text{б)}
A = \begin{pmatrix}
3 & 4\\
-2 & 1
\end{pmatrix},
\text{в)}
A = \begin{pmatrix}
2 & 1 & 1\\
3 & 1 & 0\\
0 & 1 & 2
\end{pmatrix}
\]

7. Найти $A^n$ для матрицы 
A = $\begin{pmatrix}
1 & 1\\
0 & 1
\end{pmatrix}$

\subsection*{Нетиповые задачи}	
\textbf{Определение:} Матрицы, для которых выполняется равенство $AB = BA$ называются {\it перестановочными}.

\textbf{Определение:} Целая неотрицательная степень матрицы определяется равенством $A^n = \underbrace{A \cdot A \cdot ... \cdot A}_\text{n раз}$
\textbf{Определение:} Матрица $A$ называется кососимметрической, если она удовлетворяет соотношению $A = -A^T$. Такая матрица имеет вид
\[
\begin{pmatrix}
0 & a_{12} & a_{13} & \dots & a_{1n}\\
-a_{12} & 0 & a_{23} &  \dots & a_{2n}\\
-a_{13} & -a_{23} & 0 &  \dots & a_{3n}\\
\vdots & \vdots & \vdots & \ddots & \vdots\\ 
-a_{1n} & -a_{2n} & -a_{3n} & \dots & 0\\
\end{pmatrix}
\]

1. Найти общий вид матриц А второго порядка, квадрат которых равен нулевой матрице, т.е. $A^2 = O$.

2. Найти все матрицы А второго порядка, квадрат которых равен диагональной матрице 
$\begin{psmallmatrix}a & 0\\0 & b\end{psmallmatrix}$,$a \neq b$.

3. Найти условие, при котором матрица $A$ второго порядка перестановочна со всеми матрицами второго порядка.

4. Матрица $A$ называется \emph{инволютивной}, если $A^2 = I$, и \emph{идемпотентной}, если $A^2 = A$. Найти общий вид инволютивной и идемпотентной матрицы.

5. Каким условиям должны удовлетворять элементы матрицы $A$ второго порядка, для того чтобы она была перестановочна со всеми диагональными матрицами того же порядка?

6. Найти все степени матрицы A = $\begin{pmatrix}
0 & 1 & 0\\
0 & 0 & 1\\
0 & 0 & 0
\end{pmatrix}$

7. Показать на примере матриц второго порядка, что равенство $AB - BA = I$ невозможно.

8. Найти общий вид матрицы $A$ третьего порядка, для которой
\[\begin{pmatrix}
0 & 1 & 0\\
0 & 0 & 1\\
0 & 0 & 0
\end{pmatrix} \cdot A = O
\]

9. Найти все матрицы, перестановочные с данными:
\[
\text{a)}\begin{pmatrix}
0 & 1\\
-1 & 0\\
\end{pmatrix},
\text{б)}\begin{pmatrix}
2 & 0 & 0\\
0 & 1 & 1\\
0 & 0 & 1
\end{pmatrix},
\text{в)}\begin{pmatrix}
0 & 1 & 0 & 0\\
0 & 0 & 1 & 0\\
0 & 0 & 0 & 1\\
0 & 0 & 0 & 0
\end{pmatrix}
\]

10. Показать, что матрицы A и B - перестановочны, если
\[
A = \begin{pmatrix}
3 & 1 & -2\\
3 & -2 & 4\\
-3 & 5 & -1
\end{pmatrix}
B = \begin{pmatrix}
2 & 1 & 0\\
1 & 1 & 2\\
-1 & 2 & 1
\end{pmatrix}
\]

11. Показать, что для любой матрицы $A$ матрица $ K = A - A^T$ - кососимметрическая

\renewcommand{\thefootnote}{ }
\footnote{Author: Svyatoslav Nikitin}
\end{document} 