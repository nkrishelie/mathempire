\part*{Уровень Бета}
\addcontentsline{toc}{part}{Уровень Бета}

\renewcommand{\thechapter}{B.\arabic{chapter}}

\setcounter{chapter}{0}


\newchapter{Математическая логика}
Логика высказываний, правила вывода, формальные системы, теорема Гёделя (формулировка)

\newchapter{Аксиоматическая теория множеств}
Парадокс Расселла, аксиомы, отношения и функции, факторы, построение натуральных чисел, ординалы и кардиналы, аксиома выбора и континуум-гипотеза, теорема Гудстейна

\newchapter{Общая алгебра}
Алгебраическая структура, идеалы, делимость, ОТА в области целостности, модуль над кольцом/полем





\section{*Вычеты и операции Минковского}\label{Faktor}




\begin{enumerate}
\item Вернемся к арифметическим операциям над множествами. Пусть задано целое число $m>1$, тогда
$$
m\Z = \{mk\mid k\in \Z\}.
$$
\item Как мы помним, это --- кольцо, т.е. в $m\Z$ можно складывать, вычитать и умножать, но нельзя делить любое число на любое ненулевое. Что будет если сдвинуть его на некоторое целое число? Т.е. взять множество
$$
m\Z+n = \{mk+n\mid k\in\Z\}
$$
\item При каких $n$ множество $m\Z+n$ останется кольцом? В кольце должен быть ноль, следовательно, если $m\Z+n$ --- кольцо, то при некотором $k$ имеем $mk+n=0$, откуда следует, что $n$ кратно $m$. Обратно, если $n$ кратно $m$, то $m\Z+n=m\Z$. Действительно, $n=km$, и тогда $ml+n=m(l+k)\in m\Z$, т.е. $m\Z+n\subseteq m\Z$. Кроме того, $ml=m(l-k)+mk=m(l-k)+n$, откуда $m\Z\subseteq m\Z+n$. Таким образом, $m\Z+n=m\Z$.
\item Итак, $m\Z+n$ остается кольцом тогда и только тогда, когда $n$ кратно $m$, причем это все то же кольцо $m\Z$.
\item Пусть теперь $n=mk+d$, где $d$ --- остаток от деления $n$ на $m$.
\item В этом случае $m\Z+n=m\Z+mk+d=m\Z+d$. Отсюда легко получить следующее совйство
$$
m\Z+n = m\Z+n' \iff n\equiv n' \pmod m,
$$
т.е. сложение с $m\Z$ в каком-то смысле напоминает операцию сложения по модулю $m$ --- оно <<забывает>> все, что кратно $m$, оставляя только остаток.
\item Это значит, что существует ровно $m$ различных множеств вида $m\Z+n$, а именно:
$$
m\Z,\quad m\Z+1,\quad\dots,\quad m\Z+m-1.
$$
\item Далее, эти множества попарно не пересекаются и в сумме дают все $\Z$. Это утверждение предлагается доказать самостоятельно.
\item \textbf{Важный логический шаг!} Рассмотрим теперь множества $m\Z+n$ как новые элементы (т.е. мы забываем их природу и считаем их отдельными точками, такими же, как до этого считали целые числа) и соберем из них новое множество
$$
\Z/m\Z = \{m\Z,\quad m\Z+1,\quad\dots,\quad m\Z+m-1\},
$$
которое в алгебре называется \textbf{фактормножеством}.
\item Наконец, вспомним о том, что мы можем умножать и складывать множества, т.е. определны операции Минковского
$$
(m\Z+n)+(m\Z+n'),\quad (m\Z+n)(m\Z+n').
$$
\item Нетрудно показать следующие свойства этих операций:
\begin{enumerate}[Z1]
\item $(m\Z+n)+(m\Z+n') = m\Z+(n+n'\mod m)$
\item $(m\Z+n)(m\Z+n') = m\Z+(nn'\mod m)$
\end{enumerate}
Действительно, $mk+n+mk'+n'\equiv n+n'\pmod m$ и $(mk+n)(mk'+n')\equiv nn'\pmod m$.
\item Это значит, что операции Минковского над элементами $\Z/m\Z$ в точности дают алгебру остатков, которую мы рассматривали выше.
\item То есть $\Z/m\Z$ --- кольцо, построенное на фактормножестве, причем его операциями являются операции Минковского, определенные через операции исходного кольца. Такое кольцо назвается \textbf{факторкольцом} кольца $\Z$.\index{Кольцо!факторкольцо}
\item \textbf{Зафиксируем}: в исходном кольце (например, $\Z$) рассматривается подкольцо (например, $m\Z$) и все его сдвиги, полученные смещением на элементы этого же кольца, получается набор множеств, попарно не пересекающихся и дающих в сумме исходное кольцо, далее на этих множествах вводятся операции сложения и умножения, полученные как операции Минковского. Итоговая стрктура называется факторкольцом.
\item Аналогично можно построить такое понятие как факторгруппа, воспользовавшись лишь одной операцией --- сложением.
\item Факторкольца и факторгруппы являются мощным инструментом абстракции и получения общих результатов в алгебре и теории множеств.
\end{enumerate}


\subsection*{Задачи}
\begin{enumerate}
\item Доказать, что $m\Z+n\cap m\Z+n'=\emptyset$, если $0\le n<n'\le m-1$.
\item Доказать, что
$$
m\Z\cup (m\Z+1)\cup\dots\cup (m\Z+m-1) = \Z.
$$
\item Построить факторкольцо $(\Z/6\Z)/2(\Z/6\Z)$. Алгебру остатков по какому модулю мы получим?
\item Построить факторкольцо $(\Z/6\Z)/5(\Z/6\Z)$. Почему получается одноэлементное фактормножество, т.е. тривиальное кольцо, состоящее из одного нуля?
\end{enumerate}





\section{Конечные группы}

\lesson{Повторение: группы. Единственность единицы и обратного элемента, порядок элемента, система образующих, циклическая группа}

\begin{enumerate}
\item Рассмотрим группу $G$, состоящую из $n$ элементов, с операцией $\cdot$ (обозначение которой мы часто будем пропускать для удобства). В терминах функций операция $\cdot$ --- это функция из $G\times G$ в $G$:
$$
\cdot: G\times G\to G.
$$

Напомним аксиомы группы:\index{Группа}
\begin{enumerate}[G1]
\item $ab\in G$ для всех $a,b\in G$ (группоид);
\item для любых $a,b,c\in G$ имеем тождество $(ab)c=a(bc)$ (ассоциативность);
\item существует элемент $\e\in G$ такой, что $a\e=\e a=a$ для всех $a\in G$ (единица);
\item для всякого $a\in G$ существует обратный элемент $a^{-1}\in G$ такой, что $aa^{-1}=a^{-1}a=\e$ (обратный элемент).
\end{enumerate}
Кроме того, группа называется абелевой (или коммутативной), если $ab=ba$ для всех $a,b\in G$. Количество элементов в группе называется ее порядком.
\item В группе существует только одна единица. Действительно, если их две $\e$ и $\e'$, то в силу их же свойств получим
$$
\e' = \e\e' = \e
$$
(при первом равенстве мы рассматривали $\e$ как единицу, а при втором $\e'$).
\item Обратный элемент для каждого $a\in G$ определен однозначно. Предположим, что для элемента $a$ нашлось два обратных элемента $b,c$, т.е. $ab=ba=\e$ и $ac=ca=\e$. Тогда
$$
b=b\e=b(ac)=(ba)c=\e c=c.
$$
\item Степень элемента $\underbrace{a\cdots a}_{k\mbox{ раз}}$ корректно определяется в силу закона ассоциативности и обозначается $a^k$, где $k\in\N$. Кроме того, по определению, $a^0=\e$.
\item Отрицательная степень элемента по определению: $a^{-k}=(a^{-1})^k$, $k\in\N$.
\item Операции со степенями:
$$
(a^k)(a^m)=a^{k+m},
$$
где $k,m\in \Z$. Если $k$ и $m$ одного знака, то это очевидно, а если разного, то пусть $k>0$, $m<0$, тогда
$$
(a^k)(a^m) = \underbrace{a\cdots a}_{k\mbox{ раз}}\underbrace{a^{-1}\cdots a^{-1}}_{|m|\mbox{ раз}}.
$$
Пользуясь ассоциативностью, начинаем сворачивать пары $aa^{-1}$, стоящие в середине, заменяя их на $\e$, а затем выбрасывая $\e$. В итоге либо ничего не останется (когда $k=-m$), либо останутся только $a$ в количестве $k+m$ (если $k>-m$), либо останутся только $a^{-1}$ в количестве $-m-k$ (когда $k<-m$). В любом случае это записывается как $a^{k+m}$ ($(a^{-1})^{-m-k}=a^{m+k}$ по определению).

\item В конечной группе каждый элемент в некоторой конечной степени обращается в $\e$. Действительно, все степени $a^k$ лежат в конечном множестве $G$, а число $k$ пробегает бесконечный науральный ряд. Следовательно, хотя бы два разных $k$ дадут один и тот же элемент (принцип Дирихле): $a^k=a^{k'}$, где $k<k'$. Домножим это равенство на $a^{-k}$ и получим $a^{k'-k}=\e$. Наименьший положительный показатель степени $m$ для элемента $a$, дающий равенство $a^m=\e$, называется порядком элемента $a$ в группе $G$.

Таким образом, в конечной группе у всякого элемента --- конечный порядок.

\item Отсюда следует, что всякую отрицательную степень элемента в конечной группе можно записать как положительную, поскольку
$$
a^k = a^{k\pmod p},
$$
где $p$ --- порядок элемента $a$.

\item Подмножество $T\subseteq G$, все возможные произведения степеней элементов которого, т.е. выражения вида $t_1^{k_1}\cdots t_m^{k_m}$, где $t_j\in T, k_j\in\N$ , образуют всю группу $G$, называется \textbf{системой образующих}\index{Группа!система образующих} или \textbf{порождающим множеством} группы $G$. При этом пишут $G=\langle T\rangle$ или $G=\langle t_1,\dots,t_m\rangle$. Элементы системы образующих называются \textbf{образующими} группы.
\item Если система образующих состоит из одного нетривиального элемента, то группа называется \textbf{циклической}.\index{Группа!циклическая} При этом ее можно записать так: $G=\langle g\rangle$, где $T=\{g\}$. Иначе говоря, циклическая группа состоит из степеней одного своего элемента.
\item Например, группа $\Z/n\Z$ вычетов по модулю $n$ с операцией сложения является циклической: $\Z/n\Z=\langle 1\rangle$, поскольку все ее элементы --- это конечные суммы единиц (от одной до $n$ штук). Группа вращений правильного $n$-угольника является циклической, где образующим элементом является поворот на угол $2\pi/n$. Группа $(\Z/5\Z)^*$, состоящая из элементов $1,2,3,4$, с операцией умножения по модулю 5 является циклической, $(\Z/5\Z)^*=\langle 2\rangle=\langle 3\rangle$.
\item Циклические группы являются абелевыми. Действительно, любые два элемента такой группы --- это некоторые степени образующего элемента, поэтому $(a^k)(a^m)=a^{k+m}=a^{m+k}=(a^m)(a^k)$. Здесь коммутативность наследуется от сложения в группе $\Z$, где $k+m=m+k$.


\lesson{Подгруппа, классы смежности, теорема Лагранжа}


\item Подмножество $H\subseteq G$ называется подгруппой группы $G$, если $H$ само является группой с той же операцией, которая определена в $G$. Например, $\{0,2\}$ образует подгруппу группы $\Z_4$. Тривиальная подгруппа $\{\e\}$ является подгруппой любой группы.

\item Операция Минковского умножения элемента группы на ее подмножество порождает <<смежные классы>>:
$$
gH=\{gh\mid h\in H\},\quad Hg=\{hg\mid h\in H\},
$$
где $gH$ называется левым, а $Hg$ --- правым \textbf{смежным классом} (или классом смежности), порожденным элементом $g\in G$.

\item Классы смежности по данной подгруппе $H$ содержат одинаковое количество элементов.

Действительно, пусть $h_1\ne h_2$, где $h_1,h_2\in H$. Предположим, что $gh_1=gh_2$. Домножая слева на $g^{-1}$, находим, что $h_1=h_2$. Противоречие. Следовательно, умножение на $g$ слева различные элементы переводит в различные. Аналогично --- для умножения справа. Т.о. $|gH|=|Hg|=|H|$ для любой подгруппы $H\subseteq G$ и любого элемента $g\in G$.

\item Классы смежности подгруппы $H$ либо совпадают, либо не пересекаются, а их объединение равно $G$. Иными словами, классы смежности образуют разбиение множества $G$. Такую ситуацию мы уже наблюдали в связи с подгруппами $m\Z$ и их сдвигами внутри $\Z$ и получали там $m$ классов смежности.

Пусть классы $g_1H$ и $g_2H$ имеют общий элемент $g$. Этот элемент будет иметь два представления: $g=g_1h_1=g_2h_2$, где $h_1,h_2\in H$, откуда $g_1=g_2h_2(h_1)^{-1}$. Возмем любой элемент $g_1h$ из первого класса, тогда
$$
g_1h = g_2h_2(h_1)^{-1}h,
$$
где $h_2(h_1)^{-1}h\in H$, т.к. $H$ --- подгруппа. Следовательно, $g_1h\in g_2H$, и $g_1H\subseteq g_2H$. Аналогично рассуждая, находим, что $g_2H\subseteq g_1H$, т.е. $g_1H=g_2H$.

Тот факт, что любой элемент $G$ находится в каком-то классе смежности, следует из того, что $\e\in H$, так что для любого $g\in G$ имеем $g\in gH$. И аналогично для правых классов.

\item Итак, множество $G$ есть объединение непересекающихся классов одного размера, причем размер классов равен порядку подгруппы $H$. Следовательно, порядок подгруппы делит подрядок группы. Это утверждение называется \textbf{теоремой Лагранжа}.\index{Теорема!Лагранжа о порядке группы}

\item $g_1H=g_2H$ тогда и только тогда, когда $(g_1)^{-1}g_2\in H$.

Пусть $g_1H=g_2H$, тогда любой элемент из этого множества можно записать двумя способами: $g_1h_1=g_2h_2$, где $h_1,h_2\in H$. Домножая это равенство на $h_2^{-1}$ справа и на $g_1^{-1}$ слева, получаем, что $h_1(h_2)^{-1}=(g_1)^{-1}g_2$. В левой части равенства стоит элемент подгруппы $H$, стало быть, элемент $(g_1)^{-1}g_2$ --- это элемент подгруппы $H$.

Обратно, пусть $(g_1)^{-1}g_2=h\in H$, тогда $g_1h=g_2\e$. Элемент слева принадлежит $g_1H$, элемент справа --- $g_2H$, т.е. эти классы имеют общий элемент, а значит, совпадают.

\item Если группа имеет порядок $p$, где $p$ простое число, то такая группа является циклической.

Действительно, возьмем элемент $g\ne\e$ (поскольку $p>1$, то такое всегда возможно). Тогда $H=\langle g\rangle$ --- циклическая подгруппа $G$. Ее порядок делит порядок группы $G$, т.е. простое число $p$. В то же время, порядок $H$ отличен от 1, т.к. $H$ содержит как минимум два элемента $\e,g$. Но так как $p$ делится только на 1 и на $p$, то порядок группы $H$ равен $p$. Итого, $H\subseteq G$ и содержит столько же элементов (напомним, что группа $G$ --- конечная). Следовательно, $G=H=\langle g\rangle$.


\lesson{Нормальная подгруппа, факторгруппа, изоморфизм. Примеры: $\Z_2\times\Z_2$ и $\Z_8^*$}


\item Подгруппа $H$ группы $G$ называется \textbf{нормальной}\index{Группа!нормальная подгруппа}, если для любого $g\in G$ верно равенство $gH=Hg$, т.е. левые и правые классы не различаются. Обозначение: $H\vartriangleleft G$.

В абелевых группах любая подгруппа будет нормальной. В частности, $m\Z$ --- нормальная подгруппа в $\Z$.

\item Ранее нами был получен критерий нормальности подгруппы: $H\vartriangleleft G$ тогда и только тогда, когда имеет место вложение
$$
\forall g\in G\;g^{-1}Hg\subseteq H,\quad (gHg^{-1}\subseteq H).
$$



\item О тесной связи гомоморфизмов и нормальных подгрупп говорит следующая
\begin{thrm}[Основная теорема о гомоморфизмах групп]\index{Теорема!основная о гомоморфизмах}
Пусть $h:G\to G'$ --- гомоморфизм группы $G$ в группу $G'$, тогда
фактор-группа $G/\Ker(h)$ изоморфна области значений $h$ в группе $G'$.

Наоборот, если $H\vartriangleleft G$, то существует гомоморфизм $h:G\to G'$ группы $G$ в некоторую группу $G'$ такой, что $H=\Ker(h)$.
\end{thrm}
\pf
Обозначим за $K$ ядро $\Ker(h)$, а область значений $h$ в группе $G'$ обозначим за $H'$. Необходимо построить изоморфизм между $G/K$ и $H'$.

Возьмем какой-нибудь смежный класс $L\in G/K$. Этот класс можно записать в виде $L=gK$ при некотором $g\in G$. Положим тогда $f(L)=h(g)$. Для начала необходимо показать корректность такого определения, а точнее, однозначность $f$, т.е. что значение $f$ не зависит от выбора элемента $g$, задающего класс $L$.

Пусть $L=g_1K=g_2K$, т.е. класс $L$ определяется как с помощью элемента $g_1$, так и с помощью элемента $g_2$. Необходимо показать, что в этом случае $h(g_1)=h(g_2)$.

Равенство $g_1K=g_2K$ означает, что $g_1k=g_2k'$ при некоторых $k,k'\in K$. Тогда $h(g_1k)=h(g_1)h(k)=h(g_1)$ и $h(g_2k')=h(g_2)h(k')=h(g_2)$ по свойствам гомоморфизма, откуда $h(g_1)=h(g_2)$, т.е. $f$ определено корректно.

Далее, $f$ является сюръекцией на множество $H'$, поскольку для всякого $h'\in H'$ существует $g$ такой, что $h(g)=h'$, а значит, существует и класс $L=gK$ такой, что $f(L)=h'$.

Кроме того, $f$ является инъекцией в $H'$. Действительно, если для некоторых $L_1,L_2\in G/K$ имеет место равенство $f(L_1)=f(L_2)$, то запишем эти классы как $L_1=g_1K$ и $L_2=g_2K$, откуда по определению функции $f$ получим равенство $h(g_1)=h(g_2)$. Отсюда следует, что $h(g_1g_2^{-1})=h(g_1)h(g_2)^{-1}=\e$, следовательно, по определению ядра гомоморфизма получим, что $g_1g_2^{-1}\in K$. Но тогда $g_1=kg_2$ при некотором $k\in K$, т.е. $g_1\in Kg_2=g_2K$ (последнее равенство следует из нормальности ядра $K$). Тогда $g_1K\cap g_2K\ne\emptyset$, а значит, эти классы равны. Следовательно, $f$ --- инъекция.

Таким образом, $f$ --- биекция между $G/K$ и $H'$. Осталось показать, что она сохраняется групповую операцию.
$$
f(g_1Kg_2K)=f(g_1g_2K)=h(g_1g_2)=h(g_1)h(g_2)=f(g_1K)f(g_2K).
$$
Итак, $f$ --- изоморфизм между $G/K$ и $H'=\ran(h)$.

Пусть $H\vartriangleleft G$. Обозначим за $G'$ группу $G/H$. Определим гомоморфизм из $G$ в $G'$ следующим образом:
$$
h(g)=gH, \quad g\in G.
$$

$h$ сохраняет групповую операцию, т.к. $h(g_1g_2)=(g_1g_2)H=(g_1H)(g_2H)=h(g_1)h(g_2)$. Поскольку единицей группы $G/H$ является $H$ и функция $h$ переводит в $H$ все элементы $H$, и только их, то очевидно, что $\Ker(h)=H$, т.е. $h$ --- гомоморфизм с ядром $H$.
\epf






\end{enumerate}





\newchapter{Конечные поля}
структура конечного поля, теорема о цикличности, конечное поле как линейное пространство над $\Z/(p)$

\newchapter{Комплексные числа}
Числа Гаусса, числа Эйзенштейна

\newchapter{Линейная алгебра}
Линейные операторы, движения в $\R^n$, группа $SO(n)$, однородные координаты

\newchapter{Геометрия от Евклида до Римана}
Пятый постулат Евклида, геометрия Лобачевского, геометрия на сфере, проективная геметрия

\newchapter{Многочлены}
Нормальные расширения полей, начала теории Галуа

\newchapter{Вероятность от Байеса до Колмогорова}
Игровые сюжеты, дискретная вероятность, аксиомы вероятности, предельные теоремы, марковские цепи

\newchapter{Графы}
Перечислительные теоремы конечных графов, случайный граф. предельные теоремы для графов.
