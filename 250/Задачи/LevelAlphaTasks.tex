\begin{comment}
\chapter{1. Числа, символы и фигуры}
\end{comment}
\newchapter{Числа, символы и фигуры}


\vrezka{<<Дети и наука>>: \href{https://childrenscience.ru/courses/sav/1/}{Урок 1. Числа, символы, фигуры}.

Конспект: Глава 1, разделы 1.1 Запись действий с отрезками, 1.2 Понятие натурального числа, 1.3 Визуальные доказательства. Глава 7, раздел 7.1 Построение рациональных чисел.}

\subsection*{Справочные сведения}

Операции сложения и умножения мы визуально ассоциируем со смещением по прямой вправо или влево. Вправо --- со знаком $+$, влево --- со знаком $-$. Смещение на несколько единиц вправо или влево --- это смещение на одноименное число шагов в данном направлении. В итоге операцию сложения или вычитания можно представить как путь по прямой дороге, который складывается из шагов, равных $+1$ или $-1$ в зависимости от направления.

Умножение задается с помощью прямоугольной сетки на плоскости. Имеем две координатные оси, на которых отложены, как и в одномерном случае, шаги-числа в обе стороны от точки $O$ с соответствующими знаками. Откладываем перемножаемые числа по обеим осям, получаем прямоугольник, состоящий из единичных квадратов. Число этих квдаратов, т.е. площадь прямоугольника, и есть значение произведения (см. рис. \ref{prod}).
\begin{figure}[hbt!]
\begin{center}
\includegraphics[scale=0.2]{../prod.png}
\end{center}
\caption{Произведение $5\cdot 3$.}\label{prod}
\end{figure}

В том случае, когда умножаются числа, оснащенные знаками, применяется правило ориентированной площади, т.е. знак выбирается в зависимости от направления оси наблюдателя, для которого порядок множителей всегда соответствует повороту против часовой стрелки (см. рис. \ref{mult}).
\begin{figure}[hbt!]
\begin{center}
\includegraphics[scale=0.2]{../mult.png}
\end{center}
\caption{произведение $a\cdot b$.}\label{mult}
\end{figure}


\subsection*{Задачи}
\begin{enumerate}
\item Нанести на прямой метки, соответствующие шагам вправо и влево, считая начальной точкой $O$, а все шаги равновеликими (т.е. каждый шаг равен выбранной единице длины). Дойти до точки 5, а затем от точки 5 до точки -5. Записать последовательность шагов с помощью $\pm 1$, предполагая, что шаг вправо записывается как +1, шаг влево --- как -1.
\item Описать в терминах одномерного путешественника операции сложения: $5+3$, $8-4$, $3-5$, $-2-6$. Сколько шагов и в какую сторону он прошел и в каком порядке? Записать в каждом случае путь с помощью $\pm1$ и расставить скобки, объединяя в них указанные слагаемые.
\item \textit{Путь} --- это последовательность единичных шагов, обозначаемых $+1$ (шаг вправо) и $-1$ (шаг влево). Путь может начинаться в любой точке прямой.
Записать пути, соответствующие операциям $-2+7$, $10-5$, $11-2-4$, $-8+3+10$. 
\item Выберем точку $O$ в качестве начала отсчета, затем нанесем на прямую точки, которые получаются в результате отсчета шагов влево и вправо, т.е. точки $\pm1, \pm2, \pm3$ и т.д. Назовем эти точки \textit{целыми}.
\begin{enumerate}[a)]
\item В какой точке окажется путешественник, если он стартует в точке $-3$ и проходит путь $4-1$? Изобразить графически.
\item В какой точке окажется путешественник, если он стартует в точке $1$ и проходит путь $11-4+7$? Изобразить графически.
\end{enumerate}
\item Два пути назовем \textit{эквивалентными}, если, стартуя в одной и той же точке, они и закончатся в одной и той же точке. Эквивалентны ли пути $-2+7$, $10-5$, $11-2-4$, $-8+3+10$?
\item Путь $a$ назовем \textit{обратным} к пути $b$, если, стартовав там, где путь $b$ заканчивается, он повторяет все шаги пути $b$ в обратном порядке и с противоположным знаком (например, путь $1+1+1-1-1-1$ обратен к пути $-1-1-1+1+1+1$). Построить пути, соответствующие операциям $5+3$, $8-4$, $3-5$, $-2-6$, построить обратные к ним пути, выразить обратные пути в виде суммы или разности двух чисел (использовать те же цифры, что у исходного пути).
\item Изобразить ориентированные площади, соответствующие произведениям $3\cdot 5$ и $5\cdot 3$, $(-2)\cdot 6$ и $6\cdot (-2)$, $(-3)\cdot(-4)$ и $(-4)\cdot(-3)$
\end{enumerate}



\begin{comment}
\chapter{2. Соизмеримость отрезков, алгоритм Евклида}
\end{comment}
\newchapter{Соизмеримость отрезков}


\vrezka{<<Дети и наука>>: \href{https://childrenscience.ru/courses/sav/2/}{Урок 2. Соизмеримость и несоизмеримость отрезков}.

Конспект: Глава 1, разделы 1.2 Понятие натурального числа, 1.4 Соизмеримость отрезков, алгоритм Евклида.}

\subsection*{Справочные сведения}

На этот раз у нас имеется два путешественника (кузнечика), каждый из которых имеет свою меру длины (длину шага), соответственно, у каждого из них получаются свои собственные ометки на прямой, расставленные через каждый шаг. Пусть у первого путешественника шаг равен $a$, а у второго --- $b$. Таким образом, первый может придти в точки $\pm a, \pm 2a, \pm 3a$ и т.д., а второй --- в точки $\pm b, \pm 2b, \pm 3b$ и т.д. Точка начала отсчета у них общая --- точка $O$. 

Длины шагов этих путешественников, т.е. числа $a$ и $b$ \textit{соизмеримы}, если существует такая длина $c$ (\textit{общая мера отрезков} $a$ и $b$), которая целое число раз укладывается в том и другом шаге: $a=nc$, $b=mc$. 

\textit{Графический алгоритм Евклида}: о прямоугольника со сторонами $a$ и $b$ отрезают квадраты со стороной, равной меньшей из длин $a$ и $b$, столько раз, сколько возможно (будем называть это <<операцией Евклида>>). К оставшемуся прямоугольнику снова применяют операцию Евклида, и так далее (см. рис. \ref{soizmer}).
\begin{figure}[hbt!]
\begin{center}
\includegraphics[scale=0.3]{../soizmer.png}
\end{center}
\caption{Графический алгоритм Евклида.}\label{soizmer}
\end{figure}

\subsection*{Задачи}
\begin{enumerate}
\item Доказать, что $a$ и $b$ соизмеримы тогда и только тогда, когда существует отрезок $d$ такой, что отрезки $a$ и $b$ укладываются в нем целое число раз: $d=ka=lb$. Верно ли, что это также равносильно тому, что два путешественника могут встретиться в какой-то точке прямой, отличной от точки $O$?
\item Верно ли, что отрезки $a$ и $b$ соизмеримы тогда и только тогда, когда $a$ и $2b$ соизмеримы?
\item Сколько и каких квадратов получится в результате применения графического алгоритма Евклида к прямоугольнику со сторонами 75 и 21?
\item Применяя операцию Евклида, прямоугольник разрезали на большой квадрат, два квадрата поменьше и два совсем маленьких. Найти отношение сторон исходного прямоугольника.
\item Доказать, что если стороны прямоугольника соизмеримы, то, применяя операцию Евклида, мы в конце концов разрежем его на квадраты (применить метод бесконечного спуска).
\item Доказать, что если применение операции Евклида разрезает прямоугольник на некоторое конечное число квадратов, то стороны прямоугольника соизмеримы, и сторона самого маленького квадрата будет их общей мерой.
\item От прямоугольника отрезали квадрат и получили прямоугольник, подобный исходному. Соизмеримы ли стороны исходного прямоугольника?
\end{enumerate}


\begin{comment}
\chapter{3. Визуальная арифметика}
\end{comment}
\newchapter{Визуальная арифметика}


\vrezka{<<Дети и наука>>: \href{https://childrenscience.ru/courses/sav/3/}{Урок 3. Визуальное представление бинома Ньютона}.

Конспект: Глава 1, раздел 1.3 Визуальные доказательства.}

\subsection*{Справочные сведения}

Теорема Пифагора (см. рис. \ref{pithagor}) и куб суммы (см. рис. \ref{kub}).

\begin{figure}[hbt!]
\begin{center}
\includegraphics[scale=0.25]{../pithagor.png}
\end{center}
\caption{$(a+b)^2=a^2+2ab+b^2$ и $a^2+b^2=c^2$.}\label{pithagor}
\end{figure}

\begin{figure}[hbt!]
\begin{center}
\includegraphics[scale=0.25]{../kub.png}
\end{center}
\caption{$(a+b)^3 = a^3+3a^2b+3ab^2+b^2$.}\label{kub}
\end{figure}



\subsection*{Задачи}
\begin{enumerate}
\item Найти с помощью графического метода сумму подряд идущих нечетных чисел от 1 до $n$, где $n$ --- нечетное.
\item Рассмотрим последовательность уголков (см. рис. \ref{ugolki}). Сколько клеток в $k$-м уголке? Чему равна суммарная площадь первый $k$ уголков?
\begin{figure}[hbt!]
\begin{center}
 \includegraphics[scale=0.3]{../ugolki.png}
\end{center}
\caption{}\label{ugolki}
\end{figure}
\item Найти графически сумму первых $k$ четных и первых $k$ нечетных чисел.
\item Треугольные числа Диофанта  \includegraphics[scale=0.1]{../triangle.png} обозначим по порядку $T_1,T_2,T_3,T_4$ и т.д.
\begin{enumerate}[a)]
\item Сложите из двух последовательных треугольных чисел квадрат.
\item Что получится при сложении $T_n$ с $T_n$?
\item Выразив $T_n$ через $n$, найдите $1+2+\dots+n$.
\item Докажите геометрически, что $T_{n+m}=T_n+T_m+nm$.
\end{enumerate}
\item Докажите геометрически, что $1+2+\dots+(n-1)+n+(n-1)+\dots+2+1=n^2$.
\item Получите геометрически выражение для $(a+b+c)^2$, $(a+b+c)^3$.
\item Объясните равенство на рис. \ref{sumquad} и получите формулу для суммы квадратов $1^2+2^2+3^2+\dots+n^2$.
\begin{figure}[hbt!]
\begin{center}
\includegraphics[scale=0.75]{../sumquad.png}
\end{center}
\caption{}\label{sumquad}
\end{figure}

\item С помощью рис. \ref{sumquad2} получите еще один способ найти формулу для суммы квадратов.
\begin{figure}[hbt!]
\begin{center}
\includegraphics[scale=0.6]{../sumquad2.png}
\end{center}
\caption{}\label{sumquad2}
\end{figure}
\end{enumerate}



\begin{comment}
\chapter{4. Бесконечные суммы}
\end{comment}
\newchapter{Бесконечные суммы}


\vrezka{<<Дети и наука>>: \href{https://childrenscience.ru/courses/sav/4/}{Урок 4. Бесконечные суммы}.
}

\subsection*{Справочные сведения}

В данном разделе мы рассматриваем только суммы \textit{положительных} слагаемых.

Бесконечные суммы с положительными слагаемыми могут быть сходящимися и расходящисмися. Сходимость означает, что найдется такое число, что любой сколь угодно длинный конечный отрезок данной бесконечной суммы меньше этого числа. Например, сумму
$1+1/2^2+1/3^2+1/4^2+\dots$ можно оценивать так:
$$
\frac{1}{2^2}+\frac{1}{3^2}<\frac{1}{2^2}+\frac{1}{2^2}=\frac12,
$$
$$
\frac{1}{4^2}+\frac{1}{5^2}+\frac{1}{6^2}+\frac{1}{7^2}<\frac{1}{4^2}+\frac{1}{4^2}+\frac{1}{4^2}+\frac{1}{4^2}=\frac14,
$$
и т.д. То есть, сумму можно разбить на отрезки длиной 2, 4, 8, 16 и т.д. слагаемых, причем сумма по каждому такому отрезку будет оцениваться сверху дробью $1/2^k$. Остается заметить, что ряд
$$
1+\frac{1}{2}+\frac{1}{4}+\frac{1}{8}+\frac{1}{16}+\dots
$$
сходится. А это легко обнаружить на картинке \ref{geomseq} последовательным делением квадрата $1\times 1$ пополам.
\begin{figure}[hbt]
\begin{center}
\includegraphics[scale=0.3]{../geomseq.png}
\end{center}
\caption{}\label{geomseq}
\end{figure}
Таким образом, для суммы обратных квдаратов справедлива оценка:
$$
1+1/2^2+1/3^2+1/4^2+\dots \le 1+\frac{1}{2}+\frac{1}{4}+\frac{1}{8}+\frac{1}{16}+\dots \le 2.
$$

Обратно, для некоторых рядов можно найти такую оценку снизу, которая будет заведомо бесконечной, а значит, и сумма исходного ряда также будет бесконечной. Такое верно, например, для гармонического ряда:
$$
1+\frac{1}{2}+\frac{1}{3}+\frac{1}{4}+\frac{1}{5}+\dots \ge 
1+\frac{1}{2}+\frac{1}{4}+\frac{1}{4}+4\cdot\frac{1}{8}+8\cdot\frac{1}{16}+\dots,
$$
а это --- бесконечная сумма одинаковых слагаемых, равных $1/2$ (кроме первого слагаемого). Ясно, что какое бы большой число мы ни выбрали, можно взять столь много раз $1/2$, что их сумма будет больше выбранного числа. А значит, и сумма гармонического ряда равна бесконечности.

\subsection*{Задачи}

\begin{enumerate}
\item Выведите формулу суммы геометрической прогрессии $1+x+x^2+x^3+\dots$ ($0<x<1$) путем домножения этой сумм ына $x$. Найти:
\begin{enumerate}[a)]
\item $\displaystyle \frac{1}{10}+\frac{1}{100}+\frac{1}{1000}+\dots$;
\item $\displaystyle 1+0.2+(0.2)^2+(0.2)^3+\dots$;
\item $\displaystyle \frac{1}{0.99}+\frac{1}{0.99^2}+\frac{1}{0.99^3}+\dots$.
\end{enumerate}
\item Исследовать ряды на сходимость:
\begin{enumerate}[a)]
\item $1+1/3+1/5+1/7+\dots$;
\item $1+1/3^2+1/5^2+1/7^2+\dots$;
\item $\displaystyle \frac{1}{1001}+\frac{1}{2001}+\frac{1}{3001}+\dots+\frac{1}{1000n+1}+\dots$;
\item $\displaystyle 1+\frac{1}{2!}+\frac{1}{3!}+\frac{1}{4!}+\dots$;
\item $\displaystyle 1+\frac{2}{3}+\frac{3}{5}+\frac{5}{9}+\dots+\frac{n}{2n-1}+\dots$.
\end{enumerate}
\item Доказать, что если ряды $\displaystyle \sum_na_n^2$ и $\displaystyle \sum_nb_n^2$ сходятся, то сходятся также и ряды:
$$
\sum_na_nb_n,\quad \sum_n(a_n+b_n)^2.
$$
Здесь все $a_n,b_n\ge 0$.
\item Доказать сходимость ряда
$$
a_0+\frac{a_1}{10}+\frac{a_2}{10^2}+\dots+\frac{a_n}{10^n}+\dots,
$$
где $0\le a_n<10$.
\end{enumerate}





\begin{comment}
\chapter{5. Движения прямой: работа с понятием}
\end{comment}
\newchapter[работа с понятием]{Движения прямой:}


\vrezka{<<Дети и наука>>: \href{https://childrenscience.ru/courses/sav/5/}{Урок 5. Начальные представления о движении}.

Конспект: Глава 2, разделы 2.1 Сдвиг, композиция сдвигов, группа и раздел 2.2 Отражение.}

\subsection*{Справочные сведения}

\textit{Движением} называется такое преобразование (прямой, фигуры, плоскости, области пространства и т.д.), которое сохраняет расстояния. Т.е. если между точками $A$ и $B$ расстояние равно $x$, то между точками $A'$ и $B'$, в которые переходят исходные точки $A$ и $B$ при некоторойм движении, расстояние также будет равно $x$.

На прямой рассматриваются следующие два вида движений:
\begin{itemize}
\item Сдвиг на $x$, когда все точки, как по команде, сдвигаются на число $x$ (если $x>0$, то вправо, а если $x<0$, то влево). Сдвиг на $x$ обозначается за $T_x$. Сдвиг на вектор $AB$ обозначается $T_{AB}$.
\item Отражение относительно точки $O$, когда все точки переходят в симметричные себе относительно точки $O$. Отражение относительно точки $O$ обозначается за $S_O$.
\end{itemize}

Частный случай сдвига --- тождественное движение $\id$, которое ничего не меняет (все точки остаются на своих местах). $\id=T_0$ (сдвиг на нулевой вектор).

Композиция движений $G$ и $Q$ записывается как $G\circ Q$, что означает последовательное применение движений: сначала ко всем точкам прямой применяется движение $Q$, а затем к результату предыдущего движения применяется движение $G$. Композиция движений есть движение.



\subsection*{Задачи}

Пусть на прямой даны 4 точки $A,B,C,D$, поставленные друг за другом с одинаковым шагом (см. рис. \ref{ABCD}).
\begin{figure}[hbt!]
\begin{center}
\includegraphics[scale=0.7]{../ABCD.png}
\end{center}
\caption{}\label{ABCD}
\end{figure}

\begin{enumerate}
\item Куда перейдет точка $A$ при отражении $S_B$?
\item Куда перейдут точки $B,C,D$ при преобразовании $T_{AB}\circ T_{CA}$?
\item Куда перейдут точки $A,B,C$ при преобразовании $S_C\circ T_{AB}$?
\item Какое движение переводит $A$ в $C$ и $B$ в $D$?
\item Существует ли движение, которое переводит $A$ в $B$ и $B$ в $D$?
\item Опишите все движения, которые переводят $A$ в $C$, используя только буквы $A,B,C,D$ и обозначения сдвига и отражения.
\end{enumerate}




\begin{comment}
\chapter{6. Движения прямой: классификация}
\end{comment}
\newchapter[классификация]{Движения прямой:}


\vrezka{<<Дети и наука>>: \href{https://childrenscience.ru/courses/sav/6/}{Урок 6. Классификация движений прямой}.

Конспект: Глава 2, раздел 2.4 Теорема о гвоздях, аналог теоремы Шаля.}

\subsection*{Справочные сведения}

Всякое движение прямой --- это либо сдвиг, либо отражение. При этом любое движение --- это либо одно отражение, либо композиция двух отражений.

Всякое движение прямой есть \textit{взаимно однозначное соответствие} точек прямой, т.е. оно переводит разные точки в разные, и какова бы ни была точка прямой, найдется точка, переходящая в нее под действием движения.

\textit{Обратное движение} для дивжения $G$ --- это такое движение $G^{-1}$, что $G\circ G^{-1}=G^{-1}\circ G=\id$.

Обращение композиции: $(G\circ Q)^{-1} = Q^{-1}\circ G^{-1}$.

\subsection*{Задачи}

Введем координату на прямой, отметим там точки с целыми координатами: $\dots,-2,-1,0,1,2,\dots$. Через $S_n$ обозначим отражение относительно точки $n$, через $T_n$ --- сдвиг на число $n$.
\begin{enumerate}
\item Известно, что при некотором преобразовании $G$ точка $0$ переходит в $2$, а $2$ --- в $3$. Может ли оно быть движением? Каким?
\item Известно, что при некотором преобразовании $G$ точка $0$ переходит в $3$, а $2$ --- в $1$. Может ли оно быть движением? Каким?
\item Известно, что при некотором преобразовании $G$ точка $0$ переходит в $2$, а при обратном преобразовании $G^{-1}$ точка $3$ переходит в $-1$. Может ли $G$ быть движением? Каким?
\item Дано движение $G$. Известно, что $G^{-1}(0)=1$ и при этом у $G^{-1}$ нет неподвижных точек. Чему равно $G$?
\item Назовем \textit{четностью движения} прямой четность количества отражений, с помощью которых это движение может быть выражено. Какова четность следующих движений: $S_0$, $T_x$, $T_x\circ T_y$, $S_0\circ T_x$, $S_0\circ S_1\circ T_x\circ T_y$, $T_x^{-1}$, $S_9^{-1}$, $S_0\circ S_1\circ\dots\circ S_n$?
\item Доказать, что
\begin{enumerate}[a)]
\item Четность обратного движения  $G^{-1}$ совпадает с четностью исходного движения $G$.
\item Четность композиции движений равна сумме четностей (по модулю 2) компонентов.
\item Четность движения не зависит от его представления в виде композиций каких-либо движений.
\end{enumerate}
\end{enumerate}






\begin{comment}
\chapter{7. Движения прямой: таблица композиций}
\end{comment}
\newchapter[таблица композиций]{Движения прямой:}


\vrezka{<<Дети и наука>>: \href{https://childrenscience.ru/courses/sav/7/}{Урок 7. Таблица композиций движений прямой}.

Конспект: Глава 2, раздел 2.3 Таблица композиций движений прямой.}

\subsection*{Справочные сведения}


Таблица композиций отражений и сдвигов:\index{Группа!движений прямой}
\begin{center}
\begin{tabular}{c|c|c|}
  & $T_a$ & $S_O$ \\
 \hline
$T_b$ & $T_{a+b}$ & $S_{O+b/2}$ \\
 \hline
$S_C$ & $S_{C-a/2}$ & $T_{2OC}$ \\
\hline
\end{tabular}
\end{center}

Таблицу композиций следует читать слева наверх, т.е. если в левом столбце стоит движение $F$, а в верхней строке --- движение $G$, то в соответствующей ячейке стоит композиция $F\circ G$.


\subsection*{Задачи}

Введем координату на прямой, отметим там точки с целыми координатами: $\dots,-2,-1,0,1,2,\dots$. Через $S_n$ обозначим отражение относительно точки $n$, через $T_n$ --- сдвиг на число $n$.
\begin{enumerate}
\item Какое движение получится при композиции
\begin{enumerate}[a)]
\item $S_0\circ S_1$?
\item $S_0\circ S_1\circ S_2$?
\item $S_0\circ S_2\circ S_1$?
\end{enumerate}
\item Построить сдвиг на 7 единиц вправо с помощью композиции двух отражений.
\item Каким движением является следующая композиция?
$$
S_{n}\circ S_{n-1}\circ \dots\circ S_{1}\circ S_0.
$$
Ответ получить в зависимости от четности $n$.
\item При каких $n$ сдвиг $T_n$ выражается в виде композиций $S_0$ и $S_1$?
\item При каких $n$ сдвиг $S_n$ выражается в виде композиций $S_0$ и $S_1$?
\item Пусть $G$ и $Q$ --- два движения прямой, причем $G\circ Q=Q\circ G$ и $G\ne Q$. Какими могут быть $G$ и $Q$?
\item Пусть $G$ и $Q$ --- два движения прямой, причем $G\circ Q=\id$ и $G\ne Q$. Какими могут быть $G$ и $Q$?
\item Вывести равенства $S_C\circ T_a = S_{C-a/2}$ и $T_b\circ S_O = S_{O+b/2}$ из соотношения $S_C\circ S_O=T_{2OC}$ алгебраическим путем.
\item Доказать, что никакая композиция движений $S_n$ и $T_m$ с целыми индексами $n,m$ не может быть равна сдвигу $T_x$ с нецелым $x$ и отражению $S_y$ с неполуцелым $y$.
\end{enumerate}



\begin{comment}
\chapter{8. Движения окружности: классификация}
\end{comment}
\newchapter[классификация]{Движения окружности:}


\vrezka{<<Дети и наука>>: \href{https://childrenscience.ru/courses/sav/8/}{Урок 8. Движения окружжности}.

Конспект: Глава 3, раздел 3.1 Движения окружности, раздел 3.2 Группа движений окружности, теорема Шаля.}

\subsection*{Справочные сведения}

Чтобы корректно говорить о движениях в криволинейном пространстве, нужно сначала договориться о метрике на нем. \textit{Расстояние} (метрика) между двумя точка окружности --- это длина меньшей из дуг данной окружности, соединяющих эти точки. Таким образом, движение окружности по определению должно сохранять длину дуги, переводя точки окружности в точки этой же окружности.

В отличие от прямой, на окружности расстояния имеют максимально допустимное значение, а именно, половину длины этой окружности. На масимальном расстоянии находятся диаметрально противоположные точки.

Движения на окружности являются:
\begin{itemize}
\item \textit{Отражение относительно диаметра} (произвольного). Отражение обозначается $S_l$, где $l$ --- диаметр. Если на окружности зафиксировано нулевое положение диаметра, то любой диаметр можно определить через угол наклона относительно улевого диаметра (угол откладывается против часовой стрелки). Если диаметр $l$ имеет наклон $\ph$ относительно нулевого диаметра ($0\le\ph<\pi$), то отражение относительно данного диаметра мы также записваем как $S_\ph$.
\item \textit{Поворот окружности} относительно ее центра. Поворот обозначается $R_\ph$, где $\ph$ --- угол поворота относительно центра окружности, осуществляемый против часовой стрелки, $0\le\ph<2\pi$.
\end{itemize}
В обоих случаях можно рассматривать и другие значения угла $\ph$, приводя его по модулю $\pi$ в случае отражений и по модулю $2\pi$ в случае поворотов, т.к. наклон диаметра на угол $\phi\pm\pi$ приводит к диаметру с углом $\ph$, а поворот на угол $\pi\pm2\pi$ --- это поворот на угол $\ph$.

Углы измеряются в радианах. 1 радиан --- это угол, соответствующей дуге, длина которой равна радиусу окружности. Угол в $180^o$, соответствующий дуге, равной половине длины окружности, он же --- развернутый угол, --- имеет радианную меру, равную числу $\pi$. Если окружность имеет радиус, равный 1, то мера угла в радианах численно совпадает с длиной соответствующей этому углу дуги данной окружности.

Частным случаем поворота является \textit{тождественное движение} $\id$, оставляющее все точки окружности на месте. $\id=R_0=R_{2\pi k}$.

Других движений окружности не существует (теорема Шаля). Как и в случае прямой, любое движение окружности можно представить как композицию одного или двух отражений.

\subsection*{Задачи}

\begin{enumerate}
\item Доказать, что $\pi>3$.
\item Пусть $G$ --- движение окружности. Сколько у $G$ может быьт неподвижных точек (имеется ввиду общее количество, найдите все возможные варианты)?
\item Пусть $G$ --- движение окружности. Известно, что $G(A)=A$ и $G(B)\ne B$. Какой вид может иметь $G$?
\item Пусть диаметры $l$ и $k$ перпендикулярны. Найдите $S_l\circ S_k$.
\item Известно, что точка $A$ переходит при движении $G$ окружности в точку $A'$, диаметрально противоположную точке $A$. Каким может быть движение $G$?
\item Движение назовем \textit{четным}, если оно является композицией двух отражений, а в противном случае --- \textit{нечетным}. Верно ли, что:
\begin{enumerate}[a)]
\item Композиция четных движений --- четное движение, композиция двух нечетных движений --- четное движение, композиция четного движения с нечетным движением --- нечетное движение?
\item $G$ четно тогда и только тогда, когда $G^{-1}$ нечетно?
\end{enumerate}
\end{enumerate}





\begin{comment}
\chapter{9. Движения окружности: таблица композиций}
\end{comment}
\newchapter[таблица композиций]{Движения окружности:}


\vrezka{<<Дети и наука>>: \href{https://childrenscience.ru/courses/sav/9/}{Урок 9. Таблица умножения движений окружности}.

Конспект: Глава 3, раздел 3.2 Группа движений окружности, теорема Шаля.}

\subsection*{Справочные сведения}

Таблица композиций движений окружности:
\begin{center}
\begin{tabular}{c|c|c|}
  & $R_\al$ & $S_\psi$ \\
 \hline
$R_\be$ & $R_{\al+\be}$ & $S_{\psi+\be/2}$ \\
 \hline
$S_\ph$ & $S_{\ph-\al/2}$ & $R_{2(\ph-\psi)}$ \\
\hline
\end{tabular}
\end{center}

Таблицу композиций следует читать слева наверх, т.е. если в левом столбце стоит движение $F$, а в верхней строке --- движение $G$, то в соответствующей ячейке стоит композиция $F\circ G$.

\subsection*{Задачи}
\begin{enumerate}
\item Центральная симметрия --- это какое движение?
\item Композицией каких отражений можно выразить центральную симметрию?
\item С помощью отражения относительно оси $Ox$ (горизонтальной оси) и вращений выразить отражение относительно оси $Oy$ (вертикальной оси).
\item Возьмем некоторый угол $\ph>0$. Найдите:
\begin{enumerate}[a)]
\item $S_0\circ S_\ph$;
\item $S_0\circ S_\ph\circ S_{2\ph}$;
\item $S_0\circ S_{2\ph}\circ S_{\ph}$;
\item $\displaystyle S_0\circ S_\ph\circ S_{2\ph}\circ S_{3\ph}\circ\dots\circ S_{n\ph}$.
\item Чему равно последнее выражение, если $\ph=\pi/2$, $\ph=\pi$, $\pi=2\pi$?
\end{enumerate}
\item Построить поворот на угол $90^o$ при помощи двух отражений.
\item При каких $n$ поворот на угол $n\ph$ выражается в виде композиций $S_0$ и $S_\ph$?
\item Пусть $G$ и $Q$ --- движения окружности, причем $G\circ Q=Q\circ G$. Какими могут быть $G$ и $Q$?
\item Пусть $G$ и $Q$ --- движения окружности, причем $G\circ Q=\id$. Какими могут быть $G$ и $Q$?
\end{enumerate}







\begin{comment}
\chapter{10. Конечные подгруппы движений прямой и окружности}
\end{comment}
\newchapter[ прямой и окружности]{Конечные подгруппы движений}


\vrezka{<<Дети и наука>>: \href{https://childrenscience.ru/courses/sav/10/}{Урок 10. Конечные подгруппы движений прямой и окружности}.

Конспект: Глава 2, раздел 2.5 Все конечные подгруппы движений прямой, раздел 5.3 Подгруппы движений окружности.}

\subsection*{Справочные сведения}

Все движения прямой и все движения окружности образуют группы с операцией композиции. Напомним определение группы.
Пусть на множестве $G$ задана операция $\circ$. Множество $G$ с данной опеарцией называется \textit{группой}, если:
\begin{enumerate}[G1]
\item $a\circ b\in G$ для всех $a,b\in G$ (группоид);
\item для любых $a,b,c\in G$ имеем тождество $(a\circ b)\circ c=a\circ (b\circ c)$ (ассоциативность);
\item существует элемент $\id\in G$ такой, что $a\circ \id=\id\circ a=a$ для всех $a\in G$ (единица);
\item для всякого $a\in G$ существует обратный элемент $a^{-1}\in G$ такой, что $a\circ a^{-1}=a^{-1}\circ a=\id$ (обратный элемент).
\end{enumerate}

Кроме того, группа называется \textit{абелевой} (или \textit{коммутативной}), если $a\circ b=b\circ a$ для всех $a,b\in G$. Количество элементов в группе называется ее \textbf{порядком}.

\textit{Конечная подгруппа} может быть определена следующим образом: это --- \textit{конечное подмножество группы, замкнутое относительно групповой операции}. Такого определения достаточно, чтобы вывести из него тот факт, что данное подмножество само по себе является группой, т.е. содержит единицу (исходной группы), обратные элементы, а также удовлетворяет требованию ассоциативности операции (т.к. операция та же самая).

Всякая конечная подгруппа группы движений прямой имеет вид либо $\{\id\}$, либо $\{\id,S_A\}$, где $A$ --- некоторая точка прямой.

Всякая конечная подгруппа группы движений окружности имеет один из видов:
\begin{enumerate}
\item тривиальная подгруппа $\{\id\}$;
\item группа поворотов правильного $n$-угольника (включая случай вырожденного 2-угольника);
\item подгруппа одного отражения $\{\id,S_\ph\}$;
\item группа движений правильного $n$-угольника (включает повороты, совмещающие углы многоугольника, и отражения относительно осей, проходящих через его вершины и центр окружности).
\end{enumerate}


\subsection*{Задачи}

\begin{enumerate}
\item Выпишите все конечные подгруппы группы движений окружности порядка не выше 6, содержащие отражение $S_0$ (относительно горизонтальной оси).
\item Какова группа движений правильного треугольника, квдарата, пятиугольника?
\item Пусть задан правильный треугольник $ABC$ с осями симметрии $a,b,c$ и центром $O$. Заполните таблицу композиций движений данного треугольника:
\begin{table}[htb!]\begin{center}
\begin{tabular}{c||c|c|c||c|c|c|}
             & $\id$        & $R_{2\pi/3}$ & $R_{4\pi/3}$ & $S_a$        & $S_b$        & $S_c$  \\
\hline\hline
$\id$        & \phantom{$R_{2\pi/3}$} & \phantom{$R_{2\pi/3}$} & \phantom{$R_{2\pi/3}$} & \phantom{$R_{2\pi/3}$} & \phantom{$R_{2\pi/3}$} & \phantom{$R_{2\pi/3}$} \\  \hline
$R_{2\pi/3}$ &  &  &  &  &  &  \\  \hline
$R_{4\pi/3}$ &  &  &  &  &  &  \\  \hline
$S_A$        &  &  &  &  &  &  \\  \hline
$S_B$        &  &  &  &  &  &  \\  \hline
$S_C$        &  &  &  &  &  &  \\  \hline
\end{tabular}
\end{center}\end{table}
Таблицу композиций следует читать слева наверх, т.е. если в левом столбце стоит движение $F$, а в верхней строке --- движение $G$, то в соответствующей ячейке стоит композиция $F\circ G$.
\end{enumerate}







\begin{comment}
\chapter{11. Арифметика остатков}
\end{comment}
\newchapter{Арифметика остатков}


\vrezka{<<Дети и наука>>: \href{https://childrenscience.ru/courses/sav/11/}{Урок 11. Введение в арифметику остатков}.

Конспект: Глава 8, раздел 8.1 Арифметика остатков.}

\subsection*{Справочные сведения}

Посмотрим на шкалу целых чисел $0,\pm1,\pm2,\dots$ через некоторый трафарет. Этот трафарет является непрозрачной полоской, в которой проделано две дырки на расстоянии $m$ (где $m$ --- целое положительное число). Например, пусть $m=7$, тогда если в одной дырке мы видим число 0, то в другой --- число 7. Если мы сместим трафарет вправо на единицу, то увидим пару чисел 1 и 8, далее --- 2 и 9, и т.д. Аналогично, если мы начнем его двигать влево, то будем наблюдать пары $-1$ и $6$, $-2$ и $5$ и т.д. Таким образом, в массиве всех целых чисел мы сможем выделять такие, которые связаны друг с другом через этот трафарет. Например, все числа кратные 7, т.е. $0,\pm7,\pm14,\dots$. В другой класс войдут все числа, смещенные от них на 1 вправо, т.е. $1,\pm7+1,\pm14+1,\dots$.

Эти классы называются классами вычетов по модулю $m$. Простой иллюстрацией из жизни является пример с днями недели. Все понедельники отстоят друг от друга на кратное 7 число дней. Поэтому на шкале дней их можно увидеь через трафарет с шагом 7. Аналогично --- все вторники, среды, четверги, пятницы, субботы и воскресенья.

Как только мы отождествляем целые числа, входящие в один класс, их арифметика становится модульной. Это значит, что арифметические операции мы выполняем с точностью до класса. Так, если сложить $2+5$, то в обычной арифметике мы получим число 7, но оно находится в том же классе. что и число 0 по модулю $m=7$, поэтому в модульной арифметике $2+5=0\mod 7$. Проще говоря, в модульной арифметике мы всякий раз отбрасываем максимально возможную часть числа, кратную модулю, и оставляем лишь остаток от деления на модуль. Поэтому она и называется арифметикой остатков.

\begin{figure}[hbt!]
\begin{center}
\includegraphics[scale=0.4]{../weekdays.png}
\end{center}
\caption{Арифметика остатков по модулю 7.}\label{weekdays}
\end{figure}

Попадание чисел $a$ и $b$ в один класс по модулю $m$ обозначается так: $a\equiv b\mod m$. Формально это означает, что $a-b=km$ при некотором целом $k$.

Сравните: среда(3) + 5дней = понедельник(1). Это позволяет вычислять день недели для любой даты, отстоящей от нас на известное число дней. В частности, можно легко посчитать день недели даты, отстоящей от нас на 1 год. На рис
\ref{weekdays} приведен пример с високосным годом, содержащим 366 дней. $366=350+14+2\equiv 2\mod 7$.



\subsection*{Задачи}

\begin{enumerate}
\item Отметить на числовой оси целые числа, которые при делении на 7 дают остаток 2 (на рисунке должны поместиться числа от -20 до 20).
\item Книги на столе пытались связывать в пачки по 2, по 3, по 4 и по 5 книг, и каждый раз оставалась одна лишняя. Сколько книг было на столе? (Известно, что их было не больше 100.)
\item Одному брату 6 лет, другому --- 10. Значит, сумма из возрастов четная. Какой она будет в следующем году?
\item Если сегодня понедельник, то какой день недели будет через 10 дней, через 90 дней, через 2 года (рассмотреть случай без високосных лет и с високосным годом)?
\item Найти день недели через месяц, квартал, полгода и год, отправляясь от текущей даты.
\item Поезд Москва--Владивосток отправляется из Москвы в 7:00 и находится в пути 166 часов. Определите время прибытия (московское) поезда во Владивосток.
\item Построить таблицы сложения и умножения для модулей: 2,3,4,5,6,10.
\item Найти число, которое при делении на 2 даёт остаток 1, при
делении на 3 остаток 2, при делении на 4 остаток 3, при делении
на 5 остаток 4, при делении на 6 остаток 5 и при делении на 7 даёт
остаток 6.
\item (a) Квадрат целого положительного числа оканчивается на ту
же цифру, что и само число. Что это за цифра? (Указать все воз-
можности.) (б) Квадрат целого положительного числа оканчива-
ется на те же две цифры, что и само число. Что это за цифры?
(Указать все возможности.) (в) Пятая степень числа оканчивается
на ту же цифру, что и само число. Почему? Для каких ещё степеней
это верно?
\item Доказать, что для любого целого a число 10a даёт при делении
на 9 тот же остаток, что и само a.
\item Число a даёт остаток 5 при делении на 9, число b даёт остаток
7 при делении на 9. Можно ли по этим данным определить, какой
остаток дают числа a + b и ab при делении на 9?
\item Доказать, что число и его сумма цифр дают одинаковые остат-
ки при делении на 3 и 9.
\item Сформулировать и доказать признаки делимости на 2, 3, 4, 5,
9, 11.
\item Верен ли такой признак делимости на 27: число делится на 27
тогда и только тогда, когда сумма его цифр делится на 27?
\item Целое положительное число увеличили на 1. Могла ли сумма
его цифр (а) возрасти на 8? (б)Уменьшиться на 8? (в) Уменьшиться
на 10?
\item Какие остатки может давать точный квадрат при делении на 4?
\end{enumerate}




\begin{comment}
\chapter{12. Таблицы умножения остатков}
\end{comment}
\newchapter{Таблицы умножения остатков}


\vrezka{<<Дети и наука>>: \href{https://childrenscience.ru/courses/sav/12/}{Урок 12. Таблицы умножения остатков}.

Конспект: Глава 8, раздел 8.1 Арифметика остатков, раздел 8.2 Свойства арифметики остатков.}

\subsection*{Справочные сведения}



\subsection*{Задачи}

\begin{enumerate}
\item Последняя цифра точного квадрата равна 6. Доказать, что его
предпоследняя цифра нечётна.
\item Остаток от деления простого числа на 30 --- простое число или 1. Почему?
\item Какое наибольшее число различных целых чисел можно выбрать, если требуется, чтобы сумма и разность любых двух из них не делились на 15?
\item Существуют ли целые $x, y$, для которых (а) $x^2 + y^2 = 99$? (b)
$x^2 + y^2 = 33333$? (c) $x^2 + y^2 = 5600$?
\item Докажите, что из любых $n$ целых чисел всегда можно выбрать
несколько, сумма которых делится на $n$ (или одно число, делящееся
на $n$).
\item На какую цифру оканчивается число $33^{77}+77^{33}$?
\item Могут ли среди $m$ последовательных целых чисел какие-то два иметь равные остатки от
деления на $m$?
\item Пусть $5x\equiv 6\pmod 8$. Найти $x$.
\item Найти последнюю цифру $7^{100}$, $7^{1942}$.
\item Пусть $a \equiv b \pmod m$, $c \equiv d \pmod m$. Докажите, что сравнения по одному и тому же модулю
\begin{enumerate}[a)]
\item можно складывать и вычитать: $a + c \equiv b + d \pmod m$, $a - c \equiv b - d \pmod m$;
\item можно перемножать: $ac \equiv bd \pmod m$;
\item можно возводить в натуральную степень $n$: $a^n \equiv b^n \pmod m$;
\item можно домножать на любое целое число $k$: $ka \equiv kb \pmod m$.
\end{enumerate}
\item Найдите остаток от деления \textbf{а)} числа $1 + 31 + 331 + \dots + 3333333331$ на $3$; \textbf{б)} $6100$ на $7$.
\item Найдите остаток от деления числа $1 - 11 + 111 - 1111 + \dots - 1111111111$ на $9$.
\item Найдите остаток от деления \textbf{а)} $10!$ на $11$; \textbf{б)} $11!$ на $12$.
\item \textbf{а)} Какой цифрой оканчивается $8^{18}$? \textbf{б)} При каких натуральных $k$ число $2^k-1$ кратно $7$?
\item Найдите три последние цифры числа $1999^{2000}$.
\item Найти \textbf{а)} $3^{31}\pmod 7$, \textbf{б)} $2^{35}\pmod 7$, \textbf{в)} $128^{129}\pmod 17$.
\item Докажите, что \textbf{а)} $30^{99} + 61^{100}$ делится на 31; \textbf{б)} $43^{95} + 57^{95}$ делится на 100.
\item Докажите, что $1^n + 2^n + \dots + (n - 1)^n$ делится на $n$ при нечётном $n$.
\item Числа $x$ и $y$ целые, причем $x^2 + y^2$ делится на 3. Докажите, что $x$ и $y$ делятся на 3.
\item *Докажите, что существует бесконечно много натуральных чисел, не представимых как сумма трёх или менее точных квадратов.
\item Даны 20 целых чисел, ни одно из которых не делится на 5. Докажите, что сумма двадцатых
степеней этих чисел делится на 5.
\item Какие целые числа дают при делении на 3 остаток 2, а при делении на 5 --- остаток 3?
\item Докажите, что остаток от деления простого числа на 30 есть или простое число или 1.
\item *Сколько есть способов записать 2018 как сумму натуральных слагаемых, любые два из
которых равны или различаются на 1? (Способы лишь с разным порядком слагаемых считаем равными.)
\item Докажите, что из любых 52 целых чисел всегда можно выбрать два таких числа, что
\textbf{а)} их разность делится на 51; \textbf{б)} их сумма или разность делится на 100.
\item *Докажите, что из любых $n$ целых чисел всегда можно выбрать несколько, сумма которых
делится на $n$ (или одно число, делящееся на $n$).
\item *\textbf{а)} Докажите, что для любого натурального $N$ существует делящееся на $N$ натуральное
число, все цифры которого только 0 и 1. \textbf{б)} Найдётся ли такое число вида $1\dots10\dots0$?
\item *Шайка из $K$ разбойников отобрала у купца мешок с $N$ монетами. Каждая монета стоит
целое число грошей. Оказалось, что какую монету ни отложи, оставшиеся монеты можно поделить
между разбойниками так, что каждый получит одинаковую сумму. Докажите, что $N-1$ делится на $K$.
\end{enumerate}







%Глава 5

%\item Доказать, что $\langle g_0\rangle\cap h\langle g_0\rangle = \emptyset$, т.е. конечная группа движений распадается на два непересекающихся класса, один из которых получается применением отражения ко второму.
%\item Пусть $G$ --- коммутативная группа, $g\in G$ и $H$ --- подгруппа группы $G$. Доказать, что множество $gH$ равномощно множеству $H$, т.е. отображение $f(h)=gh$ разным $h$ ставит в соответствие разные значения, и при том все элементы $gH$ являются значениями отображения $f$.
%\item Вывести из предыдущего утверждения \textbf{теорему Лагранжа}: порядок подгруппы делит порядок группы.\index{Теорема!Лагранжа о порядке группы}
%\item Обобщить результат на некоммутативные группы.

%Глава 8

%\item В группе $\Z_8^*$ найти обратные элементы: $3^{-1}, 5^{-1}, 7^{-1}$.
%\item Проверить, что $\Z_m$ удовлетворяет аксиомам кольца.
