\documentclass[12pt]{article}
\usepackage[utf8]{inputenc}
\usepackage[T2A]{fontenc}
\usepackage[russian]{babel}
\usepackage{amsmath}
\usepackage{amssymb}
\usepackage{setspace}
\usepackage{indentfirst}
\usepackage{mathtext}

\newenvironment{psmallmatrix}
  {\left(\begin{smallmatrix}}
  {\end{smallmatrix}\right)}
  

\begin{document}
\section*{Линейные отображения векторов плоскости}
Ниже представлены основные определения и задачи к лекции «Линейные отображения векторов плоскости» к курсу «100 уроков математики» Алексея Владимировича Саватеева.
Задачи разделены на 2 вида: типовые и нетиповые.
\\

\textbf{Определение.} Пусть $\mathbb V$ - линейное пространство над полем $\mathbb F$. Отображение $A$, действующее из $\mathbb V$ в $\mathbb V$, называется \textit{линейным оператором}, если для любых чисел $\alpha, \beta \in \mathbb F$ и любых векторов $\textbf{x},\textbf{y} \in \mathbb V$ выполняется равенство $A(\alpha \textbf{x} + \beta \textbf{y}) = \alpha A(\textbf{x}) + \beta A(\textbf{y})$.

\subsection*{Типовые задачи}


1. Пусть  $x = (x_{1}, x_{2}, ..., x_{n})^T $ - произвольный вектор n-мерного арифметического пространства. Исследовать линейность преобразования $\phi$, если:

а) $\phi(x) = (x_{2}, x_{1} - x_{2})^T$ (n = 2)

б) $\phi(x) = (x_{2}, x_{1}x_{2})^T$ (n = 2)

в) $\phi(x) = (x_{2}, x_{1}-3, x_{3})^T$ (n = 3)

г) $\phi(x) = (2x_{3} + x_{1},2x_{3}x_{1},x_{1} - x_{2})^T$ (n = 3)

д) $\phi(x) = (0, 0, ..., 0)^T$

е) $\phi(x) = (0, x_{1} + 3x_{2}, x_{2}^2)^T$ (n = 3)

ж) $\phi(x) = (0, 0, ..., 0, 1)^T$

з) $\phi(x) = (\sin x_{1}, \cos x_{2}, x_{3})^T$ (n = 3)

и) $\phi(x) = (x_{n}, x_{n-1}, ..., x_{1})^T$

к) $\phi(x) = (2x_{1}, 2|x_{2}|, 2x_{3})^T$ (n = 3)

2. Пусть $\phi, \psi, \chi$ - линейные отображения арифметических линейных пространств, $\alpha$ - число. При каких условиях на размерности пространств справедливо каждое из следующих равенств?

а) $\phi(\psi\chi) = (\phi\psi)\chi$

б) $\phi(\psi + \chi) = \phi\psi + \phi\chi$

в) $(\phi + \psi)\chi = \phi\chi + \psi\chi$

г) $\alpha(\phi + \psi) = \alpha\phi + \alpha\chi$

Показать, что матрицы данных отображений удовлетворяют тем же равенствам.

3. Даны линейные отображения $\phi: \mathbb R_{n} \to \mathbb R_{m}$, $\psi: \mathbb R_{l} \to \mathbb R_{k}$.

а) Указать условия на $m, n, l, k$, необходимые и достаточные для существования произведений $\phi\psi$ и $\psi\phi$.

б) Пусть $\chi = \phi\psi$. Показать, что $\chi$ - линейное отображение. Как связаны матрицы отображения $\phi, \psi, \chi$?

3. Доказать, что произведение (композиция) линейных отображений есть линейное отображение. Проверить свойства ассоциативности и дистрибутивности.

4. Доказать, что ядро и образ линейного отображения являются линейными пространствами.

5. Являются ли линейными следующие отображения $A: L_{1} \to L_{2}:$

а) $Ax = 0$;

б) $L_1 = L_2, Ax = x$ (тождественное отображение; обозначение: Id или E);

в) $L_1 = \mathbb R^4, L_2 = \mathbb R^3, A(x,y,z,t) = (x+y, y+z, z+t)$;

г) $L_1 = L_2 = \mathbb R^3, A(x,y,z) = (x+1, y+1, z+1)$;

д) $L_1 = L_2 = F[x], (Ap)(x) = p(ux^2+v), u,v $ - фиксированные элементы F[x];

e) $L_1 = L_2 = F[x], (Ap)(x)= q(x)p(x), q(x) $- фиксированный элемент F[x];

ж) $L_1—пространство сходящихся последовательностей действи-
тельных чисел, L_2 = \mathbb R, A(x_i) = $$ \lim_{n \to \infty}
\sum_{k=1}^n \frac{1}{k^2}
= \frac{\pi^2}{6} $,

6. Найти ядра и образы линейных отображений задачи 5.

7. Пусть $A$ — отображение пространства многочленов сте-
пени не выше $n$ с действительными коэффициентами в пространство
функций на $M \subset \mathbb R$, которое переводит многочлен в его ограничение
на M. 

а) Доказать, что A линейно. 

б) При каких $M ker(A) = 0$?

8. Показать, что каждое из следующих отображений, действующих в линейном пространстве $\mathbb R^3$, является линейным оператором, найти его матрицу в базисе $\textbf e_1$ = (1,0,0), $\textbf e_2 = (0,1,0)$, $\textbf e_3 = (0,0,1)$ и его определитель. Для любого вектора $\textbf{x} = (x_1, x_2, x_3) \in \mathbb R^3$.

a) $A\textbf{x} = (-3x_1 + 3x_2 - 2x_3, x_1 + 2x_2 - x_3, -x_1 - 3x_2 + 2x_3)$

б) $A\textbf{x} = (-2x_2-x_3, 3x_1 + 2x_2 + 3x_3, x_1 + 2x_2 + 2x_3)$

в) $A\textbf{x} = (-3x_1 + 3x_2 - 2x_3, x_1 + 2x_2 - x_3, -x_1 - 3x_2 + 2x_3)$

г) $A\textbf{x} = (-3x_1 + 3x_2 - 2x_3, x_1 + 2x_2 - x_3, -x_1 - 3x_2 + 2x_3)$

д) $A\textbf{x} = (-3x_1 + 3x_2 - 2x_3, x_1 + 2x_2 - x_3, -x_1 - 3x_2 + 2x_3)$

е) $A\textbf{x} = (-3x_1 + 3x_2 - 2x_3, x_1 + 2x_2 - x_3, -x_1 - 3x_2 + 2x_3)$

9. Для операторов, действующих в $\mathbb R^2$ найти:
а) $A-B$; б) $2A+3B$; в) $AB$; г) $BA$; д) $A^2$ e) $B^3$, если
$A\textbf{x} = \begin{pmatrix}1 & 2\\3 & -1\end{pmatrix}\textbf{x}$, $B\textbf{x} = \begin{pmatrix}-1 & 0 \\ 2 & 1\end{pmatrix}\textbf{x}$, где
$\textbf{x} = (x_1, x_2)$

10. Вычислить размерности ядра и образа линейного отображения, заданного матрицей:
а) $\begin{pmatrix}
1 &2 &4\\
1 &3 &5\\
1 &1 &3\\
1 &4 &6\\
1 &1 &3
\end{pmatrix}$, 
б) $\begin{pmatrix}
2 & 1 & 0 & 3 & 0\\
3 & 1 & 0 & 2 & 0\\
1 & 1 & 1 & 1 & 1\\
2 & 3 & 1 & 4 & 1
\end{pmatrix}$
\subsection*{Нетиповые задачи}	
1. Доказать, что для всякого линейного отображения $\phi$ существует пара базисов, в которых матрица отображения имеет простейший вид $\begin{psmallmatrix} E & 0\\0 & 0 \end{psmallmatrix}$. Чему равен порядок матрицы E?

2. В линейном пространстве $\mathbb F$ дан базис $\textbf{e}$. Является ли группой относительно умножения данное множество линейных преобразований пространства $\mathbb F$?

1) множество всех линейных преобразований;

2) множество всех преобразований, матрицы которых диагональны в базисе \textbf{е};

3) множество всех невырожденных преобразований, которые в базисе \textbf{e} задаются целочисленными матрицами, т.е. матрицами $||a_{ij}||$, где $a_{ij}$ - целые числа;

4) множество всех преобразований, матрицы которых в базисе \textbf{e} целочисленны и имеют определители, равные 1 или -1;

5) множество всех преобразований с данным определителем \textit{d};

6) множество всех невырожденных преобразований, имеющих в базисе \textbf{е} матрицы, каждая строка и каждый столбец которых содержат ровно по одному ненулевому элемнету?

3. Пусть $A$ - оператор на конечномерном пространстве. Доказать, что существует многочлен $f \in k[X]$, такой что $f(A) = 0$. 

4. Доказать, что оператор обратим тогда и только тогда, когда его определитель не равен нулю.

5. Верно ли, что любое линейное отображение $\phi$ из векторного пространства $n \times n$ матриц в себя представляется в виде $\phi(X) = A \times B$ для некоторой пары матриц $A, B$? Однозначен ли выбор A,B?

6. Пусть $A$ и $B$ – две квадратные матрицы одинакового порядка $n \times n$.
Доказать, что размерность образа отображения $\phi : \mathbb K^2n \to \mathbb K^2n$, заданного матрицей удвоенного размера равна сумме размерностей образов отображений $A$ и $B$.

а) $\begin{pmatrix}
A & B\\
3A & 2B
\end{pmatrix}$, 
б) $\begin{pmatrix}
A & AB\\
B & B + B^2
\end{pmatrix}$
\\

7. Линейный оператор на двумерном линейном пространстве имеет в некотором базисе матрицу $\begin{pmatrix}1&1\\0&1\end{pmatrix}$. Доказать, что ни в каком базисе этот оператор не записывается диагональной матрицей. 

8. Пусть X - матрица размера $m \times n$, где $mn > 1$. Доказать, что преобразование $X \to X^T$ нельзя представить в виде $X \to A \times B$.
\end{document} 