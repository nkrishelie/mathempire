\item *Установить коммутативность произвольного кольца, в котором каждый элемент $x$ удовлетворяет уравнению $x^2=x$.
\item Докажите, что $m\Z$ --- подкольцо (без единицы) кольца $\Z$, т.е. в нем также можно складывать, вычитать и умножать, $m$ --- произвольное целое положительное число.


\item В какие точки переходят точки $(0,3)$ и $(4,0)$ при повороте на $90$ градусов? На $-90$ градусов?
\item Каков угол поворота, если точка $(a,b)$ перешла в точку $(-a,-b)$? В точку $(-b,a)$? В точку $(b,-a)$?
\item Чему равен тангенс угла наклона прямой $3x-5y=7$?
\item Какой угол наклона у прямой $y=-x+3$?
\item Доказать, что орбиты группы вращений с центром $O$ не пересекаются.
\item *Пусть $G$ --- подгруппа группы биекций некоторого множества $X$. Доказать, что орбиты группы $G$ в $X$ попарно не пересекаются и в объединении дают все множество $X$.


\item Сравнить таблицу симметрий ромба с таблицей умножения группы $\Z_8^*$.
\item Проверить, что $\Z_m$ удовлетворяет аксиомам кольца.

\end{enumerate}



\begin{comment}
\chapter{11. Движения плоскости и пространства}
\end{comment}
\newchapter[и пространства]{Движения плоскости}

\section{Виды движений плоскости. Теорема Шаля}

\subsection*{Справочные сведения}

\subsection*{Задачи}


\begin{enumerate}
\item Найти композицию отражения относительно вертикальной оси и поворота на $180^o$ относительно точки, не лежащей на оси симметрии.
\item Пусть дан произвольный треугольник. На его сторонах построим правильные треугольники и соединим их центры. Доказать, что полученный треуголник --- правильный.
\end{enumerate}


\section{Сравнение движений прямой, окружности и плоскости}

\subsection*{Справочные сведения}

\subsection*{Задачи}




\section{Пара слов о движениях сферы}

\subsection*{Справочные сведения}

\subsection*{Задачи}

\begin{enumerate}
\item Построить таблицу движений сферы аналогично таблице движений плоскости (символику придумайте сами).
\item **Доказать, что других движений на сфере не существует (лемма о гвоздях).
\end{enumerate}


\section{Пара слов о движениях пространства}

\subsection*{Справочные сведения}

\subsection*{Задачи}

\begin{enumerate}
\item Построить таблицу движений пространства аналогично таблице движений плоскости (символику придумайте сами).
\item *Показать, что центральная симметрия пространства --- это зеркальное вращение.
\item **Доказать, что других движений в пространстве не существует (лемма о гвоздях).
\end{enumerate}



\begin{comment}
\chapter{12. Комплексная арифметика и алгебра}
\end{comment}
\newchapter[и алгебра]{Комплексная арифметика}


\section{Алгебра комплексных чисел}

\subsection*{Справочные сведения}

\subsection*{Задачи}


\begin{enumerate}
\item Докажите, что если $\la>0$, то $|\la-1|<\la+1$.
\item Вычислить, нарисовать на плоскости и указать модуль и аргумент следующих комплексных чисел:
$$
i^2,\;i^3,\;i^4,\;1/i,\;(1+2i)(2-i),\;(1+i)(1+2i)(1+3i),\;\frac{1}{1+i},\;\frac{5}{2-i}.
$$
\end{enumerate}


\section{Гауссовы целые числа}

\subsection*{Справочные сведения}

\subsection*{Задачи}

\begin{enumerate}
\item Найти все 4 остатка от деления $2+3i$ на $1+i$.
\item Найти минимальный по норме остаток от деления $3+7i$ на $1+2i$.
\end{enumerate}


\begin{comment}
\chapter{13. Введение в линейную алгебру}
\end{comment}
\newchapter[линейную алгебру]{Введение в}\label{linalg}


\section{Преобразования}

\subsection*{Справочные сведения}

\subsection*{Задачи}




\section{Подобия прямой и плоскости}


\subsection*{Справочные сведения}

\subsection*{Задачи}



\section{Линейное пространство}

\subsection*{Справочные сведения}

\subsection*{Задачи}



\section{Линейные операторы}

\subsection*{Справочные сведения}

\subsection*{Задачи}



\section{Арифметика матриц}

\subsection*{Справочные сведения}

\subsection*{Задачи}



\section{Матрицы и комплексные числа}

\subsection*{Справочные сведения}

\subsection*{Задачи}




\section{Действие линейных отображений на векторном пространстве}

\subsection*{Справочные сведения}

\subsection*{Задачи}



\begin{comment}
\chapter{14. Алгебраические числа}
\end{comment}
\newchapter{Алгебраические числа}


\section{*Упорядоченные множества}\label{Ordering}

\subsection*{Справочные сведения}

\subsection*{Задачи}

\begin{enumerate}
\item Пусть множество $X$ не пусто. Верно ли, что $\emptyset<X$? Верно ли, что $X<\emptyset$? Верно ли, что $\emptyset<\emptyset$?
\item Каким отношением (антирефлексивным, транзитивным, связным) является отношение несобственного вложения множеств? $X$ есть несобственное подмножество $Y$ (обозначение: $X\subset Y$), если $X\subseteq Y$ и $X\ne Y$.
\item Каким отношением является отношение делимости $x|y$ на положительных целых числах?
\item Является ли всюду плотным множество всех десятично рациональных чисел, т.е. чисел вида $k/10^n$, где $k\in \Z$ и $n\in\N$?
\item Выпишите полный список аксиом упорядоченного поля.
\end{enumerate}


\section{Плотные множества}

\subsection*{Справочные сведения}

\subsection*{Задачи}



\section{Зазоры между рациональными числами}

\subsection*{Справочные сведения}

\subsection*{Задачи}



\section{О построениях циркулем и линейкой}

\subsection*{Справочные сведения}

\subsection*{Задачи}

\begin{enumerate}
\item Построить циркулем и линейкой отрезки длины $\sqrt 2$ и $\sqrt 3$.
\item Доказать, что $\Q[\sqrt[3]{2}]=\{a+b\sqrt[3]{2}+c\sqrt[3]{4}\;|\;a,b,c\in\Q\}$ и имеет размерность 3 над $\Q$.
\end{enumerate}





\section{Многочлены и алгебраические числа}

\subsection*{Справочные сведения}

\subsection*{Задачи}




\begin{comment}
\chapter{15. Континуум}
\end{comment}
\newchapter{Континуум}


\section{Мощности множеств}\label{powers}

\subsection*{Справочные сведения}

\subsection*{Задачи}

\begin{enumerate}
\item Найти мощность множества всех функций из $X=\{1,2,3\}$ в $Y=\{1,2,3,4\}$.
\item Какова мощность множества $\{(n,m)\mid n,m\in\Z, n<m\}$?
\item Пусть множество $C$ счетное, а множество $X$ бесконечное.
\begin{enumerate}[a)]
\item Доказать, что $X\cup C\leftrightarrow X$ (равномощны).
\item Доказать, что если $X\setminus C$ --- бесконечное, то $X\setminus C\leftrightarrow X$.
\end{enumerate}
\end{enumerate}

\section{Изоморфизмы}

\subsection*{Справочные сведения}

\subsection*{Задачи}


\section{Действительные числа}

\subsection*{Справочные сведения}

\subsection*{Задачи}

\begin{enumerate}
\item Доказать, что между любыми двумя рациональными числами $r\ne q$ лежит какое-то двоично-рациональное.
\end{enumerate}


\section{Модели действительных чисел}

\subsection*{Справочные сведения}

\subsection*{Задачи}

\begin{enumerate}
\item Дать определение вещественного числа $\sqrt[3]{-5}$ с помощью дедекиндового сечения, т.е. построить соответствующую пару подмножеств множества $\Q$.
\end{enumerate}




\begin{comment}
\chapter{16. Элементы математического анализа}
\end{comment}
\newchapter[математического анализа]{Элементы}

\section{Оценки и пределы}

\subsection*{Справочные сведения}

\subsection*{Задачи}


\section{Экспонента}

\subsection*{Справочные сведения}

\subsection*{Задачи}

\begin{enumerate}
\item Пусть дана последовательность
$$
x_n = \left(1+\frac xn\right)^n.
$$
доказать, что она монотонно возрастает и ограничена сверху.
\item А как ведет себя последовательность
$$
y_n = \left(1+\frac xn\right)^{n+1}?
$$
\item Определим логарифм числа $x>0$ по основанию $a>0$ как такое число $y$, что $a^y=x$. Обозначение: $y=\log_a x$. Вывести формулу:
$$
\log_a x = \frac{\ln x}{\ln a} = \frac{\log_b x}{\log_b a}
$$
при любом положительном основании $b$.
\item Сравнить два числа:
$$
5^{\log_7 3}\quad vs\quad   3^{\log_7 5}.
$$

\end{enumerate}



\section{Комплексная экспонента}

\subsection*{Справочные сведения}

\subsection*{Задачи}
\begin{enumerate}
\item Найти решения уравнения $\sin z=4$.
\item Чему равны выражения
$$
\frac{e^{ix}+e^{-ix}}{2},\quad \frac{e^{ix}-e^{-ix}}{2i}?
$$
\end{enumerate}



