\documentclass[12pt]{article}
\usepackage[utf8]{inputenc}
\usepackage[T2A]{fontenc}
\usepackage[russian, english]{babel}
\usepackage{amsmath}
\usepackage{amssymb}
\usepackage{setspace}
\usepackage{mathtext}
\pagestyle{empty}
\usepackage{bm}
\usepackage{esvect}
\newenvironment{psmallmatrix}
	{\left(\begin{smallmatrix}}
	{\end{smallmatrix}\right)}

\begin{document}
\section*{67-68 Линейные преобразования}
1. Являются ли линейными следующие отображения $A: L_1 \to L_2$:\\
a) $Ax = 0$;\\
б) $L_1 = L_2, Ax = x$ (такое отображение называется тождественным; обозначение: $id$ или $E$);\\
в) $L_1 = \mathbb{R}^4, L_2 = \mathbb{R}^3, A(x,y,z,t) = (x+y, y+z, z+t)$;\\
г) $L_1 = L_2 = \mathbb{R}^3, A(x,y,z) = (x+1, y+1, z+1)$;\\
д) $L_1 = L_2 = F[x], (Ap)(x) = p(\lambda x^2 + v)$, $\lambda , v$ - фиксированные элементы F;\\
е) $L_1 = L_2 = F[x], (Ap)(x) = q(x)p(x)$, $q(x)$ - фиксированный элемент F[x];\\
ж) $L_1$ - пространство сходящихся последовательностей действительных чисел, $L_2 = \mathbb{R}, A(x_i) = \lim\limits_{i \to \infty}{x_i}$.\\
\\
2. Доказать, что произведение (композиция) линейных отображений есть линейное отображение. Проверить свойства ассоциативности и дистрибутивности.\\
\\
3. Доказать, что линейное отображение однозначно задается образами любых двух непропорциональных векторов\\
\\
4. Доказать, что любое линейное отображение\\
а) оставляет начало координат на месте;\\
б) переводит прямые в прямые;\\
в) сохраняет параллельность прямых\\
\\
5. Всякое ли\\
а) биективное;\\
б) произвольное преобразование плоскости, оставляющее на месте начало координат и переводящее прямые в прямые, является линейным?\\
\\
6. Доказать, что\\ 
а) Линейное преобразование сохраняет углы тогда и только тогда, когда
его матрица имеет вид либо $\begin{pmatrix}a & b\\ -b & a\end{pmatrix}$, либо $\begin{pmatrix}a & b\\ b & -a\end{pmatrix}$. Что это за преобразования геометрически?\\
б) Какие линейные преобразования сохраняют расстояния?\\
\\
7. Найти:\\ а) матрицу поворота на $90^\circ$\\
б) матрицу $R(\phi)$ поворота на угол $\phi$.\\
\\
8. Пусть линейное преобразование $\mathbb{R}^2 \xrightarrow{f}\mathbb{R}^2$ переводит базисные векторы $e_1 = (1,0)$ и $e_2 = (0,1)$ в векторы $(a,c)$ и $(b,d)$ соответственно. Куда это линейное отображение переведёт вектор $(x,y)$?\\
\\
9. Опиcать все линейные и аффинные отображения:\\
a) $\mathbb{R}^n \to \mathbb{R}^1$\\
б) $\mathbb{R}^1 \to \mathbb{R}^n$\\
в) $\mathbb{R}^2 \to \mathbb{R}^2$\\
г) $\mathbb{R}^n \to \mathbb{R}^m$\\
\\
10*. Изменим в определении аффинного отображения фразу «существует $a \in \mathbb{R}^m$» на фразу «$a \in \mathbb{R}^m$». Будет ли новое определение эквивалентно исходному?
\end{document}