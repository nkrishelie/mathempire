\documentclass[12pt]{article}
\usepackage[utf8]{inputenc}
\usepackage[T2A]{fontenc}
\usepackage[russian, english]{babel}
\usepackage{amsmath}
\usepackage{amssymb}
\usepackage{setspace}
\usepackage{mathtext}
\pagestyle{empty}
\usepackage{bm}
\usepackage{esvect}

\begin{document}
\section*{65-66 Векторы}
1. Вычислить модуль вектора $\bm{a} = \{6, 3, -2\}.$\\
2. Даны две координаты вектора $X = 4, Y = -12$. Найти третью координату $Z$, при условии что $|\bm{a}| = 12$\\
3. По данным векторам $\bm{a}$ и $\bm{b}$ построить каждый из следующих векторов:\\
а) $\bm{a+b}$;
б) $\bm{a-b}$;
в) $\bm{b-a}$;\\
4. Дано: $|\bm{a}| = 13$, $|\bm{b}| = 19$, $|\bm{a+b}| = 24$. Вычислить $|\bm{a-b}|$\\
5. Какому условию должны удовлетворять векторы $\bm{a}$ и $\bm{b}$, чтобы имели место следующие соотношения:\\
а)$|\bm{a-b}|$ = $|\bm{a+b}|$;\\
б)$|\bm{a-b}|$ > $|\bm{a+b}|$;\\
в)$|\bm{a-b}|$ < $|\bm{a+b}|$;\\
6. Какому условию должны удовлетворять векторы $\bm{a}$ и $\bm{b}$, чтобы вектор $|\bm{a+b}|$ делил угол пополам между векторами $\bm{a}$ и $\bm{b}$?\\
7. Даны векторы $\bm{a} = \{4,-2,4\}$ и $\bm{b} = \{6, 3, -2\}$. Вычислить:\\
а)$\bm{a}\bm{b} $\\
б)$\sqrt{\bm{a}^2}$\\
в)$\sqrt{\bm{b}^2}$\\
г)$(2\bm{a} - 3\bm{b})(\bm{a}+2\bm{b})$\\
д)$(\bm{a}+\bm{b})^2$\\
е)$(\bm{a}-\bm{b})^2$\\
\\
8. Доказать, что если прямая содержит центр гомотетии, то она отображается в себя.\\
9. Доказать, что если прямая не содержит центр гомотетии, то она отображается в параллельную прямую.\\
10.  Доказать, что при гомотетии окружность переходит в окружость.\\
\\
11. Доказать, что линейное подпространство является линейным пространством (относительно тех же операций сложения и умножения на число).\\
12. Пусть $L_1, L_2$ - линейные подпространства. Являются ли линейными подпространствами следующие множества:\\
а) $L_1 + L_2$;\\
б) $L_1 \cap L_2$;\\ 
в) $L_1 \cup L_2$.\\
\end{document}