\newchapter{Визуальная арифметика}

%\renewcommand{\sectionname}{Тема}

\section{Сложение}

\subsection{План}
\begin{enumerate}\setlength{\itemsep}{1pt}
\item На прямой откладываем отрезки друг за другом ВПРАВО --- это сложение!
\item Нулевой отрезок
\item Сложение коммутативно
\item Сложение ассоциативно
\item Сложение интерпретируется как сумма однонаправленных векторов
\end{enumerate}
\subsection{Задачи}



\section{Вычитание}

\subsection{План}
\begin{enumerate}\setlength{\itemsep}{1pt}
\item Вычитание --- это <<сложение налево>>, откладывание отрезка влево.
\item Сложение и вычитание интерпретируем как две команды: сделать шаг $a$ вправо, сделать шаг $b$ влево (на сцену выходит кузнечик!)
\item Сложение и вычитание коммутируют
\item Вычитание можно интерпретировать как сложение, если смотреть на прямую с противоположной стороны доски
\item Вычитание --- это сложение противоположно направленных векторов
\item Вычитание или сложение --- вопрос ориентации вектора!
\item Векторы, направленные вправо, идут со знаком $+$ (или вовсе без знака), а векторы, направленные влево, --- со знаком $-$
\end{enumerate}

\subsection{Задачи}



\section{Умножение}

\subsection{План}
\begin{enumerate}\setlength{\itemsep}{1pt}
\item Строим две перпендикулярно направленные оси
\item Умножение --- это площадь, построенная на векторах. $2\times 2=4$
\item Знак умножения определяется направлением векторов
\item Таблица умножения знаков
\begin{table}[htb!]\begin{flushright}
\begin{tabular}{c|c|c|}
  & $+$ & $-$ \\
 \hline
$+$ & $+$ & $-$ \\
 \hline
$-$ & $-$ & $+$ \\
\hline
\end{tabular}
\end{flushright}\end{table}
\item Понятие группы на данном примере. Элемент + является нейтральным элементом
\item Умножение коммутативно и ассоциативно
\item Умножение на нулевой отрезок (мультипликативное свойство нуля)
\item Дистрибутивный закон, в том числе при разнонаправленных векторах
\item Единичный отрезок --- способ свести многократное сложение одного вектора к умножению на сумму единичных отрезков! Прямоугольник единичной высоты и длины $an$ перекладывается в прямоугольник $a\times n$, тем самым сложение превращается в умножение
\item Сложение отрезков --- это также сложение прямоугольников единичной высоты
\item Умножение отрезков --- это не только площадь, но также и объем, который заметает вертикальный единичный отрезок на площади $a\times b$, поэтому $ab=a\times b\times 1$
\item \textit{Степень}: многократное умножение отрезка самого на себя. Иллюстрация --- отрезок, квадрат, куб
\end{enumerate}
\subsection{Задачи}


\section{Натуральные числа}

\subsection{План}
\begin{enumerate}\setlength{\itemsep}{1pt}
\item Кратность операций сложения и умножения: $a+a+a+a+a+\dots$, $a\cdot a\cdot a\cdot\ldots$ Натуральное число вводится для обозначения кратности одинаковых операций!
\item Нулевая кратность: в случае сложения ничего не складываем, остаемся на месте в начальной точке, поэтому $a\cdot 0=0$,
\item Нулевая степень: в случае умножения ничего не умножаем, от умножения остается только кратность 1, наследуемая от сложения, т.е. в произведении $1\times a\times a\times\dots$ выбрасываем все, остается только 1. Поэтому $a^0=1$, кроме того, это согласуется с законом ассоциативности умножения
\item \textbf{Натуральные числа} --- это показатели кратностей (сложения и умножения)
\item С другой стороны, натуральные числа можно рассматривать как суммы единичных отрезков
$$
n=\underbrace{1+1+\dots+1}_{n\mbox{ раз}}
$$
\item Чудо, но это вполне согласуется с операциями сложения и умножения, сохраняет все законы арифметики: ассоциативность, коммутативность, дистрибутивность
\item Поэтому натуральные числа, привязанные к единичным отрезкам, можно также считать мерой длины, площади, объема и т.д.
\item Ноль --- натуральное число, поскольку мы рассматриваем нулевую кратность для однородности законов арифметики.
\item[NotaBene] Натуральные числа --- это и кратности операций, и единицы измерения, т.е. числа
\item Натуральные числа отвечают за соизмеримость и кратность: $a$ \textbf{кратно} $b$ ($a\mathop{\vdots} b$), если $a=bn$ или $a=(-b)n$ при некотором натуральном $n$. Ноль кратен любому числу! Нулю кратен только ноль!
\end{enumerate}
\subsection{Задачи}



\section{Теорема Пифагора графически}

\subsection{План}
\begin{enumerate}\setlength{\itemsep}{1pt}
\item Строим квадрат $a+b\times a+b$ и внутри квадраты $a\times a$ и $b\times b$
\item Строим квдарат $a+b\times a+b$ и внутри квадрат $c\times c$
\item Делаем вывод, перекладывая треугольники
\item *Построение $\sqrt 2$, $\sqrt 7$ (используются признаки подобия треугольников, отношения строн)
\item Примеры пифагоровых троек (анонс теоремы!)
\end{enumerate}
\subsection{Задачи}

\section{Бином Ньютона и другие формулы визуально}

\subsection{План}
\begin{enumerate}\setlength{\itemsep}{1pt}
\item визуализация $(a-b)(a+b)=a^2-b^2$
\item сумма подряд идущих чисел $1,2,\dots,n$ с помощью сложения прямоугольников
\item сумма подряд идущих нечетных чисел
\item Вывод формулы $(a+b)^3 = a^3+3a^2b+3ab^2+b^3$
\item Разрезание сырного кубика на 8 частей тремя плоскостями
\end{enumerate}
\subsection{Задачи}



\section{Соизмеримость отрезков, алгоритм Евклида}

\subsection{План}
\begin{enumerate}\setlength{\itemsep}{1pt}
\item Два отрезка $a$ и $b$, кузнечики прыгают, один на $a$ и $-a$ сколько угодно раз, второй на $b$ и $-b$ сколько угодно раз
\item Кузнечики стартуют в одной и той же точке (назовем ее $O$). Могут ли они попасть в одну точку, отличную от $O$, когда-нибудь?
\item Ответ --- да, если есть такая точка $A$, что отрезок $OA$ кратен и $a$, и $b$ одновременно, т.е. при некотрых натуральных $n,m$, не равных нулю, будет верно равенство $an=bm$:
$$
\underbrace{a+a+\dots+a}_{n\mbox{ раз}}=\underbrace{b+b+\dots+b}_{m\mbox{ раз}}
$$
\item Отрезки, которые имеют общий кратный отрезок, называются \textbf{соизмеримыми}
\item Иллюстрация: строим прямоугольник $a\times b$ ($a<b$), начинаем отсекать в нем квадраты: сначала отсекаем квадраты $a\times a$, пока можем, останется кусок $a\times b_1$ ($b_1<a$), затем отсекаем квадраты $b_1\times b_1$, пока можем, останется кусок $a_1\times b_1$ ($a_1<b_1$), и т.д.
\item Если исходные отрезки соизмеримы, то процесс остановится: исходный прямоугольник будет разбит на конечное число квадратиков.
\item Финальный квадратик будет иллюстрировать НОД отрезков $a$ и $b$, т.к. это максимальный квадрат, которым можно замостить прямоугольник $a\times b$
\item Такой процесс называется \textbf{алгоритмом Евклида}, к нему мы еще вернемся с более формальной точки зрения
\item Заметим, что числа $a$ и $b$ при этом вовсе не обязан быть натуральными
\item Несоизмеримость стороны квадрата и его диагонали: 1 и $\sqrt 2$
\item Алгоритм Евклида никогда не остановится. НОДом будет бесконечно малое число
\end{enumerate}
\subsection{Задачи}



\newchapter{Движения на прямой}

\section{Сдвиг, композиция сдвигов}

\subsection{План}
\begin{enumerate}\setlength{\itemsep}{1pt}
\item Рассмотрим аффинную прямую, т.е. набор точек и векторов на прямой
\item Сумма точки и вектора есть точка, сумма векторов есть вектор, разность точек есть вектор
\item Команда <<прибавить ко всем точкам вектор $a$>> называется \textbf{сдвигом} прямой на вектор $a$
\item Сдвиг на $a$ --- это операция сложения с вектором без указания конкретной точки приложения, она применяется сразу ко всем точкам! В итоге вся прямая смещается как единое целое
\item Сдвиг является движением (не случайно это однокоренные слова!)
\item Вообще, движение --- это преобразование, сохраняющее расстояния (размеры и форму): если между точками $A$ и $B$ было расстояние $x$, то после преобразования движения расстояние между точками $A'$ и $B'$, в которые перешли исходные точки, тоже будет $x$, и так для любой пары точек!
\item Математическое движение --- это результат физического движения (есть только начальное и конечное состояние системы)
\item Сдвиг на вектор $a$ будем обозначать $T_a$: $T_a(A)$ --- это точка $B$ такая, что $AB$ есть вектор $a$ (совпадает по направлению и длине)
\item Композиция сдвигов --- это их последовательное применение: $$(T_b\circ T_a)(A)=T_b(T_a(A))$$
\item Композиция сдвигов соответствует сумме векторов: $T_b\circ T_a=T_{a+b}$
\item Композиция сдвигов перестановочна в силу коммутативности сложения: $$T_b\circ T_a=T_a\circ T_b$$
\item Кратность сдвига обозначается как степень
$$
\underbrace{T_a\circ\dots\circ T_a}_{n\mbox{ раз}}=T_a^n
$$
и соответствует кратности сложения или умножению на степень кратности: $T_a^n=T_{an}$
\item Нулевой сдвиг $T_0=\id$ --- это \textbf{тождественное преобразование}, которое ничего не меняет
\item Обратный сдвиг $T_a^{-1}$ --- это сдвиг на вектор $-a$, т.е. сдвиг в обратном направлении на ту же величину
\item Вообще, если есть какие-то два преобразования $u$ и $v$ и операция композиции $\circ$, то эти преобразования \textbf{взаимно обратны}, если $u\circ v=\id$ и $v\circ u=\id$, т.е. последовательное применение этих преобразований является тождественным преобразованием
\item Очевидно, что всякий сдвиг имеет обратный, причем $T_a\circ T_a^{-1}=T_a^{-1}\circ T_a=\id$
\item Нулевой сдвиг сам себе обратен
\item Все сдвиги с операцией композиции образуют группу (композиция сдвигов есть сдвиг, ассоциативность выполняется, обратимость имеется)
\item Мало того, группа сдвигов коммутативна (абелева)
\item Кратность обратного сдвига: $T_a^{-n}\rightleftharpoons (T_a^{-1})^n=T_{-a}^n=T_{-an}$
\item На основе только одного сдвига $T_a$ можно построить подгруппу сдвигов $$\{T_a^n, T_a^{-n}\;|\; n=0,1,2,\dots\}$$
\item Эта подгруппа --- реализация целых чисел $\Z$, к которым мы еще вернемся позже
\end{enumerate}
\subsection{Задачи}



\section{Отражение}

\subsection{План}
\begin{enumerate}\setlength{\itemsep}{1pt}
\item Еще один вид движений прямой --- \textbf{отражение}
\item Отражение связано с выделенной точкой --- центром отражения, и все точки переводит в симметричные относительно данного центра. Взяли прямую и перевернули ее на $180^o$, оставляя центр отраженя на месте
\item Отражение с центром $O$ будем обозначать $S_O$
\item Композиция отражений: $$S_O\circ S_C=T_{2CO},\quad S_C\circ S_O=T_{2OC}$$
\item Видим, что композиция отражений является сдвигом и при этом не коммутативна!
\item Композиция отражения и сдвига: $$S_O\circ T_a = S_{O-a/2},\quad T_a\circ S_O = S_{O+a/2}$$
\item Такая композиция является отражением и при этом не коммутативна!
\item Кратность отражения $S_O^n$ определяется четностью числа $n$. В случае четного $n$ это $\id$, в случае нечетного --- исходное $S_O$
\item Отражение обратно самому себе: $S_O\circ S_O=\id$
\item Пара $\{\id, S_O\}$ образует самую маленькую нетривиальную группу движений, которая к тому же является абелевой и циклической (т.е. все ее элементы есть степени какого-то одного, а именно $S_O=S_O^1$, $\id=S_O^2$)
\begin{table}[htb!]\begin{center}
\begin{tabular}{c|c|c|}
  & $\id$ & $S_O$ \\
 \hline
$\id$ & $\id$ & $S_O$ \\
 \hline
$S_O$ & $S_O$ & $\id$ \\
\hline
\end{tabular}
\end{center}\end{table}
\item Видим, что таблица полностью повторяет таблицу умножения знаков, причем $\id$ является нейтральным элементом
\end{enumerate}
\subsection{Задачи}



\section{Таблица Кэли движений прямой}

\subsection{План}
\begin{enumerate}\setlength{\itemsep}{1pt}
\item Еще пример группы: рассмотрим класс всех сдвигов $\T$ и класс всех отражений $\S$
\item Мы можем определить композицию классов $\T\circ \T$, $\T\circ \S$, $\S\circ \T$ и $\S\circ \S$ как все возможные композиции движений из этих классов в указанном порядке
\item Из произведенных выше вычислений легко видеть таблицу композиций этих классов:
\begin{table}[htb!]\begin{center}
\begin{tabular}{c|c|c|}
  & $\T$ & $\S$ \\
 \hline
$\T$ & $\T$ & $\S$ \\
 \hline
$\S$ & $\S$ & $\T$ \\
\hline
\end{tabular}
\end{center}\end{table}
\item Видим полную аналогию с таблицей знаков и таблицей для $\id, S_O$. Здесь класс $\T$ является нейтральным элементом
\item Если теперь собрать в одну кучу все сдвиги и отражения, то получим группу движений прямой
\item Наша цель --- доказать, что других движений нет, т.е. что мнжество $\{T_a,S_O\}_{a,O}$ полностью исчерпывает все возможные движения прямой
\end{enumerate}
\subsection{Задачи}



\section{Теоема о гвоздях, аналог теоремы Шаля}

\subsection{План}
\begin{enumerate}\setlength{\itemsep}{1pt}
\item Анализ движений проводится на основе наблюдений за количеством стационарных точек
\item Пусть движение $M$ таково, что оно оставляет на месте две точки $A\ne B$.
\item $M(A)=A$ и $M(B)=B$. Пусть $C'=M(C)$. $M$ сохраняет расстояния $AC$ и $BC$, откуда $AC=AC'$ и $BC=BC'$, откуда $C=C'$. Т.е. $M(C)=C$ для любых точек $C$, т.е. $M=\id$
\item Пусть движение $M$ оставляет на месте ровно одну точку $O$. В этом случае $A'=M(A)$ и $A\ne A'$ и $OA=OA'$, тогда $A'$ --- отражение $A$ относительно $O$. Следовательно, $M=S_O$
\item Пусть движение $M$ не оставляет на месте ни одной точки и пусть $B=M(A)$ ($B\ne A$). Обозначим $x=AB$. Тогда $T_{x}^{-1}\circ M(A)=A$, т.е. $T_{x}^{-1}\circ M$ оставляет на месте хотя бы одну точку. Если оно оставляет на месте ровно одну точку $A$, то это некоторая симметрия $S_O$, но тогда $M=T_x\circ S_O=S_{O+x/2}$. Получается, что $M$ сохраняет точку $O+x/2$ на месте. Противоречие. Остается вариант, что $T_{x}^{-1}\circ M$ оставляет на месте две точки, но тогда $T_{x}^{-1}\circ M=\id$, откуда $M=T_x\circ \id=T_x$ --- сдвиг.
\item Таким образом, все движения прямой --- это либо сдвиги (в частности, $\id$), либо отражения (теорема Шаля)
\item При этом, любое движение --- это либо одна симметрия, либо композиция двух симметрий
\end{enumerate}
\subsection{Задачи}




\newchapter{Вокруг окружности}

\section{Симметрии окружности}

\subsection{План}
\begin{enumerate}\setlength{\itemsep}{1pt}
\item Берем окружность (обруч). Какие у нее есть движения, переводящие его в самого себя?
\item Очевидно, вращение вокруг центра окружности, а также симметрии относительно прямых, проходящих через центр
\item Окружность --- аналог прямой. Только эту прямую взяли за 2 конца и замкнули где-то на бесконечности
\item Поэтому вращение окружности соответствует сдвигу прямой, а симметрия окружности относительно прямой --- отражению на прямой относительно точки (можно считать ее симметрией относительно перпендикулярной прямой)
\item Если представить, что на окружности большого радиуса живут маленькие одномерные математики, то для них окружность будет практически не отличима от прямой, и движения окружности они будут воспринимать именно как движения прямой
\item Поворот на угол $\al$ обозначим $R_\al$ (положительный --- против часовой стрелки), симметрию относительно прямой, имеющей угол наклона $\ph$, обозначим $S_\ph$ ($0\le\ph<\pi$)
\item Вновь замечаем, что композиция поворотов есть поворот на суммарный угол: $R_\al\circ R_\be=R_{\al+\be}$
\item У каждого поворота есть обратный: $R_\al^{-1}=R_{-\al}$
\item Повороты коммутируют
\item Есть нейтральный поворот $\id=R_0$
\item Так что все повороты образуют группу относительно операции композиции
\item Тем не менее, есть одна особенность: поворот на угол $2\pi k$ --- это тоже $\id$
\item Вообще, повороты, заданные углами с шагом $2\pi$, равны: $R_\al=R_{\al\pm 2\pi k}$, где $k$ --- натуральное число
\item Некоторые повороты дают $\id$ в некоторой кратности, например, $R_{90^o}^4=\id$, $R_{60^o}^6=\id$ и т.д.
\item Если угол, выраженный в градусах, соизмерим с величиной $360^o$, то поворот на данный угол имеет положительную степень, в которой он обращается в $\id$
\item Но есть угол, не обладающий таким свойством, это угол в 1 радиан. Если бы он был соизмерим с полным оборотом, то число $\pi$ оказалось бы соеизмеримым с 1, а это не так!
\item Поэтому некоторые вращения образуют конечные циклические подгруппы в группе движений, а некоторые --- нет.
\end{enumerate}
\subsection{Задачи}



\section{Таблица Кэли для окружности}

\subsection{План}
\begin{enumerate}\setlength{\itemsep}{1pt}
\item Композиция симметрий: 
$$
S_\psi\circ S_\ph=R_{2(\psi-\ph)},\quad S_\ph\circ S_\psi=R_{2(\ph-\psi)}
$$
\item Видим, что композиция симметрий является поворотом и при этом не коммутативна!
\item Композиция симметрии и поворота:
$$
S_\ph\circ R_\al = S_{\ph-\al/2},\quad R_\al\circ S_\ph = S_{\ph+\al/2}
$$
\item Такая композиция является отражением и при этом не коммутативна!
\item По аналогии с прямой обозначим $\R$ класс всех вращений окружности, $\S$ --- класс всех симметрий окружности
\item Получаем аналогичную таблицу композиций:
\begin{table}[htb!]\begin{center}
\begin{tabular}{c|c|c|}
  & $\R$ & $\S$ \\
 \hline
$\R$ & $\R$ & $\S$ \\
 \hline
$\S$ & $\S$ & $\R$ \\
\hline
\end{tabular}
\end{center}\end{table}

где $\R$ является нейтральным элементом
\item Снова наблюдаем все ту же группу умножения знаков!
\item Существуют ли другие движения окружности? Ответ --- нет!
\item Если движение сохраняет на месте две точки окружности, не являющиеся диаметрально противоположными, то это $\id$
\item Если движение сохраняет на месте ровно две диаметрально противоположные точки, то это симметрия
\item Если движение не имеет неподвижных точек. то это поворот на угол, не кратный $360^o$
\item Всякое движение окружности --- это либо поворот, либо симметрия (теорема Шаля)
\item Причем всякое движение окружности можно представить как симметрию или композицию симметрий
\end{enumerate}
\subsection{Задачи}



\section{Наматывание прямой на окружность}

\subsection{План}
\begin{enumerate}\setlength{\itemsep}{1pt}
\item Совместим теперь окружность с прямой иным способом. Выделим на окружности точку $O$ и начнем ее обход (вращение) в положительном направлении.
\item Выше мы видели, что углы поворота, кратные $360^o$, т.е. полном обороту, соответствуют тождественному движению, т.е. приведут нас в точку отправления $O$.
\item Однако, если с точки зрения математического движения ничего не изменилось, физически мы проделали путь, равный длине окружности. Для удобства будем считать, что радиус окружности есть единичный вектор, так что ее длина равна $2\pi$, и с каждым полным оборотом мы будем <<наматывать>> расстояние $2\pi$.
\item Вообще, расстояние, пройденное по окружности единичного радиуса, когда этот радиус заметает угол $\al$, равно $\al(2\pi/360^o)$. Чтобы каждый раз не переводить единицы измерения радиуса в градусы и наоборот, углы также приняот измерять в единицах длины --- радианах. А именно, \textit{угол в} 1 \textit{радиан соответствует повороту, при котором точка проделает по окружности путь, равный по длине радиусу данной окружности}. Нетрудно видеть, что в градусах 1 радиан будет иметь выражение $360^o/(2\pi)$ или $180^o/\pi$.
\item В дальнейшем условимся все углы измерять в радианах, если не потребуется иное.
\item Известно, что число $\pi$ не соизмеримо с целыми числами, так что поворот $R_1$ на 1 радиан ни в какой положительной степени не приведет нас снова в точку исхода $O$.
\item Зато поворот $R_{2\pi}$ в точности возвращает нас в точку отправления $O$.
\item При каждом таком повороте мы проделываем путь, равный углу поворота, т.е. $2\pi$ (радиус равен 1).
\item Следовательно степени такого поворота $R_{2\pi}^n$ дадут прохождение пути длиной $2\pi n$.
\item Представим эту картину не с точки зрения жителей окружности, бегающих по замкнутой траектории, а с точки зрения жителей прямой, которая наматывается на окружность. С их точки зрения все выглядит несколько иначе и больше напоминает движение оклеса по дорожному полотну: окружность катится по прямой и через равные промежутки касается точкой $O$ данной прямой.
\item Если при этом два друга --- один из мира окружности, второй из мира прямой, --- двигаются с одинаковой скоростью в одном направлении, то они могут синхронизироваться в точке касания окружности и прямой и разговаривать друг с другом.
\item Итак, колесо катится, два друга беседуют, точка $O$ то и дело, а именно через каждые $2\pi$ метров соприкасается с прямой. Каждый раз, когда точка $O$ касается прямой, наш ученый друг из мира прямой ставит на прямой отметину и считает их по порядку, т.е. приравнивает к степени совершенного поворота: в начальный момент времени это был 0, затем 1 оборот, затем 2 оборота, и т.д.
\item Что же мы видим на прямой? Мы видим не что иное как шкалу натуральных чисел, в точности соответствующую степеням вращений окружности.
\item Представим теперь, что в какой-то момент касания точки $O$ с прямой физика мира изменилась, и вращение начало осуществляться в обратную сторону!
\item Наши друзья-ученые при этом продолжат совместное путешествие, но только назад. Они пойдут отсчитывать уже проставленные отметки на прямой в убывающем порядке, пока не вренутся в точку 0. Но здесь состоится чудо, и движение продолжится дальше.
\item Как все это записать на языке вращений и сдвигов?
\item Предположим, что сначала окружность повернулась на $n$ полных оборотов вперед, а затем на $m$ полных оборотов назад.
\item Мы получаем итоговое вращение, записываемое как $R_{2\pi n}\circ R_{2\pi m}^{-1}$.
\item А что мы имеем с точки зрения движения на прямой?
\item Сначала был произведен сдвиг $T_{2\pi n}$, затем сдвиг $T_{-2\pi m}$.
\item И мы видим, что индекс, определяющий итоговое вращение и итоговый сдвиг, --- один и тот же!
\item Причем, если $n>m$, то сдвиг будет вправо на расстояние $2\pi(n-m)$, а поворот будет положительным на угол $2\pi(n-m)$.
\item Если же $n<m$, то сдвиг будет влево на расстояние $2\pi(m-n)$, а поворот будет отрицательным (по часовой стрелке) на угол $2\pi(m-n)$.
\item Ранее мы уже договаривались, что перед векторами, направленными влево, будем ставить знак '-'. Так же будем поступать и с углами вращений в отрицательную сторону.
\item Соответственно, при $n<m$ мы будем иметь итоговый сдвиг $T_{-2\pi(m-n)}$ и итоговый поворот $R_{-2\pi(m-n)}$, которые также можно записать в виде степеней:
$$
T_{-2\pi(m-n)}=T_{2\pi}^{-(m-n)}\mbox{ и }R_{-2\pi(m-n)}=R_{2\pi}^{-(m-n)}.
$$
\item Осталось добавить маленький штрих к портрету, а именно: в случае $n<m$ под разностью $n-m$ будем понимать запись $-(m-n)$.
\item Тогда уже независимо от того, $n<m$, или $m<n$, или $n=m$, композиция поворотов и сдвигов сначала на $n$ вправо и затем на $m$ влево будет записываться одинаково:
$$
T_{2\pi(n-m)}=T_{2\pi}^{n-m}\mbox{ и }R_{2\pi(n-m)}=R_{2\pi}^{n-m}.
$$
\item В итоге мы приходим к тому, что называется \textbf{целыми числами}, включающими натуральные числа и отрицательные натуральные числа (при этом $-0=0$).
\item Сколько бы мы ни вращали окружность на $2\pi$ в ту или иную сторону с помощью поворота $R_{2\pi}$, мы совершаем поворот на целую степень полного оборота. При этом как бы мы ни катали окружность по прямой, точка $O$ будет ставить отметки в точках $2\pi k$, где $k$ --- целое число.
\item Последнее замечание про отрицательные числа:
$$
T_{2\pi}^{-k}=S_0\circ T_{2\pi}^k\mbox{ и }R_{2\pi}^{-k}=S_O\circ R_{2\pi}^k.
$$
\item То есть отрицательные повороты и сдвиги --- это всего лишь отражение положительных (в случае прямой центром отражения будет точка, помеченная как 0, а в случае окружности --- прямая, проходящая через точку $O$ и центр окружности)
\end{enumerate}
\subsection{Задачи}




\section{Целые числа. Кольцо}

\subsection{План}
\begin{enumerate}\setlength{\itemsep}{1pt}
\item Итак, совмещение вращений со сдвигами дает нам полную свободу перемещений в положительном и отрицательном направлении. При этом с точки зрения окружности ничего не меняется --- происходит итоговое движение $\id$, а с точки зрения прямой --- происходит разметка точек с равным шагом. Ясно, что сам шаг при этом не имеет значения. Мы могли бы взять окружность радиуса $R$, и тогда шаг был бы равен $2\pi R$. В частности, можно взять радиус $R=1/2\pi$, и тогда точки на прямой расположатся с шагом 1.
\item Такую же картину можно получить, если взять все точки, получаемые из выделенной точки 0 степенями сдвига на единичный вектор, используя положительные и отрицательные, т.е. целые, степени.
\item Как видим, целые числа, как и натуральные, можно интерпретировать и как степени движений (и вообще любых преобразований, имеющих обратные), и как векторы сдвигов на прямой, а значит, к ним применимы определенные ранее операции сложения, вычитания и умножения. При этом результат умножения получает такой знак, который определяется из таблицы умножения знаков.
\item Множество всех целых чисел принято обозначать $\Z$. Вместе с операциями сложения (вычитания) и умножения структура $(\Z,+,\cdot)$ называется кольцом целых чисел. Кольцо --- это структура, где можно складывать, вычитать и умножать.
\item Понятие кольцо является расширением понятия группы, т.к. добавляется операция умножения.
\item Ранее мы уже видели такие группы, как группа движений прямой, группа умножения знаков, группа композиций классов сдвигов и симметрий, группа вращений окружности. Все они обладали одной операцией --- композицией, которая соответствовала сложению параметров сдвигов и вращений.
\item Кроме того, мы ввели такое понятие как кратность, заменяя тем самым многократное сложение умножением на целое число.
\item Кратность операций нельзя рассматривать как умножение сдвигов или вращений, поскольку это сущности разного рода. Поэтому движения в общем случае образуют только лишь группу.
\item Однако, уже сами кратности, как самостоятельные сущности, можно и складывать, и умножать. Например, если мы рассмотрим сдвиг $T_1$ и композицию его кратностей $T_1^n\circ T_1^m$, то получим тот же сдвиг но в суммарной кратности $T_1^{n+m}$, где $n,m\in\Z$. Но ничто не мешает нам рассмотреть кратность $m$ сдвига $T_1^n$, т.е. сдвиг $(T_1^n)^m$, а это уже будет не что иное, как сдвиг кратности $nm$, т.е. $T_1^{nm}$.
\item Иначе говоря, умножение на целых числах можно представить как кратности кратностей сдвигов!
\end{enumerate}
\subsection{Задачи}




\section{Звездочки велосипеда. Соизмеримость и кузнечик}

\subsection{План}
\begin{enumerate}\setlength{\itemsep}{1pt}
\item Теперь представим, что рядом с колесами прыгает кузнечик, который может прыгать либо только на $n$ вперед или назад, либо только на $m$, так что он всегда попадает ровно в те точки, которые колеса трактора отмечают на полотне дороги.
\item Таким образом, кузнечик может достичь точек $kn$, где $k$ --- целое, и точек $lm$, где $l$ --- целое. Как описать все точки, в которые будет попадать кузнечик?
\item Обозначим $n\Z$ множество всех точек, кратных $n$, т.е. $n\Z=\{nk\;|\;k\in\Z\}$. Аналогично, $m\Z$.
\item Ясно, что кузнечик попадает в точки множества $n\Z\cup m\Z$.
\item Теперь определим сумму Минковского: $n\Z+m\Z=\{nk+ml\;|\;k,l\in\Z\}$.
\item Ясно, что $n\Z\cup m\Z = n\Z+m\Z$
\item 
\item 
\end{enumerate}
\subsection{Задачи}





\newchapter{Симметрии фигур}


\section{Симметрии правильного треугольника}

\subsection{План}
\begin{enumerate}\setlength{\itemsep}{1pt}
\item 
\item 
\item 
\item 
\item 
\item 
\item 
\item 
\item 
\item 
\end{enumerate}
\subsection{Задачи}



\section{Симметрии ромба, группа Клейна}

\subsection{План}
\begin{enumerate}\setlength{\itemsep}{1pt}
\item 
\item 
\item 
\item 
\item 
\item 
\item 
\item 
\item 
\item 
\end{enumerate}
\subsection{Задачи}



\section{Симметрии правильного многоугольника}

\subsection{План}
\begin{enumerate}\setlength{\itemsep}{1pt}
\item 
\item 
\item 
\item 
\item 
\item 
\item 
\item 
\item 
\item 
\end{enumerate}
\subsection{Задачи}



\section{Подгруппы вращения окружности}

\subsection{План}
\begin{enumerate}\setlength{\itemsep}{1pt}
\item 
\item 
\item 
\item 
\item 
\item 
\item 
\item 
\item 
\item 
\end{enumerate}
\subsection{Задачи}





\newchapter{Движения плоскости}

\section{Виды движений плоскости}

\subsection{План}
\begin{enumerate}\setlength{\itemsep}{1pt}
\item 
\item 
\item 
\item 
\item 
\item 
\item 
\item 
\item 
\item 
\end{enumerate}
\subsection{Задачи}




\section{Теорема Шаля}

\subsection{План}
\begin{enumerate}\setlength{\itemsep}{1pt}
\item 
\item 
\item 
\item 
\item 
\item 
\item 
\item 
\item 
\item 
\end{enumerate}
\subsection{Задачи}




\section{Таблица движений}

\subsection{План}
\begin{enumerate}\setlength{\itemsep}{1pt}
\item 
\item 
\item 
\item 
\item 
\item 
\item 
\item 
\item 
\item 
\end{enumerate}
\subsection{Задачи}




\newchapter{Исчисление остатков}

\section{Простые числа, их бесконечность}

\subsection{План}
\begin{enumerate}\setlength{\itemsep}{1pt}
\item 
\item 
\item 
\item 
\item 
\item 
\item 
\item 
\item 
\item 
\end{enumerate}
\subsection{Задачи}




\section{Таблица сложения остатков}

\subsection{План}
\begin{enumerate}\setlength{\itemsep}{1pt}
\item 
\item 
\item 
\item 
\item 
\item 
\item 
\item 
\item 
\item 
\end{enumerate}
\subsection{Задачи}




\section{Умножение остатков. Поле}

\subsection{План}
\begin{enumerate}\setlength{\itemsep}{1pt}
\item 
\item 
\item 
\item 
\item 
\item 
\item 
\item 
\item 
\item 
\end{enumerate}
\subsection{Задачи}



\section{Малая теорема Ферма}

\subsection{План}
\begin{enumerate}\setlength{\itemsep}{1pt}
\item 
\item 
\item 
\item 
\item 
\item 
\item 
\item 
\item 
\item 
\end{enumerate}
\subsection{Задачи}



\section{Многочлены}

\subsection{План}
\begin{enumerate}\setlength{\itemsep}{1pt}
\item 
\item 
\item 
\item 
\item 
\item 
\item 
\item 
\item 
\item 
\end{enumerate}
\subsection{Задачи}



\newchapter{Основная теорема арифметики и ее следствия}


\section{Алгоритм Евклида визуально}

\subsection{План}
\begin{enumerate}\setlength{\itemsep}{1pt}
\item 
\item 
\item 
\item 
\item 
\item 
\item 
\item 
\item 
\item 
\end{enumerate}
\subsection{Задачи}



\section{Соизмеримость и НОД}

\subsection{План}
\begin{enumerate}\setlength{\itemsep}{1pt}
\item 
\item 
\item 
\item 
\item 
\item 
\item 
\item 
\item 
\item 
\end{enumerate}
\subsection{Задачи}



\section{ОТА в целых числах}

\subsection{План}
\begin{enumerate}\setlength{\itemsep}{1pt}
\item 
\item 
\item 
\item 
\item 
\item 
\item 
\item 
\item 
\item 
\end{enumerate}
\subsection{Задачи}



\section{Корни и разрешимость уравнений}

\subsection{План}
\begin{enumerate}\setlength{\itemsep}{1pt}
\item 
\item 
\item 
\item 
\item 
\item 
\item 
\item 
\item 
\item 
\end{enumerate}
\subsection{Задачи}


\section{Рациональные дроби}

\subsection{План}
\begin{enumerate}\setlength{\itemsep}{1pt}
\item 
\item 
\item 
\item 
\item 
\item 
\item 
\item 
\item 
\item 
\end{enumerate}
\subsection{Задачи}



\section{Цепные дроби}

\subsection{План}
\begin{enumerate}\setlength{\itemsep}{1pt}
\item 
\item 
\item 
\item 
\item 
\item 
\item 
\item 
\item 
\item 
\end{enumerate}
\subsection{Задачи}



\section{Расширение поля рациональных чисел}

\subsection{План}
\begin{enumerate}\setlength{\itemsep}{1pt}
\item 
\item 
\item 
\item 
\item 
\item 
\item 
\item 
\item 
\item 
\end{enumerate}
\subsection{Задачи}



\newchapter{Комплексные числа и Гаусс}

\section{Комплексные числа}

\subsection{План}
\begin{enumerate}\setlength{\itemsep}{1pt}
\item 
\item 
\item 
\item 
\item 
\item 
\item 
\item 
\item 
\item 
\end{enumerate}
\subsection{Задачи}



\section{Реализация движений с помощью комплексных чисел}

\subsection{План}
\begin{enumerate}\setlength{\itemsep}{1pt}
\item 
\item 
\item 
\item 
\item 
\item 
\item 
\item 
\item 
\item 
\end{enumerate}
\subsection{Задачи}



\section{Гомотетии прямой и плоскости}

\subsection{План}
\begin{enumerate}\setlength{\itemsep}{1pt}
\item 
\item 
\item 
\item 
\item 
\item 
\item 
\item 
\item 
\item 
\end{enumerate}
\subsection{Задачи}



\section{Основная теорема Алгебры}

\subsection{План}
\begin{enumerate}\setlength{\itemsep}{1pt}
\item 
\item 
\item 
\item 
\item 
\item 
\item 
\item 
\item 
\item 
\end{enumerate}
\subsection{Задачи}



\section{Числа Гаусса}

\subsection{План}
\begin{enumerate}\setlength{\itemsep}{1pt}
\item 
\item 
\item 
\item 
\item 
\item 
\item 
\item 
\item 
\item 
\end{enumerate}
\subsection{Задачи}




