\setcounter{chapter}{-1}

\renewcommand{\chapter}[1]{\newchapter{#1}}

\chapter{Логика и множества (факультативно)}

\vrezka{В этой главе обсуждаются основы математической логики и теории множеств, построение высказываний и множеств на бытовых примерах. Вводится понятие суммы и произведения числовых множеств по Минковскому.

Данная глава носит справочный характер и может быть пропущена при первом чтении конспекта. Тем не менее, настоятельно рекомендуется регулярно возвращаться к ней по мере освоения материала.
}

\section{Суждения и силлогизмы}

\subsection*{Конспект}
\begin{enumerate}\setlength{\itemsep}{1pt}
\item Типовая конструкция суждения: \textbf{Посылки} $\vdash$ \textbf{Вывод}.
\item Пример:

\begin{tabular}{lr}
\begin{minipage}[b]{0.6\linewidth}
(\textit{Все птицы --- животные}) и (\textit{все воробьи --- птицы}),

вывод: (\textit{все воробьи --- животные}).

Такой вывод является правильным независимо от того, правильные ли посылки.

(\textit{Все птицы --- животные}) и (\textit{все цветы --- птицы}), вывод: (\textit{все цветы --- животные}).

Это суждение истинно независимо от ложности посылок. Суждение показывает \textit{только взаимосвязь} посылок и вывода. Принцип <<чушь на входе --- чушь на выходе>>.
\end{minipage}&  \includegraphics[scale=0.2]{piaf.png}
\end{tabular}
\item При построении суждения посылки могут быть ложными. Более того, в математической логике из ложной посылки следует все, что угодно. Например, \textit{если снег черный, то лес зеленый}. Лес при этом может быть зеленым (летом) и не быть таковым (зимой), но суждение остается истинным, т.к. посылка про снег является ложной.
\item Сравните: если запись числа $a$ оканчивается на 0, то оно кратно 5. Здесь мы ничего не знаем про число $a$, но если для него выполняется посылка, то выполняется и вывод. А если не выполняется, то истинность самого сужджения при этом никак не страдает. Более того, мы знаем, что на 5 также делятся и другие числа, и это значит, что путать местами посылки и вывод ни в коем случае нельзя! Ведь \textbf{не всегда верно}, что если число делится на 5, то его запись заканчивается на 0.
\item Другой~пример:\nopagebreak

\begin{tabular}{lr}
\begin{minipage}[b]{0.5\linewidth}
(\textit{Некоторые французы --- блондины}) и (\textit{некоторые ученики --- французы}), следовательно, (\textit{некоторые ученики --- блондины}). \textbf{Такое суждение неверно}. Поскольку слово <<некоторые>> не гарантирует, что таковым признаком обладают все французы. А значит, из свойства <<быть французом>> не всегда следует <<быть блондином>>.
\end{minipage}& \includegraphics[scale=0.19]{some.png}
\end{tabular}
\item Здесь обе посылки истинные, но вывод ложный. Хотя легко представить ситуацию, когда некоторые ученики действительно будут блондинами. Но это --- лишь предположение, а не строгое рассуждение.
\item В этом примере нарушается именно связка между посылками и выводом, т.к. две посылки не склеиваются по общему признаку. В первой посылке стоит <<некоторые французы>>, а во второй просто <<французы>>, это разные \textbf{множества}, а потому связать две посылки вместе мы не можем!
\item Для построения \textbf{силлогизма} принципиально, чтобы связующее звено было одинаковым:

\begin{center}
\textbf{если} ($A$ есть $B$) и ($B$ есть $C$), \textbf{то} ($A$ есть $C$)
\end{center}
здесь связывание посылок происходит по свойству $B$, и если в нем допустить какое-то искажение, то можно придти к неверным выводам!
\end{enumerate}
\subsection*{Задачи}
\begin{enumerate}
\item Постройте вывод из посылок: (Сократ человек) И (все люди смертны).
\end{enumerate}


\section{Высказывания и предикаты}

\subsection*{Конспект}
\begin{enumerate}\setlength{\itemsep}{1pt}
\item \textbf{Высказывание} --- это любое утверждение на любом языке, которое может быть либо только истинным, либо только ложным.
\item Примеры высказываний: <<\textit{Шесть больше трех}>>, <<\textit{Дважды два --- пять}>>, <<$\sqrt 2$ --- \textit{число иррациональное}>>, <<\textit{среди натуральных чисел существует наибольшее}>>, <<\textit{всякое четное число является суммой двух простых чисел}>>.
\item Все эти высказывания имеют либо истинное, либо ложное значение, хотя про последнее мы не знаем точный ответ. Но мы точно знаем, что их значения не могут быть переменными, т.е. зависеть от каких-то внешних факторов или других высказываний.
\item Из выше приведенных примеров: <<\textit{все птицы --- животные}>> и <<\textit{все воробьи --- птицы}>> есть истинные высказывания.
\item Но эти высказывания можно разобрать на составляющие. Для чего нам понадобятся предикаты.
\item \textbf{Предикат} --- это суждение, зависящее от переменных, обозначающих объекты данного суждения.
\item Например, <<$x$ \textit{есть воробей}>>, <<$x$ \textit{есть птица}>>, <<$x$ \textit{есть животное}>>. Каждое из них может быть истииным или ложным, смотря что подставить вместо $x$. При $x=$ <<\textit{рыба}>> первые два будут ложными, а при $x=$ <<\textit{ромашка}>> ложными будут все три предиката.
\item Аналогично, <<$x$ \textit{есть ученик}>>, <<$x$ \textit{есть француз}>>, <<$x$ \textit{есть блондин}>>. Заметим, что если ранее мы оперировали \textbf{свойствами} (быть учеником, французом, блондином), то теперь перешли к оперированию \textbf{объектом} $x$, который может обладать тем или иным свойством.
\item Из предикатов можно построить новые предикаты, используя логические связки: И($\land$), ИЛИ($\lor$), НЕ($\neg$), СЛЕДУЕТ($\to$).
\item Например, <<($x$ \textit{есть воробей})$\to$($x$ \textit{есть птица}>>), <<($x$ \textit{есть птица})$\to$($x$ \textit{есть животное}>>). Эти предикаты содержат переменную $x$, но они всегда истинны. Такие тождественно истинные предикаты называются \textbf{тавтологиями}. Тавтологии отличаются от истинных высказываний тем, что содержат переменные, которые можно считать фиктивными. Чтобы тавтологию сделать высказыванием, достаточно перед ним сказать <<\textit{для любого} $x$>>, тогда $x$ перестанет быть параметром, а выражение превратится в истинное высказывание:

\begin{center}
<<(\textit{для любого} $x$) ($x$ \textit{есть воробей})$\to$($x$ \textit{есть птица})>>
\end{center}
\item Это называется правилом введения \textbf{квантора всеобщности}.
\item Далее, рассмотрим высказывание <<\textit{некоторые французы блондины}>>. Поступить аналогично предыдущему и заменить его на  предикат <<($x$ \textit{есть француз})$\to$($x$ \textit{есть блондин})>> нельзя! Дело в том, что высказывание <<\textit{все воробьи --- птицы}>> говорит о вложении одного свойства в другое: быть воробьем означает также быть птицей. Но при слове <<\textit{некоторые}>> мы понимаем, что речь идет не о свойстве <<\textit{быть французом}>>, а о том, что некоторые из французов обладают свойством <<\textit{быть блондином}>>. То есть, мы утверждаем, что существует хотя бы один такой объект $x$, который есть и француз и блондин одновременно!
\item Иначе говоря, мы имеем дело со связкой И:
\begin{center}
($x$ \textit{есть француз})$\land$($x$ \textit{есть блондин}),
\end{center}
данный предикат не всегда является истиной, его истинность зависит от конкретного $x$.
\item Тем не менее, и такой предикат можно превратить в высказывание, причем истинное. Для этого нужно слово <<некоторые>> превратить в <<\textit{существует} $x$>>, так что получится истинное высказывание
\begin{center}
<<(\textit{существует} $x$) ($x$ \textit{есть француз})$\land$($x$ \textit{есть блондин})>>
\end{center}
\item Это называется правилом введения \textbf{квантора существования}.
\item Примеры перевода высказываний с языка свойств на язык объектов:\hfill\;

\begin{tabular}{p{0.4\linewidth}p{0.5\linewidth}}\hline
\textit{Все птицы --- животные}  & (\textit{для любого} $x$) ($x$ \textit{есть птица})$\to$($x$ \textit{есть животное}) \\\hline
\textit{Все воробьи --- птицы}  & (\textit{для любого} $x$) ($x$ \textit{есть воробей})$\to$($x$ \textit{есть птица}) \\\hline
\textit{Все воробьи --- животные}  & (\textit{для любого} $x$) ($x$ \textit{есть воробей})$\to$($x$ \textit{есть животное}) \\\hline
Если число заканчивается на 0, то оно кратно 5 &
(\textit{для любого} $a$) ($a$ \textit{заканчивается на $0$})$\to$($a$ \textit{кратно $5$}) \\\hline
Некоторые французы --- блондины 
& (\textit{существует} $x$) ($x$ \textit{есть француз})$\land$($x$ \textit{есть блондин}) \\\hline
Некоторые ученики --- французы 
& (\textit{существует} $x$) ($x$ \textit{есть ученик})$\land$($x$ \textit{есть француз}) \\\hline
Некоторые ученики --- блондины 
& (\textit{существует} $x$) ($x$ \textit{есть ученик})$\land$($x$ \textit{есть блондин}) \\\hline
\end{tabular}

\item Видим, что построить вывод можно только в том случае, когда две посылки склеиваются по общему предикату <<$x$\textit{ есть птица}>>, при этом сами посылки являются импликациями (следование).
\item Можно комбинировать общие и частные суждения:
\begin{center}
<<($x$ \textit{есть птица})$\land$(\textit{все птицы --- животные})>>,
\end{center}
откуда следует вывод <<($x$ \textit{есть животное})>>.

Здесь мы объединили в посылке предикат, что-то говорящий о свойстве объекта $x$, с высказыванием, которое что-то говорит о связи двух свойств, и нашли новое свойство объекта $x$. Это типичное рассуждение от общего к частному.
\item Построение выводов из заданных или полученных ранее истинных высказываний и предикатов называется \textbf{дедукцией} и является основным методом рассуждений при получении математических теорем.
\item Иногда для построения нужного вывода требуетя перебрать сотни комбинаций ранее доказанных посылок. Но часто для нащупывания правильной цепочки доказательства хватает вспомогательных иллюстраций или опыта исследователя, погруженного в данную тему.
\item Ранее мы отмечали, что рассуждения в обратную сторону --- от вывода к посылкам --- неверны. Однако очень часто это верно отчасти. Например, мы знаем дедуктивный вывод: если число оканчивается на 0, то оно делится на 5. На основе этого мы не можем доказать точно, но \textbf{можем предположить}, что если число делится на 5, то оно, вероятно, может оканчиваться на 0. Как мы знаем, это верно примерно в половине случаев. Если бы такое \textit{разворачивание импликации} было бы всегда абсолютно невозможным, то дедукция представляла бы собой простейший случай вывода, когда ложь влечет любое суждение. Для построения теорий это абсолютно бесполезно.
\item Метод \textit{рассуждения назад}, к уже известной посылке, называется \textbf{абдукцией}. Именно таким методом, как правило, пользовался Шерлок Холмс в своих умозаключениях. Именно поэтому его выводы всегда носят вероятностный характер и сопровождаются словами <<вероятно>>, <<скорее всего>> и т.п. Искусство Шерлока Холмса заключается в том, чтобы из всех возможных посылок в данной конкретной ситуации выбрать наиболее вероятную.
\item Например, цитируем из рассказа <<Этюд в багровых тонах>> (Конан Дойль),

\textit{<<Этот человек по типу --- врач, но выправка у него военная. Значит, военный врач. Он только что приехал из тропиков --- лицо у него смуглое, но это не природный оттенок его кожи, так как запястья у него гораздо белее. Лицо изможденное, --- очевидно, немало натерпелся и перенес болезнь. Был ранен в левую руку --- держит ее неподвижно и немножко неестественно. Где же под тропиками военный врач-англичанин мог натерпеться лишений и получить рану? Конечно же, в Афганистане>>. Весь ход мыслей не занял и секунды. И вот я сказал, что вы приехали из Афганистана.}

\item Рассмотрим только часть умозаключений Холмса и сравним их с арифметическим примером\hfill\;

\begin{tabular}{p{0.55\linewidth}p{0.4\linewidth}}\hline
Ватсон --- военный врач с изможденным лицом и загорелый & 
Число 30 --- делится на 5 \\
Воевавшие в Афганистане --- военные с изможденным лицом и загорелые &
Оканчивающее на 0 число --- делится на 5 \\\hline

Вывод: Ватсон прибыл из Афганистана & Вывод: 30 оканчивается на 0\\\hline
\end{tabular}

\item Как видим, нам дано две посылки, в одной из которых дается некая связь между свойствами (воевашие есть военные и т.д., а также оканчивающиеся на 0 делятся на 5), а в другой дается свойство конкретного объекта (Ватсон и число 30). Это свойство общее в обеих посылках, но по нему нельзя склеить их в силлогизм, т.к. свойство всегда стоит в конце посылки. Но Холмс знает, что практически все военные с изможденным лицом и загорелые --- это воевашие в Афганистане (хотя это и неверно на 100\%), и на основании этого он предполагает(!), что и Ватсон такой же, раз он обладает таким же свойством.

\item На примере числа 30 это тоже сработало, однако стоит нам подставить 25 вместо 30, как вся цепочка рассуждений порушится! Поэтому абдуктивные умозаключения нельзя считать математическими, однако они могут навести на правильное дедуктивное умозаключение, в результате чего либо появляется теорема (\textit{Все военные с изможденным лицом воеали в Афганистане}), либо обнаруживается контрпример (в нашем случае это число 25, которое опровергает предположение о том, что все делящиеся на 5 числа оканчаиваются на 0).
\end{enumerate}

\subsection*{Задачи}
\begin{enumerate}
\item Какое абдуктивное предположение можно сделать из следующих посылок: (Зимой выпадает снег) И (Сейчас есть снег) ?
\end{enumerate}


\section{Связь предикатов и множеств}

\subsection*{Конспект}
\begin{enumerate}
\item Выше мы оперировали такими понятиями как свойство и объект, обладающий свойством, на основе чего вводили различные высказывания и предикаты. Посмотрим, как они связаны с понятием \textbf{множество}.
\item Пусть $M$ --- множество всех людей, живущих на планете. Тогда предикат $h(x)$ <<$x$ есть человек>> можно переписать следующим способом: $h(x)=(x\in M)$. Это одновременно означает и то, что $x$ находится в множестве $M$, и то, что $x$ обладает свойством <<быть человеком>>. Говорят также, что $M$ есть область истинности предиката $h(x)$. Таким образом, множество олицетворяет собой свойство, а элементы множества --- объекты, обладающие данным свойством.
\item Если множество $X$ является частью множества $Y$, (например, множество всех женщин есть часть множества $M$), то мы пишем $X\subseteq Y$ ($X$ \textit{содержится в} $Y$, $Y$ \textit{включает} $X$). Важно не путать значки $\in$ и $\subseteq$, т.к. первй говорит о принадлежности объекта к свойству, а второй --- о вложении свойств (о том, что одно свойство меньше или равно другому). Используется также символ строгого вложения $\subset$, означающий, что вложение имеется, но при этом множества не равны.
\item Вложение множеств выражается с помощью принадлежности:
$$
X\subseteq Y\mbox{ эквивалентно }(\forall x) (x\in X)\to(x\in Y)
$$
По сути, это ровно то же самое, что мы ранее делали при переводе языка свойств на язык объектов: \textit{все $X$ есть $Y$} равносильно высказыванию (\textit{для любого} $x$) ($x$ \textit{обладает свойством} $X$)$\to$($x$ \textit{обладает свойством} $Y$).
\item Обозначим далее: $p(x)$ предикат <<$x$ \textit{есть воробей}>>, $o(x)$ предикат <<$x$ \textit{есть птица}>>, $a(x)$ предикат <<$x$ \textit{есть животное}>>. Ранее мы получали следующий вывод:
$$
(\forall x) (p(x)\to o(x))\land (\forall x) (o(x)\to a(x))\vdash(\forall x) (p(x)\to a(x))
$$
\item Попробуем то же самое выразить множествами. Обозначим через $P$ область истинности предиката $p(x)$, т.е. множество всех воробьев, $O$ --- множество всех птиц, $A$ --- множество всех животных. Тогда написанный выше с помощью предикатов вывод можно записать на языке множеств так:
$$
(P\subseteq O\subseteq A) \vdash (P\subseteq A),
$$
поскольку все воробьи есть птицы, все птицы есть животные, а в итоге все воробьи есть животные.
\item На самом деле, существует намного более тесная связь между логическими связками и операциями над множествами.
Вернемся снова к картинке про французов, блондинов и учеников. На ней есть три множества, обозначенные соответствующими овалами. Обозначим их следующим способом:
$$
F = \{x\mid x\mbox{ --- француз}\},\quad B = \{x\mid x\mbox{ --- блондин}\},\quad E = \{x\mid x\mbox{ --- ученик}\}
$$
\item Здесь можно увидеть примеры \textbf{пересечений} множеств:
\begin{gather*}
F\cap B = \{x\mid (x\mbox{ --- француз})\land (x\mbox{ --- блондин})\},\\
F\cap E = \{x\mid (x\mbox{ --- француз})\land (x\mbox{ --- ученик})\},\\
E\cap B = \{x\mid (x\mbox{ --- ученик})\land (x\mbox{ --- блондин})\}.
\end{gather*}
Видим, что они соответствуют логической связке И соответствующих предикатов, выражающих свойства.
\item На той же схеме мы можем усмотреть и такие теоретико-множественные конструкции, как:
$$
F\setminus B = \{x\mid (x\mbox{ --- француз})\land \neg(x\mbox{ --- блондин})\},
$$
т.е. множество французов, не являющихся блондинами. $F\setminus B$ есть операция \textbf{вычитания} множеств.
\item Наконец, множество 
$$
F\cup E  = \{x\mid (x\mbox{ --- француз})\lor (x\mbox{ --- ученик})\}
$$
представляет собой свойство быть французом ИЛИ учеником. Оно содержит в себе как всех французов, так и всех учеников, причем среди них есть как французы, не являющиеся учениками, так и французы, являющиеся уничениками, а также ученики, не являющиеся французами. \textbf{Объединение} множеств соответствует логической связке ИЛИ.
\item Итак, мы можем легко оперировать предикатами, представляя, что они выражают свойство объекта принадлежать некоторому множеству, и наоборот, оперировать множествами, представляя, что оперируем предикатами, для которых эти множества суть область истинности. При этом И соответствует пересечению, ИЛИ --- обединению множеств. Отрицание соответствует вычитанию множеств, причем разность $X\setminus Y$ можно рассматривать как пересечение $X\cap(\neg Y)$. Наконец, вложение множеств соответствует импликации предикатов.
\end{enumerate}



\subsection*{Задачи}
\begin{enumerate}
\item Выразить свойство <<\textit{быть учеником и блондином одновременно}>> через множества $E$ и $B$.
\item Написать множество, соответствующее всем <<\textit{птицам, не являющимся воробьями}>> через множества $O$ и $P$.
\item Какие элементы содержит множество $P\setminus A$, множество $M\cap F$, множество $(F\cup B)\setminus (F\cap B)$?
\item Что выражает высказывание $(M\setminus F)\subseteq (M\setminus B)$?
\item Докажите: $(E\subseteq F)\vdash (M\setminus F)\subseteq (M\setminus E)$ (от противного).
\end{enumerate}


\section{Построение множеств}

\subsection*{Конспект}
\begin{enumerate}
\item Построение множеств прямо наследует из их связи с предикатами. Тем не менее, важно знать язык, позволяющмй компактно и наглядно записывать конструктивные примеры построения множеств.
\item Конечное множество, элементами которого являются объекты $a,b,\dots,z$ (их не обязательно 26, просто какой-то набор), обозначается
$$
\{a,b,\dots,z\},
$$
при этом неважно, в каком порядке записаны элементы внутри скобок, и есть ли там дубликаты. Если в списке один и тот же элемент повторяется несколько раз, то его дубли можно спокойно выбрасывать.\footnote{В математике существует понятие \textbf{мультимножество}, в котором как раз количество дубликатов имеет значение и называется кратностью элемента. Мультимножество удобно, например, для записи разложения числа по степеням простых.}
\item Примеры: $\{0\}$, $\{0,1\}$, $\{0,1,2,3\}$, $\{0,0,1,1,1\}$. Последнее множество равно множеству $\{0,1\}$ (убрали кратные вхождения). Еще пример: $\{\}$ --- \textbf{пустое множество}, обозначаемое также символом $\emptyset$.
\item Как мы уже видели ранее, множество можно задать в \textbf{предикативной форме}, общий вид которой такой:
$$
\{x\mid \ph(x)\},\quad \{f(x)\mid \ph(x)\},
$$
где $\ph(x)$ --- это предикат, выражающий свойство объекта $x$, а $f(x)$ --- некоторое преобразование объекта $x$ (функция). 

В первом случае даное множество является областью истинности предиката $\ph(x)$ и содержит в себе все элементы, и только их, для которых $\ph(x)$ истинно. Во втором случае множество содержит все значения функции $f(x)$, примененные к объектам из области истинности $\ph(x)$. Очевидно, что
$$
\{f(x)\mid \ph(x)\} = \{y\mid(y=f(x))\land\ph(x)\}
$$

\item Конечное множество в предикативной форме записывается так:
$$
\{a,b,\dots,z\} = \{x\mid (x=a)\lor(x=b)\lor\dots\lor(x=z)\},
$$
где предикат $\ph(x)=(x=a)\lor(x=b)\lor\dots\lor(x=z)$ выражает свойство $x$ входить в список объектов $a,b,\dots,z$.
\item Объединение (или сумма) множеств:
$$
A\cup B = \{x\mid (x\in A)\lor(x\in B)\},
$$
например, $\{a,b\}\cup\{b,c\}=\{a,b,c\}$.
\item Пересечение множеств:
$$
A\cap B = \{x\mid (x\in A)\land(x\in B)\},
$$
например, $\{a,b\}\cup\{b,c\}=\{b\}$.
\item Разность множеств:
$$
A\setminus B = \{x\mid (x\in A)\land(x\notin B)\},
$$
например, $\{a,b\}\setminus\{b,c\}=\{a\}$. Заметим, что $A\setminus B$ не всегда равно $B\setminus A$.
\item Если элементы множеств --- это числа, то с ними можно производить арифметические операции:
$$
A+B = \{x+y\mid (x\in A)\land(y\in B)\}, \quad kA = \{kx\mid x\in A\},
$$
здесь первое множество --- это сумма по Минковскому двух множеств, оно содержит все возможные суммы $x+y$, где первое слагаемое берется из первого множества, второе --- из второго.

Легко видеть также, что $A+\emptyset=\emptyset$, т.к. предикат $y\in B$ в случае $B=\emptyset$ тождественно ложный.

\textbf{Важно}: не следует путать $A+A$ и $2A$! Например,
$$
\{0,1\}+\{0,1\}=\{0,1,2\},\mbox{ но }2\{0,1\}=\{0,2\}.
$$

\item Аналогично можно определить произведение множеств по Минковскому:
$$
AB = \{xy\mid (x\in A)\land(y\in B)\},
$$
откуда легко определяется степень множества $A^k$, а также его экспонента $\exp(A)=\sum_k(1/k!)A^k$.

Аналогично сумме видим, что $A\emptyset=\emptyset$.
\end{enumerate}

\subsection*{Задачи}
\begin{enumerate}
\item Найти объединение, пересечение и разность множеств $\{0,1,2,3\}$ и $\{1,2,5\}$ (разность как в прямом, так и в обратном порядке).
\item Записать множество $\{0,1,2\}$ в предикативной форме.
\item Записать множество всех простых чисел в предикативной форме.
\item Доказать, что $A+\{0\}=A$, $A\cdot\{1\}=A$.
\item ${}^{**}$Когда $A\setminus B=B\setminus A$?
\item ${}^{***}$Доказать, что $\max\exp(\{0,x\}) = e^x$.
\end{enumerate}


\chapter{Визуальная арифметика}

\vrezka{В данной главе закладывается фундамент арифметики с помощью визуальных образов. Действия с отрезками и прямоугольниками являются иллюстрацией действий с числами. Цель --- дать наглядное обоснование законам арифметики и получить некоторые навыки арифметических операций и сравнений чисел.

Попутно вводится понятие натурального числа как количества применяемых операций композиции, а также как меры длины, площади, объема относительно заданной мерной единицы.
}
\section{Сложение и вычитание}

\subsection*{Конспект}
\begin{enumerate}\setlength{\itemsep}{1pt}
\item Берем произвольную прямую, и на ней будем откладывать отрезки --- вправо и влево.
\item Откладывание вправо есть прибавление длины, а откладывание влево --- вычитание (уменьшение) длины.
\item Можно откладывать ноль, т.е. ничего не делать. В этом случае все равно --- прибавляем или вычитаем ноль.
\item Мы можем комбинировать откладывание отрезков вправо и влево, т.е. производить серию последовательных откладываний отрезков (они могут быть разными по длине), на каждом шаге --- от текущей точки положения.
\item Результат \textit{серии откладываний} равносилен одному откладыванию отрезка, соединяющего стартовую и финишную точки, причем финишная точка:
\begin{itemize}
\item может быть справа от стартовой (результатом является одно откладывание вправо, т.е. прибавление длины),
\item может совпадать с ней (результатом оказалось нулевое откладывание)
\item или быть слева от стартовой точки (результатом является одно откладывание влево, т.е. вычитание).
\end{itemize}
\item Откладывание \textit{изотропно}, т.е. одинаковые серии откладываний, приложенные к разным стартовым точкам, приводят к одинаковым результирующим отрезкам, отложенным от этих стартовых точек. Иначе говоря, величина и направление откладывания не зависит от начального местоположения!
\end{enumerate}
\begin{tabular}{ll}
\begin{minipage}{0.6\linewidth}
\begin{enumerate}\setlength{\itemsep}{1pt}\setcounter{enumi}{6}
\item Серии откладываний можно проиллюстрировать складным метром. Раскладывание колена на $180^o$ означает прибавление его длины к общей серии откладываний, а складывание --- вычитание его длины из общей серии откладываний. При этом от стартовой точки можно уйти как вправо, так и влево, или остаться на месте.
\end{enumerate}
\end{minipage}
&
\begin{minipage}{0.4\linewidth}
%\begin{figure}[h]%[htb!]
%\vspace*{3mm}
%\begin{center}
\includegraphics[scale=0.3]{meter.png}
%\end{center}
%\label{meter}
%\end{figure}
\end{minipage}
\end{tabular}
\begin{enumerate}\setlength{\itemsep}{1pt}\setcounter{enumi}{7}
\item С помощью этой же линейки нетрудно продемонстрировать, что композиция откладываний \textbf{ассоциативна} и \textbf{коммутативна}: можно сначала сложить/разложить одну линейку, затем вторую, затем приложить вторую к первой или первую ко второй --- результат будет один и тот же!
\item Кроме того, очевидно, что у каждого откладывания существует обратное, приводящее в результате к нулевому откладыванию. Для этого нужно произвести ровно ту же самую серию откладываний, только поменять ось направления. Или, что то же самое. пройтиь по линейке в обратную сторону.
\item Далее любое откладывание будем записывать буквами $a,b,c,\dots$, имея ввиду под ними как прибавления, так и вычитания.
\item Откладывание, противоположное $a$, будем обозначать $-a$. При этом комбинация откладываний соединяется знаком '+', а если встречается комбинация $a+(-b)$, то пишем проще: $a-b$.
\item Обратные откладывания --- это просто перевернутые в обратную сторону <<линейки>>!
\item Результат откладывания (конфигурацию линейки с учтом ее направления) будем называть \textbf{вектором}. Если вектор смотрит влево (финишная точка левее стартовой), то вектор называется \textit{отрицательным}, а если вправо --- \textit{положительным}. Нулевой вектор --- когда финиш и старт совпадают.
\item Композицию откладываний будем называть \textbf{суммой векторов} или просто суммой, а процедуру откладывания --- \textbf{сложением}.
\end{enumerate}

\textbf{Свойства сложения}:
\begin{enumerate}[label=S\arabic*]
\item $(a+b)+c=a+(b+c)$ (ассоциативность);
\item $a+b=b+a$ (коммутативность);
\item $a+0=0+a=a$ (аддитивное свойство нуля);
\item $a+(-a)=0$ (обратный элемент);
\item если $a+x=b+x$, то $a=b$ (правило сокращения);
\item верно одно и только одно: либо $a=b$, либо $a=b+x$, либо $a=b-x$, где $x$ --- откладывание вправо (трихотомия)
\end{enumerate}
\subsection*{Задачи}
\begin{enumerate}
\item Вывести свойства сложения.
\end{enumerate}



\section{Сравнение}

\subsection*{Конспект}
\begin{enumerate}
\item Понятие отрицательного и положительного векторов позволяют ввести сравнение на векторах.
\item Для начала скажем, что положительный вектор больше нуля: $x>0$.
\item Далее, если $b=a+x$, где $x>0$, то пишем $a<b$.
\end{enumerate}

\textbf{Свойства сравнения} (можно вывести из определения):
\begin{enumerate}[label=O\arabic*]
\item не верно, что $x<x$ (антирефлексивность);
\item если $a<b$ и $b<c$, то $a<c$ (транзитивность);
\item верно одно и только одно: либо $a=b$, либо $a<b$, либо $b<a$ (трихотомия);
\item $a<b\Leftrightarrow a+x<b+x$, где $x>0$ (изотропность сравнения)
\end{enumerate}

\subsection*{Задачи}
\begin{enumerate}
\item Вывести свойства сравнения.
\end{enumerate}





\section{Умножение}

\subsection*{Конспект}
\begin{enumerate}
\item Строим две перпендикулярно направленные оси $Ox$ и $Oy$. На каждой оси --- свой собственный мир векторов и линеек.
\item Умножение --- это площадь, построенная на перпендикулярных векторах. Картинка $2\times 2=4$.
\item Поскольку векторы у нас двух знаков, умножение также бывает двух знаков.
Знак умножения определяется знаком (направлением) векторов и таблицей перемножения знаков:
\begin{table}[htb!]\begin{flushright}
\begin{tabular}{c|c|c|}
  & $+$ & $-$ \\
 \hline
$+$ & $+$ & $-$ \\
 \hline
$-$ & $-$ & $+$ \\
\hline
\end{tabular}
\end{flushright}\end{table}
\item Понятие группы на данном примере. Элемент '+' является нейтральным элементом группы знаков. Многократные умножения знаков не выводят за пределы группы.
\item Умножение коммутативно и ассоциативно --- можно продемонстрировать на картинках с квадратами и кубами.
\item Умножение на нулевой отрезок (мультипликативное свойство нуля) --- очевидно из равенства и свойств сложения:
$$
0 + a\times 0 = a\times 0 = a\times (0+0) = (a\times 0) + (a\times 0)\Rightarrow 0 = (a\times 0)
$$
\item Дистрибутивный закон, в том числе при разнонаправленных векторах проверяется непосредственно на картинке: $a\times (b+c)=a\times b+a\times c$.
\item \textbf{Единичный отрезок} --- способ свести многократное сложение одного вектора к умножению на сумму единичных отрезков! Прямоугольник единичной высоты и длины $an$ перекладывается в прямоугольник $a\times n$, тем самым сложение превращается в умножение.
\item Умножение на единичный отрезок: $a\times 1=a$.
\item Сложение отрезков --- это также сложение прямоугольников единичной высоты.
\item Умножение отрезков --- это не только площадь, но также и объем, который заметает вертикальный единичный отрезок на площади $a\times b$, поэтому $ab=a\times b\times 1$.
\item \textit{Степень}: многократное умножение отрезка самого на себя. Иллюстрация --- отрезок, квадрат, куб.
\item В дальнейшем умножение векторов в смысле нахождения площади/объема, т.е. $a\times b$, и умножение чисел как таковых, т.е. $ab$, будем считать одним и тем же понятием, так что $a\times b=ab$.
\end{enumerate}

\textbf{Свойства умножения}:
\begin{enumerate}[label=P\arabic*]
\item $(a\times b)\times c = a\times (b\times c)$ (ассоциативность);
\item $a\times b=b\times a$ (коммутативность);
\item $a\times 0=0\times a=0$ (мультипликативное свойство нуля);
\item $a\times 1=1\times a=a$ (нейтральный элемент по умножению);
\item $a\times(b+c)=a\times b+a\times c$ (дистрибутивный закон);
\item если $a\times b=0$, то $a=0$ или $b=0$ (отсутствие делителей нуля);
\item если $a\times c=b\times c$ и $c\ne 0$, то $a=b$ (правило сокращения);
\item если $a\times c<b\times c$, то $a<b$ (монотонность);
\item если $a<b$ и $c>0$, то $a\times c<b\times c$.
\end{enumerate}


\subsection*{Задачи}
\begin{enumerate}
\item Вывести свойства умножения.
\end{enumerate}

\section{Натуральные числа}

\subsection*{Конспект}
\begin{enumerate}
\item Кратность операций сложения и умножения: $a+a+a+a+a+\dots$, $a a a\ldots$ Натуральное число вводится для обозначения кратности одинаковых операций!
\item Нулевая кратность: в случае сложения ничего не складываем, остаемся на месте в начальной точке, поэтому
$$
\underbrace{a+\dots+a}_{0\mbox{ раз}}=0.
$$
\item Нулевая степень: в случае умножения ничего не умножаем, от умножения остается только кратность 1, наследуемая от сложения, т.е. в произведении $1\times a\times a\times\dots$ выбрасываем все, остается только 1. Поэтому
$$
\underbrace{a\times\dots\times a}_{0\mbox{ раз}}=1,
$$
кроме того, это согласуется с законом ассоциативности умножения. Многие правила в математике для крайних значений определяются с целью сохранить общий вид формул, если это не приводит к противоречию!
\item \textbf{Натуральные числа} --- это показатели кратности операций (сложения и умножения).
\item С другой стороны, натуральные числа можно рассматривать как суммы единичных отрезков.
$$
n=\underbrace{1+1+\dots+1}_{n\mbox{ раз}}
$$
\item Чудо, но это вполне согласуется с операциями сложения и умножения, сохраняет все законы арифметики: ассоциативность, коммутативность, дистрибутивность.
\item Поэтому натуральные числа, привязанные к единичным отрезкам, можно также считать мерой длины, площади, объема и т.д.
\item Ноль --- натуральное число, поскольку мы рассматриваем нулевую кратность для однородности законов арифметики.
\item[NotaBene] Натуральные числа --- это и кратности операций, и единицы измерения, т.е. числа.
\item Натуральные числа отвечают за соизмеримость и арифметическую кратность: $a$ \textbf{кратно} $b$ ($a\mathop{\vdots} b$), если $a=bn$ или $a=(-b)n$ при некотором натуральном $n$. Ноль кратен любому числу! Нулю кратен только ноль!
\item Если $a$ кратно $b$, то говорят также, что $b$ делит $a$, или что $b$ является делителем $a$ ($b|a$).
\item Если $a>0$ кратно $b>0$, то $a=kb=b+(k-1)b$, где $k>0$. Здесь $x=(k-1)b$. Поэтому $a\ge b$. Так что для положительных векторов кратность означает превосходство в смысле сравнения. И наоборот, если $b$ делит $a$, то $b\le a$. Аналогичные неравенства можно получить и для отрицательных векторов.
\end{enumerate}
\subsection*{Задачи}
\begin{enumerate}
\item Доказать, что если $a|b$ и $b|c$, то $a|c$.
\item Доказать, что если $a|b$ и $b|a$, то $a=\pm b$ ($a,b$ --- натуральные).
\end{enumerate}


\section{Теорема Пифагора графически}

\subsection*{Конспект}
\begin{enumerate}
\item Строим квадрат $a+b\times a+b$ и внутри квадраты $a\times a$ и $b\times b$
\item Строим квдарат $a+b\times a+b$ и внутри квадрат $c\times c$
\item Делаем вывод, перекладывая треугольники
\item *Построение $\sqrt 2$, $\sqrt 7$ (используются признаки подобия треугольников, отношения строн)
\item Примеры пифагоровых троек (анонс теоремы!)
\end{enumerate}

\section{Бином Ньютона и другие формулы визуально}

\subsection*{Конспект}
\begin{enumerate}\setlength{\itemsep}{1pt}
\item Визуализация $(a-b)(a+b)=a^2-b^2$.
\item Сумма подряд идущих чисел $1,2,\dots,n$ с помощью сложения прямоугольников.
\item Сумма подряд идущих нечетных чисел.
\item Вывод формулы $(a+b)^3 = a^3+3a^2b+3ab^2+b^3$.
\item Разрезание сырного кубика на 8 частей тремя плоскостями.
\end{enumerate}
\subsection*{Задачи}
\begin{enumerate}
\item Вывести формулу квадрата суммы визуально.
\end{enumerate}


\section{Соизмеримость отрезков, алгоритм Евклида}

\subsection*{Конспект}
\begin{enumerate}\setlength{\itemsep}{1pt}
\item Два отрезка $a$ и $b$, кузнечики прыгают, один на $a$ и $-a$ сколько угодно раз, второй на $b$ и $-b$ сколько угодно раз
\item Кузнечики стартуют в одной и той же точке (назовем ее $O$). Могут ли они попасть в одну точку, отличную от $O$, когда-нибудь?
\item Ответ --- да, если есть такая точка $A$, что отрезок $OA$ кратен и $a$, и $b$ одновременно, т.е. при некотрых натуральных $n,m$, не равных нулю, будет верно равенство $an=bm$:
$$
\underbrace{a+a+\dots+a}_{n\mbox{ раз}}=\underbrace{b+b+\dots+b}_{m\mbox{ раз}}
$$
\item Отрезки, которые имеют общий кратный отрезок, называются \textbf{соизмеримыми}
\item Иллюстрация: строим прямоугольник $a\times b$ ($a<b$), начинаем отсекать в нем квадраты: сначала отсекаем квадраты $a\times a$, пока можем, останется кусок $a\times b_1$ ($b_1<a$), затем отсекаем квадраты $b_1\times b_1$, пока можем, останется кусок $a_1\times b_1$ ($a_1<b_1$), и т.д.
\item Если исходные отрезки соизмеримы, то процесс остановится: исходный прямоугольник будет разбит на конечное число квадратиков.
\item Финальный квадратик будет иллюстрировать НОД отрезков $a$ и $b$, т.к. это максимальный квадрат, которым можно замостить. прямоугольник $a\times b$.
\item Такой процесс называется \textbf{алгоритмом Евклида}, к нему мы еще вернемся с более формальной точки зрения.
\item Заметим, что числа $a$ и $b$ при этом вовсе не обязан быть натуральными.
\item Несоизмеримость стороны квадрата и его диагонали: 1 и $\sqrt 2$.
\item Алгоритм Евклида никогда не остановится. НОДом будет бесконечно малое число.
\end{enumerate}
\subsection*{Задачи}
\begin{enumerate}
\item Найти НОД(10,6) методом прямоугольников.
\item Сколько и каких шагов должен сделать кузнечик НОД(10,6), чтобы попасть в точку НОД(10,6)?
\end{enumerate}



\chapter{Движения прямой}

\vrezka{В этой главе мы переходим к более формальной работе с точками и векторами на прямой. Целью является знакомство с понятиями <<движение>>, <<композиция движений>>. Проводится полный анализ видов движений и свойств их композиций.

Попутно вводится понятие группы и подгруппы в приложении к группе движений на прямой. Изучаются все конечные подгруппы движений прямой.
}

\section{Сдвиг, композиция сдвигов, группа}

\subsection*{Конспект}
\begin{enumerate}\setlength{\itemsep}{1pt}
\item Рассмотрим аффинную прямую, т.е. набор точек и векторов на прямой.
\item Сумма точки и вектора есть точка, сумма векторов есть вектор, разность точек есть вектор.
\item Команда <<прибавить ко всем точкам вектор $a$>> называется \textbf{сдвигом} прямой на вектор $a$.
\item Сдвиг на $a$ --- это операция сложения с вектором без указания конкретной точки приложения, она применяется сразу ко всем точкам! В итоге вся прямая смещается как единое целое.
\item Сдвиг является движением (не случайно это однокоренные слова!)
\item Вообще, \textbf{движение --- это преобразование, сохраняющее расстояния} (размеры и форму): если между точками $A$ и $B$ было расстояние $x$, то после преобразования движения расстояние между точками $A'$ и $B'$, в которые перешли исходные точки, тоже будет $x$, и так для любой пары точек!
\item Математическое движение --- это результат физического движения (есть только начальное и конечное состояние системы).
\item Сдвиг на вектор $a$ будем обозначать $T_a$: $T_a(A)$ --- это точка $B$ такая, что $AB$ есть вектор $a$ (совпадает по направлению и длине).
\item Композиция сдвигов --- это их последовательное применение: $$(T_b\circ T_a)(A)=T_b(T_a(A))$$.
\item Композиция сдвигов соответствует сумме векторов: $T_b\circ T_a=T_{a+b}$.
\item Композиция сдвигов перестановочна в силу коммутативности сложения: $$T_b\circ T_a=T_a\circ T_b$$.
\item Кратность сдвига обозначается как степень
$$
\underbrace{T_a\circ\dots\circ T_a}_{n\mbox{ раз}}=T_a^n
$$
и соответствует кратности сложения или умножению на степень кратности: $T_a^n=T_{an}$.
\item Нулевой сдвиг $T_0=\id$ --- это \textbf{тождественное преобразование}, которое ничего не меняет.
\item Обратный сдвиг $T_a^{-1}$ --- это сдвиг на вектор $-a$, т.е. сдвиг в обратном направлении на ту же величину.
\item Вообще, если есть какие-то два преобразования $u$ и $v$ и операция композиции $\circ$, то эти преобразования. \textbf{взаимно обратны}, если $u\circ v=\id$ и $v\circ u=\id$, т.е. последовательное применение этих преобразований является тождественным преобразованием.
\item Очевидно, что всякий сдвиг имеет обратный, причем $T_a\circ T_a^{-1}=T_a^{-1}\circ T_a=\id$.
\item Нулевой сдвиг сам себе обратен.
\item Обобщая свойства сдвигов, фиксируем понятие \textbf{группы}. Это --- множество $G$ с одной бинарной операцией $\circ$, для которой выполняются законы:
\begin{enumerate}[{\bf G}1)]
\item Результат групповой операции снова лежит в этом же множестве (например, композиция сдвигов есть сдвиг):
$$
u,v\in G \Rightarrow u\circ v\in G.
$$
\item Групповая операция \textbf{ассоциативна} (сочетательный закон): для любых элементов $u,v,w$ группы $G$
$$
(u\circ v)\circ w = u\circ (v\circ w)
$$
(например, $(T_a\circ T_b)\circ T_c=T_a\circ(T_b\circ T_c)$.
\item Существует \textbf{нейтральный элемент} $\id$ такой, что для любого элемента $u$ имеет место равенство
$$
u\circ\id = u = \id\circ u.
$$
\item Групповая операция \textbf{обратима}: для всякого элемента $u$ существует обратный ему элемент $v$ такой, что
$$
u\circ v=\id = v\circ u
$$
(например, обратный сдвиг --- это сдвиг в противоположную сторону:  $T_{a}^{-1}=T_{-a}$). Элемент $v$ в таком случае обозначается как $u^{-1}$ и называется \textbf{обратным} к элементу $u$.
\end{enumerate}
\item Множество всех сдвигов образует группу относительно операции композиции!
\item Мало того, группа сдвигов \textbf{коммутативна} (абелева), т.е. для ее групповой операции выполняется переместительный закон:
\begin{enumerate}[resume*]
\item $u\circ v=v\circ u$ для всех $u,v$ из группы $G$.
\end{enumerate}
\item Кратность обратного сдвига: $T_a^{-n}\rightleftharpoons (T_a^{-1})^n=T_{-a}^n=T_{-an}$
\item На основе только одного сдвига $T_a$ можно построить подгруппу сдвигов
$$
\langle T_a\rangle = \{T_a^n, T_a^{-n}\mid n=0,1,2,\dots\}
$$
\item Эта подгруппа --- реализация целых чисел $\Z$, к которым мы еще вернемся позже.
\item Фиксируем понятие \textbf{подгруппы}. Это --- подмножество группы, на котором групповая операция удовлетворяет групповым законам, т.е. подгруппа сама является группой с той же операцией, которая задана в группе.
\item Каждый сдвиг $T_a$ порождает (с помощью его многократного тиражирования) свою подгруппу в группе всех сдвигов.
\end{enumerate}



\section{Отражение}

\subsection*{Конспект}
\begin{enumerate}\setlength{\itemsep}{1pt}
\item Еще один вид движений прямой --- \textbf{отражение}
\item Отражение связано с выделенной точкой --- центром отражения, и все точки переводит в симметричные относительно данного центра. Взяли прямую и перевернули ее на $180^o$, оставляя центр отраженя на месте
\item Отражение с центром $O$ будем обозначать $S_O$
\item Композиция отражений: $$S_O\circ S_C=T_{2CO},\quad S_C\circ S_O=T_{2OC}$$
\item Видим, что композиция отражений является сдвигом и при этом не коммутативна!
\item Композиция отражения и сдвига: $$S_O\circ T_a = S_{O-a/2},\quad T_a\circ S_O = S_{O+a/2}$$
\item Такая композиция является отражением и при этом не коммутативна!
\item Таблица композиций отражений и сдвигов:
\begin{center}
\begin{tabular}{c|c|c|}
  & $T_a$ & $S_O$ \\
 \hline
$T_b$ & $T_{a+b}$ & $S_{O+b/2}$ \\
 \hline
$S_C$ & $S_{C-a/2}$ & $T_{2OC}$ \\
\hline
\end{tabular}
\end{center}
\item Кратность отражения $S_O^n$ определяется четностью числа $n$. В случае четного $n$ это $\id$, в случае нечетного --- исходное $S_O$
\item Отражение обратно самому себе: $S_O\circ S_O=\id$
\item Пара $\{\id, S_O\}$ образует самую маленькую нетривиальную группу движений, которая к тому же является абелевой и циклической (т.е. все ее элементы есть степени какого-то одного, а именно $S_O=S_O^1$, $\id=S_O^2$)
\begin{table}[htb!]\begin{center}
\begin{tabular}{c|c|c|}
  & $\id$ & $S_O$ \\
 \hline
$\id$ & $\id$ & $S_O$ \\
 \hline
$S_O$ & $S_O$ & $\id$ \\
\hline
\end{tabular}
\end{center}\end{table}
\item Видим, что таблица полностью повторяет таблицу умножения знаков, причем $\id$ является нейтральным элементом.
\item Суммируя, находим, что вообще все сдвиги и отражения вместе образуют группу (относительно операции композиции), т.е. для них выполняются аксиомы группы G1--G4. При этом данная группа не является абелевой (не выполняется G5), поскольку, как мы видели, далеко не все композиции движений перестановочны.
\end{enumerate}



\section{Таблица Кэли движений прямой}

\subsection*{Конспект}
\begin{enumerate}\setlength{\itemsep}{1pt}
\item Еще пример группы: рассмотрим класс всех сдвигов $\T$ и класс всех отражений $\S$
\item Мы можем определить композицию классов $\T\circ \T$, $\T\circ \S$, $\S\circ \T$ и $\S\circ \S$ как все возможные композиции движений из этих классов в указанном порядке. Иначе говоря, композиции классов --- это их умножение по Минковскому:
$$
\T\circ \T = \{t\circ t'\mid (t\in\T)\land(t'\in\T)\},\quad \T\circ \S = \{t\circ s\mid (t\in\T)\land(s\in\S)\}
$$
$$
\S\circ \T = \{s\circ t\mid (s\in\S)\land(t\in\T)\},\quad \S\circ \S = \{s\circ s'\mid (s\in\S)\land(s'\in\S)\}
$$
\item Из произведенных выше вычислений легко видеть таблицу композиций этих классов:
\begin{center}
\begin{tabular}{c|c|c|}
  & $\T$ & $\S$ \\
 \hline
$\T$ & $\T$ & $\S$ \\
 \hline
$\S$ & $\S$ & $\T$ \\
\hline
\end{tabular}
\end{center}
\item Видим полную аналогию с таблицей знаков и таблицей для $\id, S_O$. Здесь класс $\T$ является нейтральным элементом
\item Если теперь собрать в одну кучу все сдвиги и отражения, то получим группу движений прямой
\item Наша цель --- доказать, что других движений нет, т.е. что мнжество $\{T_a,S_O\}_{a,O}$ полностью исчерпывает все возможные движения прямой
\end{enumerate}

\subsection*{Задачи}

Пусть на прямой даны 4 точки $A,B,C,D$, поставленные друг за другом с одинаковым шагом (см.рис).
\begin{center}
\includegraphics[scale=0.7]{ABCD.png}
\end{center}

\begin{enumerate}
\item Куда перейдет точка $A$ при преобразовании $S_B$?
\item Куда перейдут точки $B,C,D$ при преобразовании $T_{AB}\circ T_{CA}$?
\item Куда перейдут точки $A,B,C$ при преобразовании $S_C\circ T_{AB}$?
\end{enumerate}



\section{Теорема о гвоздях, аналог теоремы Шаля}

\subsection*{Конспект}
\begin{enumerate}
\item Анализ движений проводится на основе наблюдений за количеством стационарных точек
\item Пусть движение $M$ таково, что оно оставляет на месте две точки $A\ne B$.
\item $M(A)=A$ и $M(B)=B$. Пусть $C'=M(C)$. $M$ сохраняет расстояния $AC$ и $BC$, откуда $AC=AC'$ и $BC=BC'$, откуда $C=C'$. Т.е. $M(C)=C$ для любых точек $C$, т.е. $M=\id$

\begin{center}
\includegraphics[scale=0.35]{LineMoving.png}
\end{center}
\item Пусть движение $M$ оставляет на месте ровно одну точку $O$. В этом случае $A'=M(A)$ и $A\ne A'$ и $OA=OA'$, тогда $A'$ --- отражение $A$ относительно $O$. Следовательно, $M=S_O$
\begin{center}
\includegraphics[scale=0.35]{LineMovingO.png}
\end{center}
\item Пусть движение $M$ не оставляет на месте ни одной точки и пусть $B=M(A)$ ($B\ne A$). Обозначим $x=AB$. Тогда $T_{x}^{-1}\circ M(A)=A$, т.е. $T_{x}^{-1}\circ M$ оставляет на месте хотя бы одну точку $A$. Если оно оставляет на месте ровно одну точку $A$, то это некоторая симметрия $S_A$, но тогда $M=T_x\circ S_A=S_{A+x/2}$. Получается, что $M$ сохраняет точку $A+x/2$ на месте. Противоречие. Остается вариант, что $T_{x}^{-1}\circ M$ оставляет на месте как минимум две точки, но тогда $T_{x}^{-1}\circ M=\id$, откуда $M=T_x\circ \id=T_x$ --- сдвиг.
\begin{center}
\includegraphics[scale=0.35]{LineMovingx.png}
\end{center}
\item Таким образом, все движения прямой --- это либо сдвиги (в частности, $\id$), либо отражения (теорема Шаля)
\item При этом, любое движение --- это либо одна симметрия, либо композиция двух симметрий
\end{enumerate}
\subsection*{Задачи}
\begin{enumerate}
\item Построить сдвиг на 7 единиц вправо с помощью композиции двух симметрий.
\item Каким движением является следующая композиция?
$$
S_{O+n}\circ S_{O+n-1}\circ \dots\circ S_{O+1}\circ S_O.
$$
Ответ получить в зависимости от четности $n$.
\end{enumerate}


\section{Все конечные подгруппы движения прямой}


\subsection*{Конспект}
\begin{enumerate}
\item 
\end{enumerate}


\chapter{Вокруг окружности}

\vrezka{В этой главе мы расширяем сферу деятельности и переходим к движениям окружности. Снова изучаем виды движений, строим таблицу композиций, доказываем теорему Шаля.

Попутно сопоставляем движения окружности с движениями прямой, выходим на отрицательные степени композиций и их арифметические свойства, как следствие, получаем целые числа.

По аналогии с натуральными числами говорим о том, что целые числа --- это и степени композиций движений, и мера длины, только оснащенная знаком, т.е. направлением измерения длины.
}

\section{Движения окружности}

\subsection*{Конспект}
\begin{enumerate}\setlength{\itemsep}{1pt}
\item Берем окружность (обруч). Какие у нее есть движения, переводящие его в самого себя?
\item Прежде всего, повторим, что движение --- это преобразование, сохраняющее расстояния (изометрия). Поэтому, если мы говорим о движении, переводящем фигуру (прямую, круг, квадрат, многоугольник, плоскость и т.д.) в саму себя, то это значит, что мы берем копию этой фигуры и накладываем ее на оригинал до полного совмещения контуров. При этом допускается вертеть ее как угодно, лишь бы наложение фигур оказалось идеальным --- без выступов и впадин, без какой-либо деформации.
\item Для того, чтобы уточнить смысл определения движения, нужно зафиксировать способ измерения расстояний на окружности.  Расстоянием между точками окружности $A$ и $B$  будем называть длину меньшей из дуг, соединяющих эти точки.
\item Очевидно, что движениями окружности являются как минимум: вращение вокруг ее центра, а также симметрии относительно прямых, проходящих через ее центр.
\item В некотором смысле окружность --- аналог прямой. Только эту прямую взяли за 2 конца и замкнули где-то на бесконечности.
\item Поэтому вращение окружности соответствует сдвигу прямой, а симметрия окружности относительно прямой --- отражению на прямой относительно точки (можно считать ее симметрией относительно перпендикулярной прямой).
\item Если представить, что на окружности большого радиуса живут маленькие одномерные математики, то для них окружность будет практически не отличима от прямой, и движения окружности они будут воспринимать именно как движения прямой.
\item Поворот на угол $\al$ обозначим $R_\al$ (положительный --- против часовой стрелки), симметрию относительно прямой, имеющей угол наклона $\ph$, обозначим $S_\ph$ ($0\le\ph<180^o$). Угол наклона прямой измеряется от некоторого заданного раз и навсегда радиуса окружности, который можно считать точкой отсчета (аналог нуля на прямой).
\item Ось симметрии $S_\ph$ мы будем обозначать $l_\ph$ (см. рис.)

\begin{center}
\includegraphics[scale=0.2]{Rund3.png}
\end{center}

\item Вновь замечаем, что композиция поворотов есть поворот на суммарный угол: $R_\al\circ R_\be=R_{\al+\be}$
\item У каждого поворота есть обратный: $R_\al^{-1}=R_{-\al}$, т.н. поворот в противоположном направлении.
\item Повороты коммутируют: $R_\al\circ R_\be=R_\be\circ R_\al$.
\item Есть нейтральный поворот $\id=R_0$.
\item Так что все повороты образуют группу относительно операции композиции.
\item Тем не менее, есть одна особенность: поворот на угол $360^o k$ --- это тоже $\id$.
\item Вообще, повороты, заданные углами с шагом $360^o$, равны: $R_\al=R_{\al\pm 360^ok}$, где $k$ --- натуральное число.
\item Некоторые повороты дают $\id$ в некоторой степени, например, $R_{90^o}^4=\id$, $R_{60^o}^6=\id$ и т.д.
\item Если угол, выраженный в градусах, соизмерим с величиной $360^o$, то поворот на данный угол имеет положительную степень, в которой он обращается в $\id$.
\item Но есть угол, не обладающий таким свойством, это угол в 1 радиан. Если бы он был соизмерим с полным оборотом, то число $\pi$ оказалось бы соизмеримым с 1, а это не так! Доказательство этого факта является сложной математической теоремой!
\item В зависимости от соизмеримости угла поворота с полным оборотом некоторые повороты порождают конечные циклические подгруппы в группе движений, а некоторые --- нет.
\end{enumerate}


\section{Группа движений окружности, теорема Шаля}

\subsection*{Конспект}
\begin{enumerate}\setlength{\itemsep}{1pt}
\item Композиция симметрий: 
$$
S_\psi\circ S_\ph=R_{2(\psi-\ph)},\quad S_\ph\circ S_\psi=R_{2(\ph-\psi)}
$$
Этот факт легко увидеть из картинки, где точка $A$ переходит в $A'$ под действием симметрии $S_\psi$ относительно оси $l_\psi$, а затем $A'$ переходит в $A''$ под действием симметрии $S_\ph$ относительно оси $l_\ph$:

\begin{center}
\includegraphics[scale=0.25]{Rund.png}
\end{center}

Суммарный угол поворота точки $A$ при переходе в точку $A''$ можно разбить на 2 пары углов так, что в каждой паре углы равны в силу свойств симметрии (разные пары отмечены разным цветом), и в то же время угол между осями состояит как раз из суммы углов, принадлежащих разным парам. Нетрудно убедиться в аналогичном результате и в том случае, если точка лежит между осями симметрии.

\item Итак, композиция симметрий является поворотом на двойной угол между их осями. Отсюда видно также, что композиция симметрий не коммутативна! Перестановка симметрий приводит к смене направления вращения.
\item Композиция симметрии и поворота:
$$
S_\ph\circ R_\al = S_{\ph-\al/2},\quad R_\al\circ S_\ph = S_{\ph+\al/2}
$$

Это легко доказать из предыдущего равенства для композиции симметрий. Рассмотрим композицию $S_\ph\circ R_\al$. Пусть также $\psi = \ph-\al/2$. Домножая слева равенство $S_\ph\circ S_\psi=R_{2(\ph-\psi)}$ на симметрию $S_\ph$, получим
$$
S_\ph\circ R_{2(\ph-\psi)}=S_\ph\circ (S_\ph\circ S_\psi)=(S_\ph\circ S_\ph)\circ S_\psi=S_\psi,
$$
откуда
$$
S_\ph\circ R_\al = S_\ph\circ R_{2(\ph-\psi)} = S_\psi = S_{\ph-\al/2}.
$$
Аналогично доказывается второе равенство.
\item Итак, композиция симметрии и поворота является симметрией и при этом тоже не коммутативна!
\item Запишем полную таблицу композиций симметрий и вращений окружности:
\begin{center}
\begin{tabular}{c|c|c|}
  & $R_\al$ & $S_\psi$ \\
 \hline
$R_\be$ & $R_{\al+\be}$ & $S_{\psi+\be/2}$ \\
 \hline
$S_\ph$ & $S_{\ph-\al/2}$ & $R_{2(\ph-\psi)}$ \\
\hline
\end{tabular}
\end{center}
\item По аналогии с прямой обозначим $\T$ класс всех вращений окружности, $\S$ --- класс всех симметрий окружности
\item Получаем аналогичную таблицу композиций классов:
\begin{center}
\begin{tabular}{c|c|c|}
  & $\T$ & $\S$ \\
 \hline
$\T$ & $\T$ & $\S$ \\
 \hline
$\S$ & $\S$ & $\T$ \\
\hline
\end{tabular}
\end{center}

\item Снова наблюдаем все ту же группу умножения знаков!
\item Существуют ли другие движения окружности? Ответ --- нет!

\item Анализ движений проводится, как и в случае прямой, на основе наблюдений за количеством стационарных точек.
\item Для начала заметим, что если при движении окружности одна точка точка остается на месте, то неподвижной будет и диаметрально противоположная ей точка. Если бы это было не так, то, очевидно, расстояние между этими точками (равное половине дуги окружности) не сохранялось бы --- оно стало бы меньше. А это невозможно при движении.
\item Поэтому при анализе движений окружности всегда нужно иметь ввиду, что пары противоположных точек ведут себя одинаково --- либо они обе стационарны, либо обе двигаются.
\item Пусть движение $M$ таково, что оно оставляет на месте две точки $A\ne B$, не являющиеся диаметрально противоположными.
\item $M(A)=A$ и $M(B)=B$. Пусть $C'=M(C)$. Здесь могут быть два варианта: либо $C$ лежит на малой дуге $AB$, либо на большой. Эти дуги не могут быть равны по длине, т.к. $A$ и $B$ не являются противоположными (см.рис.). Точка $C'$ может лежать строго на одной из этих дуг.

Поскольку $M$ сохраняет расстояния, дуги $AC$ и $AC'$ равны, дуги $BC$ и $BC'$ равны. А значит, равный и суммы длин дуг $AC+CB$ и $AC'+C'B$. Отсюда следует, что $C$ и $C'$ могут лежать только на одной и той же дуге. Но тогда, в силу равенства дуг $AC$ и $AC'$ точки $C$ и $C'$ также должны совпадать (они лежат на одной дуге и на равных расстояниях от концов). Таким образом, $M(C)=C$ для любых точек $C$, т.е. $M=\id$.

\begin{center}
\includegraphics[scale=0.2]{Rund1.png}
\end{center}
\item Пусть движение $M$ оставляет на месте ровно одну пару противоположных точек $A$ и $A'$. В этом случае $C'=M(C)$, $C\ne C'$ и $AC=AC'$, тогда $C'$ --- отражение $C$ относительно оси симметрии $AA'$. Следовательно, $M=S_\ph$, где $\ph$ --- угол наклона прямой $AB$.
\item Пусть движение $M$ не оставляет на месте ни одной точки и пусть $B=M(A)$ ($B\ne A$). Обозначим за $\al$ угол дуги $AB$.

\begin{center}
\includegraphics[scale=0.2]{Rund2.png}
\end{center}
Тогда $R_{\al}^{-1}\circ M(A)=A$, т.е. $R_{\al}^{-1}\circ M$ оставляет на месте хотя бы одну точку $A$ (а точнее, пару противоположных точек $A$ и $A'$). Если оно оставляет на месте ровно одну пару точек $A$ и $A'$, то это некоторая симметрия $S_\ph$ (на рис. ось симметрии $l_\ph$), но тогда $M=R_\al\circ S_\ph=S_{\ph+\al/2}$. Получается, что $M$ сохраняет точку $C$ на месте ($C$ есть середина дуги $AB$). Противоречие с тем, что $M$ не оставляет на месте ни одной точки. Остается вариант, что $R_{\al}^{-1}\circ M$ оставляет на месте как минимум две точки, не являющихся противоположными, но тогда $R_{\al}^{-1}\circ M=\id$, откуда $M=R_\al\circ \id=R_\al$ --- поворот.
\item Таким образом, всякое движение окружности --- это либо поворот (в частности, $\id$), либо симметрия относительно оси, проходящей через центр окружности (теорема Шаля).
\item При этом, любое движение --- это либо одна симметрия, либо композиция двух симметрий.
\end{enumerate}
\subsection*{Задачи}
\begin{enumerate}
\item Центральная симметрия --- это какое движение?
\item Композицией каких симметрий можно выразить центральную симметрию?
\item С помощью симметрии относительно оси $Ox$ и вращений выразить симметрию относительно оси $Oy$.
\end{enumerate}


\section{Наматывание прямой на окружность}

\subsection*{Конспект}
\begin{enumerate}\setlength{\itemsep}{1pt}
\item Совместим теперь окружность с прямой иным способом. Выделим на окружности точку $O$ и начнем ее обход (вращение) в положительном направлении.
\item Выше мы видели, что углы поворота, кратные $360^o$, т.е. полном обороту, соответствуют тождественному движению, т.е. приведут нас в точку отправления $O$.
\item Однако, если с точки зрения математического движения ничего не изменилось, физически мы проделали путь, равный длине окружности. Для удобства будем считать, что радиус окружности есть единичный вектор, так что ее длина равна $2\pi$, и с каждым полным оборотом мы будем <<наматывать>> расстояние $2\pi$.
\item Вообще, расстояние, пройденное по окружности единичного радиуса, когда этот радиус заметает угол $\al$, равно $\al(2\pi/360^o)$. Чтобы каждый раз не переводить единицы измерения радиуса в градусы и наоборот, углы также приняот измерять в единицах длины --- радианах. А именно, \textit{угол в} 1 \textit{радиан соответствует повороту, при котором точка проделает по окружности путь, равный по длине радиусу данной окружности}. Нетрудно видеть, что в градусах 1 радиан будет иметь выражение $360^o/(2\pi)$ или $180^o/\pi \approx 57^o$.
\item В дальнейшем условимся все углы измерять в радианах, если не потребуется иное.
\item Известно, что число $\pi$ не соизмеримо с целыми числами, так что поворот $R_1$ на 1 радиан ни в какой положительной степени не приведет нас снова в точку исхода $O$.
\item Зато поворот $R_{2\pi}$ в точности возвращает нас в точку отправления $O$.
\item При каждом таком повороте мы проделываем путь, равный углу поворота, т.е. $2\pi$ (радиус равен 1).
\item Следовательно степени такого поворота $R_{2\pi}^n$ дадут прохождение пути длиной $2\pi n$.
\item Представим эту картину не с точки зрения жителей окружности, бегающих по замкнутой траектории, а с точки зрения жителей прямой, которая наматывается на окружность. С их точки зрения все выглядит несколько иначе и больше напоминает движение оклеса по дорожному полотну: окружность катится по прямой и через равные промежутки касается точкой $O$ данной прямой.
\item Если при этом два друга --- один из мира окружности, второй из мира прямой, --- двигаются с одинаковой скоростью в одном направлении, то они могут синхронизироваться в точке касания окружности и прямой и разговаривать друг с другом.
\item Нужно заметить при этом, что если колесо вращается по часовой стрелке, т.е. в отрицательном направлении, то вдоль прямой оно движется направо, т.е. в положительном направлении. Но фокус в том, что житель окружности для синхронизации с жителем прямой должен идти навстречу вращению колеса, т.е. тоже в положительном направлении! Таким обраом, движения обоих друзей имеют одинаковый знак! На рис. ниже мы отметили синей стрелкой направление движения жителя окружности, а черной --- встречное вращение самой окружности.
\item Итак, колесо катится, два друга беседуют, точка $O$ то и дело, а именно через каждые $2\pi$ метров соприкасается с прямой. Каждый раз, когда точка $O$ касается прямой, наш ученый друг из мира прямой ставит на ней отметины и считает их по порядку, т.е. приравнивает к степени совершенного поворота колеса: в начальный момент времени это был 0, затем 1 оборот, затем 2 оборота, и т.д.

\begin{center}
\includegraphics[scale=0.2]{RundLine.png}
\end{center}

\item Что же мы видим на прямой? Мы видим не что иное как шкалу натуральных чисел, в точности соответствующую степеням вращений окружности. Число $2\pi$, фигурирующее как коэффициент, является не более чем единицей измерения. Кто-то измеряет в метрах, кто-то -- в ярдах, а мы измеряем в длинах единичной окружности.
\item Представим теперь, что в какой-то момент касания точки $O$ с прямой физика мира изменилась, и вращение начало осуществляться в обратную сторону!
\item Наши друзья-ученые при этом продолжат совместное путешествие, но только назад. Они пойдут отсчитывать уже проставленные отметки на прямой в убывающем порядке, пока не вренутся в точку 0. Но здесь состоится чудо, и движение продолжится дальше.
\item Как все это записать на языке вращений и сдвигов?
\item Предположим, что сначала окружность повернулась на $n$ полных оборотов вперед, а затем на $m$ полных оборотов назад.
\item Мы получаем итоговое вращение, записываемое как $R_{2\pi n}\circ R_{2\pi m}^{-1}$.
\item А что мы имеем с точки зрения движения на прямой?
\item Сначала был произведен сдвиг $T_{2\pi n}$, затем сдвиг $T_{-2\pi m}$.
\item И мы видим, что индекс, определяющий итоговое вращение и итоговый сдвиг, --- один и тот же!
\item Причем, если $n>m$, то сдвиг будет вправо на расстояние $2\pi(n-m)$, а поворот будет положительным на угол $2\pi(n-m)$.
\item Если же $n<m$, то сдвиг будет влево на расстояние $2\pi(m-n)$, а поворот будет отрицательным (по часовой стрелке) на угол $2\pi(m-n)$.
\item Ранее мы уже договаривались, что перед векторами, направленными влево, будем ставить знак '-'. Так же будем поступать и с углами вращений в отрицательную сторону.
\item Соответственно, при $n<m$ мы будем иметь итоговый сдвиг $T_{-2\pi(m-n)}$ и итоговый поворот $R_{-2\pi(m-n)}$, которые также можно записать в виде степеней:
$$
T_{-2\pi(m-n)}=T_{2\pi}^{-(m-n)}\mbox{ и }R_{-2\pi(m-n)}=R_{2\pi}^{-(m-n)}.
$$

\begin{center}
\includegraphics[scale=0.15]{RundLine1.png}
\end{center}

\item Осталось добавить маленький штрих к портрету, а именно: в случае $n<m$ под разностью $n-m$ будем понимать запись $-(m-n)$.
\item Тогда уже независимо от того, $n<m$, или $m<n$, или $n=m$, композиция поворотов и сдвигов сначала на $n$ вправо и затем на $m$ влево будет записываться одинаково:
$$
T_{2\pi(n-m)}=T_{2\pi}^{n-m}\mbox{ и }R_{2\pi(n-m)}=R_{2\pi}^{n-m}.
$$
\item В итоге мы приходим к тому, что называется \textbf{целыми числами}, включающими натуральные числа и отрицательные натуральные числа (при этом $-0=0$).
\item Сколько бы мы ни вращали окружность на $2\pi$ в ту или иную сторону с помощью поворота $R_{2\pi}$, мы совершаем поворот на целую степень полного оборота. При этом как бы мы ни катали окружность по прямой, точка $O$ будет ставить отметки в точках $2\pi k$, где $k$ --- целое число.
\end{enumerate}



\chapter{Целые числа и ОТА}

\vrezka{
Это --- первая глава, где мы по-настоящему погружаемся в арифметику, используя тот понятийный аппарат, который был наработан в предыдущих главах. Здесь вводится обозначение множества целых чисел, дается строгое определение алгебраического понятия <<кольцо>>, обосновывается алгоритм Евклида.

Ключевым моментом является получение теоремы о том, что НОД двух чисел можно записать в виде их линейной комбинации с целыми коэффициентами. Этот факт выводится как непосредственно из алгоритма Евклила, так и с помощью сумм Минковского (что отсылает нас к главе 0).

Далее отсюда выводится основная теорема арифметики и некоторые ее следствия.
}

\section{Целые числа. Кольцо}

\subsection*{Конспект}
\begin{enumerate}\setlength{\itemsep}{1pt}
\item Итак, совмещение вращений со сдвигами дает нам полную свободу перемещений в положительном и отрицательном направлении. При этом, с точки зрения окружности ничего не меняется --- происходит итоговое движение $\id$, а с точки зрения прямой --- происходит разметка точек с равным шагом. Ясно, что сам шаг при этом не имеет значения. Мы могли бы взять окружность радиуса $R$, и тогда шаг был бы равен $2\pi R$. В частности, можно взять радиус $R=1/2\pi$, и тогда точки на прямой расположатся с шагом 1.
\item Такую же картину можно получить, если взять все точки, получаемые из выделенной точки 0 степенями сдвига на единичный вектор, используя положительные и отрицательные, т.е. целые, степени.
\item Как видим, целые числа, как и натуральные, можно интерпретировать и как степени движений (и вообще любых преобразований, имеющих обратные), и как векторы сдвигов на прямой, а значит, к ним применимы определенные ранее операции сложения, вычитания и умножения. При этом результат умножения получает такой знак, который определяется из таблицы умножения знаков.
\item Множество всех целых чисел принято обозначать $\Z$. Вместе с операциями сложения (вычитания) и умножения структура $(\Z,+,\cdot)$ называется \textbf{кольцом целых чисел}. Кольцо --- это структура, где можно складывать, вычитать и умножать.
\item Понятие кольцо является расширением понятия группы, т.к. добавляется операция умножения.
\item Ранее мы уже видели такие группы, как группа движений прямой, группа умножения знаков, группа композиций классов сдвигов и симметрий, группа вращений окружности. Все они обладали одной операцией --- композицией, которая соответствовала сложению параметров сдвигов и вращений.
\item Кроме того, мы ввели такое понятие как кратность, заменяя тем самым многократное сложение умножением на целое число.
\item Кратность операций нельзя рассматривать как умножение сдвигов или вращений, поскольку это сущности разного рода. Поэтому движения в общем случае образуют только лишь группу.
\item Однако, уже сами кратности, как самостоятельные сущности, можно и складывать, и умножать. Например, если мы рассмотрим сдвиг $T_1$ и композицию его кратностей $T_1^n\circ T_1^m$, то получим тот же сдвиг но в суммарной кратности $T_1^{n+m}$, где $n,m\in\Z$. Но ничто не мешает нам рассмотреть кратность $m$ сдвига $T_1^n$, т.е. сдвиг $(T_1^n)^m$, а это уже будет не что иное, как сдвиг кратности $nm$, т.е. $T_1^{nm}$.
\item Иначе говоря, умножение на целых числах можно представить как кратности кратностей сдвигов!
\item Целые числа, если их рассматривать как счетчик витков по окружности, образуют так называемую \textbf{фундаментальную группу} окружности, которая является важным топологическим свойством окружности и ей подобным (в топологии) фигурам. Зная фундаментальную группу, можно определить, насколько схожи фигуры в топологическим смысле --- можно ли из одной получить другую путем деформации без разрывов и склеиваний.
\item Фиксируем понятие \textbf{кольцо}. Это --- множество $K$ с двумя бинарными операциями $+$ (плюс) и $\cdot$ (точка), которые подчинены следующим законам:
\begin{enumerate}[{\bf K}1)]
\item $a,b\in K\Rightarrow a+b\in K, a\cdot b\in K$ (замкнутость операций);
\item $a,b,c\in K\Rightarrow (a+b)+c=a+(b+c), (a\cdot b)\cdot c = a\cdot (b\cdot c)$ (ассоциативность операций);
\item существует элемент $0\in K$ такой, что $a+0=0+a=a$ для всех $a\in K$ (аксиома нуля);
\item для всякого элемента $a\in K$ существует противоположный $-a$ такой, что $a+(-a)=0$ (аксиома противоположного элемента);
\item для всех $a,b,c\in K$ имеем $(a+b)\cdot c=(a\cdot c)+(b\cdot c)$, $c\cdot(a+b)=(c\cdot a)+(c\cdot b)$ (правая и левая дистрибутивность);
\item для всех $a,b\in K$ имеем $a+b=b+a$ (коммутативность сложения).
\end{enumerate}

Обычно изучаются \textbf{кольца с единицей}, т.е. такие кольца, для которых
\begin{enumerate}[resume*]
\item существует элемент $1\in K$ такой, что $a\cdot 1=1\cdot a=a$ для всех $a\in K$ (аксиома единицы),
\end{enumerate}
а также \textbf{коммутативные кольца}, т.е. такие кольца, для которых
\begin{enumerate}[resume*]
\item для всех $a,b\in K$ имеем $a\cdot b=b\cdot a$ (коммутативность умножения).
\end{enumerate}

Иначе говоря, в коммутативном кольце с единицей можно складывать, вычитать и умножать по обычным правилам.
\end{enumerate}
\subsection*{Задачи}
\begin{enumerate}
\item Докажите, что $m\Z$ --- подкольцо кольца $\Z$, т.е. в нем также можно складывать, вычитать и умножать. $m$ --- положительное целое число.
\end{enumerate}

\section{Кузнечик НОД и алгоритм Евклида}

\subsection*{Конспект}
\begin{enumerate}\setlength{\itemsep}{1pt}
\item Поработаем теперь непосредственно с целыми числами. Пусть у нас есть кузнечик, стоящий в точке 0, который умеет прыгать с шагом $a$ и с шагом $b$ в любую сторону. Числа $a,b$ --- натуральные.
\item Ясно, что он может попасть в любую точку вида $ka+mb$, где кратности $k,m$ --- целые. Как понять, в какие точки он может попасть, а в какие --- нет?
\item Пусть $d$ --- наименьшее положительное число, в которое кузнечик может попасть, т.е. оно имеет вид $d=ka+mb$ при некоторых $k,m$. Тогда он может попасть и в любое число вида $nd$, поскольку $nd=(nk)a+(nm)b$, где $n\in\Z$. Следовательно, кузнечик может попасть во все целые числа, кратные $d$ (множество $d\Z$).
\item Но в любые другие целые числа он не сможет попасть. Действительно, если он попадает в какое-то число $x$, лежащее между двумя соседними кратностями $d$, т.е. в число $x=nd+y$, где $0<y<d$, то тогда он момжет попасть в число $y$, т.е. остаток от деления $x$ на $d$. Но $y<d$ и притом положительное, а это противоречит выбору числа $d$. Таким образом, кузнечик попадает во все точки $d\Z$, и только в эти точки!
\item Что такое $d$ на самом деле?
\item Для ответа на этот вопрос вспомним про алгоритм Евклида (с отсечениями квадратов). Пусть $a<b$. Вычтем из $b$ столько $a$, сколько сможем: $b=k_0a+r_1$, где $0\le r_1<a$. Далее, из $a$ вычитаем столько $r_1$, сколько сможем, если $r_1>0$. Получим $a=k_1r_1+r_2$, где $0\le r_2<r_1$. Снова, если $r_2>0$, вычитаем из $r_1$ столько $r_2$, сколько можем: $r_1=k_2r_2+r_3$, где $0\le r_3<r_2$. И так далее.
\item Видим, что всякий раз, если $r_i>0$, то мы приходим к $r_{i+1}<r_i$. Проблема в том, что это не может продолжаться бесконечно долго, т.к. от всякого натурального числа в сторону нуля можно спуститься за конечное число шагов (а ведь остатки у нас все положительные!). Так что рано или поздно случится $r_{n+1}=0$, и на этом алгоритм Евклида остановится! Это значит, что прямоугольник $a\times b$ можно сложить квадратами $r_n\times r_n$.
\item Если теперь раскрутить равенства $r_{i-1}=k_ir_i+r_{i+1}$ в обратную сторону, то мы получим, во-первых, что $a$ и $b$ кратны $r_n$, и во-вторых, что $r_n=Ka+Mb$ при некоторых целых $K,M$. То есть, $r_n$ есть общий делитель исходных чисел $a$ и $b$, и наш кузнечик способен попасть в точку $r_n$ (а значит, и во все точки, ему кратные, т.е. в $r_n\Z$).
\item С другой стороны, если какое-то $q$ является общим делителем $a$ и $b$, то $q$ делит $r_1=b-k_0a$, делит $r_2=a-k_1r_1$, делит $r_3=r_1-k_2r_2$, и т.д., и, наконец, делит $r_n$. Стало быть, $q\le r_n$, и $r_n$ --- наибольшой общий делитель $a$ и $b$.
\item Итак, кузнечик способен попасть в НОД($a,b$), следовательно, $d\le\mbox{НОД}(a,b)$. С другой стороны, выбор $d$ таков, что $d=ka+mb$ при некоторых целых $k,m$, но тогда всякий делитель $a$ и $b$ является и делителем $d$, в частности НОД($a,b$) делит $d$, откуда $\mbox{НОД}(a,b)\le d$. Таким образом, минимальный шаг, на который способен сдвинуться кузнечик, --- это наибольший общий делитель чисел $a$ и $b$. Поэтому кузнечика с ногами $a$ и $b$ можно назвать НОД($a,b$). Он способен прыгнуть (в несколько прыжков) во ВСЕ точки, кратные НОД($a,b$), и ТОЛЬКО в эти точки!
\end{enumerate}
\subsection*{Задачи}
\begin{enumerate}
\item С помощью алгоритма Евклида найти $\gcd(2020,555)$.
\end{enumerate}


\section{Простые числа и ОТА}\label{PrimeNumbers}

\subsection*{Конспект}
\begin{enumerate}\setlength{\itemsep}{1pt}
\item У кузнечика НОД может получиться уникальная ситуация, когда при достаточно больших числах $a$ и $b$ он способен прыгнуть в любое целое число! Это верно в том и только том случае, когда НОД$(a,b)$=1. При этом говорят, что $a$ и $b$ взаимно просты. Например, 125 и 63 взаимно просты.
\item Взаимная простота также обеспечивается, если одно из чисел само по себе \textbf{простое}, т.е. не делится ни на что, кроме 1 и самого себя. Например, 101 --- простое, так что в паре с любым другим числом (кроме кратного 101) оно будет взаимно просто, и наш кузнечик сможет прыгнуть в любую целую точку! Например, он умеет прыгать на 101 и 62, значит, он умеет прыгать в любое целое число!
\item Любое число можно представить как произведение степеней простых. Действительно, 1 есть произведение нулевых степеней простых чисел, например, $2^0$. Предположим, что для всех чисел от 1 до $n$ утверждение о разложимости справедливо (внимание! индукция!) и рассмотрим число $n+1$. Оно либо уже простое, либо делится на число меньше $n$, отличное от 1. Тогда $n+1=mk$, причем $m,k\le n$, а они есть произведение степеней простых по предположению индукции, но тогда и $n+1$ есть произведение степеней простых!
\item Простых чисел бесконечно много. Предположим, что это не так, и пронумеруем все простые числа:
$$
p_1=2,\;p_2=3,\;p_3=5,\;p_4=7,\;p_5=11,\;\dots,\;p_n
$$
Далее рассмотрим число $m=p_1p_2\dots p_n+1$. Оно не кратно никакому простому числу из ряда $p_1,\dots,p_n$, иначе бы 1 также было бы кратно этому простому. Следовательно, оно простое, но не входит в данный ряд. Противоречие.
\item Если простое число $p$ делит произведение чисел $ab$, то оно по крайней мере делит одно из них. Доказательство: допустим, что $p$ не делит $a$, тогда НОД$(p,a)=1$, но тогда, как мы уже видели выше, $1=kp+ma$ при некоторых целых $k,m$. Умножим это равенство на $b$: $b=kpb+mab$. Справа оба слагаемых делятся на $p$, значит, и $b$ делится на $p$.
\item Из этого свойства легко получить \textbf{основную теорему арифметики}: каждое натуральное число единственным образом представляется в виде произведения степеней простых чисел:
$$
n=p_1^{k_1}p_2^{k_2}\dots
$$
Набор степеней $k_1,k_2,\dots$ уникален для каждого числа $n$. Действительно, если бы было два разложения, то после сокращения на одинаковые сомножители мы бы получили равенство
$$
p_1^{k_1}p_2^{k_2}\dots p_m^{k_m} = q_1^{s_1}q_2^{s_2}\dots q_t^{s_t}
$$
Но каждое простое слве делит все число справа, значит, делит один из его множителей, а значит, совпадает с одним из $q_i$, что по предположению невозможно. Противоречие! Следовательно, разложение по степеням простых единственно.
\item Здесь еще нужно сделать оговорку про $\Z$. Любое целое число также единственным образом раскладывается по степеням порстых, но с точностью до знака $\pm$ перед этим разложением.
\end{enumerate}

Основную теорему арифметики можно доказать разными способами. Покажем еще один способ, который использует множества и операции Минковского с этими множествами.

\begin{enumerate}[T1]
\item Пусть $P,Q\subseteq\Z$. Суммой и разностью по Минковскому называются, соответственно, множества:
$$
P\oplus Q=\{x+y\mid x\in P, y\in Q\},\quad P\ominus Q=\{x-y\mid x\in P, y\in Q\}.
$$
\item Множества вида $a\Z$ замкнуты относительно операций сложения и умножения (являются подкольцами кольца $\Z$), поэтому для любых $P,Q\subseteq a\Z$ и любых $k,n\in \Z$ имеет место вложение:
$$
kP\oplus nQ\subseteq a\Z.
$$
\item $a|b$ тогда и только тогда, когда $b\Z\subseteq a\Z$.

Действительно, если $a|b$, то $b=ka$. Если $x\in b\Z$, то $x=by=aky\in a\Z$.

Пусть $b\Z\subseteq a\Z$, тогда $b\in b\Z$ и, следовательно, $b\in a\Z$, т.е. $b=ka$ при некотором целом $k$, тогда
$a|b$.

\item Решим неравенство $P\ominus P\subseteq P$, где $P\subseteq \Z$.

1) Пустое множество удовлетворяет этому неравенству.

2) Множество $P=\{0\}$ также удовлетворяет данному неравенству.

3) Пусть $c\in P$ и $c\ne 0$. В этом случае ясно, что в $P$ есть положительные числа ($0=c-c$, а значит, есть $c$ и $-c$). 
Пусть $a=\min\{x\mid (x\in P)\land (x>0)\}$. Легко видеть, что $a\Z\subseteq P\ominus P\subseteq P$. Но если $P\setminus a\Z$ не пусто, то существует $x\in P\setminus a\Z$, причем $x=ka+d$, где $0<d<a$. Но $d=x-ka\in P\ominus P$, т.е. $d\in P$, что противоречит выбору $a$. Следовательно, $P=a\Z$.

Таким образом, если $P\ominus P\subseteq P$, то либо $P=\emptyset$, либо $P=a\Z$ при некотором целом $a$.

\item $a\Z\oplus b\Z=\gcd(a,b)\Z$.

Действительно, $P=a\Z\oplus b\Z$ удовлетворяет неравенству $P\ominus P\subseteq P$, и значит, по свойству T4 $a\Z\oplus b\Z$ совпадает с множеством $c\Z$ при некотором $c$ (причем, если $a,b>0$, то и $c>0$), т.е.
$$
a\Z\oplus b\Z=c\Z.
$$

Отсюда, с одной стороны, следует, что $a\Z,b\Z\subseteq c\Z$, откуда (свойство T3) $c|a$ и $c|b$. С другой стороны, если $d|a$ и $d|b$, то $a\Z,b\Z\subseteq d\Z$, откуда (свойство T2) $c\Z\subseteq d\Z$, откуда (свойство T3) $d|c$. То есть, любой делитель $a$ и $b$ не превосходит $c$, а $c$ также является делителем $a$ и $b$. Следовательно, $c=\gcd(a,b)$.

\item Если простое $p$ делит произведение $ab$, то или $p|a$, или $p|b$.

Предположим, что $p\not|a$, тогда $\gcd(p,a)=1$ и (по свойству T5) $p\Z\oplus a\Z=\Z$. Откуда $1=kp+ma$ при некоторых целых $k,m$. Тогда $b=kbp+mab$, откуда следует, что $p|b$.

Если предположить, что $p\not|b$, то аналогично выводим соотношение $p|a$.
\item Отсюда, как уже отмечалось выше, легко выводится Основная теорема арифметики.
\end{enumerate}


\subsection*{Задачи}
\begin{enumerate}
\item Докажите, что если $P\ominus P\subseteq P$, то выполняется равенство $P\ominus P=P$.
\item Докажите, что неравенство $P\ominus P\subseteq P$ определяет все подгруппы $\Z$ по сложению.
\item Натуральное число называется \textbf{совершенным}, если сумма всех его делителей, меньших его, равно ему самому. Например, 6 и 28 --- совершенные числа. Докажите, что число $2^{n-1}(2^n-1)$ будет совершенным, если $2^n-1$ --- простое число.
\end{enumerate}


\section{Некоторые следствия ОТА}

\begin{comment}

2015_09_10 - 15-я лекция д. ф.-м.н. А. В. Савватеева Основная теорема арифметики: следствия ч. ¼
https://www.youtube.com/watch?v=znkH8vFwy6s

2-30 для чего нужна основная теорема арифметики?
4-00 пифагоровы треугольники
6-00 5 12 13
8-30 разбиваем плоскость на кватдратики, рисуем прямую

2015_09_10 - 15-я лекция д. ф.-м.н. А. В. Савватеева Основная теорема арифметики: следствия ч. 2/4
https://www.youtube.com/watch?v=S9G6DLcB4as

0-40 какие целые координаты попадают на прямую
3-00 аб делится на ц, а и ц взаимопросты, то б делится на ц
4-40 произведение взаимнопростых чисел равно степени, то каждое является той же степенью какого-то числа
9-00 два целых числа

2015_09_10 - 15-я лекция д. ф.-м.н. А. В. Савватеева Основная теорема арифметики: следствия ч. ¾
https://www.youtube.com/watch?v=6Dve5JifC7M

0-00 кузнечик прыгает на а или в, куда он может попасть?

2015_09_10 - 15-я лекция д. ф.-м.н. А. В. Савватеева Основная теорема арифметики: следствия ч. 4/4
https://www.youtube.com/watch?v=OocW2VaNzOE
0-00 математически запишем точки куда может попасть кузнечик
\end{comment}


\chapter{Симметрии фигур}

\vrezka{В этой главе мы снова возвращаемся к геометрии и занимаемся полным описанием групп движений правильных многоугольников, а заодно и всех конечных подгрупп движений окружности. В конце главы рассматривается нестандартный пример группы движений ромба и вводится определение четверной группы Клейна.
}

\section{Симметрии правильного треугольника}

\subsection*{Конспект}
\begin{enumerate}\setlength{\itemsep}{1pt}
\item Вернемся на окружность и рассмотрим на ней вращение $R_{2\pi/3}$, т.е. на $120^o$.
\item Множество вращений $R^3=\{R_{2\pi/3},R_{2\pi/3}^2,R_{2\pi/3}^3\}$ образует циклическую группу. Видим, что
$$
R^3 = \{\id,R_{2\pi/3},R_{4\pi/3}\}.
$$
\item Зафиксируем точку $A$ на окружности и найдем ее образы при действии этой группы: $B=R_{2\pi/3}(A)$, $C=R_{4\pi/3}(A)$. Набор точек $\{A,B,C\}$ образует орбиту точки $A$ при действии группы $R^3$.
\item Посмотрим теперь на треугольник $ABC$. Какие движения переводят его в себя? Очевидно, вращения из группы $R^3$, но также есть и симметрии $S^3=\{S_A, S_B, S_C\}$ относительно осей, проходящих через центр окружности и вершины треугольника.
\item Можем проверить, что объединение $R^3\cup S^3$, состоящее из трех вращений и трех симметрий, образует группу относительно операции композиции движений.
\item Выпишем полную таблицу Кэли для этой группы:
\begin{table}[htb!]\begin{center}
\begin{tabular}{|c|c|c||c|c|c|}
\hline
$\id$        & $R_{2\pi/3}$ & $R_{4\pi/3}$ & $S_A$        & $S_B$        & $S_C$  \\  \hline
$R_{2\pi/3}$ & $R_{4\pi/3}$ & $\id$        & $S_B$        & $S_C$        & $S_A$  \\  \hline
$R_{4\pi/3}$ & $\id$        & $R_{2\pi/3}$ & $S_C$        & $S_A$        & $S_B$  \\  \hline\hline
$S_A$        & $S_C$        & $S_B$        & $\id$        & $R_{4\pi/3}$ & $R_{2\pi/3}$  \\  \hline
$S_B$        & $S_A$        & $S_C$        & $R_{2\pi/3}$ & $\id$        & $R_{4\pi/3}$  \\  \hline
$S_C$        & $S_B$        & $S_A$        & $R_{4\pi/3}$ & $R_{2\pi/3}$ & $\id$   \\  \hline
\end{tabular}
\end{center}\end{table}
\item На примере этой группы мы можем заметить, во-первых, что в группе можно выделить подгруппу вращений (верхний левый квадрат $3\times 3$), во-вторых, что группа движений треугольника конечна и некоммутативна, поскольку ее таблица умножения несимметрична. Кроме того, в полном сооветствии с таблицей умножения классов $\R$ и $\S$ видим, что композиция вращений есть вращение, композиция вращения и симметрии есть симметрия, композиций двух симметрий есть вращение.
\item В группе симметрий треугольника можно выделить базовые элементы: либо пара $(R_{2\pi/3}, S_A)$, либо пара $(S_A,S_C)$. Понятно, что здесь можно заменить поворот и симметрии на другие.
\item Вопрос: есть ли еще какие-то движения окружности, переводящие правильный треугольник в себя?
\item Заметим, что при движении, переводящем треугольник в себя, вершины обязательно переходят в вершины. Если бы это было не так, то какая-то вершина перешла бы в точку на стороне треугольника, но тогда преобразование не сохранило бы угол при этой вершине. Таким образом, преобразований треугольника не может быть больше, чем всех возможных перестановок трех вершин:
$$
\begin{pmatrix}
A & B & C \\
A & B & C
\end{pmatrix},
\begin{pmatrix}
A & B & C \\
B & C & A
\end{pmatrix},
\begin{pmatrix}
A & B & C \\
C & A & B
\end{pmatrix},
$$
$$
\begin{pmatrix}
A & B & C \\
A & C & B
\end{pmatrix},
\begin{pmatrix}
A & B & C \\
C & B & A
\end{pmatrix},
\begin{pmatrix}
A & B & C \\
B & A & C
\end{pmatrix}
$$
Нетрудно видеть, что эти перестановки в точности соответствуют преобразованиям $\id, R_{2\pi/3}, R_{4\pi/3}, S_A, S_B, S_C$. Так что данными преобразованиями исчерпываются все возможные движения, переводящие правильный треугольник в себя.
\end{enumerate}
\subsection*{Задачи}
\begin{enumerate}
\item Выписать все перестановки на 4 символах $A,B,C,D$.
\end{enumerate}



\section{Симметрии правильного многоугольника}

\subsection*{Конспект}
\begin{enumerate}\setlength{\itemsep}{1pt}
\item Рассмотрим еще один случай преобразований фигуры в себя. Пусть имеется правильный $n$-угольник. Тогда очевидными преобразованиями, сохраняющими форму и размеры фигуры, будут:
$$
R_{2\pi k/n},\quad S_k,\quad k=\overline{1,n}
$$
\item В случае четного $n$ в многоугольнике все вершины разбиваются на пары противоположных, лежащих на общей оси симметрии, поэтому имеется $n/2$ осей симметрии, проходящих через вершины, и $n/2$ осей, проходящих через середины строн. В случае нечетного $n$ на каждую вершину приходится своя ось симметрии.
\item Как и в случае треугольника, несложно показать, что этими $2n$ преобразованиями исчерпываются все преобразования правильного многоугольника в себя, что, как видим, сильно меньше общего числа перестановок вершин, которое равно $n!$ (совпадение получается только при $n=3$).
\item Однако и в этом случае в качестве базисных можно выбрать всего два преобразования: $R_{2\pi/n}$ и $S_1$, либо две симетрии, оси которых являются соседними.
\end{enumerate}
\subsection*{Задачи}
\begin{enumerate}
\item Составить полную таблицу Кэли для группы движений правильного 4-угольника.
\item Выразить поворот на 90 градусов с помощью двух симметрий.
\end{enumerate}


\section{Подгруппы движений окружности}

\subsection*{Конспект}
\begin{enumerate}\setlength{\itemsep}{1pt}
\item Правильные $n$-угольники дают приблизительное представление о подгруппах движений окружности. Приблизительное --- именно в том смысле, что движения $n$-угольников с любой наперед заданной точностью (при достаточно большом $n$) будут представлять движения окружности.
\item Вопрос: все ли конечные подгруппы движений окружности задаются движениями правильных $n$-угольников?
\item Ответ: да, но с оговоркой. Некоторые конечные подгруппы совпадают с группами движений $n$-угольников, другие же являются их собственными подгруппами.
\item Действительно, пусть $G$ --- некоторая подгруппа движений окружности, причем конечная, т.е.
$$
G=\{g_1,g_2,\dots,g_m\}.
$$
\item Возьмем произвольный элемент $g_k$ и рассмотрим множество всех его целых степеней:
$$
\langle g_k\rangle=\{\dots,g_k^{-1},g_k^0,g_k,g_k^2,\dots\}
$$
\item Данное множество, очевидно, является подгруппой группы $G$, а значит, конечно. Но тогда среди степеней $g_k$ точно есть два совпадающих значения: $g_k^s=g_k^t$ при $t\ne s$. Пусть для определенности $t>s$. Тогда, умножая равенство на $g_k^{-s}$, получаем $g_k^{t-s}=g_k^0=\id$. Иначе говоря, $g_k$ в некоторой положительной степени превращается в $\id$.
\item \textbf{Порядком элемента} $g\in G$ называется минимальное натуральное число $s$ такое, что $g^s=\id$. Как видим, для всякого $g_k\in G$ такой порядок существует.
\item При этом, как мы установили ранее, $g_k$ --- это либо поворот окружности, либо отражение относительно оси, проходящей через ее центр. В первом случае порядок может быть любым начиная с 1. В случае, когда порядок элемента $g_k$ равен 1, получаем, что $g_k=\id$, т.е. поворот на нулевой угол (или угол $2\pi$). Во втором случае, очевидно, что порядок $g_k$ строго равен 2, т.к. отражение само себе обратно.
\item Если $g_k$ --- поворот, то это поворот на угол $2\pi/s$, где $s$ --- порядок $g_k$.
\item Порядок элемента является одновременно и порядком подгруппы $\langle g_k\rangle$. Действительно, если $s$ --- порядок элемента $g_k$, то все $g_k$ в степенях меньше $s$ различны (иначе порядок оказался бы меньше $s$), а все бОльшие степени сводятся к меньшим сокращением на $g_k^s$. Так что в подгруппе $\langle g_k\rangle$ ровно $s$ элементов!
\item Конечная группа $\langle g_k\rangle$, порожденная степенями одного своего элемента, называется \textbf{циклической}. Это название вполне соответствует тому, что все элементы группы в нашем случае есть повороты окружности на определенный угол, нацело делящий $2\pi$.
\item Итак, мы видим, что в $G$ есть подгруппы вида $\langle g_k\rangle$, которые либо тривиальны (состоят из одного элемента $\id$), либо соответствуют группам вращения многоугольников (если $g_k$ --- поворот, причем десь стоит оговориться, что при $g_k=R_\pi$ многоугольника как такового нет, это вырожденный двуугольник), либо соответствуют группам отражений вида $\{\id,S_\ph\}$ при некотором угле наклона $\ph$ оси отражения. Наша задача состоит в том, чтобы показать, что все эти подгруппы, а равно и сама группа $G$, есть подгруппы движений какого-то одного $n$-угольника.
\item Пусть $G'=\{g\in G\mid g\mbox{ --- поворот или }\id\}$. Ясно, что $G'$ --- подгруппа группы $G$. Предположим далее, что $G'\ne G$, т.е. в группе $G$ существует хотя бы одно отражение $h$. В этом случае, как мы видели ранее, все элементы произведения Минковского $hG'$ также являются отражениями. Предположим, что существует отражение $h'\in G\setminus (hG'\cup G')$. Но ранее мы установили, что $hh'$ есть поворот, причем $hh'=g\in G'$, т.к. $hh'\in G$.
 Но тогда $h'=h^{-1}g=hg\in hG'$ (отражение обладает свойством $h=h^{-1}$), а это противоречит выбору $h'$.
\item Итак, если в группе $G$ есть отражения, то все они находятся в одном классе $hG'$, причем этот класс не зависит от выбора отражения $h$. Иначе говоря, все отражения порождены каким-то одним отражением и всеми поворотами. При этом может оказаться, что в группе $G$ есть только один поворот --- $\id$, а значит, там есть и только одно отражение.
\item Осталось разобраться с подгруппой $G'$ всех поворотов.
\item Возьмем из $G'$ самый маленький поворот $g_0$, т.е. такой, у которого порядок наибольший. Угол поворота $g_0$ обозначим через $x_0$, а порядок $g_0$ --- через $s_0$. Так что $x_0s_0=2\pi$.
\item Пусть $g$ --- произвольный поворот из $G'$ и его угол поворота равен $x>0$ (если угол поворота отрицательный, то можно рассмотреть $g^{-1}$, который также принадлежит $G'$). Если $x$ не делится нацело на $x_0$, то имеет место представление
$$
x = kx_0+y,
$$
где $0<y<x_0$. Кроме того, углу $y$ соответствует поворот $g'=g(g_0)^{-k}$, который, очевидно, принадлежит группе $G'$, а значит, имеет конечный порядок.
\item Каков порядок этого поворота? Ясно, что $s_0y<s_0x_0=2\pi$, следовательно, порядок поворота $g'$ должен быть больше $s_0$. Но $s_0$ --- наибольшоий порядок среди всех поворотов группы $G'$. Противоречие! Значит, $y=0$, т.е. $x$ нацело делится на $x_0$: $x=kx_0$ при некотором целом положительном $k$.
\item Таким образом, подгруппа $G'$ группы $G$ состоит из поворотов, являющихся степенями поворота $g_0$ --- самого маленького поворота! В частности, отсюда следует и то, что порядок самой группы $G'$ равен порядку этого наименьшего поворота $g_0$ (т.е. поворота с наибольшим порядком).
\item Итак, произвольная конечная группа движений окружности:
\begin{enumerate}[a)]
\item либо тривиальна, т.е. совпадает с $\{\id\}$,
\item либо является циклической группой поворотов $\langle g_0\rangle$, совпадающей с группой поворотов правильного $n$-угольника, где $n$ --- порядок этой группы (включая вырожденный случай 2-угольника),
\item либо является группой одного отражения $\{\id,S_\ph\}$,
\item либо есть объединение $\langle g_0\rangle\cup h\langle g_0\rangle$, где $h$ --- некоторое отражение того же самого правильного $n$-угольника.
\end{enumerate}
\item Наконец, заметим, что и тривиальная группа, и циклическая конечная группа поворотов порядка $n$, и группа одного отражения $\{\id,S_\ph\}$ (здесь важно отметить, что для согласования $S_\ph$ с многоугольником нужно, чтобы ось отражения проходила через вершину или середину строны многоугольника), и наиболее полная группа $\langle g_0\rangle\cup h\langle g_0\rangle$  --- все они являются подгруппами группы движений правильного $m$-угольника, где $m\vdots n$. Отсюда следует, что все конечные группы движений окружности являются подгруппами движений правильных многоугольников, лежащих на данной окружности.
\end{enumerate}

\subsection*{Задачи}
\begin{enumerate}
\item Доказать, что $\langle g_0\rangle\cap h\langle g_0\rangle = \emptyset$, т.е. группа движений распадается на два равномощных класса. один из которых получается применением отражения ко второму.
\item Пусть $G$ --- коммутативная группа, $g\in G$ и $H$ --- подгруппа группы $G$. Доказать, что множество $gH$ равномощно множеству $H$.
\item Вывести из предыдущего \textbf{теорему Лагранжа}: порядок подгруппы делит порядок группы.
\item Обобщить результат на некоммутативные группы.
\end{enumerate}

\section{Симметрии ромба, группа Клейна}

\subsection*{Конспект}
\begin{enumerate}\setlength{\itemsep}{1pt}
\item Рассматриваем ромб, не являющийся квадратом.
\item Движения ромба состоят из:
\begin{enumerate}[a)]
\item двух симметрий: относительно его диагоналей, обозначим эти симметрии $S_1$ и $S_2$;
\item одного вращения: на угол $\pi$, обозначим это вращение $R$;
\item тождественного преобразования $\id$.
\end{enumerate}
\item Других движений ромба не существует. Докажем это.

Пронумеруем вершины ромба цифрами 1,2,3,4 (1 и 3 противоположны). Предположим, что при некотором преобразовании 1 переходит в 1. В этом случае 3 не может перейти ни в 1, ни в 2 или 4, иначе произойдет потеря инцидентности --- вершина 3 либо совпадет с 1, либо будет соседней. Стало быть, 3 также останется на месте. Но тогда остается ровно два преобразования: $\id$ и симметрия относительно оси 13 (обозначим ее $S_1$).

Очевидно также, что 1 не может перейти в 2 или 4, т.к. в противном случае расстояние 1--3 перейдет в расстояние 2--4, а это невозможно для ромба с различными диагоналями. Остается вариант перехода 1 в 3, который дает два оставшихся преобразования: поворот на $180^o$ и симметрию относительно диагонали 24 (обозначим ее $S_2$).

Если провести аналогичный анализ для остальных вершин, то мы получим те же самые преобразования.
\item Таблица Кэли группы движений ромба:\hfill
\begin{tabular}{c|c|c|c|}
$\id$     & $R$ & $S_1$ & $S_2$ \\
\hline
$R$ & $\id$     & $S_2$ & $S_1$ \\
\hline
$S_1$     & $S_2$     & $\id$ & $R$ \\
\hline
$S_2$     & $S_1$     & $R$ & $\id$ \\
\hline
\end{tabular}

\item Отличие данной группы от группы движений правильного $n$-угольника состоит в том, что группа ромба является коммутативной (абелевой).
\item Эта группа нам еще встретится позднее под именем <<четвернаяя группа Клейна>>, когда мы будем говорить об арифметике остатков.
\end{enumerate}





\chapter{Движения плоскости и пространства}

\vrezka{Данная глава продолжает тему групп движений. Здесь мы получаем теорему Шаля (для движений плоскости), а затем широкими мазками освещаем тему движений сферы и пространства. 

Разделы о сфере и пространстве могут быть пропущены при первом ознакомлении с конспектом.
}

\section{Виды движений плоскости. Теорема Шаля}

\subsection*{Конспект}
\begin{enumerate}\setlength{\itemsep}{1pt}
\item Разбираем движения, попутно доказывая лемму <<о гвоздях>>.
\item Пусть на плоскости три точки, не лежащие на одной прямой, остаются неподвижными при движении. Вывод: это $\id$.
\item Пусть на плоскости неподвижны 2 точки и вся прямая, проходящая через них, остальные точки подвижны. Тогда это симметрия относительно данной прямой.
\item Пусть неподвижна лишь одна точка. Такое возможно лишь при вращении вокруг этой точки на угол, не кратный полному обороту.
\item Пусть вообще нет неподвижных точек. Берем любую точку, смотрим, куда она переходит, применяем сдвиг (параллельный перенос). Оставшееся преобразование имеет как минимум 1 неподвижную точку, а значит, является либо $\id$, либо симметрией, либо поворотом. Интересно, что поворот в данном случае можно исключить, т.к. композиция сдвига и поворота есть просто поворот, а значит, в исходном преобразовании была как минимум одна неподвижная точка. Следовательно, исходное движение есть либо сдвиг, либо смещенная симметрия (композиция сдвига и симметрии).
\item Таким образом, движение плоскости можно рассматривать как комбинацию параллельного переноса (в частности, на нулевой вектор), поворота (в частности, на нулевой угол) и симметрии относительно произвольной прямой.
\item \textbf{Теорема Шаля}. Произвольное движение (без разложения его на компоненты) есть движение одного из следующих классов:
\begin{enumerate}[a)]
\item класс параллельных переносов (на произвольный вектор), который мы обозначим $\rightrightarrows$;
\item класс поворотов относительно произвольного центра, который мы обозначим $\circlearrowleft$;
\item класс \textbf{скользящих симметрий} (сдвиг на произвольный вектор с последующей симметрией относительно оси данного вектора), который мы обозначим $\leftharpoonup\leftharpoondown$.
\end{enumerate}
\item Таблица композиций для таких классов выглядит следующим образом:
\begin{center}
\begin{tabular}{c|ccc}
 & $\rightrightarrows$ & $\circlearrowleft$ &  $\leftharpoonup\leftharpoondown$ \\ \hline
$\rightrightarrows$ & $\rightrightarrows$ &  $\circlearrowleft$ &  $\leftharpoonup\leftharpoondown$  \\ 
$\circlearrowleft$ & $\circlearrowleft$ & $\rightrightarrows$ или $\circlearrowleft$ & $\leftharpoonup\leftharpoondown$  \\ 
$\leftharpoonup\leftharpoondown$ & $\leftharpoonup\leftharpoondown$ & $\leftharpoonup\leftharpoondown$ & $\rightrightarrows$ или $\circlearrowleft$  \\ 
\end{tabular}
\end{center}
\item Аналогично одномерным случаям (прямая и окружность) можно выбирать различные базовые преобразования для построения с их помощью всех движений. 
\item Всякое движение есть композиция не более трех симметрий (относительно разных и, вообще говоря, не обязательно параллельных осей).
\item Сдвиг можно представить как композицию двух симметрий (относительно параллельных осей).
\item Поворот можно представить как композицию двух симметрий (относительно пересекающихся осей).
\item Скользящую симметрию можно представить как композицию трех симметрий (две на сдвиг и одна собственно симметрия).
\end{enumerate}


\subsection*{Задачи}
\begin{enumerate}
\item Показать, что композиция поворотов (относительно разных центров) есть либо сдвиг, либо поворот (вычислить его центр).
\item Показать, что композиция сдвига и поворота есть поворот.
\end{enumerate}



\section{Сравнение движений прямой, окружности и плоскости}
\subsection*{Конспект}
\begin{enumerate}
\item Отметим несколько общих свойств рассмотренных нами движений прямой, окружности и плоскости.
\item Во-первых, их всех можно свести к композиции симметрий. Для одномерных объектов (прямая и окружность) --- не более двух, для двумерных --- не более трех.
\item Во-вторых, все движения можно разделить на два класса: сохраняющие и меняющие \textbf{ориентацию}. Те движения, которые сводятся к композиции четного числа симметрий, сохраняют ориентацию фигур, а те, которые сводятся к композиции нечетного числа симметрий, --- меняют ориентацию фигур. Изменение ориентации означает, что право и лево меняются местами, т.е. мы как бы переходим в зазеркалье. 
\item При этом нужно отметить, что преобразования, меняющие ориентацию, обязательно требуют выхода в пространство, если мы хотим осуществить их непрерывным движением.
\item В-третьих, есть и более глубинная связь движений прямой, окружности и плоскости. Мы уже отмечали, что окружность можно рассматривать как прямую, у которой склеили противоположные концы (где-то на бесконечности). И с этой точки зрения сдвиг на прямой является прямой аналогией вращения окружности. Особенно, если величина сдвига сильно меньше радиуса.
\item А симметрия прямой при этом естественным образом превращается в симметрию окружности. Только ось симметрии должна проходить через место склейки двух бесконечностей. Остальные же симметрии можно получить дополнительным сдвигом, т.е. вращением.
\item Далее, окружность находится на плоскости. И поэтому вращение окружности полностью аналогично вращению плоскости, если при этом совместить их центры.
\item Еще проще увидеть совпадения понятий сдвига на прямой и плоскости. В обоих случаях мы просто смещаем все точки на какой-то вектор.
\item Тем не менее, на плоскости появляется новый вид движения, который комбинирует в себе сдвиг и отражение относительно оси сдвига. Это --- скользаящая симметрия, т.е. симметрия с последующим применением сдвига вдоль оси симметрии. На одномерных объектах такое движение в принципе невозможно. На прямой симметрия относительно этой же прямой ничего не дает, т.е. является $\id$, а на окружности симметрия относительно самой окружности вообще требует специального построения в геометрии плоскости.
\end{enumerate}


\section{Векторно-числовое представление движений плоскости}

\subsection*{Конспект}

\begin{enumerate}
\item \textbf{Аффинное пространство} --- множество точек и векторов. В аффинном пространстве мы работаем сразу с двумя сортами объектов --- точками и векторами, на которых заданы операции сложения и вычитания. При этом в сумме $a+b$ и разности $a-b$ могут быть такие комбинации:
\begin{enumerate}[1)]
\item $a$ --- точка, $b$ --- вектор, результатом $a+b$ будет точка, соответствующая концу вектора $b$, когда он отложен от точки $a$, результатом $a-b$ будет точка $c$ такая, что $c+b=a$;
\item $a$ и $b$ --- векторы, результатом $a+b$ будет вектор, построенный по правилу параллелограмма, результатом $a-b$ будет вектор $c$ такой, что $c+b=a$;
\item $a$ и $b$ --- точки, результатом $a-b$ будет вектор с началом в точке $b$ и концом в точке $a$.
\end{enumerate}
\item Движения --- это преобразования точек. Параметром движения может быть вектор и/или угол (число).
\item Сдвиг на плоскости на вектор $a$ обозначим $T_a$. Операция $T_a$ осуществляет прибавление вектора $a$ к точкам плоскости. Композиция сдвигов соответствует сумме векторов сдвига: $T_a\circ T_b=T_{b+a}$.
\item Поворот вокруг нуля мы ранее обозначали $R_\al$, где $\al$ --- угол в радианах.
\item Поворот на угол $\al$ относительно произвольной точки $M$ можно выразить так:
$$
R_{M,\al} = T_{O+M}\circ R_\al\circ T_{O-M},
$$
т.е. сначала сдвигаем точку $M$ в центр вращения, отмеченный точкой $O$, затем производим вращение, затем возвращаем точку $M$ на место обратным сдвигом.
\item Наконец, у нас остается такой вид движения, который осуществляет отражение относительно произвольной прямой на плоскости. Обозначим его $S_l$.
\item Предположим, что на плоскости помимо точки $O$ мы также зафиксировали некоторую прямую, проходящую через $O$ с выделенным направлением $OA$ ($A$ лежит на этой прямой и не совпадает с $O$). Зафиксируем отражение $S_{OA}$ относительно данной выбранной оси $OA$. Отметим, что $S_{OA}=S_{AO}$, т.е. отражение не зависит от направления оси отражения.
\item Выразим произвольное отражение через базовое отражение $S_{OA}$ и другие движения. Для этого обозначим через $M$  произвольную точку прямой $l$, через $\al$ --- угол наклона прямой $l$ относительно направления $OA$. Тогда
$$
S_l = T_{O+M}\circ R_\al \circ S_{OA} \circ R_{-\al}\circ T_{O-M},
$$
т.е. сначала мы сдвигаем плоскость так, чтобы точка $M$ оказалась в точке $O$, затем выполняем поворот на угол $-\al$, далее выполняем стандартное отражение, а затем производим обратные операции, которые возвращают прямую $l$ на место.
\item Соответственно, скользящая симметрия, при которой выполняется отражение относительно оси $l$ и сдвиг на вектор $MM'$ ($M,M'\in l$), записывается так:
$$
S_l = T_{O+M}\circ R_\al \circ S_{OA} \circ R_{-\al}\circ T_{O-M}\circ T_{M'-M},
$$
\item В терминах движений $T,R,S$ можно записать все возможные виды движений плоскости, т.е. сдвиг на произвольный вектор, поворот на произвольный угол относительно произвоьлной точки, скользящую симметрию относительно произвольной прямой $l$ со сдвигом на произвольный вектор, лежащий на этой прямой.
\item Если мы вернемся на окружность, то нам потребуется исключить сдвиги, оставив только вращения и симметрии.
\end{enumerate}









\section{Пара слов о движениях сферы}
\subsection*{Конспект}
\begin{enumerate}
\item Имея опыт перехода от прямой к окружности, мы можем легко найти движения сферы, отправляясь от движений плоскости.
\item Представим себе сферу как плоскость, у которой бесконечно удаленный край был стянут в точку (метод <<хинкали>>).
\item Во что превращаются при этом движения плоскости?
\item Сдвиг, он же параллельный перенос, превращается в такое движение, при котором все точки движутся по параллельным траекториям. С точки зрения географии это есть движение вдоль широтных линий. Да, проходят они при этом разное расстояние! Из-за чего, кстати, и появляются силы Кориолиса, создающие океанические течения вроде Гольфстрима. Но собственные расстояния между точками сохраняются, и это, несомненно, движение.
\item Вращение, которое, как мы помним, на окружности соответствует сдвигу на прямой, в случае сферы в прямом смысле слова совпадает со сдвигом! Дело в том, что вращение сферы вокруг оси, --- это вращение вокруг полюса, при котором угол поворота измеряется меридианом. Но ведь то же самое движение около экватора есть то, что мы только что отнесли к сдвигам вдоль широтных линий.
\item Таким образом, сдвиг прямой и вращение окружности в случае сферы чудесным образом объединяются в один вид движений --- осевое вращение. И это делает движения сферы чуть проще, чем движения плоскости, где сдвиг можно представить лишь как композицию двух вращений.
\item Далее, симметрия плоскости относительно прямой естественным образом переходит в отражение сферы относительно центральной секущей плоскости или, иначе говоря, относительно окружности большого круга. При такой симметрии полюса сферы меняются местами (полюса определются пересечением со сферой прямой, пересекающей плоскость отражения в центре сферы и перпендикулярной ей), а плоскость отражения остается на месте.
\item Наконец, скользящая симметрия плоскости есть композиция сдвига и осевой симметрии, и ей на сфере соответствует \textbf{зеркальное вращение}, т.е. композиция отражения и вращения параллельно плоскости отражения.
\item Таким образом, все движения сферы распадаются на два класса: вращения и зеркальные вращения. При этом, все движения есть композиция не более чем трех отражений.
\item Этот аналог теоремы Шаля для сферы можно доказать, используя очередную лемму о гвоздях, предполагая неподвижность пары противоположых точек (случай одной точки на плоскости), неподвижность целой окружности большого круга (случай двух точек на плоскости), отсутствие неподвижных точек.
\end{enumerate}

\subsection*{Задачи}
\begin{enumerate}
\item Построить таблицу движений сферы аналогично таблице движений плоскости (символику придумайте сами).
\item **Доказать, что других движений на сфере не существует (лемма о гвоздях).
\end{enumerate}


\section{Пара слов о движениях пространства}
\subsection*{Конспект}
\begin{enumerate}
\item Наконец, мы можем от сферы перейти к пространству. На самом деле, переход в пространство сопровождается лишь добавлением сдвига в пространстве. Т.е. любое движение сферы можно рассматривать как движение пространства с одной неподвижной точкой --- центром сферы. После чего можно применить сдвиг этого центра, и получить новые движения. Понятно, что никаких других движений тут быть не может.
\item Тем не менее, классификация движений пространства становится сложнее примерно так же, как классификация движений плоскости превосходит классификацию движений окружности. А именно, в пространстве появляется \textbf{винтовое движение} как композиция осевого вращения и сдвига вдоль оси вращения. Это --- обобщение скользящей симметрии на плоскости (если винт осуществляет поворот на $180^0$, мы как раз получаем скользящую симметрию).
\item Есть также и собственно \textbf{скользящая симметрия пространства}. Это --- отражение относительно плоскости с последующим сдвигом вдоль направления, параллельного данной плоскости. Такое движение также является обобщением скользящей симметрии на плоскости.
\item Заметим, что более сложное движение винт включает в себя более простые. Так, если винт имеет нулевой сдвиг, то он доставляет осевое вращение, а если винт имеет нулевой поворот, то он доставляет сдвиг. Понятно, что в случае полного зануления параметров винта мы получим $\id$.
\item Точно так же, \textbf{зеркальное вращение}, как и в случае сферы, при нулевом повороте доставляет просто симметрию.
\item Наконец, скользящая симметрия своим частным случаем имеет просто симметрию относительно плоскости.
\item Таким образом, классификация движений пространства включает следующие виды движений:
\begin{enumerate}[a)]
\item винт (в частности, сдвиг, осевое вращение, $\id$);
\item зеркальное вращение (в частности, отражение);
\item скользящая симметрия (в частности, отражение).
\end{enumerate}
\end{enumerate}

\subsection*{Задачи}
\begin{enumerate}
\item Построить таблицу движений пространства аналогично таблице движений плоскости (символику придумайте сами).
\item *Показать, что центральная симметрия пространства --- это зеркальное вращение.
\item **Доказать, что других движений в пространстве не существует (лемма о гвоздях).
\end{enumerate}


\begin{sidewaystable}
\caption{Сравнение движений.}
\label{Transitions}
\begin{tabular}{p{2cm}|p{2.5cm}p{2.5cm}p{2.5cm}p{2cm}p{2.5cm}p{2.5cm}}
\rowcolor{darkred}
& \multicolumn{3}{P{8.5cm}}{\textcolor{white}{\bfseries Собственные движения\linebreak (не меняют ориентацию)}} & \multicolumn{3}{P{8cm}}{\textcolor{white}{\bfseries Несобственные движения\linebreak (меняют ориентацию)}} \\ 
& Перенос & Поворот & Смещение поворота & Симметрия & \multicolumn{2}{p{5cm}}{Смещенная симметрия} \\ \hline \hline
Прямая     & сдвиг на число & & & относи\-тель\-но точки & & \\  \hline
Окруж\-ность & \multicolumn{2}{p{5cm}}{\centerline{вращение}} & & осевая симметрия & & \\ \hline
Плос\-кость  & параллель\-ный перенос & относи\-тель\-но точки & & осевая симметрия & скользящая симметрия (перенос+ сим\-мет\-рия) & \\  \hline
Сфера & вращение вблизи экватора & вращение вблизи полюса & & отражение относительно плоскости & \multicolumn{2}{p{5cm}}{зеркальное вращение (вращение+симметрия)} \\ \hline
Прост\-ранство & параллель\-ный перенос & осевое вращение & винт (перенос + вращение) & отражение относительно плоскости & скользящая симметрия (перенос+ сим\-мет\-рия) & зеркальное вращение (вращение+ сим\-мет\-рия) \\ \hline \hline
\end{tabular}
\end{sidewaystable}













