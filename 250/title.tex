{\thispagestyle{empty}
%\clearpage
%\setcounter{page}{0}


\begin{flushright}
\quad

\vspace{2cm}

{\fontsize{100pt}{0pt}\bfseries\sffamily ВВЕДЕНИЕ В\\[20pt] МАТЕМАТИКУ}

\vspace{15mm}

{\fontsize{25pt}{0pt}\bfseries\sffamily Уровень АЛЬФА}

\vspace{2cm}

{\fontsize{20pt}{22pt}\sffamily конспект лекций для учителей математики\\
по мотивам <<100 уроков математики>> Алексея Савватеева\\[5pt]
}

\vspace{2cm}

{\Large\bfseries\sffamily 	Н.~И.~Казимиров, А.~В.~Савватеев}

\vfill

{\Large\bfseries\sffamily 	Москва, \number\year}
\end{flushright}
}



\shorttableofcontents{Оглавление}{0}
\markboth{}{}

\clearpage
\renewcommand*\contentsname{\vspace{-20mm}\quad\hfill\Large\bfseries\sffamily\MakeUppercase{Содержание}\vspace{2mm}\textcolor{darkred}{\hrule}\thispagestyle{empty}}
\tableofcontents
 

\nochapter{Аннотация к курсу}

Данный курс (уровень Альфа) рассчитан на широкую аудиторию школьников, студентов, учителей математики и физики и всех, кто интересуется математикой. Компоновка и содержание курса во многом повторяют <<100 уроков математики>>, прочитанные А.~В.~Савватеевым в Филипповской школе в 2014-2018 годах.

Начальные главы курса представляют собой графическое введение в базовые математические концепции, такие как сложение, умножение, упорядочение чисел. Кроме того, на примере движений прямой, окружности и плоскости формируется понятие числа как преобразования.
Практически сразу появляется алгоритм Евклида, основная теорема арифметики и цепные дроби.
С первых же глав вводятся понятия группы и кольца чисел на примере групп движений и отражений.

Главы 1--5 доступны на уровне школьной программы 5-6 классов общеобразовательной школы.

Далее мы приступаем к построению рациональных чисел, решению линейных уравнений в целых числах, разрабатываем теорию делимости, изучаем кольцо вычетов по модулю. Довольно подробно в курсе изучаются перестановки, гауссовы целые числа, подобия плоскости.

Главы 6--11 доступны на уровне 7-9 классов общеобразовательной школы.

Начиная с 12 главы углубление в математику становится необратимым. Мы рассматриваем векторные пространства, линейные операторы, матрицы и постепенно подбираемся к построению континуума. Рассматривается понятие плотного множества, непрерывного упорядоченного поля, дается несколько формулировок аксиомы полноты.

Заканчивается курс введением в математический анализ и построением комплексной экспоненты с доказательством формулы Эйлера.

Главы 12--15 доступны старшим школьникам общеобразовательной школы, планирующим поступать в физико-математические вузы.


Помимо разбиения на главы и разделы, курс разделен на уроки, которые пронумерованы и снабжены краткой аннотацией содержания. Разбиение на уроки примерно соответствует распределению материала в видеокурсе Алексея Савватаеева.

Некоторые разделы отмечены звездочкой, что означает их повышенную сложность. Обычно они связаны с теорией множеств или высшей алгеброй. Дополнительно такие разделы помечены на полях цветной полоской.

В каждой главе есть набор задач, которые можно решать самостоятельно или рекомендовать преподавателям математики для разбора на занятиях. 

Большое внимание в курсе уделяется подготовке читателя к языку, методам и символике высшей математики. Глава 0, стоящая особняком в курсе, посвящена именно этой цели. Эта глава не является неотъемлемой частью курса, но будет полезна всем, кто хочет разбираться в математических текстах и связать свое дальнейшее образование с математикой.

Авторы выражают глубокую благодарность за помощь и конструктивную критику при составлении курса Павлу Иванову, Егору Кузьмичеву, Михаилу Бочкареву и команде проекта \href{http://childrenscience.ru/}{<<Дети и наука>>}.

\quad

\quad \hfill \today



